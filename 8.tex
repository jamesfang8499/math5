
\chapter{空间解析几何初步}
\section{空间向量的坐标运算}
\subsection{空间直角坐标系与向量运算}
任取一点$O$(图8.1), 一个单位长,通过$O$点建立
三条互相垂直的数轴,$X$轴、$Y$轴、$Z$轴,并且使这三个数
轴的正方向构成右手系。这样我们
就说在空间建立了一个空间右手坐
标系,并用$OXYZ$来表示。$O$点
叫做坐标系的原点。$X$轴、$Y$轴、
$Z$轴总称为坐标轴。三个坐标轴每
两个决定一平面叫做坐标平面。坐标平面共有三个$OXY$、$OYZ$、
$OZX$,它们互相垂直并且把空间分为八个区域,每个区域叫做一个\textbf{卦限}。

\begin{figure}[htp]\centering
    \begin{minipage}[t]{0.48\textwidth}
    \centering
\begin{tikzpicture}[>=latex, scale=1]
\draw[<->](0,3.5)node[right]{$Z$}--(0,0)node [below right]{$O$}--(3,0)node[right]{$Y$};  
\draw[dashed](-2,0)--(0,0)--(1.5,1.5);
\draw[dashed](0,0)--(0,-1);
\draw[->](0,0)--(-1.5,-1.5)node[right]{$X$};
    \end{tikzpicture}
    \caption{}
    \end{minipage}
    \begin{minipage}[t]{0.48\textwidth}
    \centering
    \begin{tikzpicture}[>=latex, scale=1]
\draw[<->](0,3.5)node[right]{$Z$}--(0,0)--(3,0)node[right]{$Y$};  
\draw[->](0,0)--(-1.5,-1.5)node[left]{$X$};
\tkzDefPoints{0/0/O, 2/0/B, 2/2.5/P', 0/2.5/C, -1/-1/A}
\tkzDefPointsBy[translation= from O to A](B,P',C){B',P,C'}
\tkzDrawPolygon(B',P,C',A)
\tkzDrawPolygon(B,P',C,O)
\tkzDrawSegments(P,P' C,C' B,B')
\tkzLabelPoints[below](A,O,B)
\tkzLabelPoints[right](P)
\tkzLabelPoints[left](C)
    \end{tikzpicture}
    \caption{}
    \end{minipage}
    \end{figure}

设$P$是空间中任一点,通过$P$点作平面分别与坐标平面
$OYZ$、$OZX$、$OXY$平行(图8.2),并且分别与$X$
轴、$Y$轴、$Z$轴相交于$A$、$B$、$C$三点,如果$A$、$B$、$C$在
各坐标轴上的坐标分别为$x$、$y$、$z$, 则这三个有序实数组
$(x,y,z)$叫$P$点的\textbf{空间坐标}。简称坐标。$P$
点的坐标是$(x,y,z)$, 
记作$P(x,y,z)$. $x$、$y$、$z$分别叫做$P$点
的$X$坐标,$Y$坐标,$Z$坐标。

































































































































































































































