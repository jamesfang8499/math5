\documentclass[b5paper, openany]{ctexbook}


\usepackage[margin=2.5cm]{geometry}


\usepackage{pifont}
\usepackage[perpage,symbol*]{footmisc}
\DefineFNsymbols{circled}{{\ding{192}}{\ding{193}}{\ding{194}}
{\ding{195}}{\ding{196}}{\ding{197}}{\ding{198}}{\ding{199}}{\ding{200}}{\ding{201}}}
\setfnsymbol{circled}

\usepackage{ulem}

\usepackage{amsmath,amsfonts,mathrsfs,amssymb}
\usepackage{graphicx}

\usepackage[font=bf,labelfont=bf,labelsep=quad]{caption}

\usepackage{tikz}
\usepackage{tikz-3dplot} 
\usepackage{pgfplots}
\usetikzlibrary{3d, calc, math, intersections, patterns}

\makeatletter
\tikzoption{canvas is plane}[]{\@setOxy#1}
\def\@setOxy O(#1,#2,#3)x(#4,#5,#6)y(#7,#8,#9)%
{\def\tikz@plane@origin{\pgfpointxyz{#1}{#2}{#3}}%
	\def\tikz@plane@x{\pgfpointxyz{#4}{#5}{#6}}%
	\def\tikz@plane@y{\pgfpointxyz{#7}{#8}{#9}}%
	\tikz@canvas@is@plane
}
\makeatother  

\tikzset{
	mark position/.style args={#1(#2)}{
		postaction={
			decorate,
			decoration={
				markings,
				mark=at position #1 with \coordinate (#2);
			}
		}
	}
}

\usepackage{ntheorem}
\theoremseparator{\;}



\usepackage{blkarray}
\usepackage{bm}
\usepackage[colorlinks=true, linkcolor=black]{hyperref}

\usepackage{enumerate}


\theoremstyle{plain}
\theoremheaderfont{\normalfont\bfseries} 
\theorembodyfont{\normalfont}


\usepackage[framemethod=tikz]{mdframed}


\newtheorem{example}{\bf 例}[chapter]
\newenvironment{solution}{\noindent {\bf 解:}}{}
\newenvironment{analyze}{\noindent {\bf 分析:}}{}
\newenvironment{rmk}{\noindent {\bf 注意:}}{}
\newenvironment{note}{\noindent {\bf 说明:}}{}



\renewcommand{\proofname}{\bf 证明:}
\newenvironment{proof}{{\noindent \bf 证明:}}{}%{\hfill $\square$\par}

\DeclareMathOperator{\E}{\mathbb{E}}
\renewcommand{\Pr}{\mathbb{P}}
\newcommand{\EP}{\mathbb{E}^{\mathbb{P}}}
\newcommand{\EQ}{\mathbb{E}^{\mathbb{Q}}}
\newcommand{\dif}{\,{\rm d}}
\DeclareMathOperator{\Var}{Var}
\DeclareMathOperator{\Cov}{Cov}
\newcommand{\x}{\times}


\usepackage{tcolorbox}
\tcbuselibrary{breakable}
\tcbuselibrary{most}



\newtcolorbox{ex}[1][]
{colback = white, colframe = cyan!75!black, fonttitle = \bfseries,
	colbacktitle = cyan!85!black, enhanced,
	parbox = false,
	attach boxed title to top center={yshift=-2mm},breakable, 
	title=练习, #1}

\newtcolorbox{blk}[2][]
{colback = white, colframe = magenta!75!black, fonttitle = \bfseries,
	colbacktitle = magenta!85!black, enhanced,
	parbox = false,
	attach boxed title to top left={xshift=5mm, yshift=-2mm},breakable, 
	title=#2, #1}


\setcounter{tocdepth}{2}

\setcounter{secnumdepth}{3}



\ctexset {
	section = {
		name = {第,节},
		number = \chinese{section}},
	subsection = {
		name = {,、\hspace{-1em}},
		number = \chinese{subsection}
	},
	subsubsection = {
		name = {(,)\hspace{-1em}},
		number = \chinese{subsubsection},
	}
}



\renewcommand{\contentsname}{目~~录}

\newcommand{\poly}{\polynomial[reciprocal]}



\usepackage{mathtools}

\setlength{\abovecaptionskip}{0.cm}
\setlength{\belowcaptionskip}{-0.cm}

\usetikzlibrary{decorations.pathmorphing, patterns}
\usetikzlibrary{calc, patterns, decorations.markings}
\usetikzlibrary{positioning, snakes}


\usepackage{tkz-base, tkz-euclide}
\usepackage{yhmath}   % \wideparen
\usepackage{longdivision}
\usepackage{polynom}
\usepackage{polynomial}
\usepackage{cancel}

\renewcommand{\frac}{\dfrac}
\newcommand{\oc}{$^{\circ}{\rm C}$}
\renewcommand{\Vec}{\overrightarrow}

\usepackage{multicol}
\usepackage{cases}

\newcommand\parallelogram{%
	\mathord{\text{%
			\tikz[baseline]
			\draw(0em,.1ex)--++(.9em,0ex)--++(.2em,1.2ex)--++(-.9em,0ex)--cycle;
	}}
}

\renewcommand\parallel{\mathrel{/\mskip-2.5mu/}}


\newcommand{\eY}{\vec{e}_y}
\newcommand{\eX}{\vec{e}_x}
\newcommand{\eZ}{\vec{e}_z}
\newcommand{\va}{\vec{a}}
\newcommand{\expval}[1]{\left\langle #1 \right\rangle}





\begin{document}




\title{\Huge\bfseries 中学数学实验教材\\第五册(下)}



\author{\Large 中学数学实验教材编写组编}
\date{\Large 1985年5月}

\maketitle




\frontmatter

\input{preface.tex}
\tableofcontents


\mainmatter

%  \chapter{直线和平面}
\section{空间图形}
在初中几何课中,我们讨论的几何图形和几何问题,主
要局限在一个平面上。然而,我们生活着的现实世界却是具
有长、宽、高的所谓“三维空间”。而平面只是我们生活空
间的某个断面。我们日常所见的形体,如桌椅、房屋、球、
漏斗等等,它们并不局限在一个平面上,都是立体的。为了
精确地认识这些实际存在的形体,以便为我们的生产和生活
服务,我们必须学习有关空间图形的知识。

我们知道平面图形是同一平面上的点的集合,而\textbf{空间图
形却是不全在同一平面上的点的集合}。例如正方体、三棱
锥、圆柱、球等都是空间图形。空间图形也叫立体图形。我
们将在平面几何知识的基础上,来研究空间图形的性质、画
法以及有关的计算和应用。

\begin{figure}[htp]
    \centering
%\includegraphics[scale=.4]{fig/1-1.png}
    \caption{}
\end{figure}

空间图形是在平面上(纸上)画出来的表示立体的图。
如在图1.1中,各图都表示出了某种立体。这种图形叫直
观图。除了用直观图表示立体以外,还有其他表示立体的方
法,如展开图和投影图等等。下面我们作些简要的介绍。

\subsection{直观图}
\subsubsection{水平放置的平面图形的画法}
当我们把一个正方形和圆放置在水平位置观察时,我们
的视觉会产生一些变化,总觉得它们像平行四边形和椭圆。
它们会变成怎样的平行四边形和椭圆呢?由于观察的角度不
同,会有不同的形状,这就涉及到了水平平面图形的画法问
题,下面就是较常用的两种画法。

第一种画法

\begin{example}
    画正方形的直观图。
\end{example}

\begin{figure}[htp]
    \centering
    \begin{tikzpicture}[>=latex,scale=1.5]
     \begin{scope}
\draw[<->](1.5,0)node[right]{$X$}--(0,0)--(0,1.5)node[right]{$Y$};
\draw(0,0)node[below]{$A$}--(1,0)node[below]{$B$}--(1,1)node[right]{$C$}--(0,1)node[left]{$D$};
\node at (.8,-.5,0){$(1)$};
\end{scope}   
\begin{scope}[xshift=3cm, x={(0:1cm)},y={(45:.5cm)},z={(90:1cm)}]
    \draw[<->](2,0,0)node[right]{$X'$}--(0,0,0)--(0,2,0)node[right]{$Y'$};
    \draw(0,0,0)node[below]{$A'$}--(1,0,0)node[below]{$B'$}--(1,1,0)node[right]{$C'$}--(0,1,0)node[left]{$D'$};
    \node at (1.5,-1.414,0){$(2)$};
    \end{scope}      
    \end{tikzpicture}
    \caption{}
\end{figure}

\textbf{画法}
\begin{enumerate}
    \item 在正方形$ABCD$上,分别取$AB$、$AD$为$X$轴
和$Y$轴,它们互相垂直于$A$点。

画对应的$X'$轴和$Y'$轴,使$\angle X'A'Y'=45^{\circ}$.
\item 在$X'$轴上截取$A'B'=AB$, 在$Y'$轴上截取
$A'D'=\frac{1}{2}AD$

\item 以$A'B'$、$A'D'$为邻边画平行四边形
$A'B'C'D'$就是正方形$ABCD$的直观图。(图1.2)。
\end{enumerate}

\begin{example}
    画正五边形的直观图。
\end{example}


\begin{figure}[htp]
    \centering
    \begin{tikzpicture}[>=latex,scale=1.2]
     \begin{scope}
\draw[<-](0,4)node[right]{$Y$}--(0,-.5); 
\draw[->](-2,0)--(2,0)node[right]{$X$};
\draw(-1,0)node[below]{$A$}--(1,0)node[below]{$B$}--(1.62,1.9)node[right]{$C$}--(0,3.08)node[right]{$D$}--(-1.62,1.9)node[left]{$E$}--(-1,0);
\draw[dashed](-1.62,1.9)--(-1.62,0)node[below]{$E_1$};
\draw[dashed](1.62,1.9)--(1.62,0)node[below]{$C_1$};
\node at (.25,-.25){$O$};
\node at (0,-1){$(1)$};
\end{scope}   
\begin{scope}[xshift=5cm, x={(0:1cm)},y={(45:.5cm)},z={(90:1cm)}]
    \draw[<-](0,6)node[right]{$Y'$}--(0,-1); 
    \draw[->](-2,0)--(3,0)node[right]{$X'$};
    \draw(-1,0)node[below]{$A'$}--(1,0)node[below]{$B'$}--(1.62,1.9)node[right]{$C'$}--(0,3.08)node[right]{$D'$}--(-1.62,1.9)node[left]{$E'$}--(-1,0);
    \draw[dashed](-1.62,1.9)--(-1.62,0)node[below]{$E'_1$};
    \draw[dashed](1.62,1.9)--(1.62,0)node[below]{$C'_1$};
    \node at (.25,-.5){$O'$};
    \node at (1,-2.828,0){$(2)$};
    \end{scope}      
    \end{tikzpicture}
    \caption{}
\end{figure}

\textbf{画法}
\begin{enumerate}
    \item 取正五边形$ABCDE$的$AB$所在直线为$X$轴,
    取$AB$的中垂线为$Y$轴($Y$轴必过$D$点),两轴交于$O$点.
    (图1.3(1))

    画对应的$X'$轴,$Y'$轴,使$\angle X'O'Y'=45^{\circ}$
    \item 作$CC_1\bot X$轴于$C_1$, 作$EE_1\bot X$轴于$E_1$, 在
    $X'$轴上取$A'B'=AB$, 且使$O'$为$A'B'$中点,并在$X'$轴
    上分别取$C'_1$点和$E'_1$点,使$O'C'_1=OC_1$, $O'E'_1=OE_1$.
    \item 在$Y'$轴上截取$O'D'=\frac{1}{2}OD$, 并引$E'_1E'\parallel
    O'Y'$, 且使$E'_1E=\frac{1}{2}E_1E$, 引$C_1'C'\parallel O'Y'$, 且使
    $C'_1C'=\frac{1}{2}C_1C$.
    \item 连结$A'E'$、$E'D'$、$D'C'$、$C'B'$, 则五边
    形$A'B'C'D'E'$就是正五边形$ABCDE$的直观图。(图
    1.3(2))
\end{enumerate}

通过上面例题,我们可以得出这种画法的规则如下:
\begin{enumerate}
\item 在图形上取互相垂直的$X$轴、$Y$轴,并画出与之
对应的$X'$轴,$Y'$轴,使$\angle X'O'Y'=45^{\circ}$(或$135^{\circ}$), $X'$
轴和$Y'$轴所确定的平面表示水平平面。
\item 图形中平行于$X$轴或$Y$轴的线段(包括在轴上的
线段)分别画成平行于$X'$轴和$Y'$轴的线段。
\item 平行于$X$轴的线段,长度不变;平行于$Y$轴的线
段,长度变为原来的一半。
\end{enumerate}

第二种画法

\begin{example}
   画圆的直观图 

   圆的直观图是一个椭圆,常采用如下近似画法:
\begin{enumerate}
    \item 取$\odot O$的一对互相垂直的直径$AB$, $CD$分别为$X$
轴、$Y$轴,并画出对应的$X'$轴、$Y'$轴,使$\angle X'O'Y'=120^{\circ}$.
\item 在$O'X'$上取$O'A'=OA$, 并取$A'$关于$O'$的对称
点$B'$, 然后在$O'Y'$轴上取$O'C'=OC$, 并取$C'$关于$O'$的对
称点$D'$.
\item 过$A'$、$B'$作$O'Y'$的平行线,过$C'$、$D'$作$O'X'$
的平行线,所作四条直线交出一个菱形,设$E$、$F$是菱形关
于$O'$点对称的顶点。
\item 连$ED'$, $FA'$, 交于$G$; 连$EB'$、$FC'$交于$H$.
\item 以$E$为心,以$ED'$为半径,作$\wideparen{D'B'}$, 以$F$为心,
以$FA'$为半径作$\wideparen{A'C'}$, 以$G$为心,以$GA'$为半径作$\wideparen{D'A}$,
以$H$为心,以$HC'$半径作$\wideparen{B'C'}$, 则此四弧连接成一个近似椭
圆.(图1.4(2))
\end{enumerate}

\begin{figure}[htp]
    \centering
    \begin{tikzpicture}[>=latex,scale=1.1]
     \begin{scope}
\draw[<-](0,2)node[right]{$Y$}--(0,-2); 
\draw[->](-2,0)--(2,0)node[right]{$X$};
\draw(-1.5,-1.5)--(-1.5,1.5)--(1.5,1.5)--(1.5,-1.5)--(-1.5,-1.5);
\draw (0,0) circle (1.5);
\node at (-1.7,0)[above]{$A$};
\node at (1.7,0)[above]{$B$};
\node at (0,1.7)[left]{$C$};
\node at (0,-1.7)[left]{$D$};
\node at (-.25,-.25){$O$};
\node at (0,-2.5){$(1)$};
\end{scope}   
\begin{scope}[xshift=6cm, x={(-150:1cm)},y={(150:1cm)}]
    \draw[->](0,2.5)--(0,-2.5)node[right]{$Y'$}; 
    \draw[->](-2.5,0)--(2.5,0)node[left]{$X'$};
    \draw(-1.5,-1.5)--(-1.5,1.5)--(1.5,1.5)--(1.5,-1.5)--(-1.5,-1.5);
    \draw (0,0) circle (1.5);
    \node at (-1.5,0)[above]{$B'$};
    \node  at (1.5,0)[below]{$A'$};
    \node  at (0,1.5) [left]{$D'$};
    \node  at (0,-1.5)[below]{$C'$};
    \node  at (-.25,-.25){$O'$};
    \node  at (.55,.65){$G$};
    \node  at (-.65,-.55){$H$};
    \draw[dashed](0,1.5)--(1.5,-1.5)node[below]{$E$}--(-1.5,0);
    \draw[dashed](1.5,0)--(-1.5,1.5)node[above]{$F$}--(0,-1.5);
    \node at (2.25,-2.5){$(2)$};
    \end{scope}      
    \end{tikzpicture}
    \caption{}
\end{figure}

从上述画法中,我们看出圆的中心$O$, 变成了椭圆的中
心$O'$, 圆的一对互相垂直的直径(如$AB$、$CD$)变成椭圆
的一对直径(如$A'B'$, $C'D'$),它们叫做椭圆的共轭直
径,实际上如果知道了一对椭圆的共轭直径,就可以把椭圆
近似地画出来了。

更省事的办法是用椭圆模板(图1.5)经过椭圆的一对
共轭直径端点来画椭圆。
\begin{figure}[htp]
    \centering
%\includegraphics[scale=.7]{fig/1-5.png}
    \caption{}
\end{figure}

从上例中可以看到:\textbf{第二种画法与第一种画法不同的,只
是$\angle X'O'Y'=120^{\circ}$;并且平行于$Y$轴的线段的长度不变}.

注意:直观图画好以后,要擦去辅助线。
\end{example}

\begin{ex}
\begin{enumerate}
\item 任意画一个三角形,然后用两种方法画出它们的直观图。
\item 任意画一个长方形,然后用两种方法画出它们的直观图。
\item 已知椭圆的一对共轭直径$AA'$、$BB'$, 近似地画出这个
椭圆。
\item 画一个边长为1.5cm的正六边形,然后用两种方法画出
它的直观图。
\end{enumerate}

(注意:以上练习只要求精确而美观地画出图形,不要
求写作法。)
\end{ex}

\subsubsection{空间形体的直观图}
\begin{example}
    画一个长、宽、高分别为3cm, 2cm和1.5cm的长
方体的直观图。
\end{example}

第一种画法:
\begin{enumerate}
    \item 在水平平面上画$X'$轴和$Y'$轴,两轴交于$O'$点,
    且使$\angle X'O'Y'=135^{\circ}$.
    \item 在$O'X'$上取$O'P'=1$cm, 在$O'Y'$上取$O'S'=
    3$cm, 作$S'R'\parallel O'X'$, $P'R'\parallel O'Y'$, 则平行四边形$O'P'R'S'$
    为已知长方体的底面的直观图。
    \item 过$O'$点作$Z'$轴垂直于$Y'$轴,并在$O'Z'$上取$OO'
    =1.5$cm, 过$P'$、$R'$、$S'$分别作$PP'\parallel O'Z'$, $RR' \parallel O'Z'$,
    $SS'\parallel O'Z'$, 并且,$PP'=RR'=SS'=1.5$cm.
    \item 连$OP$、$PR$、$RS$、$SO$, 则$PRSO-P'R'S'O'$
    为已知长方体的直观图。(图1.6)    
\end{enumerate}

\begin{figure}[htp]
    \centering
    \begin{tikzpicture}[>=latex]
 \tkzDefPoints{0/0/O', 3/0/S', 2.25/-.75/R', -.75/-.75/P', 0/1.5/O}
\tkzDefPointsBy[translation = from O' to O](S',P',R'){S,P,R}
\tkzDrawPolygon(P,R,S,O)
\tkzDrawSegments[dashed](O,O' O',P' O',S')
\tkzDrawSegments(P',R' R',S' P,P' R,R' S,S')
\draw[->](O)--(0,2.5)node[right]{$Z'$};
\draw[->](S')--(4,0)node[right]{$Y'$};
\tkzLabelPoints[above left](P,R,S,O)
\tkzLabelPoints[below right](P',R',S',O')       
\draw[->](P')--(-1.5,-1.5)node[right]{$X'$};
    \end{tikzpicture}
    \caption{}
\end{figure}

第二种画法
\begin{enumerate}
    \item 作$\angle X'O'Y'=120^{\circ}$, 并取$X'$轴和$Y'$轴的对称
    轴$Z'$轴为铅直线。
    \item 作已知长方体底面的直观图$O'P'R'S'$; 在
    $O'Z'$上取$OO'=1.5$cm.
    \item 分别过$P'$、$R'$、$S'$作$O'Z'$的平行线$PP'$、$RR'$、
    $SS'$、并使$PP'=RR'=SS'=OO'$.
    \item 连结$OP$、$PR$、$RS$、$SO$, 则$OPRS-O'P'R'S'$为已知长方体的直观图。(图1.7)
\end{enumerate}

\begin{figure}[htp]
    \centering
    \begin{tikzpicture}[>=latex]
\tkzDefPoints{0/0/O', 0/1.5/O}
\tkzDefPoint(-150:2){P'} \tkzDefPoint(-30:3){S'}
\tkzDefPointsBy[translation = from O' to O](S',P'){S,P}
\tkzDefPointsBy[translation = from O' to P'](S'){R'}
\tkzDefPointsBy[translation = from O to P](S){R}
\tkzDrawPolygon(O,S,R,P)
\tkzDrawSegments[dashed](O,O' O',P' O',S')
\tkzDrawSegments(P',P R',R  S,S' P',R' R',S')
\tkzLabelPoints[below](P', R',S',O')
\tkzLabelPoints[above](P,R,S)        
\tkzLabelPoints[above right](O) 
\draw[->](O)--(0,2.5)node[right]{$Z'$};  
\draw[->](P')--(-150:3)node[right]{$X'$};
\draw[->](S')--(-30:4)node[right]{$Y'$};
    \end{tikzpicture}
    \caption{}
\end{figure}

通过例1.4我们看到,画空间形体的直观图规则是在平面
图形的直观图画法的基础上发展起来的。\textbf{它多了一个$Z'$轴,
并且平行于$Z'$轴的线段的平行性和长度都不变。在第一种
画法中,$O'X'$, $O'Y'$, $O'Z'$中,$\angle X'O'Y'=135^{\circ}$,
 $\angle Z'O'Y'=90^{\circ}$, 并使$O'Z'$在铅直位置。第二种画法中,
使$\angle X'O'Y'=\angle Y'O'Z'=\angle X'O'Z'=120^{\circ}$, 且使$O'Z'$轴
居铅直位置.在两种画法中,平行于$O'X'$轴和$O'Y'$轴的线
段长度的变化与画平面图形的直观图的规定相同}。在图中,
$X'O'Y'$表示水平平面,$Y'O'Z'$平面和$X'O'Z'$平面都表
示直立的平面。

\begin{example}
画底面半径为1.5cm,高为2.5cm的圆锥的直观图。

画法(略)见图1.8.
\begin{figure}[htp]
    \centering

    \caption{}
\end{figure}
\end{example}

\begin{rmk}
\begin{enumerate}
    \item 如果要画的空间形体中的面上有圆,一般采用第二种画法来画它的直观图。
    \item 第二种画法画圆的直观图,所得椭圆,由于较宽,不直观,故有时用较扁的椭圆来代替。
\end{enumerate}
\end{rmk}

\begin{ex}
\begin{enumerate}
    \item 用第一种画法画一个棱长为3cm的立方体的直观图.
    \item 用第二种画法画一个长、宽、高分别为4cm, 3cm, 2cm的长方体的直观图。
    \item 画一个底面半径1cm, 高为2.5cm的圆柱的直观图。
    \item 右面是用第一种画法画出来表示某立体的直观图,试根据图上标的尺寸,用第二种画法画出这个立体的直观图来。(单位mm)
    \item 左图是用第二种画法画
    出的某立体的直观图,试根据图上标的尺寸(单位mm),用第一种画法画出该立体的直观图。
    \item 画棱长为4cm的正四面体的直观图\footnote{正四面体是由四个正三角形围成的封闭立体。}。
\end{enumerate}
\end{ex}

\begin{figure}[htp]
    \centering
    \begin{minipage}[t]{0.48\textwidth}
    \centering
    \begin{tikzpicture}[>=latex, scale=1]

    \end{tikzpicture}
    \caption*{第4题}
    \end{minipage}
    \begin{minipage}[t]{0.48\textwidth}
    \centering
    \begin{tikzpicture}[>=latex, scale=1]
  
    \end{tikzpicture}
    \caption*{第5题}
    \end{minipage}
  \end{figure}

\subsection{投影图}

\subsubsection{二视图}

在上一节我们谈到了画空间形体的直观图的方法,这种方法,立体感觉强,但立体的各个侧面的形状和大小不容易一下看清楚,因此,也需要从不同的方向来观察空间形体,
从而画出表示该立体的方法,这种方法通常叫投影图法,是画机械和建筑物设计图的常用方法。

\begin{figure}[htp]
    \centering
    
    \caption{}
\end{figure}

在图1.9(1)中,平面$\alpha$是水平平面,平面$\beta$是铅直平面这两个平面的交线是$XY$. 当圆柱垂直于水平面的时候,从正上方向下看,它是一个圆,也可以这样想,在这种观察下,圆柱被视线投影成一个圆,这个圆在$\alpha$平面上,如果我们从平面$\beta$的正前方看圆柱,就看到一个矩形,可设想,这时圆柱被从$\beta$平面正前方发出的视线投影成一个矩形,这个矩形在$\beta$平面上,我们把$\alpha$平面叫作\textbf{俯视图平面},$\beta$平面叫\textbf{主视图平面},$\alpha$和$\beta$的相交的直线$XY$叫\textbf{基线}。画在俯视图平面上的图叫\textbf{俯视图},画在主视图平面上的图叫\textbf{主视图},俯视图和主
视图合起来叫\textbf{二视图}。

通常把俯视图平面旋转$90^{\circ}$, 使俯视图和主视图画在同一个平面内,这就成了图1.9(2), 它表示的是圆柱的二视图。

\subsubsection{三视图}

有些立体图形只用二视图表示还不够,例如像图1.10(1)那样摆法的圆柱的俯视图和主视图都是矩形,这个二视图也可看成是长方体的二视图了,为了区别起见,我们再设一个与俯视图平面和主视图平面都垂直的第三个平面$\gamma$, 从左侧面看这个圆柱,它在$\gamma$平面上的投影是个圆。这第三个平面$\gamma$叫\textbf{左视图平面},在它上面画出的图叫\textbf{左视图}。

主视图、俯视图、左视图三个视图统称三视图。
图1.10(2)就是圆柱的三视图。主视图和俯视图或主视图和左视图都叫二视图,二视图和三视图也叫投影图。

在制图中投影图是不画基线的,如图1.11表示的就是由一个大长方体上挖去一个小长方体后的三视图。

从上面的三视图中,可以清楚地看出三个视图的关系是:

\begin{verbatim}
    主俯两图长对正,
    主左两图高平齐,
    左俯两图宽相等
\end{verbatim}

了解上述三视图的基本关系,我们常常可以从二视图画出第三个视图。

画出图1.12的左视图,并把这个三视图所表示的立体的直观图画出来。

画法:
\begin{enumerate}
    \item 画左视图(略解)。
    
    根据“主左两图高平齐”和“俯左两图宽相等”,先画出左视图轮廓是个矩形,再观察主俯两图细部,看出这个视图所表示的立体是在一个大长方体的顶部正中贯通前后挖去一个小长方体,这样就必须在左视图的轮廓矩形上加一条虚线,这就画出了左视图。
    \item 画直观图。
    
    我们采用第二种画法:(略解)
    
    首先分析观察视图所表示的组成形体的各部分形状,如1中所述的是一个大长方体挖去一个小长方体.然后,根据视图所给的长、宽、高画出这两个长方体的直观图,但要特别注意在视图上所反映的这两个长方体的相互位置关系,最后擦去不需要的线便得到视图所表示的立体的直观图。(见图1.13)
\end{enumerate}

\subsection*{练习}
\begin{enumerate}
    \item 根据下列各二视图想想立体是什么形状,并画出它
    的直观图。
    \item 根据下列各投影图来想象立体的形状,并画出它们
    的直观图。
\end{enumerate}

\subsection*{习题1.1}
\begin{enumerate}
    \item 画出半径为2cm的球的三视图。
    \item 根据左面的二视图,试
    补画出它的左视图,并画视图所表示的立体的直观图。
    \item 试回答下列二视图(1)—(4)中每一个是立
    体直观图(a)—(d)中的哪一个的视图。

    (注:(2)与(3)中的点划线表示的是该视图所表示的
    立体的轴线及中心线)

    \item 补全下列各视图的投影。
    \item 根据下列三视图,用两种方法画出它们所表示的立体直观图。
\end{enumerate}



\begin{example}
    
\end{example}

\begin{example}
    
\end{example}

\begin{example}
    
\end{example}
\begin{solution}
    
\end{solution}


\begin{example}
    
\end{example}

\begin{solution}
    
\end{solution}

\begin{example}
    
\end{example}
\begin{solution}
    
\end{solution}


\begin{example}
    
\end{example}

\begin{solution}
    
\end{solution}


\begin{example}
    
\end{example}

\begin{solution}
    
\end{solution}




\begin{solution}
    
\end{solution}

\begin{solution}
    
\end{solution}


\begin{solution}
    
\end{solution}

\begin{solution}
    
\end{solution}

\begin{solution}
    
\end{solution}

\section{集合运算}
\subsection{复习}
关于集合的初步知识,我们在初中几何中已经学过了,
现在简要地复习一下要点:

\subsubsection{集合}
通常我们把一些确定的、彼此不同的“事物”
作为一个整体考虑时,便说这个整体是一个\textbf{集合},这些事物叫做该集合的\textbf{元素}。

\begin{blk}{问题1}
    下列集合在空间表示什么图形?
\[\begin{split}
 A&=\{X| OX=5{\rm cm},\; O\text{点是三维空间的一个定点,
$X$是三维空间的点}\}\\
B&=\{P| OP<5{\rm cm},\; O\text{点是三维空间的一个定点,
$P$是三维空间的点}\}\\
C&=\{P| OP>5{\rm cm},\; O\text{点是三维空间的一个定点,
$P$是三维空间的点}\}   
\end{split}\]
\end{blk}

\subsubsection{集合关系}

\paragraph{包含关系}

如果集合$A$中的每一个元素也是集合$B$的元素,则称“$A$包含于$B$”或“$B$包含$A$”,也可以说$A$是$B$的子集。记作$A\subseteq B$, 或$B\supseteq A$. 

若$x\in A$, 则$x\in B$; 但$B$中至少有一个元素$y$不属于$A$, 则称$A$为$B$\textbf{真子集}。记作$A\subset B$或$B\supset A$.


\paragraph{相等关系}

如果两个集合$A,B$由共同的元素构成,我们说这两个集合相等,记作$A=B$. 

判定两个集合相等的方法是:若$A\subseteq B$且$B\subseteq A$, 则$A=B$. 

\begin{blk}{问题2}
    若$A\subset B$且$B\subset A$, 则$A$与$B$是否相等?为什么?
\end{blk}

\paragraph{空集、全集与补集}

不含有任何元素的集合叫做\textbf{空集},用$\emptyset$表示,空集是任何集合的子集。注意空集与零集合不同,零集合包含一个零元素,记作$\{0\}$, 而空集不含有任何元素.

我们把讨论对象所涉及的整个范围称作“全集”,用$I$表示。讨论的范围如果是整数,那么$I$就表示整数集。讨论的范围如果是实数,那么$I$就表示实数集。讨论的是三维空间,则$I$就表示三维空间。

设$A\subset I$, 则集合$\{x|x\in I,\; x\notin A\}$, 称为$A$的\textbf{补集},记作$\sim A$. 也可记作$\overline{A}$.

\begin{blk}{问题3}
\begin{enumerate}
    \item 若$I$为整数集,$A$是偶数集,则$\sim A$为什么
数集?
\item 若$I$为实数集,$A$为无理数集,则$\sim A$为什
么数集?
\end{enumerate}
\end{blk}

\paragraph{集合的“交”与“并”}

由集合$A$和集合$B$的共同元素所组成的集合称作集合$A$与$B$的\textbf{交集}.记作$A\cap B$,即
\[A\cap B=\{x|x\in A\text{ 且 }x\in B\}\]

由集合$A$的元素或集合$B$的元素合并而成的集合叫作$A$和$B$的\textbf{并集},记作$A\cup B$,即
\[A\cup B=\{x|x\in A\text{ 或 }x\in B\}\]

\begin{blk}{问题4}
    设$A=\{1,a,a^2\}$, $B=\{1,a,b\}$, 假定$a,b$都是实数,并且$A\cap B=\{1, 3\}$, $A\cup B=\{1,a,2a,3a\}$, 求
$a$和$b$的值分别是多少?
\end{blk}


\subsection{集合运算}

我们学过的集合的“交”、“并”、“补”都是集合的运算。

我们学过的数的运算是有算律的,那么集合的运算有什么算律呢?

我们很容易看出以下的事实成立:


















































































%
%   \chapter{柱、锥、台、球}

\section{柱、锥、台、球的定义及性质}
本节将研究由点、直线、平面等元素构成的几何体,常
见的几何体有柱、锥、台、球。

\subsection{柱}

柱面是经常遇到的物体的表面形状。建筑物的
棱柱、圆柱的侧面都是柱面,这些侧面可以看作是由一条直
线运动所产生的。

\begin{blk}{定义}
    一条直线$\ell$在空间作平行于固定方向的运动,但
总和任一固定的曲线$C$相交,所产生的曲面叫做柱面、移动
的直线$\ell$所在的每个位置叫做柱面的母线,而在移动中始终
和母线相交的曲线$C$叫做柱面的准线。
\end{blk}

下面,我们考虑几种简单情况
\begin{enumerate}
\item 准线是一条直线,这时的柱面是一个平面(见图
2.1)。
\item 准线是一个平面多边形,这时的柱面叫做棱柱面
(见图2.2)。
\item 准线是一个圆,这时的柱面叫做圆柱面(图2.3)。
\end{enumerate}

\begin{figure}[htp]\centering
    \begin{minipage}[t]{0.3\textwidth}
    \centering
\begin{tikzpicture}[>=latex, scale=1]
\draw(0,0) rectangle (3.6,2.5);
\foreach \x in {.4,.8,1.2,...,2.8,3.2}
{
    \draw(\x,0)--(\x,2.5);
}
\draw[very thick](0,.5)--(3.6,2)node[right]{$c$};
\node at (3.2,2.2)[right]{$\ell$};
    \end{tikzpicture}
    \caption{}
    \end{minipage}
    \begin{minipage}[t]{0.3\textwidth}
    \centering
    \begin{tikzpicture}[>=latex, scale=1]
\foreach \x in {0,.4,1.2,1.6}
{
    \draw(\x,0)--(\x, 3.5);
}
\draw[dashed](.8,0)--(.8, 3.7);
\foreach \x in {0,1,2}
{
    \draw(0,.5+\x)--(.4,.25+\x)--(1.2,.25+\x)--(1.6,.7+\x);
    \draw[dashed](1.6,.7+\x)--(.8,.9+\x)--(0,.5+\x);
}
\node at (1.6,3)[right]{$\ell$};
\node at (1.2,2.25)[right]{$c$};
    \end{tikzpicture}
    \caption{}
    \end{minipage}
        \begin{minipage}[t]{0.3\textwidth}
    \centering
    \begin{tikzpicture}[>=latex, yscale=.45]
\foreach \x in {0,.4,.8,1.2,1.6}
{
    \draw(\x,0)--(\x, 7);
}
\foreach \x in {.4,.8,1.2,1.6}
{
    \draw[dashed](\x-.2,0)--(\x-.2, 6.5);
}
\foreach \x in {1,2,3}
{
   \draw[dashed](1.6,-.5+\x*2) arc (0:180:.8);
   \draw (1.6,-.5+\x*2) arc (0:-180:.8);
}
\draw (1.6,-.5+3*2) arc (0:180:.8);
\node at (1.6,5.5)[right]{$\ell$};
\node at (1.6,5.2)[below left]{$c$};
    \end{tikzpicture}
    \caption{}
    \end{minipage}
    \end{figure}


以后我们只研究母线和该圆所在的平面垂直的圆柱面,
这种圆柱面叫\textbf{直圆柱面}。

\begin{blk}{定理}
    若平行平面与柱面的母线相交,则交线所成的图
形全等。
\end{blk}

\begin{proof}
    设平行平面$\alpha_1$和$\alpha_2$与柱面的母线相交,$\ell$是柱面上
任一条母线。令$P_1=\alpha_1\cap \ell$, $P_2=\alpha_2\cap\ell$, 我们称$P_1$和$P_2$是对
应点。显然$P_1$对应于$P_2$, $P_2$也对应于$P_1$, 并且不同位置的
两点$P_1$、$P'_1$对应于不同位置的两点$P_2$、$P_2'$(图2.4).

\begin{figure}[htp]
    \centering
    \includegraphics[scale=.7]{fig/2-4.png}
    \caption{}
\end{figure}


于是在平行平面与柱面所截出的两条截线上的点与点之间建立了一一对应关系。

设在两条截线上有任意三对对应
点:$A_1$对应于$A_2$, $B_1$对应于$B_2$, $C_1$对
应于$C_2$, 因为$A_1A_2,B_1B_2,C_1C_2$都
是夹在平行平面$\alpha_1$和$\alpha_2$之间柱面的母
线的一部分,故
\[A_1A_2\mathop{=}^{\parallel} B_1B_2\mathop{=}^{\parallel}C_1C_2\]
从而四边形$A_1A_2B_2B_1$和四边形$B_1B_2C_2C_1$都是平行四边形,
它们的对边相等,所以,$A_1B_1=A_2B_2$, $B_1C_1=B_2C_2$. 又因
为$\angle A_1B_1C_1$与$\angle A_2B_2C_2$的两边平行且同向,所以
$\angle A_1B_1C_1=\angle A_2B_2C_2$。

这样,平行平面和柱面的母线相交,交出的图形总能重
合,因此它们全等。
\end{proof}


\begin{blk}{定义} 
    由封闭的柱面和两个与母线都相交的平行平面所
围成的几何体叫做柱体。

柱体的柱面夹在平行平面之间的部分叫做柱体的\textbf{侧面},
两个平行平面与柱体的相交部分叫做柱体的\textbf{底面}。
    
\end{blk}

现在,我们来研究几个特殊的柱体。

\begin{blk}{定义} 
    如果一个柱体的侧面是棱柱面,那么这个柱体叫
做棱柱。如果一个柱体的侧面是圆柱面,那么这个柱体叫圆
柱体。    
\end{blk}

显然,由这个定义可以得到以下推论:
\begin{enumerate}
\item 棱柱的两个底面是对应边互相平行的全等多边
形。

全等多边形的顶点叫做柱棱的顶点。以全等多边形对应
顶点为端点的线段称为棱柱的侧棱,相邻两侧棱及其所夹的
两底全等多边形的两条对应边所组成的四边形称为棱柱的侧
面。显然,
\item 棱柱的各侧面全是平行四边形。
\item 棱柱的各侧棱平行且相等。
\end{enumerate}


\begin{blk}{定义}
    棱柱按底面是三角形,四边形……分别叫做三棱
柱、四棱柱……

如果棱柱的侧棱垂直于底面,那么这个棱柱叫做直棱
柱,否则叫做斜棱柱。

如果棱柱是直棱柱,并且其底面是正多边形,那么这棱
柱就叫做正棱柱。在正棱柱中,各侧面都是全等的矩形。
\end{blk}
 
\begin{blk}
    {定义}棱柱上不在同一个底面上也不在同一个侧面上的
两个顶点所连结的线段叫做棱柱的对角线;过不在同一个
侧面的两条侧棱作一个平面,这平面截棱柱所得的截面叫做
棱柱的对角面。两底面之间的距离叫做棱柱的高。
\end{blk}

\begin{figure}[htp]
    \centering
    \includegraphics[scale=.7]{fig/2-5.png}
    \caption{}
\end{figure}

图2.5是五棱柱,五边形$ABCDE$和$A'B'C'D'E'$是它
的两个底面,$ABB'A'$、$BCC'B'$等是它的侧面,$AA'$、$BB'$
等是它的侧棱,$HH'$的长是它的高。$A'C$是它的一条对角
线。

棱柱的表示法是写出两个底面各顶点
的字母,中间用一条短横线隔开,如图
2.5的五棱柱就表示作棱柱$ABCDE$-$A'
B'C'D'E'$. 也有用棱柱一条对角线的两
个端点表示棱柱的,如图2.5可表示作五
棱柱$A'C$. 

\begin{blk}{定义} 
    底面是平行四边形的四棱柱叫
    做平行六面体。其中侧棱和底面斜交的叫
    斜平行六面体,侧棱和底垂直的叫直平行六面体。底面是
    矩形的直平行六面体叫做长方体。交于一个顶点的三条棱的
    长相等的长方体叫做正方体,也叫立方体。(图2.6)
\end{blk}

\begin{figure}[htp]
    \centering
\begin{tikzpicture}
\begin{scope}
\tkzDefPoints{0/0/A, 1/0/B, 1.5/.5/C, .5/.5/D}
\tkzDefPoint(.25,2.2){A_1}
\tkzDefPointsBy[translation=from A to A_1](B,C,D){B_1,C_1,D_1}
\tkzDrawPolygon(A_1,B_1,C_1,D_1)
\draw(A)--(B)--(C);
\draw[dashed](A)--(D)--(C);
\draw[dashed](D_1)--(D);
\draw(A_1)--(A);\draw(B_1)--(B);\draw(C_1)--(C);
\node at (.5,-.5){斜平行六面体};
\end{scope}

\begin{scope}[xshift=3cm]
    \tkzDefPoints{0/0/A, .65/-.15/B, 2/0/C, 1.35/.15/D}
\tkzDefPoint(0,2){A_1}
\tkzDefPointsBy[translation=from A to A_1](B,C,D){B_1,C_1,D_1}
\tkzDrawPolygon(A_1,B_1,C_1,D_1)
\draw(A)--(B)--(C);
\draw[dashed](A)--(D)--(C);
\draw[dashed](D_1)--(D);
\draw(A_1)--(A);\draw(B_1)--(B);\draw(C_1)--(C);
\node at (0.75,-.5){直平行六面体};
\end{scope}

\begin{scope}[xshift=6cm]
    \tkzDefPoints{0/0/A, 1/0/B, 1.5/.5/C, .5/.5/D}
\tkzDefPoint(0,2.5){A_1}
\tkzDefPointsBy[translation=from A to A_1](B,C,D){B_1,C_1,D_1}
\tkzDrawPolygon(A_1,B_1,C_1,D_1)
\draw(A)--(B)--(C);
\draw[dashed](A)--(D)--(C);
\draw[dashed](D_1)--(D);
\draw(A_1)--(A);\draw(B_1)--(B);\draw(C_1)--(C);
\node at (.5,-.5){长方体};
\end{scope}

\begin{scope}[xshift=8.5cm]
    \tkzDefPoints{0/0/A, 1.5/0/B, 2/.5/C, .5/.5/D}
\tkzDefPoint(0,1.5){A_1}
\tkzDefPointsBy[translation=from A to A_1](B,C,D){B_1,C_1,D_1}
\tkzDrawPolygon(A_1,B_1,C_1,D_1)
\draw(A)--(B)--(C);
\draw[dashed](A)--(D)--(C);
\draw[dashed](D_1)--(D);
\draw(A_1)--(A);\draw(B_1)--(B);\draw(C_1)--(C);
\node at (1,-.5){正方体};
\end{scope}
\end{tikzpicture}
    \caption{}
\end{figure}




\begin{blk}
    {定义}
平行六面体中不含公共棱的两个面称为相对的
面。
\end{blk}

\begin{blk}{定理} 
   平行六面体相对的两个面全等,而它们所在的平
面互相平行。 
\end{blk}

已知:图2.7平行六面体$ABCD-A_1B_1C_1D_1$中,四边形
$ABB_1A_1$和$DCC_1D_1$是相对的两个面。

求证:$\parallelogram ABB_1A\cong \parallelogram DCC_1D_1$且平面$A_1B\parallel \text{平面}D_1C$.


\begin{figure}[htp]\centering
    \begin{minipage}[t]{0.48\textwidth}
    \centering
\begin{tikzpicture}[>=latex, scale=1.5]
    \tkzDefPoints{0/0/A, 2/0/B, 2.5/.5/C, .5/.5/D}
    \tkzDefPoint(.2,1){A_1}
    \tkzDefPointsBy[translation=from A to A_1](B,C,D){B_1,C_1,D_1}
    \tkzDrawPolygon(A_1,B_1,C_1,D_1)
    \draw(A)--(B)--(C);
    \draw[dashed](A)--(D)--(C);
    \draw[dashed](D_1)--(D);
    \draw(A_1)--(A);\draw(B_1)--(B);\draw(C_1)--(C);
    \tkzLabelPoints[left](A,D,A_1,D_1)
    \tkzLabelPoints[right](B,C,B_1,C_1)
    \end{tikzpicture}
    \caption{}
    \end{minipage}
    \begin{minipage}[t]{0.48\textwidth}
    \centering
    \begin{tikzpicture}[>=latex, scale=1.5]
        \tkzDefPoints{0/0/A, 2/0/B, 2.5/.5/C, .5/.5/D}
        \tkzDefPoint(0,1.5){A'}
        \tkzDefPointsBy[translation=from A to A'](B,C,D){B',C',D'}
        \tkzDrawPolygon(A,B,B',A')
        \draw(A')--(D')--(C')--(B');
        \draw(C')--(C)--(B);
        \draw[dashed](D')--(D)--(A);
        \draw[dashed](C)--(D)--(B');
        \draw[dashed](D)--(B);
        \tkzLabelPoints[left](A,D,A',D')
        \tkzLabelPoints[right](B,C,B',C')
    \end{tikzpicture}
    \caption{}
    \end{minipage}
    \end{figure}



\begin{proof}
    平行六面体的底面是平行四边形。

    $\therefore\quad AB\displaystyle\mathop{=}^{\parallel} DC$

    又$\because\quad$它的侧棱平行且相等,

    $\therefore\quad AA_1\displaystyle\mathop{=}^{\parallel} D_1D$

    $\therefore\quad \text{平面}A_1B {\parallel} \text{平面}D_1 C,\qquad \angle A_1AB=\angle D_1DC$

$\therefore\quad \parallelogram ABB_1A\cong \parallelogram DCC_1D_1$

同理,平面$A_1D\parallel\text{平面}B_1C,\qquad \parallelogram A_1 ADD_1\cong \parallelogram B_1BCC_1$.
\end{proof}

\begin{blk}
    {推论} 平行六面体的任何一对相对的面都可以作它的底
面。
\end{blk}

\begin{example}
    长方体的一条对角线的平方,等于通过同一顶点
的三条棱的平方和。

已知:长方体$DB'$中$DB'$是一条
对角线.(图2.8)

求证:${DB'}^2=AB^2+AD^2+{AA'}^2$
\end{example}

\begin{proof}
    连结$DB$

$\because\quad  AD\bot AB$

$\therefore\quad DB^2=AB^2+AD^2$

$\because B'B\bot\text{平面}AC,\quad DB\subset \text{平面}AC$
    
$\therefore\quad B'B\bot DB$

$\therefore\quad B'D^2={BB'}^2+BD^2$

$\because\quad AA'=BB'$

$\therefore\quad DB^2={AA'}^2+{BD}^2={AA'}^2+AB^2+AD^2$
\end{proof}

\begin{blk}
   {定理} 正棱柱两底中心的连线垂直于底面(请读者自证)
   
   圆柱体的两个底面是相等的圆面,它们所在的平
面平行。
\end{blk}

\begin{blk}{定义} 
    圆柱体两底面之间的距离叫做圆柱的高。

    圆柱体的母线平行且相等,并且垂直于两个底
面,母线的长等于圆柱的高。
\end{blk}

\begin{blk}{定义}
空间图形$F$对给定直线$\ell$为轴的轴对称是具有如
下性质的映射$R_1:F\mapsto F$, 对于任意点$P\in F$, 有它的像$R_1(P)
=P_1\in F$, 并且线段$PP'$被$\ell$垂直平分。这时$\ell$称为$F$的对
称轴。$P$、$P'$叫做关于$\ell$的对称点。

圆柱两底圆心连线垂直于底面,并且是圆柱的对
称轴(简称圆柱的轴)。
\end{blk}

已知:如图2.9圆柱$AD$\footnote{圆柱可用它的两个底面内不在同一条母线上的两个点的字母来
表示。}中,
$O$、$O'$分别是两底上的圆心.

求证:
\begin{enumerate}
    \item $OO'$是圆柱$AD$的对称轴。
    \item $OO'$垂直两底。
\end{enumerate}

\begin{figure}[htp]
    \centering
\begin{tikzpicture}[xscale=.8, yscale=.32]
\draw[dashed] (2,0) arc (0:180:2);
\draw (2,0) arc (0:-180:2);
\draw (0,8) circle (2);
\draw[dashed](1.414,1.414+8)--(1.414,1.414)node[right]{$D$}--(-1.414,-1.414);
\draw(1.414,1.414+8)node[above]{$C$}--(-1.414,-1.414+8)node[above]{$A$}--(-1.414,-1.414)node[below]{$B$};
\draw(-2,0)--(-2,8);\draw(2,0)--(2,8);
\node at (.5,.5+8)[below right]{$\pi$};
\draw[dashed](0,0)node [below]{$O'$}--(0,8)node [above]{$O$};
\draw[dashed](-1.414,-1.414+4)node[left]{$P$}--(1.414,1.414+4)node[right]{$P'$};
\node at (0,4)[below right]{$P_0$};
\end{tikzpicture}
    \caption{}
\end{figure}


\begin{proof}
\begin{enumerate}
    \item 

设点$P$是圆柱$AD$的柱面上
的任一点,过$P$作$PP_0\bot OO'$于$P_0$, 并且
$PP_0$的延长线交圆柱面于另一点$P'$。因为$PP'\cap OO'=P_0$, 所
以$PP'$、$OO'$确定一个平面$\pi$, $\pi$与圆柱面的交线为$AB$ ($P\in
AB$) 和$CD$ ($P'\in CD$), 与两底面的交线为$AC$ ($O\in AC$) 及
$BD$ ($O'\in BD$).

$\because\quad $圆柱的两个底面平行。

$\therefore\quad AC\parallel BD$ 且AC、BD分别为两底面圆的直径。

$\because\quad $圆柱的两底面是全等的圆面。

$\therefore\quad AC=BD\qquad ABDC$是平行四边形。

又$\because\quad AB$垂直两底面。

$\therefore\quad ABDC$为矩形,$\angle CAB=\angle ACD=90^{\circ}$

$\therefore\quad AO\displaystyle\mathop{=}^{\parallel}BO', OC\displaystyle\mathop{=}^{\parallel}O'D$

$\therefore\quad ABO'O$和$O'ODC$是两个全等的矩形

$\because\quad PP_0\bot OO'$于$P$.

$\therefore\quad PP_0=AO=OC=P_0P'$

$\therefore\quad PP'$被$OO'$垂直平分。

如果点$P$取在圆柱的底面上,那么它的像$P'$也在该圆面
上,同样可以证明$PP'$被$OO'$垂直平分。

$\therefore\quad OO'$是圆柱$AD$的对称轴。

\item 由1的证明可知,$OO'\parallel AB$, 而$AB$垂直于圆柱的两
个底面,所以$OO'$垂直于两底面。
\end{enumerate}
\end{proof}

\begin{blk}{定义}
    通过圆柱的轴的截面称为圆柱的轴截面。
\end{blk}

\begin{blk}{推论}
\begin{enumerate}
    \item 圆柱的轴截面是矩形。
    \item 过圆柱对称轴的平面是圆柱的对称平面。
    \item 平行于圆柱的轴的截面是矩形。
\end{enumerate}
\end{blk}

\begin{example}
    一圆柱用平行于圆柱的轴的平面去截它,截面周
界是30cm, 面积为54${\rm cm}^2$, 在底面上截得的一段弧为$120^{\circ}$, 求底面半径和圆柱的高。
\end{example}

\begin{figure}[htp]
    \centering
\begin{tikzpicture}[xscale=.8, yscale=.32]
\draw[dashed] (2,0) arc (0:180:2);
\draw (2,0) arc (0:-180:2);
\draw (0,8) circle (2);
\draw[dashed](1.414,1.414+8)node[above right]{$B$}--(1.414,1.414)node[above right]{$B_1$}--(0,0)node [below]{$O'$}--(0,8)node [above]{$O$};
\draw(-2,0)--(-2,8);\draw(2,0)--(2,8);
\draw[dashed](1.414,1.414)--(1,-1.732)node[below]{$A_1$}--(0,0);
\draw(1,-1.732)--(1,-1.732+8)--(0,8)--(1.414,1.414+8)--(1,-1.732+8)node[below left]{$A$};
\end{tikzpicture}
    \caption{}
\end{figure}

\begin{solution}
    如图2.10. 平行轴$OO'$的截面是矩形$AA_1B_1B$, 其边
$AB$是$\odot O$的弦,所对的圆心角是$120^{\circ}$, 故在$\triangle AOB$中,有
\[\frac{AB}{\sin 120^{\circ}}=\frac{R}{\sin 30^{\circ}}\]
其中$R$为$\odot O$的半径。

$\therefore\quad AB=\sqrt{3}R$

$\because\quad $矩形的另一边$AA_1$为圆柱的高,

$\therefore\quad AA_1=\frac{30}{2}-R\sqrt{3}$

又$\because\quad $矩形的面积为$54{\rm cm^2}$

$\therefore\quad R\sqrt{3}\left(15-R\sqrt{3}\right)=54 \quad \Rightarrow\quad R^2-5\sqrt{3}R+18=0$

解得:$R_1=3\sqrt{3},\quad R_2=2\sqrt{3}$,相应地圆柱高$h_1=6,\quad h_2=9$

答:圆柱底面半径为$3\sqrt{3}$cm, 高为
6cm, 或者半径为$2\sqrt{3}$cm, 高为9cm.
\end{solution}


\begin{ex}
\begin{enumerate}
    \item 柱面可视为准线沿母线方向连续平行移动时所占各位置
    的轨迹。用这句话说明棱柱、圆柱。
    \item 确定一个柱面需要几个条件?
    \item 从四棱柱、五棱柱和$n$棱柱的某一个顶点出发,各能引几
    条对角线?四棱柱、五棱柱和$n$棱柱各有几条对角线?
    \item \begin{enumerate}
    \item 平行六面体是斜四棱柱,斜四棱柱是不是平行六面
体?
\item 长方体是直四棱柱,直四棱柱是不是长方体?
\item 正方体是正四棱柱,正四棱柱是不是正方体?
    \end{enumerate} 

\item 有一个侧面是矩形的棱柱是不是直棱柱?有两个相对侧
面是矩形的$2n$棱柱呢?有两个相邻侧面是矩形的棱柱
呢?
\item 求证:和棱柱各侧棱相交的两个平行截面是两个全等的
多边形。
\item 求证:用平行于圆柱的底面的平面去截圆柱所得的截面
是和底面相等的圆面。
\item 一个圆柱体有多少个对称轴?有多少个对称平面?
\item 用第一种画法画正六棱柱的直观图,用第二种画法画圆
柱的直观图。
\end{enumerate}  
\end{ex}

\subsection{锥}

在前面我们初步地讨论了柱(包括棱柱和圆
柱)的定义及其某些性质。这一节将研究锥(包括棱锥和圆
锥)的定义及其性质,我们从锥面说起。

\subsubsection{锥面}

\begin{blk}
    {定义} 给定平面曲线$C$和$C$所在平面外一点$S$, 设直线$\ell$
经过$S$点运动并且总和$C$相交,则运动的直线$\ell$所产生的曲面
叫做锥面。$S$点叫做锥面的顶点,运动着的直线$\ell$的每一位置
叫作锥面的母线,而在运动中始终和母线相交的曲线$C$叫做
锥面的准线。
\end{blk}

通常,我们只讨论通过顶点$S$总和准线$C$相交的射线运
动时所产生的锥面。

下面我们来考虑几种简单情况:
\begin{enumerate}
    \item 准线是一条线段,这时锥面是一个平面区域,其边
    界是一个角。(图2.11)
    \item 准线是一个多边形,这时的锥面叫\textbf{多面角}。它是由
    从顶点出发通过多边形的各个顶点作射线,这些射线以及每
    两条相邻射线间平面部分所组成的图形。(图2.12)

    通过多边形顶点的母线叫作多面角的\textbf{棱}。锥面的顶点叫
    做多面角的\textbf{顶点}。相邻两棱的平面部分叫做多面角的\textbf{面},在
    每个面内由两条棱组成的角叫作多面角的\textbf{面角}。
    \item 准线是个圆,如果圆面垂直于连接顶点和圆心的直
    线,那么锥面叫作直圆锥面,简称圆锥面。(图2.13)通过
    顶点和准线圆中心的直线是圆锥面的轴。锥面的表示法可以
    用它的顶点字母$S$表示,记作锥面$S$. 也可表示成锥面$S-
    ABCDE$等其中$A,B,C,D,E$等分别是锥面上不同母线上的
    点。
\end{enumerate}

\begin{figure}[htp]\centering
    \begin{minipage}[t]{0.3\textwidth}
    \centering
\begin{tikzpicture}[>=latex, scale=1]
    \foreach \x in {-60,-80,-100,-115,-130}
    {
        \draw(0,0)--(\x:3);
    }
    \node at (0,0)[above]{$S$};
    \draw(-130:2.8)node[left]{$A$}--(-60:2.6)node[right]{$B$};
    \end{tikzpicture}
    \caption{}
    \end{minipage}
    \begin{minipage}[t]{0.38\textwidth}
    \centering
    \begin{tikzpicture}[>=latex, scale=1]
\tkzDefPoints{0/0/A, 1.5/0/B, 2/.5/C, .8/1/D, -.5/.6/E, .8/2.5/S}
\draw(E)--(A)--(B)--(C);
\draw [dashed](C)--(D)--(E);
\tkzDrawLines[add=0 and .5](S,E S,A S,B S,C)
\tkzDrawLines[add=0 and 1, dashed](S,D)
\tkzLabelPoints[below](A,B)
\tkzLabelPoints[above](C,D,E,S)

    \end{tikzpicture}
    \caption{}
    \end{minipage}
    \begin{minipage}[t]{0.25\textwidth}
    \centering
    \begin{tikzpicture}[>=latex, yscale=.4]
        \draw[dashed] (1,0) arc (0:180:1);
        \draw (1,0) arc (0:-180:1);
\tkzDefPoints{1/0/A, -1/0/B, 0/6/S, 0/0/O}
\tkzDrawLines[add=0 and .5](S,A S,B)
\tkzDrawLines[add=0 and .35, dashed](S,O)
\draw(S)--+(-95:9);
\tkzLabelPoints[above](S,O)
    \end{tikzpicture}
    \caption{}
    \end{minipage}
    \end{figure}

\begin{blk}
    {定理} 不通过顶点$S$的两个平行平面与锥面相交,所
    得到的图形相似,其相似比等于顶点到两平行平面的距离之
    比。
\end{blk}

已知:锥面$S-ABC\cdots$, 平行平面$\alpha_1$与$\alpha_2$都不通过顶
点$S$与锥面相交,得到截面图形$F_1$(在$\alpha_1$上)与$F_2$(在$\alpha_2$上),
又$SO_1\bot\alpha_1$于$O_1$, $SO_2\bot\alpha_2$于$O_2$(图2.14)

求证:截面图形$F_1\backsim F_2$, 其相似比
$K=\frac{SO_1}{SO_2}$

\begin{figure}[htp]
    \centering
\includegraphics[scale=.6]{fig/2-14.png}
    \caption{}
\end{figure}

\begin{proof}
任引锥面$S-ABC\cdots$的一条母线$SX$, 由于$F_1$
和$F_2$都是锥面的截面,所以$SX$
必与$F_1$和$F_2$分别交于$X_1$和$X_2$两
点,这时,$X_2$可以看成$F_1$上$X_1$的
对应点,$X_1$也可看成是$F_2$上$X_2$
的对应点。

如果另外再取一条不同于
$SX$的母线$SY$, 同样可以得到
$F_1$和$F_2$上的另一对对应点$Y_1$和
$Y_2$, 显然$X_1\ne Y_1$, $X_2\ne Y_2$, 否
则$SX$与$SY$重合,这将与它们
是不同的母线相矛盾。因此,
$F_1$和$F_2$的点与点之间建立了一一对应关系。

又设$SX$和$SY$所确定的平面为$\pi$, 则$\pi\cap \alpha_1=X_1Y_1$,
$\pi\cap \alpha_2=X_2Y_2$, 因为$\alpha_1\parallel \alpha_2$, 所以$X_1Y_1\parallel X_2Y_2$. 因此:
\begin{equation}
    \frac{X_1Y_1}{X_2Y_2}=\frac{SX_1}{SX_2}
\end{equation}
连结$X_1O_1$, $X_2O_2$, 并令$SX$与$SO_2$所确定的平面为$\pi'$, 
则$\pi'\cap\alpha_1=X_1O_1$, $\pi'\cap\alpha_2=X_2O_2$

$\because\quad \alpha_1\parallel \alpha_2$

$\therefore\quad X_1O_1\parallel X_2O_2$
\begin{equation}
    \frac{SX_1}{SX_2}=\frac{SO_1}{SO_2}
\end{equation}
由(2.1), (2.2)有:
\[\frac{X_1Y_1}{X_2Y_2}=\frac{SO_1}{SO_2}=K\quad\text{(常数)}\]
由相似图形的定义可知:
\[F_1\backsim F_2,\quad \text{并且相似比}K=\frac{SO_1}{SO_2}\]
\end{proof}

\subsubsection{棱锥与圆锥}
\begin{blk}
   {定义} 如果一个锥面的准线是一条封闭曲线则称这个锥
面为封闭锥面。如圆锥面、多面角都是封闭锥面。

如果一个封闭锥面的所有线被一个平面所截,那么由
这个截面和锥面所围成的几何体叫做锥体。锥体平面部分叫
作锥体的底面,顶点和底面的距离叫做锥体的高。 
\end{blk}

下面我们讨论几个特殊的锥体。

\begin{blk}
    {定义}如果一个多面角的所有的棱被一个平面所截,那
么截面和多面角各面所围成的几何体叫作棱锥。原多面角的
顶点叫做棱锥的顶点,截面和多面角相交的部分显然是多边
形,它所围的平面部分叫做棱锥的底面,有公共顶点的各个
三角形的面叫做棱锥的侧面,两个相邻侧面的公共边叫做棱
锥的棱侧。

如果棱锥的底面是三角形、四边形……$n$边形,那么棱锥
就分别叫做\textbf{三棱锥,四棱锥,五棱锥……$n$棱锥}(见图2.15)、棱锥的表示法可以用表示顶点和底面顶点的
几个字母来表示,例如棱锥$S-ACD$. 
\end{blk}

如果棱锥的底面是正多边形,并且由棱锥顶点到它的底
面的垂线经过这个多边形的中心,那么这个棱锥就叫作\textbf{正棱
锥}。

\begin{figure}[htp]
    \centering
\begin{tikzpicture}
\begin{scope}
\draw(0,0)node[below]{$A$}--(2,0)node[below]{$B$}--(1.5,2)node[above]{$S$}--(0,0);
\draw[dashed](0,0)--(1.2,.5)node[left]{$C$}--(1.5,2);
\draw[dashed](1.2,0.5)--(2,0);
\node at (1,-.75){(1)};

\end{scope}
\begin{scope}[xshift=3.5cm]
    \draw(0,0)node[below]{$A$}--(2,0)node[below]{$B$}--(2.5,.75)node[right]{$C$};
\draw[dashed](0,0)--(.5,.75)node[below]{$D$}--(2.5,.75);
\draw(0,0)--(.7,2)node[above]{$S$}--(2,0);
\draw(.7,2)--(2.5,.75);
\draw[dashed](.7,2)--(.5,.75);
\node at (1,-.75){(2)};
\end{scope}
\begin{scope}[xshift=8cm]
    \draw(-.3,.6)node[left]{$E$}--(0,0)node[below]{$A$}--(1.5,0)node[below]{$B$}--(2,.75)node[right]{$C$};
    \draw(0,0)--(.7,2)node[above]{$S$}--(1.5,0);
    \draw(-.3,.6)--(.7,2)--(2,.75);

    \draw[dashed](-.3,.6)--(.75,.9)node[below]{$D$}--(2,.75);
\draw[dashed](.7,2)--(.75,.9);
    \node at (1,-.75){(3)};
\end{scope}
\end{tikzpicture}
    \caption{}
\end{figure}


图2.16是用第一种画法画出的正六棱锥的直观图。
\begin{figure}[htp]
    \centering
\includegraphics[scale=.55]{fig/2-16.png}
    \caption{}
\end{figure}

正棱锥显然有如下的性质:
\begin{enumerate}
\item 正棱锥所有的侧棱都相等。
\item 正棱锥过顶点的所有面角都相等。
\item 正棱锥各个侧面是全等的等腰三角形。

因为全等的等腰三角形底边上的高都相等,所以正棱锥侧面的等腰三角形底边上的高都叫作
正棱锥的\textbf{斜高}。例如图2.16(3)中$\triangle SBC$底边$BC$上的高$SG$就
是正棱锥$S-AD$的斜高。
\item 正棱锥所有的斜高都相等.
\item 正棱锥的棱、高、及棱在底上的射影以及斜高、高
和斜高在底面上的射影分别组成一个直角三角形。
\end{enumerate}

我们已经看到正棱锥的许多元素,如侧棱、底边(底面
正多边形的边)、高、斜高、侧棱和底面的夹角、侧面和底
面所成的角、斜高和高所成的角等等。在这些元素中只要给
出其中两个元素(至少有一线段),就可以通过解直角三角
形和底面正多边形来求出其它元素。

\begin{example}
已知棱锥底面是边长为12的三角形,它的各个侧
面和底面成$45^{\circ}$角,
\begin{enumerate}
    \item 求证这个棱锥是正三棱锥;
    \item 求这个三棱锥的高。
\end{enumerate}

已知:在棱锥$S-ABC$中,$AB=BC=CA=12$cm, 
面$SAB$、$SBC$、$SCA$都和底面$ABC$成$45^{\circ}$角.
\begin{enumerate}
    \item 求证:$S-ABC$是正三棱锥;
    \item 求三棱锥的高$SO$.
\end{enumerate}
\end{example}

\begin{solution}
    设$SO$为三棱锥的高,过$O$点在底面内作$OD$、$OE$、
$OF$分别垂直于$\triangle ABC$的各边$AB$、$BC$、$CA$, 连接$SD$、$SE$、
$SF$, 则$SD\bot BA$, $SE\bot BC$, $SF\bot AC$ (三垂线定理)(见图2.17), 即$\angle SDO$, $\angle SEO$, $\angle SFO$分别是棱锥各个侧
面和底面所成二面角的平面角。

$\therefore\quad \angle SDO=\angle SEO=\angle SFO=45^{\circ}$

从而$\triangle SOD\cong \triangle SOE\cong \triangle SOF$,
于是$OD=OE=OF$。

即:$O$点是$\triangle ABC$的内心,又因为正三角形的内心、外
心、重心、垂心重合,所以三棱锥$S-ABC$是正三棱锥。

又$OD=\frac{1}{3}CD=\frac{1}{3}\sqrt{CB^2-BD^2}=\frac{1}{3}\sqcup12^2-6^2=\frac{1}{3}\x 6\sqrt{3}=2\sqrt{3}$

$\because\quad \triangle SOD$是等腰直角三角形

$\therefore\quad SO=OD=2\sqrt{3}$ (cm)

答:棱锥的高为$2\sqrt{3}$cm.
\end{solution}

\begin{figure}[htp]\centering
    \begin{minipage}[t]{0.48\textwidth}
    \centering
\begin{tikzpicture}
\tkzDefPoints{0/0/A, 3/0/B, 4/1.5/C, 2/3/S, 2.3/.5/O}
\tkzDefMidPoint(A,B) \tkzGetPoint{D}
\tkzDefMidPoint(A,C) \tkzGetPoint{F}
\tkzDefMidPoint(C,B) \tkzGetPoint{E}
\tkzDrawSegments(A,B B,C S,A S,B S,C S,D S,E)
\tkzDrawSegments[dashed](A,C S,F F,B C,D A,E S,O)
\tkzLabelPoints[below](A,D,B,O)
\tkzLabelPoints[right](E,C)
\tkzLabelPoints[above](F,S)

\end{tikzpicture}
    \caption{}
    \end{minipage}
    \begin{minipage}[t]{0.48\textwidth}
    \centering
\begin{tikzpicture}[yscale=.4]
\tkzDefPoints{1.5/0/C, -1.5/0/A, 0/8/S, 0/0/O}
\tkzDefPoint(-45:1.5){D}
\tkzDefPoint(-45-90:1.5){B}
\draw[dashed](1.5,0) arc (0:180:1.5);
\draw(1.5,0) arc (0:-180:1.5);

\tkzDrawSegments(S,D S,B S,A S,C)
\tkzDrawSegments[dashed](D,O S,O B,O)
\tkzLabelPoints[below](D,B,O)
\tkzLabelPoints[above](S)
\tkzLabelPoints[left](A)
\tkzLabelPoints[right](C)
\node at (0,4) [right]{$h$};
\node at (-.5,3.5) [left]{$\ell$};
\node at (-135:0.5)[left]{$r$};

\end{tikzpicture}
    \caption{}
    \end{minipage}
    \end{figure}

\begin{blk}
    {定义} 如果用不经过圆锥面顶点而垂直于圆锥面的轴的
一个平面去截圆锥面,那么截面和圆锥面所围成的几何体叫
做直圆锥,简称圆锥(图2.18)。原来圆锥面的顶点和轴分别
叫做圆锥的顶点和轴,圆锥面母线夹在顶点和截面之间的部
分叫做圆锥的母线。圆锥面夹在顶点和截面之间的部分叫作
圆锥的侧面。圆锥可以用它的顶点以及它的底面上三个点的
字母来表示,如图2.18的圆锥可以记作圆锥$S-ABC$, 也可
以简记作圆锥$S$.
\end{blk}

圆锥也可以看成是以直角三角形
的一条直角边所在直线为轴,直角
三角形旋转一周而成的几何体。例
如以直角三角形$SOB$的一条直角边
所在的直线为轴,使$\triangle SOB$旋转一
周,$OB$、$SB$旋转所成的面就围成了一
个圆锥(图2.18),直角边$SO$叫做圆
锥的高,直角边$OB$旋转而成的圆面叫做圆锥的底面,斜边
旋转而成的面叫做圆锥的侧面,在侧面各个位置的斜边叫做
圆锥的母线。显然圆锥的任一条母线$\ell$和高$h$, 以及这条母
线$\ell$在底面上的射影即底面半径$r$组成一个直角三角形,根据
勾股定理有:
\[\ell^2=h^2+r^2\]

圆锥有下面的一些性质:
\begin{enumerate}
\item 圆锥的母线都经过顶点并且都相等。
\item 各母线和轴的夹角相等。
\item 垂直于圆锥的轴而且不经过圆锥顶点的平面去截圆
锥面,所得的截面是圆面。因此截面圆与底面圆相似。
\item 过轴的平面是圆锥的对称平面。
\item 过轴的所有截面是全等的等腰三角形。
\end{enumerate}


\begin{example}
求证:与圆锥底面圆相切的直线,和过切点所作的
母线互相垂直。

已知:圆锥$V-ABC$, 底面圆的切线$AT$, 母线$VA$
(图2.19)

\begin{figure}[htp]
    \centering
\begin{tikzpicture}[yscale=.4]
\draw[dashed](0.414,1) arc (0:180:1.414);
\draw(0.414,1) arc (0:-180:1.414);
\tkzDefPoints{0.414/1/B, -2.414/1/C, -1/9/V, 0/0/A,-1/1/O}
\draw(2,2)node[right]{$T$}--(-2,-2);
\tkzDrawSegments(V,C V,A V,B)
\tkzDrawSegments[dashed](V,O O,A)
\tkzLabelPoints[above](V)
\tkzLabelPoints[left](C,O)
\tkzLabelPoints[below right](A)
\tkzLabelPoints[right](B)
\end{tikzpicture}
    \caption{}
\end{figure}

求证:$AT\bot VA$
\end{example}

\begin{proof}
设$O$为圆锥底面的圆心,连接
$VO$、$OA$, 则$VO\bot$平面$ABC$.

$\therefore\quad VA$在平面$ABC$上的射影是$OA$.

$\because\quad AT$是$\odot O$的切线,

$\therefore\quad OA\bot AT$

由三垂线定理,有$AT\bot VA$.
\end{proof}

\begin{ex}
\begin{enumerate}
    \item \begin{enumerate}
        \item 底面是正多边形的棱锥是正棱锥吗?
    \item 侧棱都相等的棱锥是正棱锥吗?
    \item 侧面和底面夹角都相等的棱锥是正棱锥吗?
    \end{enumerate}
  
    \item 
    \begin{enumerate}
      \item 棱锥的侧棱和底面所成的角都相等,则顶点在底面
    上的射影是什么?
    \item 棱锥的侧面和底面所成的二面角都相等,则顶点在
    底面上的射影是什么?  
    \end{enumerate}
    
    \item 证明:六条棱长相等三棱锥是正三棱锥。
    \item 试用两种画法分别画出底面边长为6cm, 高为15cm的
    正六棱锥的直观图,比例尺为
    \item 用第二种画法画出底面半径是2cm, 母线为3cm的圆锥的
    直观图。
    \item 用第一种画法画出底面边长为2cm, 高为4cm的正五棱
    的直观图。
    \item 在正六棱锥$V-ABCDEF$中,底面多边形的边长为$a$,
    侧棱长为$2a$, 求这棱锥的高$VO$及斜高$VM$的长。
    \item 已知圆锥底面圆的半径为3cm, 高为4cm, 设$VA$是圆锥
    的一条母线,$O$点是底面中心,如果$OM\bot VA$于$M$, 求
    $OM$的长。
\end{enumerate}
\end{ex}

\subsection{台体}
\begin{blk}{定义}
    平行于棱锥底面的平面与棱锥侧面相截,棱锥在
底面和截面之间的部分叫做棱台。原棱锥的底面和截面叫做
棱台的下底面和上底面,其他各面叫侧面,两相邻侧面的公
共边叫做棱台的侧棱,两个底面之间的距离叫做棱台的高。
(图2.20)
\end{blk}

\begin{figure}[htp]
    \centering
\begin{tikzpicture}[scale=1.3]
\begin{scope}
    \tkzDefPoints{0/0/A, 2/0/B, 2.75/.75/C, .75/.75/D, 1.375/3/S, 1.375/.375/O}
    \tkzDefMidPoint(S,A) \tkzGetPoint{A'}
    \tkzDefMidPoint(S,B) \tkzGetPoint{B'}
    \tkzDefMidPoint(S,C) \tkzGetPoint{C'}
    \tkzDefMidPoint(S,D) \tkzGetPoint{D'}
    \tkzDefMidPoint(S,O) \tkzGetPoint{O'}
    \tkzDrawPolygon(A',B',C',D')
    \tkzDrawSegments(A,B B,C A,A' B,B' C,C')
    \tkzDrawSegments[dashed](A',S B',S C',S O,S A,D C,D S,D)
    \tkzLabelPoints[below](A,B)
    \tkzLabelPoints[left](A',D',D)
    \tkzLabelPoints[right](C,C',B',O,O')

\node at (1.5,-.5){(1)};
\end{scope}
\begin{scope}[xshift=5cm]
    \tkzDefPoints{0/0/A, 2/0/B, 2.75/.75/C, .75/.75/D, 1.375/3/S, 1.375/.375/O}
\tkzDefMidPoint(S,A) \tkzGetPoint{A'}
\tkzDefMidPoint(S,B) \tkzGetPoint{B'}
\tkzDefMidPoint(S,C) \tkzGetPoint{C'}
\tkzDefMidPoint(S,D) \tkzGetPoint{D'}
\tkzDrawPolygon(A',B',C',D')
\tkzDrawSegments(A,B B,C A,A' B,B' C,C')
\tkzDrawSegments[dashed](A,D C,D D',D)
\node at (1.5,-.5){(2)};
\end{scope}
\end{tikzpicture}
    \caption{}
\end{figure}


棱台可以用表示上、下底顶点的字母来表示,例如棱台
$AC'$. 也可记作棱台$ABCD-A'B'C'D'$(图2.20(1)).

由三棱锥、四棱锥、五棱锥……截得的棱台分别叫做\textbf{三
棱台、四棱台、五棱台}……

\begin{blk}
{定义} 由正棱锥截得的棱台叫做正棱台。    
\end{blk}



\begin{figure}[htp]
    \centering
\begin{tikzpicture}[scale=1.3]
\begin{scope}
    \tkzDefPoints{-.5/0/A, .5/0/B, -1.1/.4/E, 1.1/.4/C, 0/.8/D, 0/2.5/S}
    \tkzDefMidPoint(S,A) \tkzGetPoint{A'}
    \tkzDefMidPoint(S,B) \tkzGetPoint{B'}
    \tkzDefMidPoint(S,C) \tkzGetPoint{C'}
    \tkzDefMidPoint(S,D) \tkzGetPoint{D'}
    \tkzDefMidPoint(S,E) \tkzGetPoint{E'}
    \tkzDrawPolygon(A',B',C',D',E')
    \tkzDrawSegments(A,B B,C E,A A,A' B,B' C,C' E,E')
    \tkzDrawSegments[dashed](A',S S,E'  B',S C',S  C,D E,D)
    
    \node at (0,-.5){(1)};

\end{scope}
\begin{scope}[xshift=3cm]
    \tkzDefPoints{0/0/A, 2/0/B, 2.75/.75/C, .75/.75/D, 1.375/3/S, 1.375/.375/O, 2.375/.375/E}
\tkzDefMidPoint(S,A) \tkzGetPoint{A'}
\tkzDefMidPoint(S,B) \tkzGetPoint{B'}
\tkzDefMidPoint(S,C) \tkzGetPoint{C'}
\tkzDefMidPoint(S,D) \tkzGetPoint{D'}
\tkzDefMidPoint(S,E) \tkzGetPoint{E'}
\tkzDefMidPoint(S,O) \tkzGetPoint{O'}
\tkzDrawPolygon(A',B',C',D')
\tkzDrawSegments(A,B B,C A,A' B,B' C,C' E,E')
\tkzDrawSegments[dashed](A,D C,D D',D O,O' O,E O',E' O,B O',B')
\node at (1.5,-.5){(2)};
\end{scope}
\end{tikzpicture}
    \caption{}
\end{figure}

正棱台有下面一些性质:
\begin{enumerate}
\item 正棱台的两个底面及平行于底面的截面是相似的正
多边形。(图2.21)
\item 两底面中心连线垂直于底面(图2.21(2))。
\item 正棱台的各侧面是全等的等腰梯形,各等腰梯形的
高相等,它叫做正棱台的斜高。(图2.21(2))
\item 正棱台的各侧棱相等,并且延长后相交于一点。
(图2.21(1))
\item 正棱台两底面中心连线、相应的两底面的边心距和
斜高组成直角梯形;两底中心连线、侧棱和两底面相应的半
径组成一个直角梯形;同一个侧面上的上底和下底中点的连
线将这个侧面分成两个全等的直角梯形。(见图2.21(2))
\end{enumerate}

解正棱台的问题一般总可化为解5中提到的直角梯形
及底面的正多边形的问题。

\begin{example}
    设正三棱台的两个底面边长分别是2cm和5cm, 侧
棱长为5cm, 求这个棱台的高和斜高(图2.22)。
\end{example}

\begin{figure}[htp]
    \centering
\begin{tikzpicture}[scale=1.2]
\tkzDefPoints{0/0/A, 4/0/C, 3/-1.5/B, 2.5/3/S, 2.5/-.45/O, 1.5/-.75/D}
\tkzDefMidPoint(S,A) \tkzGetPoint{A_1}
\tkzDefMidPoint(S,B) \tkzGetPoint{B_1}
\tkzDefMidPoint(S,C) \tkzGetPoint{C_1}
\tkzDefMidPoint(S,O) \tkzGetPoint{O_1}
\tkzDefMidPoint(S,D) \tkzGetPoint{D_1}

\tkzDrawPolygon(A_1,B_1,C_1)
\tkzDrawSegments(A,B B,C A,A_1 B,B_1 C,C_1 A_1,O_1 O_1,D_1 D_1,D)
\tkzDrawSegments[dashed](O,D A,C A,O O,O_1)

\tkzLabelPoints[left](A,A_1,D_1)
\tkzLabelPoints[right](C,C_1,O,O_1,B_1)
\tkzLabelPoints[below](B,D)
\tkzDefPointsBy[translation = from O_1 to D_1](O){F}
\tkzDefPointsBy[translation = from O_1 to A_1](O){E}
\tkzDrawSegments[dashed](D_1,F A_1,E)
\tkzLabelPoints[below](E,F)



\end{tikzpicture}
    \caption{}
\end{figure}


\begin{solution}
设两底面的中心分别为$O_1$和$O$, 连接$O_1O$, $O_1A_1$,
 $OA$. 取$A_1B_1$的中点$D_1$, $AB$的中
点$D$, 连接$O_1D_1$, $OD$, $D_1D$. 则
$O_1O$是棱台的高,$DD_1$是棱台的
斜高;并知$O_1A_1AO$和$OO_1D_1D$是两个直角梯形。分别引$A_1E\bot
AO$于$E$, $D_1F\bot DO$于$F$. 

由于上、下两底的边长分
别为2cm和5cm.

因此:
\[O_1A_1=\frac{\sqrt{3}}{3}\x 2=\frac{2}{3}\sqrt{3}, \qquad  OA=\frac{\sqrt{3}}{3}\x 5=\frac{5}{3}\sqrt{3}\]
因此:$AE=OA-O_1A_1=\frac{5}{3}\sqrt{3}-\frac{2}{3}\sqrt{3}=\sqrt{3}$


在直角$\triangle AA_1E$中,
\[A_1E=A_1A_2-AE_2=\sqrt{5^2-(\sqrt{3})^2}=\sqrt{22}\approx 4.69\]

$\therefore\quad OO_1=A_1E=\sqrt{22}\approx 4.69$(cm)

又$\because\quad O_1D_1=\frac{\sqrt{3}}{6}\x 2=\frac{\sqrt{3}}{3},\qquad OD=\frac{\sqrt{3}}{6}\x 5=\frac{5}{6}{\sqrt{3}}$

$\therefore\quad DF=OD-O_1D_1=\frac{5}{6}{\sqrt{3}}-\frac{\sqrt{3}}{3}=\frac{\sqrt{3}}{2}$

在直角$\triangle DD_1F$中
\[\begin{split}
    D_1D&=\sqrt{DF^2+D_1F^2}=\sqrt{DF^2+O_1O^2}\\
&=\sqrt{\left(\frac{\sqrt{3}}{2}\right)^2+(\sqrt{22})^2}=\frac{1}{2}\sqrt{91}\approx 4.77({\rm cm})
\end{split}\]

答:这棱台的高约等于4.69cm, 斜高约等于4.77cm.
\end{solution}

\begin{blk}
    {定义} 平行于圆锥底面的平面与圆锥相截,圆锥的底面
和截面之间的部分叫做圆台。原来圆锥的底面和截面分别叫
做圆台的下底面和上底面,原来圆锥的轴叫做圆台的轴,原
来圆锥的母线夹在两底面之间的部分叫做圆台的母线,原来
圆锥的侧面夹在两底面之间的部分叫做圆台的侧面,两底面
之间的距离叫做圆台的高。(图2.23)
\end{blk}

圆台也可以看作是一个直角梯形绕着垂直于底边的一条
腰所在的直线为轴,旋转$360^{\circ}$时,这个直角梯形的两底边及
另一腰所形成的面围成的几何体。直角梯形的上、下两底边旋
转而成的圆面分别叫做圆台的上底面和下底面。直角梯形垂
直于底边的腰的长叫做圆台的高,而另一腰则称为圆台的母
线。(图2.24)

\begin{figure}[htp]\centering
    \begin{minipage}[t]{0.48\textwidth}
    \centering
\begin{tikzpicture}[>=latex, yscale=.4]
\tkzDefPoints{-1.5/0/A_1, 1.5/0/C_1, -1/4/A, 1/4/C, 0/0/O_1, 0/4/O, 0/12/S}
\tkzDefPoint(-120:1.5){B_1}
\draw(O) circle (1);
\draw(C_1) arc (0:-180:1.5);
\draw(C_1)[dashed] arc (0:180:1.5);
\tkzDrawSegments(A,A_1 C,C_1)
\tkzDrawSegments[dashed](A,S C,S S,O_1 S,B_1)
\tkzLabelPoints[left](A,A_1)
\tkzLabelPoints[right](C,C_1,O,O_1)
\tkzLabelPoints[below](B_1)
\tkzInterLC(S,B_1)(A,C) \tkzGetPoints{B'}{B}
\tkzDrawLines[add=0 and .3](B_1,B) \tkzLabelPoints[right](B)
\tkzLabelPoints[above](S)
    \end{tikzpicture}
    \caption{}
    \end{minipage}
    \begin{minipage}[t]{0.48\textwidth}
    \centering
    \begin{tikzpicture}[>=latex, yscale=.4]
        \tkzDefPoints{-1.5/0/A_1, 1.5/0/C_1, -1/4/A, 1/4/C, 0/0/O_1, 0/4/O, 0/12/S}
        \tkzDefPoint(-120:1.5){B_1}
\draw(O) circle (1);
\draw(C_1) arc (0:-180:1.5);
\draw(C_1)[dashed] arc (0:180:1.5);
\tkzDrawSegments(A,A_1 C,C_1)
\tkzDrawSegments[dashed](O,O_1 O_1,C_1)
\tkzDrawSegments(C_1,C O,C) 
\tkzLabelPoints[left](O, O_1)
\node at (C)[right]{$A$};
\node at (C_1)[right]{$B$};
\node at (.5,4)[above]{$R_1$};
\node at (.75,0)[above]{$R_2$};
    \end{tikzpicture}
    \caption{}
    \end{minipage}
    \end{figure}


圆台的表示法是用它两底上而不在同一条母线上的两个
点的字母来表示。如图2.23的圆台可以记作圆台$AB_1$.

圆台有下面一些性质:
\begin{enumerate}
\item 圆台的两个底面是圆,它们所在的平面平行。
\item 圆台的轴经过两个底面的圆心,并且和底面垂直,
连结两底圆心线段的长等于高。
\item 圆台的母线都相等。
\item 圆台的各母线的延长线交于一点。
\item 经过圆台的轴的截面叫圆台的轴截面,它是一个等
腰梯形。
\item 经过圆台的轴的平面是圆台的对称平面。
\end{enumerate}

如果设圆台的上、下两底面的半径分别为$R_1$和$R_2$, 母线
长为$\ell$, 高为$h$, 那么$\ell^2=(R_2-R_1)^2+h^2$ (图2.24)

\begin{example}
    已知一个圆台两底面面积分别是$1{\rm dm^2}$和$49{\rm dm^2}$, 有
    一个截面和底面平行,它的面积是$25{\rm dm^2}$, 求这个截面和两
    个底面距离的比。
\end{example}

\begin{figure}[htp]
    \centering
\begin{tikzpicture}[scale=.5]
\draw(-3,0)--(3,0)--(0,6)--(-3,0);
\draw(-2,2)--(2,2);
\draw(-1+.2,4+.4)--(1-.2,4+.4);
\draw(0,0)--(0,6);
\node at (0,1) [right]{$y$};
\node at (0,3) [right]{$x$};
\node at (0,4.7) [right]{$a$};
\end{tikzpicture}
    \caption{}
\end{figure}

\begin{solution}
    设从截得圆台的原来圆锥的顶点到圆台上底的距离
是$a$ $(a>0)$, 从截面到圆台上底面和下底
面的距离分别是$x$和$y$, 那么参看轴截面
图2.25,由圆台的性质及相似形面积的比等于相似比的平方,就得到
\[\frac{(a+x)^2}{a^2}=\frac{25}{1},\qquad \frac{(a+x+y)^2}{a^2}=\frac{49}{1}\]
由此得正数解$x=4a$, $y=2a$。因此,$x:y=4a:2a=2:1$

答:截面和上、下两底面距离的比为2:1.
\end{solution}



\begin{ex}
\begin{enumerate}
    \item 如果两个相似三角形(不在同一平面内)的对应边互相
    平行,连结它们的对应顶点而形成各面所围成的几何体
    是不是棱台。
    \item 已知正棱台上、下两底面的边长分别是$a$和$b$ $(a<b)$, 侧
    棱和底面成$45^{\circ}$角,求它的侧棱和斜高的长。
    \item 圆台的两底面半径分别是$r_1$和$r_2$ ($r_1>r_2$), 母线与底面夹
    角是$60^{\circ}$, 求它的高。
    \item 圆台的母线长为$2a$, 母线与轴的夹角为$30^{\circ}$,一个底面半径是另一个底面半径的2倍,求两底面的半径。
    \item 下图是正四棱台的二视图,
    (图中所标尺寸单位是cm)
    试用两种画法画出它的直观
    图,比例尺为
    1/3.
\end{enumerate}
\end{ex}

\begin{figure}[htp]
    \centering
\begin{tikzpicture}[>=latex]
\begin{scope}
\draw(0,0)--(4,0)--(3,2.5)--(1,2.5)--(0,0);
\draw[<->](3,2.9)--node[fill=white]{12}(1,2.9);
\draw[<->](4.5,0)--node[fill=white]{15}(4.5,2.5);
\draw(4.2,0)--(4.8,0);
\draw(3.2,2.5)--(4.8,2.5);
\draw(1,2.6)--(1,3.2);
\draw(3,2.6)--(3,3.2);
\end{scope}
\begin{scope}[xshift=6cm]
\draw(0,0) rectangle(4,4);
\draw(1,1) rectangle(3,3);
\draw[<->](4.5,0)--node[fill=white]{24}(4.5,4);
\draw(1,1)--(0,0);
\draw(3,3)--(4,4);\draw(3,1)--(4,0);
\draw(1,3)--(0,4);
\draw(4.1,0)--(4.8,0);
\draw(4.1,4)--(4.8,4);
\end{scope}
\end{tikzpicture}
    \caption*{第5题}
\end{figure}

\subsection{球}

\begin{blk}
    {定义} 在空间与定点距离相
的点的集合称做球面。球面所
包围的立体叫做球。定点叫做球
心,定点和球面上任一点所连线
段叫做球的半径。连结球面上任意两点的线段叫做球的弦,
通过球心的弦叫做球的直径。
\end{blk}

球面也可以看作半圆绕着它的直径旋转一周所成的图
形。球可以看作是一半圆面绕着它的直径旋转一周而形成的
立体。原半圆面的半径是球的半径,原半圆面的圆心是球
心。

一个球可以用表示它球心的字母来表示,例如“球$O$”。
画半径是$R$的直观图时,一般用第二种画法,先分别在$XOY$、
$XOZ$、$YOZ$三个平面的画出表示半径是$R$的圆的三个椭圆,再
在外面画一个圆和这三个椭圆相切(图2.26),通常用简便
的画法如图2.27.

\begin{figure}[htp]\centering
    \begin{minipage}[t]{0.48\textwidth}
    \centering
\begin{tikzpicture}[>=latex, scale=1]
\draw(0,0) circle(1.5);
\foreach \x in {90,-30,-150}
{
    \draw[dashed](0,0)--(\x:1.5);
    \draw[->](\x:1.5)--(\x:2.5);
}

\node at (90:2.5)[right]{$Z$};
\node at (-30:2.5)[right]{$Y$};
\node at (-150:2.5)[left]{$X$};
\node at (0,0)[below]{$O$};
\draw[dashed](0,0) ellipse [x radius=1.5, y radius= .6, rotate=45];
\draw[dashed](0,0) ellipse [x radius=1.5, y radius= .6, rotate=-45];
\draw[dashed](0,0) ellipse [x radius=1.5, y radius= .6];
\draw[thick](-1.5,0) arc [x radius=1.5, y radius= .6, start angle=180, delta angle=180];
\draw[thick, rotate=45](-1.5,0) arc [x radius=1.5, y radius= .6, start angle=180, delta angle=-180];
\draw[thick, rotate=-45](-1.5,0) arc [x radius=1.5, y radius= .6, start angle=180, delta angle=-180];
    \end{tikzpicture}
    \caption{}
    \end{minipage}
    \begin{minipage}[t]{0.48\textwidth}
    \centering
    \begin{tikzpicture}[>=latex, scale=1]
        \draw(0,0) circle(1.5);  
        \draw[dashed](0,0) ellipse [x radius=1.5, y radius= .6];
        \draw[thick](-1.5,0) arc [x radius=1.5, y radius= .6, start angle=180, delta angle=180];
        \node at (0,0)[below]{$O$};
        \tkzDrawPoint(0,0)
    \end{tikzpicture}
    \caption{}
    \end{minipage}
    \end{figure}



球有下列一些性质:
\begin{enumerate}
    \item 同球的半径(或直径)相等。
    \item 球被平面所截得截面是一圆面。
\end{enumerate}

\begin{figure}[htp]
    \centering
\includegraphics[scale=.7]{fig/2-28.png}
    \caption{}
\end{figure}

\begin{proof}
    设平面$M$通过球心$O$, 并设$A$、$B$为平面$M$和球面交线
    上任意两点。(图2.28(1))在平面$M$内连接$OA$, $OB$, 因
    $OA$、$OB$都是球$O$的半径,所以$OA=OB$. 这就是说,交线上
    任意两点,因为交线上一切点与$O$点等距离,所以交线是平
    面$M$内以$O$点为圆心的一个圆,它的半径就等于球的半径,
    因此球$O$与平面$M$的截面是以球心为圆心的圆面。

设平面$M$不通过球心(图2.28(2)),自球心$O$作$OO_1$
    垂直平面$M$于$O_1$, 设$A$、$B$为平面$M$与球$O$的交线上任意两点,
    连结$OA$, $OB$, 因为$OA=OB$, 所以它们在平面$M$内的射影相
    等,即$AO_1=BO_1$, 因此平面$M$与球$O$的交线是一个圆,它的圆
    心就是球心$O$到平面的垂线的垂足$O_1$. 如果用$R,r$分别表示球
    的半径和截面圆的半径,$d$表示由球心$O$到平面$M$的距离,
那么,
\[r=\sqrt{R^2-d^2}\]
因此,平面$M$和球$O$的截面是以$O_1$为圆心,以$\sqrt{R^2-d^2}$
为半径的圆面。
\end{proof}

\begin{blk}{推论}
\begin{enumerate}
\item 连结球心与截面圆的圆心的直线和截面垂直。
\item 与球心距离相等的截面的圆大小相等。
\item 与球心距离不等的截面,所截得的圆的大小不等,距
离球心较近的截面所截的圆较大。
\item 当$d=0$时,这时截面过球心,所以$r=R$, 也就是说,
这时截面的圆的半径最大,称这个圆为球的大圆,不过球心
截面的圆称为球的小圆。
\item 当$d=R$时,$r=0$, 这时截面的圆退化为一个点,我们
称和球面只有一个公共点的乎面叫做球的切面,球的切面和
球的公共点叫做切点。和球只有一个公共点的直线叫做球的
切线。显然在球的切面上过切点的直线是球的切线。
\end{enumerate}
    
\end{blk}

\begin{blk}{定义} 
    连结球面上$A$、$B$两点间大圆的劣孤叫做球面上
$A$、$B$两点间的球面距离。
\end{blk}

\begin{blk}
    {定理} 如果球的半径通过球面
    的切面的切点,这个半径必垂直于
    球的切面。
\end{blk}


































\subsection*{习题2.1}

\begin{enumerate}
    \item \begin{enumerate}
        \item 图示四棱柱、平行六面体、直平行六面体、长方
    体、正方体各集合之间的关系。
    \item 图示多面体、凸多面体、棱柱、棱锥、棱台、平
    行六面体各集合之间的关系。
    \end{enumerate}
   
    \item 已知长方体的对角线与其共顶点的三条棱分别成角$\alpha,\beta,\gamma$,求证:
    \[\cos^2\alpha+\cos^2\beta+\cos^2\gamma=1\]
    \item 已知长方体的一条对角线与各个面所成的角分别是$\alpha,\beta,\gamma$,求证:\[\cos^2\alpha+\cos^2\beta+\cos^2\gamma=2\]
    \item 四棱柱的对角线如果交于一点,求证:此柱是平行六面
    体。
    \item 底面是菱形的直棱柱对角线长分别是9cm和15cm, 侧棱
    是5cm, 求它的底面边长。
    \item 在直平行六面体$ABCD-A_1B_1C_1D_1$中,若$AD=1$m,
    $AB=2$m, $AA_1=3$m, $\angle DAB=60^{\circ}$,求对角线$BD_1$和$AC_1$
    的长。
    \item 圆柱有一个内接直三棱柱,直三棱柱的一个侧面经过连
    结圆柱两底中心的轴,求证:直三棱柱的其他两侧面互
    相垂直。
    \item 已知正六棱锥的底面边长是4cm, 侧棱长是8cm, 求它
    的侧面和底面所成的角。
    \item 四面体(三棱锥)的各顶点与其对面的重心的连线共有
    四条,试证这四条直线交于一点。

    \item 试证:平行六面体所有对角线交于一点,而且互相平
分。
\item 求证:正三棱锥的侧棱与底面的对边垂直。
\item 如果棱锥的侧棱相等且侧面和底面成相等的角,求证:
这个棱锥是正棱锥。
\item 如果一个正三棱锥在顶点处的三个面角都是直角,求它
的侧棱与底面所成的角。
\item 一个正三棱锥的底面边长为$a$, 侧棱和底面成$\alpha$角,求这
棱锥的高和斜高。
\item 正三棱柱的棱长及底面边长都是$a$, 过底面一边及其相
对侧棱的中点作截面,求此截面的面积。
\item 过长方体$ABCD-A_1B_1C_1D_1$的底面一条对角线$AC$作一
个平面平行于长方体的一条对角线$BD_1$, 如果长方体的
底面是一个边长为$a$的正方形,对角线$BD_1$和底面所成
的角为$\alpha$, 求这个截面的面积。
\item 圆锥的高为20cm, 过圆锥的顶点与底面成$45^{\circ}$角的平
面,把圆锥底面圆周截成1:3两部分,求截面的面积。
\item 在一个正棱锥中,求证:
\begin{enumerate}
    \item 各条侧棱与底面所成的角都相等。
    \item 各侧面和底面所成的二面角相等。
\end{enumerate}

\item 已知正四棱锥的侧棱为15, 侧棱与底面所成的角为$45^{\circ}$,
画出正四棱锥的二视图与直观图。
\item 已知圆锥底面半径为4cm, 母线与底面所成的角为$60^{\circ}$,
求圆锥内接正四棱锥的斜高。
\item 正六棱锥$S-ABCDEF$的侧棱为$4a$, 底面正六边形边长
为$a$, 求过$SA$和$SC$的对角面的面积。
\item 正三棱台的上、下两底的边长分别为2cm和6cm, 侧棱
和底面成$60^{\circ}$角,求棱台的高和斜高。
\item 正四棱台的上下底的边长是6cm和24cm, 斜高为15cm, 
求这棱台的高。
\item 圆台的高为$h$, 母线和底面成$30^{\circ}$角,求母线的长。
\item 正四棱台上、下两底边长分别为$a$和$b$ ($a<b$), 侧棱和
底面成$45^{\circ}$角,求它的侧棱和斜高。
\item 圆台的高是两个底面的面积分别是$3\pi h^2$和$12\pi h^2$, 求圆
台的母线和底面所成的角。
\item 台体上、下两底的面积各是$Q_1$和$Q_2$, 求证这台体的高
和截得这台体的原锥体高的比是
\[\frac{Q_2-\sqrt{Q_1\cdot Q_2}}{Q_2}\]
\item 台体的两底面积分别是$9{\rm cm^2}$和$25{\rm cm^2}$, 求它的中截面的
面积。
\item 求证:球的任意两个大圆互相平分。
\item 在半径是13cm的球面上,有$A,B,C$三点,$AB=6$cm, 
$BC=8$cm, $CA=10$cm, 求经过这三点的截面和球心的
距离。
\item 在半径是$r$的球面上有两点$A$、$B$, 半径$OA$和$OB$的夹角
是$n^{\circ}$($n^{\circ}<180^{\circ}$), 求$AB$两点间的球面距离。
\item 在北纬$30^{\circ}$圈上有甲、乙两地,它们的径度相差$120^{\circ}$,
求这两地间纬度线的长。
\item 一个球的两个切面所成的二面角(球在二面角内)是
$120^{\circ}$, 这两个切点的球面距离是$12\pi$cm, 求这个球的半
径。
\item 一个圆锥的高为8cm, 母线为10cm, 求它的内切球的半
径。
\item 两个底面半径分别为$r_1$和$r_2$圆台中有一个内切球,求这
个内切球的半径。
\item 已知两个球的半径分别是3cm和4cm, 球心连线长为
5cm, 求这两个球交线圆的面积。
\item 试证:经过不在同一平面内的四个点的球面只有一个,
并说明球面的作法。
\item 已知正多面体的棱长为$a$, 那么
\begin{enumerate}
\item 求它的相邻两个
面所成的角;
\item 两条不相交棱间的距离。
\end{enumerate}

\item 求证:正八面体相对的两个平面(即没有公共顶点的
面)互相平行。
\item 已知正八面体,作一个平面与相对的两个平面平行,并
且和其余六个棱相交于各棱中点,求证:此截面是正六
边形。
\item 若正四棱台$ABCD-A_1B_1C_1D_1$的对角线$AC_1$和$A_1C$互相
垂直,它们的长都等于2m, 求棱台的高和对角面的面
积。
\item 已知圆锥的底面的面积$Q=324\pi{\rm cm^2}$, 平行于底面截面
的面积$Q'=182\frac{1}{4}{\rm cm^2}$, 截面与底面间的距离$OO'$为
30cm, 求圆锥的顶角(母线间最大夹角)。
\item 圆锥底面半径为$r$, 高为$h$, 在它里面作各侧面都是正方
形的内接正三棱柱,求这棱柱的每条棱的长。
\item 求正四面体内切球的半径。
\item 高和底面直径相等的圆柱叫等边圆柱,若一个等边圆柱
的底面半径为$R$, 上底圆周上的一个点与下底圆周上的
某一个点的连线和底面所成的角等于$\alpha$, 求这条连线和
连结圆柱两底中心所成之轴间的距离,当$\alpha$等于多少度
角时,距离最短?
\item 圆锥的高是20cm, 底面半径是25cm, 经过它的顶点作
一截面,如果底面的圆心截面的距离是12cm, 求这
截面的面积。
\item 已知正四棱台的上、下底面积分别是$Q_1$和$Q_2$, 其侧面
面积为$P$, 求对角面的面积。
\item 有四个等球$A$、$B$、$C$、$D$两两相切,其半径为$r$, 把这
四个球放到桌面上,试求最上面的球心到桌面的距离。
\item 在四面体$ABCD$中,如果有两组对棱互相垂直,求证:
第三组对棱必互相垂直,而且三组对棱的平方和相等。
\item 有一个相对的棱都互相垂直的四面体,从一个顶点向它
的对面作垂线,求证:这条垂线的垂足是对面三角形的
垂心。
\item 设正四面体相对的一组棱$AC$, $BD$的中点分别是$P$、$Q$,
那么
\begin{enumerate}
\item 求$AC$、$PQ$所成的角。    
\item 在$PQ$的垂直平面
内,如果作这个正四面体的正射影,将成怎样的图形?
\end{enumerate}
\end{enumerate}

\section{柱、锥、台、球的全
面积和部分面积}
\subsection{棱柱和圆柱的侧面积和全面积}

\begin{blk}{定义}
    如果平面和棱柱的所有侧棱都相交且垂直,这
样所得的截面叫做棱柱的直截面。(如图2.42的截面$FGLM
N$)
\end{blk}

斜棱柱有时需要延长某些侧棱,以使
和截棱柱的平面相交,得到直截面。

\begin{blk}
    {定理} 棱柱的侧面积等于侧棱长和直
截面周长的乘积。
\end{blk}




















































\begin{ex}
\begin{enumerate}
    \item 如果球的大圆面积增为原来的10倍,球的体积有什么变
    化?
    \item 一种钢滚珠的半径是10mm,造50个这样的滚珠需要多
    少钢?(精确到100g,设钢的比重为7${\rm g/cm^3}$)
    \item 球缺的高是球的直径的$1/10$,
    求它们体积的比。
    \item 球缺的底面半径是球的半径的
    1/2,体积是球的几分之几?
    \item 球半径是5cm的球台,它的两个底面半径分径别是3cm
    和4cm, 求这个球台的体积(有两种情形)。
    \item 在一个直径为50mm的球的中央,以直径为轴钻一个底
    面圆的直径为30mm的圆柱孔,计算球被钻孔后剩余下
    的部分体积。
\end{enumerate}
\end{ex}

\subsection*{习题2.3}
\begin{enumerate}
    \item 一长方体的三度的比为1:2:3, 对角线长是$2\sqrt{14}$cm, 求它的体积。
    \item 将正四棱柱的底面的边三等分,过三等分点用平行于侧
    棱的平面截去四个三棱柱,得到一个八棱柱,这个八棱
    柱的体积是原四棱柱的体积的几分之几?
    \item 求证:底面是梯形直棱柱的体积等于两个平行侧面积的
    和与这两个侧面间距离的积的一半。
    \item 已知正六棱柱较长的一条对角线长是13cm, 侧面积是
    180${\rm cm^2}$, 求这棱柱的体积。
    \item 一根圆木料,长3.0m, 直径0.8m, 距离圆木的轴0.2m且
    平行轴锯去一片,求剩余木料有多少立方米?
    \item 每条棱都是$a$的正六棱柱,求它的内切圆锥的体积。
    \item 从一个正方体中如图那样截
    去四个三棱锥后,得到一个
    正三棱锥$A-BCD$, 求它的
    体积和正方体的体积之比是
    多少?

\begin{figure}[htp]
    \centering
    \begin{tikzpicture}
\tkzDefPoints{0/0/O, 2/0/B, 2/2/C', 0/2/A, .42/.5/D}
\tkzDrawPolygon(O,B,C',A)
\tkzDefPointsBy[translation=from O to D](B,C',A){B',C,A'}        
\tkzDrawSegments(A,B A,C B',C B,C A',C C',C A,A' B,B')
\tkzDrawSegments[dashed](D,B A,D C,D D,B' D,A' D,O)
\tkzLabelPoints[left](A,D)
\tkzLabelPoints[right](B,C)
    \end{tikzpicture}
    \caption*{第7题}
\end{figure}

    \item 三棱锥的三个侧面互相垂直,
    它们的面积分别是$6{\rm m^2}$
    $4{\rm m^2}$和$3{\rm m^2}$, 求它的体积。
    \item 在正三棱柱中有一个内切球,已知球的半径是$r$, 求这个
    三棱柱的体积。
    \item 在体积为$V$的棱柱内取一点$O$, 以$O$为顶点作与棱柱同
    底的两个棱锥,试求它们的体积的和。
    \item 从一切薄铁片上,裁下一个半径为24cm, 圆心角为$120^{\circ}$
    的扇形,再围成一个圆锥筒,求这个圆锥筒的容积(保
    留两位有效数字)。
    \item 在一个大圆锥内,以它的底面为底面作一个小圆锥,
    大、小圆锥的高和母线所夹的角分别为$\alpha$、$\beta$, 这两个圆
    锥的高之差为$h$, 求大小圆锥的侧面所夹部分的体积。
    \item 若圆锥的侧面积是内接圆柱侧面积的四倍,圆锥的高
    $h=2$, 母线长$\ell =3$, 求圆柱的高和体积。
    \item 有甲、乙两个容器,甲容器是圆柱形,高是2dm, 底面
    半径是1dm, 乙容器是圆锥形(锥顶向下),高是2dm.
    底面半径是$2\sqrt{3}$dm, 如果把甲容器灌满水,然后将甲
    容器内水的一部分倒入乙容器,使得两个容器的水面同
    样高,这时水面高是多少dm(分米)?
\item 三棱锥的每条侧棱长都是$\ell$, 底面三边的长分别是$a$、
$b$、$c$, 求证:它的体积
\[V=\frac{1}{12}\sqrt{16\ell^2 s(s-a)(s-b)(s-c)-a^2b^2c^2}\]
其中$s=\frac{1}{2}(a+b+c)$

\item 三棱锥$S-ABC$中,侧面$SCB\bot$侧面 $ABC$, 又$SC=SB
=1$, 在顶点处的各面角等于$60^{\circ}$, 求这个三棱锥的体
积。
\item 已知棱台两底面面积分别为$245{\rm cm^2}$, $80{\rm cm^2}$, 截得棱台
的棱锥高是35cm, 求这个棱台的体积。
\item 两底面边长分别为15cm, 10cm的正三棱台,它的侧面积
等于两底面面积的和,求这个三棱台的体积。
\item 圆台的高为3,一个底面半径是另一个底面半径的2
倍,母线与下底面所成的角为$45^{\circ}$, 求它的体积。
\item 圆台的母线与底面夹角为$\alpha$, 又它的内切球半径为$R$, 
求证:圆台的体积为$\frac{2}{3}\pi R^3\frac{4-\sin^2\alpha}{\sin^2\alpha}$
,侧面积为$\frac{4\pi R^2}{\sin^2\alpha}$.
\item 已知棱台的体积为$76{\rm cm^3}$, 高是6cm, 一个底面面积为
18cm, 求这个棱台的另一个底面面积。
\item 棱台的上、下底面的面积分别是$S_1$和$S$, 求证:这棱台的
高和截得这棱台的原棱锥高的比是
\[\frac{S-\sqrt{SS_1}}{S}\]
\item 体积$52{\rm cm^3}$的圆台,一个底面面积9倍于另一个底面面
积,求截成这圆台的原来圆锥的体积。
\item 如果一个圆柱和一个圆锥的底面直径和高都与球的直径
相等,求证:圆柱、球、圆锥体积的比是3:2:1.
\item 一个多面体的各面都与球相切,求证:多面体的体积等
于它的表面积与球的半径的积的1/3.
\item 求高为$h$, 母线为$\ell$的外接球的体积。
\item 球缺的体积是$\frac{\pi^2}{3}{\rm cm^3}$, 它的高是
$\frac{1}{2}$cm, 求截得球缺的球
的表面积和体积。
\item 一个木球浮于水中,在水面上球缺的高为2cm, 底面半
径为8cm, 求这个木球的重量。
\item 若某球的球冠面积恰等于另一个半球的球冠面积,求
证:半球的体积大于球缺的体积。
\item 已知扇形的圆心角为$30^{\circ}$, 半径为$r$, 以扇形的一边为
轴旋转一周,求所得旋转体的体积。
\item 已知圆台的母线和下底面成$60^{\circ}$角,半径为3cm的球内
切于圆台,求圆台的侧面积和球积。
\item 一个球台的两个底面的半径分别是6cm和4cm, 高是2cm, 
求这个球台的体积。
\item 两个球的半径分别是3cm和4cm, 另一个球的球面面积
等于它们两个球面面积的和,求这个球的半径。
\item 有半径都是$R$的四个球,每一个都和其他三个相切,求
和这四个球同时相切的球的体积和表面积(有两种情
形)。
\item 一个倒圆锥容器(圆锥顶点向下)它的轴截面是正三角
形,在这容器内注入水,并且放入一个半径为$R$的球,水
平面恰好和球面相切,试问将球取出后,水平面的高是
多少?
\end{enumerate}


\section*{复习题二}
\begin{enumerate}
    \item 证明:三棱锥的对棱互相垂直。
    \item 证明:正$n$棱柱每相邻两个侧面所成面二面角等于
\[\frac{(n-2)\x 180^{\circ}}{n}\]

\item 直平行六面体$ABCD-A_1B_1C_1D_1$中已知下底$ABCD$的
边$AB$及$AD$上的高$BE$分别等于26cm和24cm, 又$BE$分
$AD$成2:3, 六面体的高$AA_1$等于45cm, 求这个六面体
的截$ADC_1B_1$的面积。
\item 证明:平行六面体的所有对角线交于一点并且互相平
分。
\item 已知长方体$ABCD-A_1B_1C_1D_1$, $AA_1C_1C$是它的对角
面,它的面积$S_{AA_1C_1C}=M$, 底面面积$S_{ABCD}=Q$, 侧棱
$AA_1=h$, 求长方体的侧面积。
\item 有一个圆柱,它的高是12cm, 底面半径为5cm, 设有一
条线段长13cm, 它的两端分别在上、下底面的圆周上,
求这线段和轴的距离以及它们所成的角。
\item 
棱锥的底面是正方形,有相邻两个侧面垂直于底面。另
外两个侧面与底面成$45^{\circ}$角,最长的侧棱长为15cm, 求
这棱锥的高。
\item 过棱锥的各侧棱分别作垂直于底面的平面,求证这些平
面相交于一直线。
\item 圆锥的母线长为$L$, 它和底
面所成的角为$\theta$, 求这个圆
锥的内接正方体的棱长。
\item 有一个圆锥如图,它的底面
半径为$r$, 母线长为$L$, 在
母线$SA$上有一点$B$, $AB=a$,
求由$A$绕圆锥一周到$B$的最
短距离是多少?

\begin{figure}[htp]
    \centering
    \includegraphics[scale=.7]{fig/10ti.png}
    \caption*{第10题}
\end{figure}

\item 已知一个四棱锥的底的顺次三个角的比等于2:3:4; 又
棱锥侧面与底面构成的角皆相等,求底面各角的大小。
\item 如果四棱台的底面是平行四边形,那么四条对角线必交
于一点。
\item 三条等线段两两垂直交于一点,并且被这点所平分。
求证:这三条线段的六个端点是一个正八面体的六个顶
点。
\item 斜三棱柱的一个侧面的面积等于$S$, 这个侧面与它所
对的棱的距离等于$a$, 求证:这个棱柱的体积是$\frac{1}{2}Sa$.
\item 棱锥的底面是边长分别为20cm和36cm, 夹角为$30^{\circ}$的平
行四边形,棱锥的高等于12cm, 并且顶点在底面内的射
影为底面对角线的交点,求这棱锥的侧面积。
\item 圆台的母线长是$L$, 母线和下底面所成的角是$\theta$, 轴截面
的对角线垂直于母线,求证:这个圆台的侧面积是
$\pi L^2\sin\theta \tan\theta$
\item 圆锥的顶角为$120^{\circ}$, 有一过两母线且与轴截面等积的截
面,求这截面和圆锥的轴所成的角。
\item 底面为直角三角形的直棱柱里有一个内切球,底面三角
形斜边上的高为$h$, 且这高和直角边之一成$\alpha$角,求这棱
柱的体积。
\item 正四棱台的上、下底面积分别是$Q_1$和$Q_2$, 其侧面积为$P$, 
求对角面的面积。
\item 圆锥的高是20cm, 底面半径是25cm, 经过它的顶点,作
一截面,如果底面的圆心到截面的距离是12cm, 求这截
面的面积。
\item 两个圆锥有公共高,且它们的顶点为公共高的两端,若
第一个圆锥的母线为$\ell$, 顶角为$2\alpha$, 第二个圆锥的顶角
为$2\beta$, 求这两个圆锥公共部分的体积。
(提示,两个圆锥的交线是一圆,圆的半径$r$满足
$r(\cot\alpha+\cot\beta)=\ell\cos\alpha$)
\item 若球冠的顶点和底面圆上一点间的距离是$a$, 求球冠的
面积。
\item 在北纬$60^{\circ}$圈上有甲、乙两地,它们的纬度圈上的弧长
为$\frac{\pi}{2}R$($R$为地球半径),求这两地间的球面距离。
\item 求地球热带的面积。(注:地球上从南纬23度半到北纬
23度半之间的部分球面为热带)
\item 在一个平行六面体中,一个顶点上的三条棱长分别是
$a$、$b$、$c$, 这三条棱中每两条所成的角是$60^{\circ}$. 求平行六
面体的体积。
\item 一个棱锥的体积是$V$, 把棱锥的高三等分,过两个分点
的平行于底面的截面将这个三棱锥分成三部分,求中间
一部分的体积。
\item 三棱锥$S-ABC$中,侧棱$SA$、$SB$、$SC$的长分别是$a$、
$b$、$c$,又$\angle ASB=60^{\circ}$, $\angle ASC=\angle BSC=90^{\circ}$, 求这
个棱锥的体积。
\item 一个直角三角形的两条直角边为15cm和20cm, 以斜边
为轴旋转,求这个旋转体的体积。
\item 一个扇形的半径是10cm,用一个半径5cm的同心弧在这
扇形上截去一个小扇形,若用剩下的一块做一个圆台的
侧面,并且它的下底面的面积是$36\pi {\rm cm^2}$. 求原扇形的圆
心角和所成圆台的体积。
\item 一个球冠的面积是$40\pi {\rm m^2}$, 高是2m, 求含这面的球缺的体积。
\item 一个球的半径为7cm, 用两个平行平面截去两个高为
3cm的球缺,求剩余部分(球台)的体积。
\item 经过正四棱柱$AC'$的底面的一条对角线$AC$引一个平
面平行于对角线$BD'$, 交棱$DD'$于$P$, 如果这个正四棱
柱底面的边长为$a$, 对角线$BD'$与底面所成的角是$\theta$, 求
截面$APC$的面积。
\item 侧棱都相等底面为矩形的棱锥,已知底面边长为6cm和
8cm, 棱锥的高为2cm, 求通过它的底的一条对角线,
且平行于不与此对角线相交的一条侧棱的截面的面积。
\item 圆柱的底面半径是10cm, 高是15cm, 平行于轴的截面
在底面上截得的弦等于底面半径,求圆柱被截去部分的
体积。
\item 圆台母线长17cm, 轴截面面积为$420{\rm cm^2}$, 中截面面积为
$196\pi {\rm cm^2}$, 求这圆台的体积。
\item 分别以直角三角形的斜边、两直角边所在直线为轴,旋
转这个直角三角形所得的三个旋转体的体积为$V$、$V_1$、
$V_2$.
求证:
\[\frac{1}{V^2}=\frac{1}{V^2_1}+\frac{1}{V^2_2}\]
\item 求图中阴影部分绕轴$\ell$旋转
一周所成旋转体的全面积和
体积。

\begin{figure}[htp]\centering
    \begin{minipage}[t]{0.48\textwidth}
    \centering
\begin{tikzpicture}[>=latex, scale=.6]
\draw[|<->|](0,-.4)--node[fill=white]{$4$}(4,-.4);
\draw[|<->|](0,5.656+.4)--node[fill=white]{$2$}(2,5.656+.4);
\fill[pattern=north east lines, draw](0,0)--(4,0)--(2,5.656)--(0,5.656)--(0,0);
\draw[fill=white](0,0) arc (-90:90:2.828);
\draw[->](0,2.828)--node[fill=white]{$r$}+(30:2.828);
    \end{tikzpicture}
    \caption*{第37题}
    \end{minipage}
    \begin{minipage}[t]{0.48\textwidth}
    \centering
      \includegraphics[scale=.7]{fig/41ti.png}
    \caption*{第41题}
    \end{minipage}
    \end{figure}

\item 正方体的棱长为$a$, 过顶点
将正方体截去四个三棱锥得
到一个正四面体,求这个正四面体的体积。
\item \begin{enumerate}
    \item 体积相等的正四面体、正六面体、正八面体的全
    面积哪一个最小?哪一个最大?
    \item 正方体、等边圆柱、球这三个体积相同时,哪一
    个全面积最小?
\end{enumerate}
\item 在母线长为$\ell$, 高在$\ell$与$\ell/2$
之间的圆锥中,求侧面积最
大的那个圆锥。
\item 如图,将正方体的棱分为4等分,在1/4处,截去各棱角得到一个多面体,正方体的体积减少了几分之几?
\item 已知一个四面体,它的四个面都是各边长为$a$、$b$、$c$所
组成的四个全等三角形,求此四面体的体积。
\item 若$M$、$N$分别为立方体$ABCD-A_1B_1C_1D_1$的面$A_1C_1$和
$B_1C$的中心,求$DM$和$AN$的距离。
\item 正四棱锥$S-ABCD$中,底面的边长为$a$, 侧棱长为$b$,
\begin{enumerate}
    \item 求这棱锥的体积;
    \item 求这棱锥的侧面积;
    \item 画出过$AC$且平行于$SB$的截面,并求出这截面的周
长及面积。
\end{enumerate}

\item 已知正四棱锥的侧面与底面所夹的二面角为$\alpha$, 相邻两
个侧面所夹的二面角为$\beta$, 求证:$\cos\beta=-\cos^2\alpha$.
\end{enumerate}

\section{附录~~圆锥曲线}
古希腊的几何学家不但对于圆、球、柱、锥进行研究,
而且还对于其他的多种曲线如椭
圆、抛物线、双曲线等等的性质
进行研究并且获得了杰出的成
果,这里只介绍他们所得结果的
简要部分。关于椭圆、抛物线和
双曲线的研究工作,首推公元前
四世纪的孟奈奇姆和前三世纪的
阿波罗尼。

\subsection{圆柱与椭圆}
将一条直线绕着另一条和它
平行的轴旋转一周就产生一个圆
柱面,若用一个垂直于轴的平面
去截这个圆柱面,则所得的截痕
是一个圆;如果截平面与轴不垂
直,则截痕是一个椭圆,那么如何描述它们的几何特性呢?

若从圆柱的上方放入一个和圆柱等半径的球,它总是与
柱面相切,当小球下降到斜截面相切于一点后,就受阻而
不再下降了,同样也可把一个与圆柱等半径的球由下方向
上顶到和斜截面相切于一点的位置。如图2.81所示。设上球
和斜截面相切于$F_1$点,和柱面相切于圆$C_1$, 下球和斜截面
相切于一点$F_2$, 和柱面相切于圆$C_2$。

\begin{figure}[htp]\centering
    \begin{minipage}[t]{0.48\textwidth}
    \centering
\includegraphics[scale=.6]{fig/2-81.png}
    \caption{}
    \end{minipage}
    \begin{minipage}[t]{0.48\textwidth}
    \centering
\includegraphics[scale=.6]{fig/2-82.png}
    \caption{}
    \end{minipage}
    \end{figure}

设$P$是椭圆截痕上任一点,则过$P$点与轴线平行的直线
全在柱面上,设它交圆$C_1$和圆$C_2$于$Q_1$和$Q_2$两点,则
\begin{itemize}
    \item $PQ_1$和$PF_1$是$P$点到上球的两条切线,所以等长。
    \item   $PQ_2$和$PF_2$是$P$点到下球的两条切线,所以等长。
\end{itemize}

所以,$PF_1+PF_2=PQ_1+PQ_2=Q_1Q_2$

我们让$P$点在椭圆上
变动,则$Q_1Q_2$的变化只
是绕轴旋转,而其长度保持不变的。这就是说,
$PF_1+PF_2=Q_1Q_2=$常数。

$F_1$、$F_2$椭圆焦点,这
就发现了椭圆的特性:椭
圆是平面上一个动点到两个定点$F_1$和$F_2$的距离之
和等于定长的轨迹,$F_1$和
$F_2$叫做椭圆的焦点。

\subsection{圆锥与圆锥曲线}

设$\ell_1$、$\ell_2$是相交于$O$点的两条直线,让$\ell_2$以$\ell_1$为旋转
所得的曲面就是一个圆锥面。再用不过$O$点的平面去截割圆
锥面,所得的曲线叫圆锥曲线,如图2.82所示。设$\ell_1$、$\ell_2$的夹角
为$\alpha$, 截平面和轴的交角为$\beta$, 则不难发现$\beta>\alpha$时,截痕为
椭圆;$\beta=\alpha$时,截痕为抛物线;$\beta<\alpha$时,截痕分为上、下
两支,是双曲线。这三类曲线统称圆锥曲线(或圆锥截线)。
这样就对上述这三种曲线有了一种统一的产生办法和处理方
式。

\subsection{圆锥曲线的性质}
在一中前面我们用平面斜截圆柱面而得椭圆这一类型的曲
线,并且利用上、下切球和定点到球面的切线长相等来推导
它的性质:
$PF_1+PF_2=$定长。

在二中我们改用平面来截割圆锥面来构造三种曲线,
其中$\beta>\alpha$时,我们说所得曲线
还是椭圆,这一点还得加以证
明。现在我们把一中的证明略作
修改,证明如下:

我们依然可以作一个
上切球,它和圆锥面相切于圆
$C_1$, 和截割平面相切于$F_1$点,
也可以作一个下切球,它和圆锥
面相切于圆$C_2$, 和截割平面相切
于$F_2$点(唯一不同之点是:
在柱面的情形,上下切球大小相
同,这里的上、下切球一小,一
大。如图2.83)。

设$P$为截痕上任意一点,连结直线$OP$, 交$C_1$、$C_2$于$Q_1$
和$Q_2$两点,则同样有
\[PQ_1=PF_1,\qquad PQ_2=PF_2\]
所以$PF_1+PF_2=Q_1Q_2$.

再者,当$P$点在截痕上变动时,$Q_1Q_2$的变化只在锥面上
旋转,而长度不变。这就证明了圆锥的这种截痕($\beta>\alpha$)
满足一中所证的椭圆的性质,因此它也是椭圆。

\begin{figure}[htp]\centering
    \begin{minipage}[t]{0.48\textwidth}
    \centering
\includegraphics[scale=.6]{fig/2-83.png}
    \caption{}
    \end{minipage}
    \begin{minipage}[t]{0.48\textwidth}
    \centering
\includegraphics[scale=.7]{fig/2-84.png}
    \caption{}
    \end{minipage}
    \end{figure}

\subsubsection{双曲线的性质}
设$\beta<\alpha$, 则截割平面交圆锥于上、下两支,如图2.84所
示,我们也可以作上、下两个切球(这次它们分居于圆锥的
上、下两部),它们和圆锥分别切于圆$C_1$和圆$C_2$, 和截平面
相切于$F_1$和$F_2$点。

设$P$为截痕上的任一点,连结直线$OP$, 分别交$C_1$、$C_2$
于$Q_1$和$Q_2$点,则同样地可以得出:
\[PF_1=PQ_1,\qquad PF_2=PQ_2\]
而$QQ_2$的长度是与$P$点位置无关的一个常数,
所以,$Q_1Q_2=PQ_1-PQ_2=PF_1-PF_2=$常数。

这就证明了双曲线的性质:双曲线上的任一点和双曲线
所在平面上两个定点的距离之差等于一个常数。
即
\[PF_1-PF_2=\text{常数}\qquad \text{或}\qquad PF_2-PF_1=\text{常数}\]


\subsubsection{抛物线的性质}
设$\beta=\alpha$, 则截割平面和圆锥交于开口的一支,这时我
们只能作一个球,它和圆锥相交于$C$, 和截平面相切于$F$点。

再者,圆$C$所在的平面和截平面相交于一条直线叫作
准线,如图2.85所示。
\begin{figure}[htp]
    \centering
    \includegraphics[scale=.6]{fig/2-85.png}
    \caption{}
\end{figure}

设$P$是截痕上任一点,连结$PO$交圆$C$于$Q$点,再由$P$点
向准线$p$作垂线$PR$. 由假设,我们可以将$PQ$旋转到和$PR$
平行的位置$P'Q'$, 这样,就不难看出:
\begin{itemize}
    \item $PF=PQ$(切线长相等)
    \item $PQ=P'Q'$(移形公理)
    \item $PR=P'Q'$(平行平面间的平行线段相等)
\end{itemize}

所以$PF=PR=P$点到准线的距离。

总结上面的讨论得到抛物线的性质:
抛物线上任意一点和抛物线所在平面上的一定点和一定
直线的距离相等。即:$PF=PR$。

定点叫作抛物线的焦点,定直线叫做它的准线。


%  \chapter{向量与向量运算}
在常用的数量问题中,我们用数去表达各种量,如重
量、长度、面积、体积、密度等等;用加、减、乘、除运算
的组合去表达各种量之间的关系(通称代数通性)。在代数
中,我们已掌握了数系的基本性质(即交换律、结合律和分
配律)。并熟知了代数学的基本精神在于有效地运用数系
通性,对于各种类型的代数问题谋求通解(即以通性求通
解)。现在我们要着手把几何学的讨论也推进到定量的层
面,设法把空间结构有系统地代数化,数量化,这也就是本
章所要详加讨论的课题——\textbf{向量与向量运算}。为了便于同学
们逐步地理解向量这个基本概念,本章的前三节先对平面向
量详加分析,然后再在第四节讨论空间向量。

\section{平面位移向量及其加法运算}

\subsection{位移向量}

如图3.1所示,我们在$\overline{AB}$的端点$B$处画
上一个箭头表示 线段$\overline{AB}$具有射线$AB$的方向,这种用箭头指明方向的线段叫做\textbf{有向线段},记作$\Vec{AB}$,读作有向线段$AB$,并称A为$\Vec{AB}$的\textbf{始点},$B$为$\Vec{AB}$的\textbf{终点}。$\overline{AB}$的长叫做$\Vec{AB}$的长,并记作$|\Vec{AB}|$。

$\Vec{AB}$和$\Vec{CD}$同向且等长,那么我们称$\Vec{AB}$和$\Vec{CD}$\textbf{相等},记作$\Vec{AB}=\Vec{CD}$(图3.2)。

\begin{figure}[htp]\centering
    \begin{minipage}[t]{0.48\textwidth}
    \centering
\begin{tikzpicture}[>=latex, scale=.8]
    \draw[->](0,0)node[below]{$A$}--(2,3)node[above]{$B$};
    \end{tikzpicture}
    \caption{}
    \end{minipage}
    \begin{minipage}[t]{0.48\textwidth}
    \centering
    \begin{tikzpicture}[>=latex, scale=.8]
        \draw[->](0,0)node[below]{$A$}--(2,3)node[above]{$B$};
        \draw [dashed](0,0)--(3,.5);\draw [dashed](2,3)--(5,3.5);
        \draw[->](3,.5)node[below]{$C$}--(5,3.5)node[above]{$D$};
    \end{tikzpicture}
    \caption{}
    \end{minipage}
    \end{figure}

已知方向$ON$(图3.3), 如果平面上的每一点都沿射线
$ON$的方向移动相同的距离,那
么我们称这种平面上全体点的移
动叫做平面上全体点的一个\textbf{平移}。

\begin{figure}[htp]\centering
    \begin{minipage}[t]{0.48\textwidth}
    \centering
\begin{tikzpicture}[>=latex, xscale=.8]
       \draw[->] (0,0)node[right]{$O$}--(0,1)node[right]{$N$};
       \draw[dashed](0,1)--(0,2.5);
\draw[->] (1,0)node[right]{$P$}--(1,2)node[right]{$P_1$};
\draw[->] (2,1)--(2,3);  \draw[->] (3,.5)--(3,2.5);  
\draw[->] (4,1)--(4,3);  \draw[->] (5,.5)--(5,2.5);  
    \end{tikzpicture}
    \caption{}
    \end{minipage}
    \begin{minipage}[t]{0.48\textwidth}
    \centering
    \begin{tikzpicture}[>=latex, scale=1]
        \draw[->] (0,0)node[right]{$P_1$}--(1.5,2)node[right]{$T(P_1)$};
        \draw[->] (2,0)node[right]{$P_2$}--(3.5,2)node[right]{$T(P_2)$};
    \end{tikzpicture}
    \caption{}
    \end{minipage}
    \end{figure}

平面上全体点的一个平移通
常用字母$T$来表示,不同的平移可分别用$T_1,T_2,\ldots$来表示。
如果$P$点通过平移$T$移动到$P_1$点,则$P_1$点叫做$P$点的\textbf{象
点},记作$P_1=T(P)$, 这时有向线段$\Vec{PP_1}$; 可写作$\Vec{PT(P)}$。

为了给出一个平移,根据定义,只需给出平面上任一点
和它的象点即可。设$P_1=T(P)$, 则$P$、$P_1$这两点 就完全
确定了平移$T$的方向和距离;对平面上任一点$A$, 我们都
可作$\Vec{AA_1}=\Vec{PP_1}$, 这样$A$的象点$A_1$也就被$\Vec{PP_1}$所唯一确
定,这也就是说,给定了一个平移$T$, 如果$P_1=T(P)$, 那
么这个平移$T$完全被$\Vec{PT(P)}$所唯一确定,通常我们就用
$\Vec{PT(P)}$或$\Vec{PP_1}$表示这个平移。

显然,平面上全体点的一个移动是一个平移的充要条件
是:对平面上任意两点$P_1$、$P_2$,都有
\[\Vec{P_1T(P_1)}=\Vec{P_2T(P_2)}\]

从以上讨论知,一个平移$T$只有两个要素:\textbf{方向和距离},
因此平面上的一个平移$T$, 可用
那些同向且等长的任一条有向线段来表示。

几次连续平移的结果,叫做平移的\textbf{合成}。设$T_1$、$T_2$是
两次连续平移,这两次连续平移的合成通常记作$T_2\circ T_1$(第
一次平移写在右边),类似地$T_3\circ T_2\circ T_1$, 表示$T_1$、$T_2$、$T_3$三
次连续平移的合成。

\begin{blk}{定理}
    平移的合成还是一个平移且平移的合成满足交换
律。
\end{blk}

\begin{proof}
设$T_1$、$T_2$是两个平移,我们要证明的是:$T_2\circ T_1$也是一个平移,$T_2\circ T_1=T_1\circ T_2$.

设$P$、$Q$是平面上任意两点(图3.5),$P_1=T_1(P)$、
$P_2=T_2(P_1)$、$Q_1=T_1(Q)$、$Q_2=T_2(Q_1)$, 则$P_2=T_2(T_1(P))$, $Q_2=T_2(T_1(Q))$
\begin{figure}[htp]
    \centering
\begin{tikzpicture}[>=latex]
\tkzDefPoints{0/0/P, .85/2/P_1, 2/3/P_2, 4/0/Q}
\tkzDefPointsBy[translation= from P to Q](P_1,P_2){Q_1,Q_2}
\draw[->](P) -- node[left]{$T_1$} (P_1);
\draw[->](Q) -- node[left]{$T_1$} (Q_1);
\draw[->](P) --  (P_2);
\draw[->](Q) -- (Q_2);
\draw[->](P_1) -- node[left]{$T_2$} (P_2);
\draw[->](Q_1) -- node[left]{$T_2$} (Q_2);
\tkzDrawSegments(P,Q P_2,Q_2 P_1,Q_1)
\tkzLabelPoints[left](P_2,Q_2,P_1,Q_1)
\tkzLabelPoints[below](P,Q)
\end{tikzpicture}
    \caption{}
\end{figure}

我们要证明$T_2\circ T_1$是一个平移也就是要证
明$\overline{PP_2}$与$\overline{QQ_2}$平行且等
长。即
\[\Vec{PP_2}=\Vec{QQ_2}\]
由于$T_1$、$T_2$都是平移,则
\[\Vec{PP_1}=\Vec{QQ_1},\qquad \Vec{P_1P_2}=\Vec{Q_1Q_2}\]
连$P$、$P_2$, $Q$、$Q_2$由平行四边形基本定理(即\textbf{一个四边形的
一组对边平行且等长,则另一组对边也平行且等长}),容易
证明:
\[\Vec{PP_2}=\Vec{QQ_2}\]
这就证明了$T_2\circ T_1$也是一个平移。



设点$P$是平面上任一点,$P_1=T_1(P)$, $P_2=T_2(P_1)$,
则$P_2=T_2(T_1(P))$, 这就是说$T_2\circ T_1$把$P$点移动到$P_2$点。

设$Q=T_2(P)$, 由平移定义可知
\[\Vec{P_1P_2}=\Vec{PQ}\]
又由上述平行四边形基本定理可证
\[\Vec{PP_1}=\Vec{QP_2}\]
这就是说
\[P_2=T_1(Q)=T_1(T_2(P))\]
所以
\[T_2\circ T_1=T_1\circ T_2\]
\end{proof}

\begin{figure}[htp]\centering
    \begin{minipage}[t]{0.48\textwidth}
    \centering
\begin{tikzpicture}[>=latex, scale=1]
    \draw[->](0,0)node[below]{$P$}--node[below]{$T_2$}(3,0) node[below]{$Q$} ;
    \draw[->](0,0)--node[left]{$T_1$}(1,2)node[above]{$P_1$}  ;
    \draw[->](3,0)--node[right]{$T_1$}(4,2)node[above]{$P_2$};
    \draw[->>, thick](0,0)--node[above=1pt, rotate=30]{$T_2\circ T_1$}(4,2)  ;
    \draw[->](1,2)--node[above]{$T_2$}(4,2)  ;
    \end{tikzpicture}
    \caption{}
    \end{minipage}
    \begin{minipage}[t]{0.48\textwidth}
    \centering
    \begin{tikzpicture}[>=latex, scale=.8]
        \draw[->](0,0)--node[below]{$T_1$}(4,0);
        \draw[->](0,0)--node[left]{$T_2$}(0,4);
        \draw[->>](0,0)--node[above, rotate=45]{$5\sqrt{2}$}(4,4);
        \draw[->](4,0)--node[right]{$T_2$}(4,4);
        \draw[->](0,4)--node[above]{$T_1$}(4,4);

        \draw[->](-1,2)--(-1,3)node[left]{北};
    \end{tikzpicture}
    \caption{}
    \end{minipage}
    \end{figure}


\begin{example}
    已知$T_1$表示平移“向东5公里”,$T_2$表示平移
“向北5公里”,求$T_1\circ T_2$, $T_2\circ T_1$.
\end{example}

\begin{solution}
    $T_1\circ T_2=T_2\circ T_1$都表示平移:“向东北$5\sqrt{2}$公里”(图3.7)
\end{solution}

\begin{example}
    已知:$T_1:$ “向东5公里”,$T_2:$ “向西10公里”,求$T_1\circ T_2$和$T_2\circ T_1$.
\end{example}

\begin{solution}
   $T_1\circ T_2=T_2\circ T_1$都表示平移:“向西5公里”(图3.8)。
\end{solution}

\begin{example}
    已知$T_1:$ “向东3公里”,$T_2:$ “向西3公里”,
求$T_1\circ T_2$, $T_2\circ T_1$。
\end{example}

\begin{solution}
    $T_1\circ T_2=T_2\circ T_1$都表示“原地不动”(图3.9)。
这种原地不动的“移动”,可看作平移的特例,叫做\textbf{零平移},零平移记作$\Vec{0}$.
\end{solution}

\begin{blk}
    {定义} 平面上全体点的一个平移,叫做平面上的一个位
移向量,简称向量。
\end{blk}

\begin{figure}[htp]\centering
    \begin{minipage}[t]{0.48\textwidth}
    \centering
\begin{tikzpicture}[>=latex, scale=.9]
\draw[very thick, <->] (2,0)--node[above]{$T_1$}(0,0)--(-2,0);
\draw[thick, ->]  (-4,0)--node[below]{$T_1$}(-2,0);
\draw[thick, ->](-4,0)--(-4,1)node[above]{北};
\draw(0,0)[fill=black]circle (1.5pt);
\draw[thick, ->](2,-.25)--node[below]{$T_2$}(-2,-.25);
\draw[thick, ->](0,.25)--node[above]{$T_2$}(-4,.25);
\draw(-4,0)--(2.5,0);
    \end{tikzpicture}
    \caption{}
    \end{minipage}
    \begin{minipage}[t]{0.48\textwidth}
    \centering
    \begin{tikzpicture}[>=latex, scale=.9]
\draw[very thick, <->] (2,0)--node[above]{$T_1$}(0,0)--node[below]{$T_2$}(-2,0);
\draw[thick, ->](2,-.25)--node[below]{$T_2$}(0,-.25);
\draw[thick, ->](-2,.25)--node[above]{$T_1$}(0,.25);
\draw(0,0)[fill=black]circle (1.5pt);
\draw(-2.5,0)--(2.5,0);
    \end{tikzpicture}
    \caption{}
    \end{minipage}
    \end{figure}

如果我们用$\Vec{AB}$, $\Vec{CD}$表示平面上点的平移,我们就说向量$\Vec{AB},\Vec{CD},\ldots$。印刷时
经常用粗体字$\bf{a}$、$\bf{b}$、$\bf{c}$等表
示一个向量。向量$\bf{a}$、$\bf{b}$、$\bf{c}$在
手写时常写作$\vec{a}$、$\vec{b}$、$\vec{c}$。

如果向量$\vec{a}$, 把$A$点移
动到$B$点,$C$点移动到$D$点,$E$点移动到$F$点(图3.10),则
\[\vec{a}=\Vec{AB}=\Vec{CD}=\Vec{EF}=\cdots\]
这些同向且等长的有向线段都表示同一个向量,也就是说它
们中的任一个都可用来表示向量$\vec{a}$。
\begin{figure}[htp]
    \centering
    \begin{tikzpicture}[>=latex]
\draw[thick, ->](0,0)node[below]{$A$}--node[left]{$\vec{a}$}(1.5,3)node[above]{$B$};
\draw[thick, ->](1,0.25)node[below]{$C$}--node[left]{$\vec{a}$}(2.5,3.25)node[above]{$D$};
\draw[thick, ->](2,.5)node[below]{$E$}--node[left]{$\vec{a}$}(3.5,3+.5)node[above]{$F$};
\draw[thick, ->](3,.75)--node[left]{$\vec{a}$}(4.5,3+.75);        
    \end{tikzpicture}

    \caption{}
\end{figure}


若$\vec{a}=\Vec{AB}$, 则$\Vec{AB}$的长叫做$\vec{a}$的长度,记作$|\Vec{AB}|$或$|\vec{a}|$.
射线$\Vec{AB}$的方向叫做向量$\vec{a}$的方向。零平移又叫做\textbf{零向量},它
的方向不确定。

如果$\Vec{AB}$与$\Vec{CD}$的方向相同,那么$\Vec{AB}$, $\Vec{CD}$叫做\textbf{同向
向量},如果$\Vec{A_1B_1}$、$\Vec{C_1D_1}$的方向相反,那么$\Vec{A_1B_1}$、$\Vec{C_1D_1}$叫
做\textbf{反向向量}(图3.11)。方向相同或相反的向量,叫做\textbf{平行
向量}。
\begin{figure}[htp]
    \centering
\begin{tikzpicture}[>=latex]
\begin{scope}
    \draw[thick, ->] (0,0)node[below]{$A$}--(0,3)node[above]{$B$};
    \draw[thick, ->] (1,.5)node[below]{$C$}--(1,2.5)node[above]{$D$};    
    \draw[thick, ->] (2,1)--node[left]{$\vec{a}$}(2,3);
\end{scope}
\begin{scope}[xshift=5cm]
    \draw[thick, ->] (0,0)node[below]{$A_1$}--(0,3)node[above]{$B_1$};
    \draw[thick, <-] (1,.5)node[below]{$D_1$}--(1,3)node[above]{$C_1$};    
    \draw[thick, <-] (2,1)--node[left]{$\vec{b}$}(2,2.5);
\end{scope}
\end{tikzpicture}
    \caption{}
\end{figure}

长度为1个单位的向量叫做\textbf{单位向量},通常我们用单位
向量表示平面上的一个方向。

以后我们用到记号$\Vec{AB}$或$\vec{a}$, 它们表示的是一个有向线段
还是一个向量,一般我们都不另加说明,读者可根据实际问
题加以区分。

\begin{ex}
\begin{enumerate}
    \item 用有向线段表示以下平移:
\begin{enumerate}
    \item $T_1$: 向东5cm;
    \item $T_2$: 向南偏西$30^{\circ}$, 3cm;
    \item $T_3$: 向北偏西$60^{\circ}$, 50km.
\end{enumerate}
\item  用有向线段表示以下两个平移的合成:
\begin{enumerate}
    \item $T_1$: 向西5里,$T_2$: 向南3里。
    \item $T_1$: 向北4里,$T_2$: 向西8里。
\end{enumerate}
\end{enumerate}
\end{ex}

\subsection{向量的加法与减法}
向量$\vec{a}$与$\vec{b}$的合成就叫做$\vec{a}$与
$\vec{b}$的和,记作$\vec{a}+\vec{b}$, 即
\[\vec{a}+\vec{b}=\vec{b}\circ \vec{a}\]

由上节定理可知,$\vec{a}$与$\vec{b}$的和$\vec{a}+\vec{b}$也是一个向量。由
于每个向量可由一点和它的象点所唯一确定,所以可按下面
方法作$\vec{a}+\vec{b}$.

在平面上,任取一点$A$ (图3.12), 作$\Vec{AB}=\vec{a}$, $\Vec{BC}=\vec{b}$, 即
$\vec{a}$与$\vec{b}$的合成把$A$点移动到$C$点,因而
\[\Vec{AC}=\vec{a}+\vec{b}\]
\begin{figure}[htp]
    \centering
\begin{tikzpicture}[>=latex, scale=1.5]
\begin{scope}
    \draw[->,  thick](0,0)--node[left]{$\vec{a}$}(1,1);
    \draw[->,  thick] (1,.5)--node[above]{$\vec{b}$}(2.5,.6);
\end{scope}

\begin{scope}[xshift=3.5cm]
    \draw[->,  thick](0,0)node[below]{$A$}--node[left]{$\vec{a}$}(1,1);
    \draw[->,  thick](1,1)--node[above]{$\vec{b}$}(2.5,1.1)node[right]{$C$};
    \draw[->,  thick](0,0)--node[right]{$\vec{a}+\vec{b}$}(2.5,1.1);
\end{scope}
\end{tikzpicture}    
    \caption{}
\end{figure}

以上求和作图法叫做\textbf{三角形求和法则}。

由于平移的合成满足交换律,所以向量加法也满足交换
律,即
\[\vec{a}+\vec{b}=\vec{b}+\vec{a}\]

读者不难从图3.13
去验证,向量加法还满
足结合律,即
\[(\vec{a}+\vec{b})+\vec{c}=\vec{a}+(\vec{b}+\vec{c})\]
\begin{figure}[htp]\centering
    \begin{minipage}[t]{0.48\textwidth}
    \centering
\begin{tikzpicture}[>=latex,scale=1.5]
    \tkzDefPoints{0/0/A, 1/1.5/B, 3/1.7/C, 3.5/0/D}
\draw[->, thick](A)--node[left]{$\vec{a}$}(B);
\draw[->, thick](A)--node[above,rotate=30]{$\vec{a}+\vec{b}$}(C);
\draw[->, thick](A)--node[below]{$(\vec{a}+\vec{b})+\vec{c}=\vec{a}+(\vec{b}+\vec{c})$}(D);
\draw[->, thick](B)--node[above]{$\vec{b}$}(C);
\draw[->, thick](B)--node[above,rotate=-35]{$\vec{b}+\vec{c}$}(D);
\draw[->, thick](C)--node[right]{$\vec{c}$}(D);
    \end{tikzpicture}
    \caption{}
    \end{minipage}
    \begin{minipage}[t]{0.48\textwidth}
    \centering
    \begin{tikzpicture}[>=latex, scale=1.5]
      \draw[->, thick](0,0)--node[left]{$\vec{a}$}(1,1.5);
      \draw[<-, thick](1.25,0)--node[left]{$\vec{x}$}node[right]{$-\vec{a}$}(2.25,1.5);
    \end{tikzpicture}
    \caption{}
    \end{minipage}
    \end{figure}

容易看出,$\vec{a}+\vec{0}=\vec{0}+\vec{a}=\vec{a}$.

如果向量$\vec{a}$与$\vec{x}$反向且等长,那么由求和作图法可知
(图3.14)
\[\vec{a}+\vec{x}=\vec{0}\]

如果用$-\vec{a}$表示$\vec{x}$, 那么$-\vec{a}$叫
做$\vec{a}$的\textbf{逆向量}或\textbf{负向量},于是
\[\vec{a}+(-\vec{a})=\vec{0}\]
如图3.15所示。



\begin{example}
    

\end{example}

\begin{solution}
    
\end{solution}

\begin{example}
    



\end{example}

\begin{solution}
    
\end{solution}

\begin{solution}
    
\end{solution}


























































%  \chapter{向量几何初步}
在第三章,我们学习了向量运算与运算律,这一章我们
要用向量代数的方法来研究几何学,把对几何学的研究推进
到有效能算的定量的水平.

\section{平行与相似}
\subsection{直线的向量方程}
\begin{figure}[htp]
    \centering
\begin{tikzpicture}[>=latex, yscale=1.5]
    \tkzDefPoints{2/.5/A, .5/2/P, 0/0/O}
    \tkzDefMidPoint(A,P)
    \tkzGetPoint{B}
\draw[<->,  thick](P)--(O)--(A);
\draw[->,  thick](O)--(B);
\tkzDrawLines[add =.25 and .25](A,P)
\tkzLabelPoints[right](A,B,P)
\tkzLabelPoint[below](O){$O$}
\end{tikzpicture}
    \caption{}
\end{figure}

给定空间任意两点$A$、$B$(图4.1), 由平行向量基本定理可知,对空
间中一点$P$与$A$、$B$两点共线的充要条件是存在一实数$t$, 使
\[\Vec{AP}=t\Vec{AB}\]
对空间任一点$O$, 这个条件还可
写为
\[\Vec{OP}-\Vec{OA}=t\left(\Vec{OB}-\Vec{OA}\right)\]
\begin{equation}
    \Vec{OP}=(1-t)\Vec{OA}+t\Vec{OB}
\end{equation}
这就是说,如果$P$点在直线$AB$上,则一定存在实数$t$使(4.1)
式成立,反之,任给一实数$t$, 由等式(4.1)所确定的$P$点也
一定在直线$AB$上,方程(4.1)通常叫做\textbf{直线$AB$的向量方
程}.其中\textbf{参数}$t$的几何意义是,$|t|=|\Vec{AP}|:|\Vec{AB}|$, 当$P$点在
射线$AB$上,$t\ge 0$. 当$P$在射线$AB$的反向延长线上,$t<0$.

由直线$AB$的向量方程(4.1)可知,如果
\[\Vec{OP}=x\Vec{OA}+y\Vec{OB}\]
那么点$P$在直线$AB$上的充要条件是$x+y=1$

\begin{example}
    已知$\vec{a},\vec{b}$是两个线性无关的向量,
$\Vec{OA}=\alpha\vec{a}$, $\Vec{OB}=\beta\vec{b}\quad (\alpha\ne 0,\;\beta\ne 0)$,$\Vec{OC}=x\vec{a}+y\vec{b}$(图4.2),
则$A$、$B$、$C$三点共线的充要条件是
\[\frac{x}{\alpha}+\frac{y}{\beta}=1\]
\end{example}

\begin{figure}[htp]
    \centering
    \begin{tikzpicture}[>=latex]
  \draw(0,0)--(-40:4);
\draw[->](0,-3)node[below]{$O$}--(-40:.5)node[right]{$B$}node[below left]{$\beta\vec{b}$};
\draw[->](0,-3)--(-40:1.5)node[right]{$C$};
\draw[->](0,-3)--(-40:2.5)node[right]{$A$}node[below]{$\alpha\vec{a}$};  

\tkzDefPoint(0,-3){O}
\tkzDefPoint(-40:.5){B}
\tkzDefPoint(-40:2.5){A}
\tkzDefMidPoint(O,B) \tkzGetPoint{B1}
\tkzDefMidPoint(O,A) \tkzGetPoint{A1}
\draw[->](O)--(A1)node[right]{$\vec{a}$};
\draw[->](O)--(B1)node[left]{$\vec{b}$};

\end{tikzpicture}
    \caption{}
\end{figure}

\begin{proof}
    由直线的向量方程可知,点$C$在直线$AB$上的充要条
    件是存在实数$t$使
\[\Vec{OC}=(1-t)\Vec{OA}+t\Vec{OB}\]
即\[\Vec{OC}=(1-t)\alpha\vec{a}+t\beta\vec{b}\]
但已知$\Vec{OC}=x\vec{a}+y\vec{b}$且$\vec{a}\nparallel\vec{b}$,所以
\[x=(1-t)\alpha,\qquad y=t\beta\]
消去$t$则可得$A$、$B$、$C$三点共线的充要条件为
\[\frac{x}{\alpha}+\frac{y}{\beta}=1\]
\end{proof}

\begin{example}
    如图4.3, 设$O$、$A$、$B$三点不共线,
$\Vec{OA}=\vec{a}$, $\Vec{OB}=\vec{b}$, $\Vec{OA_1}=\alpha_1\vec{a}$, $\Vec{OA_2}=\alpha_2\vec{a}$, $\Vec{OB_1}=\beta_1\vec{b}$, $\Vec{OB_2}=\beta_2\vec{b}$
且$\alpha_1$、$\alpha_2$、$\beta_1$、$\beta_2$都不为零,又设$A_1B_2$与$A_2B_1$交于$C$点,试以$\vec{a}$、$\vec{b}$、$\alpha_1$、$\alpha_2$、$\beta_1$、$\beta_2$, 表示$\vec{c}=\Vec{OC}$
\end{example}

\begin{figure}[htp]
    \centering
\begin{tikzpicture}[>=latex]
\tkzDefPoints{0/0/O, 1/0/A, 2/0/A_1,  4.5/0/A_2}
\tkzDefPoint(45:1){B}
\tkzDefPoint(45:2.5){B_1}\tkzDefPoint(45:4.5){B_2}
\tkzDrawSegments[->, thick](O,B_1 O,B_2  O,A_1 O,A_2 A_1,B_2 B_1,A_2)
\tkzInterLL(A_1,B_2)(B_1,A_2) \tkzGetPoint{C}
\tkzDrawSegments[->, thick](O,C)
\tkzLabelPoints[below](A,A_1,A_2)
\tkzLabelPoints[above](B,B_1,B_2)
\tkzLabelPoints[right](C)
\tkzLabelPoints[left](O)
\tkzDrawPoints(A,B)
\end{tikzpicture}
    \caption{}
\end{figure}


\begin{solution}
    设$\vec{c}=x\vec{a}+y\vec{b}$, 因为$A_1$、$C$、$B_2$三点共线,$A_2$、$C$、$B_1$三点共线,由例4.1有方程组
\[\begin{cases}
    \frac{x}{\alpha_1}+\frac{y}{\beta_2}=1\\
    \frac{x}{\alpha_2}+\frac{y}{\beta_1}=1\\
\end{cases}\]
由于$A_1B_2$与$A_2B_1$相交于$C$点,可知$\alpha_1\beta_1-\alpha_2\beta_2\ne 0$, 解这个方程组可得
\[x=\alpha_1\alpha_2\frac{\beta_2-\beta_1}{\alpha_2\beta_2-\alpha_1\beta_1},\qquad y=\beta_1\beta_2\frac{\alpha_2-\alpha_1}{\alpha_2\beta_2-\alpha_1\beta_1}\]
\end{solution}

\begin{ex}
\begin{enumerate}
    \item 已知$\vec{a}$、$\vec{b}$线性无关,$\Vec{OA}=\vec{a}$, $\Vec{OB}=\vec{b}$, $\Vec{AP_1}=\frac{1}{3}\Vec{AB}$,
    $\Vec{AP_2}=\frac{1}{2}\Vec{AB}$, $\Vec{AP_3}=\frac{3}{2}\Vec{AB}$,
    试用向量$\vec{a}$、$\vec{b}$表示
    $\Vec{OP_1},\Vec{OP_2},\Vec{OP_3}$.

    \item 已知$\vec{a}$、$\vec{b}$线性无关,$\Vec{OA}=\vec{a}$, $\Vec{OB}=\vec{b}$, $P$点满足
    $\Vec{AP}=\mu \Vec{PB}\;  (\mu\in\mathbb{R})$, $P$点叫做$AB$的\textbf{定比分点}.求证:
\[\Vec{OP}=\frac{1}{1+\mu}\vec{a}+\frac{\mu}{1+\mu}\vec{b}\]
    \item 已知$\vec{a}$、$\vec{b}$线性无关,$\Vec{OA}=\vec{a}$, $\Vec{OB}=\vec{b}$,$\Vec{AP_1}=\frac{1}{2}\Vec{P_1B}$,
    $\Vec{AP_2}=-\frac{1}{2}\Vec{P_2B}$, $\Vec{AP_3}=-\frac{3}{2}\Vec{P_3B}$, 
    试用向量$\vec{a}$、$\vec{b}$表
    示$\Vec{OP_1},\Vec{OP_2},\Vec{OP_3}$.
    \item 已知$\vec{a}$、$\vec{b}$不平行且$\Vec{OP_1}=
    x_1\vec{a}+y_1\vec{b}$, $\Vec{OP_2}=x_2\vec{a}+y_2\vec{b}$, 
    $P$点在直线$P_1P_2$上且以$k$为比值定比分割$\Vec{P_1P_2}$,
    若$\Vec{OP}=x\vec{a}+y\vec{b}$, 
    试用$k$、$x_1$、
    $x_2$、$y_1$、$y_2$表示$x$、$y$.
    \item 如果$\Vec{OA}=\vec{a}$, $\Vec{OB}=\vec{b}$, $\Vec{OC}=\vec{c}$, 那么,$A$、$B$、
    $C$三点共线的充要条件是存在三个不全为零的实数$\alpha$、
    $\beta$、$\gamma$使
  \[  \alpha\vec{a}+\beta\vec{b}+\gamma\vec{c}=0\quad \text{且}\quad \alpha+\beta+\gamma=0\]
    (提示:应用直线的向量方程(4.1))
\end{enumerate}
\end{ex}

\subsection{几何证明举例}
 在上一章我们已详细的分析了平
行、相似与向量的加法、倍积运算之间的密切关系,下面我们
举例说明向量加法与倍积运算在几何证题中的应用.


\begin{example}
    证明三角形中位线定理.

已知:在$\triangle ABC$中,$D$、$E$分别是$\overline{AB}$、$\overline{AC}$的中点.

求证:$DE\parallel BC$, 且$\overline{DE}=\frac{1}{2}\overline{BC}$ (图4.4)
\end{example}

\begin{figure}[htp]
    \centering
\begin{tikzpicture}[>=latex, scale=.7]
\tkzDefPoints{0/0/B, 4/0/C, 3/4/A}
\tkzDefMidPoint(A,B) \tkzGetPoint{D}
\tkzDefMidPoint(A,C) \tkzGetPoint{E}
\tkzDrawSegments[->, thick](A,B A,C D,E B,C)
\tkzLabelPoints[below](B,C)
\tkzLabelPoints[left](D)
\tkzLabelPoints[right](E)
\tkzLabelPoints[above](A)
\end{tikzpicture}
    \caption{}
\end{figure}

\begin{proof}
    因$D$、$E$分别是$\overline{AB}$、$\overline{AC}$的中点,所以
\[\Vec{AD}=\frac{1}{2} \Vec{AB},\qquad \Vec{AE}=\frac{1}{2}\Vec{AC}\]
\[\Vec{DE}=\Vec{AE}-\Vec{AD}=\frac{1}{2}\left(\Vec{AC}-\Vec{AB}\right)=\frac{1}{2}\Vec{BC}\]
即:$DE\parallel BC$,  $\overline{DE}=\frac{1}{2}\overline{BC}$
\end{proof}

由例4.3的证明,大致可以看出用向量运算证明几何题的
主要步骤:
\begin{enumerate}
\item 选择基底向量
$\Vec{AB}$、$\Vec{AC}$, 
把已知条件($D$、$E$
是$\Vec{AB}$、$\Vec{AC}$的中点)写为向量形式($\Vec{AD}=\frac{1}{2}\Vec{AB}$, $\Vec{AE}=\frac{1}{2}\Vec{AC}$)
\item 进行向量运算,算出结果($\Vec{DE}=\frac{1}{2}\Vec{BC}$)
\item 把结果转化为几何结论.
\end{enumerate}

对例4.3的进一步分析,我们还会看到,证明中应用了倍
积分配律和向量平行的条件,这正好与几何中应用平行四边
形定理(或相似形定理)相对应.

\begin{example}
 已知五边形$ABCDE$, $M$、$N$、$P$、$Q$分别是$\overline{AB}$、
 $\overline{CD}$、$\overline{BC}$、$\overline{DE}$的中点,$K$、$L$是$\overline{MN}$与$\overline{PQ}$的中点,求
证:$\overline{KL}=\frac{1}{4}\overline{AE}$ 且$KL\parallel AE$(图4.5).
\end{example}

\begin{proof}
    在平面上任选一点$O$作为基点,则
\[\begin{split}
    \overline{KL}&=\overline{OL}-\overline{OK}=\frac{1}{2}\left(\overline{OP}+\overline{OQ}\right)-\frac{1}{2}\left(\overline{OM}+\overline{ON}\right)\\
&=\frac{1}{2}\left[\frac{1}{2}\left(\overline{OB}+\overline{OC}\right)+\frac{1}{2}\left(\overline{OD}+\overline{OE}\right)\right]-\frac{1}{2}\left[\frac{1}{2}\left(\overline{OA}+\overline{OB}\right)+\frac{1}{2}\left(\overline{OC}+\overline{OD}\right)\right]\\
&=\frac{1}{4}\left(\overline{OB}+\overline{OC}+\overline{OD}+\overline{OE}\right)-\frac{1}{4}\left(\overline{OA}+\overline{OB}+\overline{OC}+\overline{OD}\right)\\
&=\frac{1}{4}\left(\overline{OE}-\overline{OA}\right)=\frac{1}{4}\overline{AE}
\end{split}\]
即:$KL\parallel AE$且$\overline{KL}=\frac{1}{4}\overline{AE}$
\end{proof}

在例4.4中,基点选为任一点
$O$, 这样对题中各点的位置向量
表达就比较对称,计算起来就较为方便.

\begin{figure}[htp]\centering
    \begin{minipage}[t]{0.48\textwidth}
    \centering
\begin{tikzpicture}[>=latex, scale=1]
\tkzDefPoints{0/0/C, 2/0/D, 1.9/1.7/E, 0/2.5/A, -1/1.2/B, 1.3/3/O}
\tkzDefMidPoint(A,B) \tkzGetPoint{M}
\tkzDefMidPoint(C,B) \tkzGetPoint{P}
\tkzDefMidPoint(C,D) \tkzGetPoint{N}
\tkzDefMidPoint(D,E) \tkzGetPoint{Q}
\tkzDefMidPoint(M,N) \tkzGetPoint{K}
\tkzDefMidPoint(P,Q) \tkzGetPoint{L}

\tkzDrawPolygon(A,B,C,D,E)
\tkzDrawSegments(M,N P,Q)
\tkzDrawSegments[->](K,L)
\tkzDrawSegments[dashed, ->](O,K O,L)

\tkzLabelPoints[left](A,B,M,K,P)
\tkzLabelPoints[right](E,Q)
\tkzLabelPoints[below](C,N,D,L)
\tkzLabelPoints[above](O)

    \end{tikzpicture}
    \caption{}
    \end{minipage}
    \begin{minipage}[t]{0.48\textwidth}
    \centering
    \begin{tikzpicture}[>=latex, scale=1]
\tkzDefPoints{0/0/B, 3/0/C, 2.2/2.2/A}
\tkzDefPointWith[linear, K=.6](A,B) \tkzGetPoint{E}
\tkzDefPointWith[linear, K=.6](C,A) \tkzGetPoint{F}
\tkzDefPointWith[linear, K=.6](C,B) \tkzGetPoint{D}
\tkzDrawSegments[->, thick](A,D A,E A,F)
\tkzDrawSegments(E,B E,D F,D F,C B,C)
\tkzLabelPoints[below](B,D,C)
\tkzLabelPoints[left](E)
\tkzLabelPoints[right](F)
\tkzLabelPoints[above](A)
    \end{tikzpicture}
    \caption{}
    \end{minipage}
    \end{figure}

\begin{example}
    已知$\triangle ABC$, $D$是$\overline{BC}$上任一点,$\overline{DE}\parallel \overline{CA}$, 
$\overline{DF}\parallel\overline{BA}$, 

求证:$\frac{\overline{ED}}{\overline{AC}}+\frac{\overline{FD}}{\overline{AB}}=1$ (图4.6).
\end{example}

\begin{proof}
设$\frac{\overline{ED}}{\overline{AC}}=x$, $\frac{\overline{FD}}{\overline{AB}}=y$,则:
\[\Vec{ED}=x\Vec{AC},\qquad \Vec{FD}=y\Vec{AB}\]
由求和法则可得
\[\Vec{AD}=\Vec{AE}+\Vec{AF}=\Vec{ED}+\Vec{FD}=x\Vec{AC}+y\Vec{AB}\]
又因$D$在直线$BC$上,所以$x+y=1$,即:
$$\frac{\overline{ED}}{\overline{AC}}+\frac{\overline{FD}}{\overline{AB}}=1$$
\end{proof}

例4.5中的结论,实际上就是一点在直线上的一个必要条
件,换用向量表达式就一目了然了,应注意,题中若设$D$点
是直线$BC$上任一点,结论同样成立.


\begin{example}
    已知$\parallelogram ABCD$, $M$是$\overline{AB}$的中点,$\overline{DM}$交对角线
$\overline{AC}$于$H$点,

求证:
$\overline{AH}=\frac{1}{3} \overline{AC},\qquad \overline{MH}=\frac{1}{3} \overline{MD}$ (图4.7).
\end{example}

\begin{proof}
设$\Vec{AH}=x\Vec{AC}$,$\Vec{MH}=y\Vec{MD}$,则
\[\begin{split}
    \Vec{AH}&=X\Vec{AC}=x\left(\Vec{AB}+\Vec{AD}\right)=x\Vec{AB}+x\Vec{AD}\\
    \Vec{AH}&=(1-y)\Vec{AM}+y\Vec{AD}=\frac{1}{2}(1-y)\Vec{AB}+y\Vec{AD}
\end{split}\]
由于$\Vec{AB}$、$\Vec{AD}$线性无关,所以有
\[\begin{cases}
    2x+y=1\\
    y=x
\end{cases}\]
解方程组可得:$x=\frac{1}{3},\quad y=\frac{1}{3}$

所以:
\[\Vec{AH}=\frac{1}{3}\Vec{AC},\qquad \Vec{MH}=\frac{1}{3}\Vec{MD}\]
\[\overline{AH}=\frac{1}{3}\overline{AC},\qquad \overline{MH}=\frac{1}{3}\overline{MD}\]
\end{proof}

例4.6中,证明的关键是设未知数,根据已知条件,用基
向量$\Vec{AB}$, $\Vec{AD}$写出$\Vec{AH}$的两个表达式,然后由基向量的线
性无关性转化为方程组来求解.这和在代数中设未知数列方
程的解问题的方法相似.

\begin{figure}[htp]\centering
    \begin{minipage}[t]{0.48\textwidth}
    \centering
\begin{tikzpicture}[>=latex, scale=1]
\tkzDefPoints{0/0/A, 3/0/B, 1/2/D, 4/2/C}
\tkzDefMidPoint(A,B)   \tkzGetPoint{M}
\tkzDrawSegments[->](A,D A,M A,B A,C M,D)
\tkzDrawSegments(B,C D,C)
\tkzLabelPoints[below](A,M,B)
\tkzLabelPoints[above](C,D)
\tkzInterLL(A,C)(D,M) \tkzGetPoint{H}
\tkzLabelPoints[above](H))
\tkzDrawSegments[->](M,H A,H)
    \end{tikzpicture}
    \caption{}
    \end{minipage}
    \begin{minipage}[t]{0.48\textwidth}
    \centering
    \begin{tikzpicture}[>=latex, scale=1]
\tkzDefPoints{0/0/B, 3/0/C, 2/2.6/A}
\tkzDefMidPoint(A,B)   \tkzGetPoint{F}
\tkzDefMidPoint(A,C)   \tkzGetPoint{E}
\tkzDefMidPoint(C,B)   \tkzGetPoint{D}

\tkzDrawPolygon(A,B,C)
\tkzDrawSegments[->, thick](A,D B,E C,F)
\tkzInterLL(A,D)(B,E) \tkzGetPoint{G}

\tkzLabelPoints[below](D,C,B)
\tkzLabelPoints[above](A,G)
\tkzLabelPoints[left](F)
\tkzLabelPoints[right](E)
    \end{tikzpicture}
    \caption{}
    \end{minipage}
    \end{figure}


\begin{example}
    已知$\triangle ABC$, 证明:三条中线$\overline{AD}$、$\overline{BE}$、$\overline{CF}$相交于一点$G$且$\overline{AG}=\frac{2}{3} \overline{AD}$, $\overline{BG}=\frac{2}{3}\overline{BE}$, $\overline{CG}=\frac{2}{3}\overline{CF}$ (图4.8).
\end{example}

\begin{proof}
    设$\overline{AD},\overline{BE}$相交于
    $G$点,$\Vec{AG}=x\Vec{AD}$, $\Vec{BG}=y\Vec{BE}$,$\Vec{AB}=a$, $\Vec{AC}=\vec{b}$,则:
\[\Vec{AG}=x\Vec{AD}=x\x \frac{1}{2}(\vec{a}+\vec{b})=\frac{1}{2}x\vec{a}+\frac{1}{2}x\vec{b}\]
又    
\[\Vec{AG}=(1-y)\Vec{AB}+y\Vec{AE}=(1-y)\vec{a}+\frac{1}{2}y\vec{b}\]
由$\vec{a},\vec{b}$线性无关,得:
\[\begin{cases}
    \frac{1}{2}x=1-y\\
    \frac{1}{2}x=\frac{1}{2}y
\end{cases}\]
解之得:$x=\frac{2}{3},\quad y=\frac{2}{3}$,所以
\[\Vec{AG}=\frac{2}{3}\Vec{AD},\qquad \Vec{BG}=\frac{2}{3}\Vec{BE}\]
\[\overline{AG}=\frac{2}{3}\overline{AD},\qquad \overline{BG}=\frac{2}{3}\overline{BE}\]

如果$\overline{BE},\overline{CF}$相交于$G'$点,那么同样可证,
\[\overline{BG'}=\frac{2}{3} \overline{BE},\qquad \overline{CG'}=\frac{2}{3}\overline{CF}\]
于是$G$点与$G'$点重合,题中结论得证.
\end{proof}

\begin{example}
    已知三棱锥$S-ABC$, $K_1$、$L_1$、$M_1$分别是侧棱
    $\overline{SA}$、$\overline{SB}$、$\overline{SC}$的中点,$K_2$、$L_2$、$M_2$分别是$\overline{BC}$、$\overline{CA}$、
    $\overline{AB}$的中点,求证:$\overline{K_1K_2}$, $\overline{L_1L_2}$, $\overline{M_1M_2}$相交于一点,并在
    这点互相平分(图4.9).
\end{example}

\begin{proof}
设$\Vec{SA}=\vec{a}$, $\Vec{SB}=\vec{b}$, $\Vec{SC}=\vec{c}$, $O_1$为$\overline{K_1K_2}$的中点,则
\[\Vec{SO_1}=\frac{1}{2}(\Vec{SK_1}+\Vec{SK_2})=\frac{1}{2}\left[\frac{1}{2}\vec{a}+\frac{1}{2}(\vec{b}+\vec{c})\right]
=\frac{1}{4}\left(\vec{a}+\vec{b}+\vec{c}\right)\]
取$L_1L_2$的中点$O_2$, $M_1M_2$的中点$O_3$, 同理可证
\[\Vec{SO_2}=\Vec{SO_3}=\frac{1}{4}\left(\vec{a}+\vec{b}+\vec{c}\right)\]
因此三点$O_1$、$O_2$、$O_3$必重合于一点$O$并在$O$点互相平分.
\end{proof}

\begin{figure}[htp]\centering
    \begin{minipage}[t]{0.48\textwidth}
    \centering
\begin{tikzpicture}[>=latex, scale=.8]
\tkzDefPoints{-.5/0/A, 5.5/0/C, 3.5/3/S, 3.5/-1/B}
\tkzDrawPolygon(A,B,C,S)

\tkzDrawSegments(S,B)
\tkzDefMidPoint(S,A) \tkzGetPoint{K_1}
\tkzDefMidPoint(S,B) \tkzGetPoint{L_1}
\tkzDefMidPoint(S,C) \tkzGetPoint{M_1}

\tkzDefMidPoint(B,C) \tkzGetPoint{K_2}
\tkzDefMidPoint(C,A) \tkzGetPoint{L_2}
\tkzDefMidPoint(B,A) \tkzGetPoint{M_2}

\tkzDefMidPoint(M_1,M_2) \tkzGetPoint{O_1}
\tkzDrawSegments[dashed](M_1,M_2 K_1,K_2 A,C L_1,L_2)

\tkzLabelPoints[below](A,B,C,M_2,K_2,L_2)
\tkzLabelPoints[above](M_1,K_1,L_1,S,O_1)
    \end{tikzpicture}
    \caption{}
    \end{minipage}
    \begin{minipage}[t]{0.48\textwidth}
    \centering
    \begin{tikzpicture}[>=latex, scale=1]
\tkzDefPoints{0/0/O'}
\tkzDefPoint(180-15:3.5){A'}\tkzDefPoint(180-15:2.5){A}
\tkzDefPoint(180:5){C'}\tkzDefPoint(180:1.8){C}
\tkzDefPoint(180+15:4.2){B'}\tkzDefPoint(180+15:2.2){B}

\tkzDrawPolygon[pattern=north east lines](A,B,C)
\tkzDrawPolygon[pattern=north west lines](A',B',C')
\tkzLabelPoints[left](A',B',C')
\tkzLabelPoints[right](A,B,C)

\tkzInterLL(A',B')(A,B) \tkzGetPoint{P}
\tkzInterLL(A',C')(A,C) \tkzGetPoint{R}
\tkzInterLL(C',B')(C,B) \tkzGetPoint{Q}
\tkzLabelPoints[right](O')
\tkzLabelPoints[above](P,R)
\tkzLabelPoints[below](Q)
\tkzDrawSegments(Q,B' Q,B Q,P P,A' P,A R,A' R,A)
\tkzDrawSegments[dashed](O',B' O',A' O',C')

    \end{tikzpicture}
    \caption{}
    \end{minipage}
    \end{figure}
    
\begin{example}
    试证 Desargues 定理:如图4.10, 设$\triangle ABC$与$\triangle A'B'C'$的顶点连
    线$AA'$、$BB'$、$CC'$相交于一点$O$, 则对应边$AB$与$A'B'$、
    $BC$与$B'C'$、$CA$与$C'A'$的延长线分别相交于$P$、$Q$、$R$, 
    
    试证$P$、$Q$、$R$共线.
\end{example}

\begin{proof}
    设$\Vec{OA}=\vec{a}$, $\Vec{OB}=\vec{b}$, $\Vec{OC}=\vec{c}$, 则存在$\alpha,\beta,\gamma\in \mathbb{R}$,使$\Vec{OA'}=\alpha\vec{a}$, $\Vec{OB'}=\beta\vec{b}$, $\Vec{OC'}=\gamma\vec{c}$

    因$P$点既在$AB$上又在$A'B'$上,所以存在$x,y\in\mathbb{R}$,使
\[\begin{split}
    \Vec{OP}&=(1-x)\vec{a}+x\vec{b}\\
    \Vec{OP}&=(1-y)\alpha\vec{a}+\beta y\vec{b}\\
\end{split}\]
由此可得
\[x=\frac{\beta(\alpha-1)}{\alpha-\beta},\qquad y=\frac{\alpha-1}{\alpha-\beta}\]
\begin{equation}
    \Vec{OP}=\frac{\alpha(1-\beta)}{\alpha-\beta}\vec{a}+\frac{\beta(\alpha-1)}{\alpha-\beta}\vec{b}
\end{equation}
同理可求得
\begin{align}
    \Vec{OQ}&=\frac{\beta(1-\gamma)}{\beta-\gamma}\vec{b}+\frac{\gamma(\beta-1)}{\beta-\gamma}\vec{c}\\
    \Vec{OR}&=\frac{\gamma(1-\alpha)}{\gamma-\alpha}\vec{c}+\frac{\alpha(\gamma-1)}{\gamma-\alpha}\vec{a}
\end{align}
\[\begin{split}
    \Vec{PR}&=\Vec{OR}-\Vec{OP}=\frac{\alpha(\alpha-1)(\gamma-\beta)}{(\gamma-\alpha)(\alpha-\beta)}\vec{a}+\frac{\beta(1-\alpha)}{\alpha-\beta}\vec{b}+\frac{\gamma(1-\alpha)}{\gamma-\alpha}\vec{c}\\
    \Vec{PQ}&=\Vec{OQ}-\Vec{OP}=\frac{\alpha(\beta-1)}{\alpha-\beta}\vec{a}+\frac{\beta(\beta-1)(\gamma-\alpha)}{(\alpha-\beta)(\beta-\gamma)}\vec{b}+\frac{\gamma(\beta-1)}{\beta-\gamma}\vec{c}
\end{split}\]

容易验证$\vec{a},\vec{b},\vec{c}$的系数成比例,所以
$\Vec{PR}$与$\Vec{PQ}$共
线,即$P$、$Q$、$R$共线.

此题还可另证,设$x=(y-1)(\alpha-\beta)$,
 $y=(\alpha-1)(\beta-\gamma)$, $z=(\beta-1)(\gamma-\alpha)$, 则可得:
\[x\Vec{OP}+y\Vec{OQ}+z\Vec{OR}=\vec{0}\]
并且$x+y+z=0$.

$x$、$y$、$z$至少有一不为0, 
这便证明了$P$、$Q$、$R$三点共线.
\end{proof}

\begin{ex}
    试用向量运算证明以下各题.
\begin{enumerate}
\item  试证平行四边形的对角线互相平分.
\item 设$\triangle ABC$和$\triangle A'B'C'$的对应顶点连线$AA'$、$BB'$、
    $CC'$相交于一点$O$. 试证若 $AB\parallel A'B'$, $BC\parallel B'C'$, 则
    $AC\parallel A'C'$.

\item 设$\ell$、$\ell'$相交于$O$点,$A,B,C\in\ell$, $A',B',C'\in\ell'$. 试
证如果$AB'\parallel A'B$, $BC'\parallel B'C$, 则有$AC'\parallel A'C$.
\item 证明梯形中位线定理.
\item 已知:$\triangle ABC$中,$D$是$\overline{BC}$的中点,过$D$任作一直线分别交$\overline{AC}$于$E$, 交$AB$的延长线于$F$, 求证:
$\overline{AE}:\overline{EC}=\overline{AF}:\overline{FB}$.
\item 已知:梯形$ABCD$, $AB\parallel DC$, $\overline{AB}=2\overline{CD}$, $\overline{AC}$、$\overline{BD}$
相交于$E$, 求证:$\overline{CE}=\frac{1}{3}\overline{AC}$.
\item 已知:梯形$ABCD$中,$E$、$F$是上、下底$\overline{AD}$、$\overline{BC}$的
中点,$\overline{AC}$、$\overline{BD}$相交于$G$, 求证:$E$、$G$、$F$三点共线.
\end{enumerate}
\end{ex}

\begin{figure}[htp]\centering
    \begin{minipage}[t]{0.48\textwidth}
    \centering
\begin{tikzpicture}[>=latex, scale=1]
\tkzDefPoints{0/0/O, 4/0/A', 4.2/1.5/B', 3/2.5/C'}
\tkzDefPointWith[linear, K=.6](O,A') \tkzGetPoint{A}
\tkzDefPointWith[linear, K=.6](O,B') \tkzGetPoint{B}
\tkzDefPointWith[linear, K=.6](O,C') \tkzGetPoint{C}
\tkzDrawSegments(O,C' O,A' O,B')
\tkzDrawPolygon(A,B,C)
\tkzDrawPolygon(A',B',C')
\tkzLabelPoints[below](A,A')\tkzLabelPoints[above](C,C')
\tkzLabelPoints[below right](B,B')\tkzLabelPoints[left](O)
    \end{tikzpicture}
    \caption*{第2题}
    \end{minipage}
    \begin{minipage}[t]{0.48\textwidth}
    \centering
    \begin{tikzpicture}[>=latex, scale=.7]
\tkzDefPoints{0/0/O, 2/0/A, 3.5/0/B, 5/0/C}
\tkzDefPoint(40:2){C'}\tkzDefPoint(40:3.1){B'}
\tkzDefPoint(40:5){A'}
\tkzLabelPoints[above](A',B',C')
\tkzLabelPoints[below](A,B,C)
\draw(0,0)node[left]{$O$}--(6,0)node[right]{$\ell$};
\draw(0,0)--(40:6)node[right]{$\ell'$};
\tkzDrawSegments(A,B' A,C' B,C' B,A' C,B' C,A')
    \end{tikzpicture}
    \caption*{第3题}
    \end{minipage}
    \end{figure}

\section*{习题4.1}
\addcontentsline{toc}{subsection}{习题4.1}

\begin{enumerate}
    \item 已知:$O$是一个定点,$\vec{a}$、$\vec{b}$是两个线性无关的向量,
$\Vec{OP_1}=2\vec{a}+\vec{b}$, $\Vec{OP_2}=\vec{a}-\vec{b}$, $P$是直线$P_1P_2$上任意一
    点且$\Vec{OP}=x\vec{a}+y\vec{b}$, 求$x$、$y$满足的代数关系式.
    \item 已知:在$\triangle ABC$中,$D$、$E$分别是$\overline{BC}$、$\overline{AC}$边上的点且
    $\overline{BD}:\overline{DC}=1:2$, $\overline{CE}:\overline{EA}=1:1$, $\overline{AD}$与$\overline{BE}$相交于$O$
    点,设$\Vec{AB}=\vec{a}$, $\Vec{AC}=6$, $\Vec{CO}=x\vec{a}+y\vec{b}$, 求$x$、$y$.
    \item 设$\vec{a}$、$\vec{b}$是两个线性无关的向量,$\Vec{OD}=h\vec{a}+k\vec{b}\ne \vec{0}$.
    $P$是直线$OD$上任意一点,令$\Vec{OP}=x\vec{a}+y\vec{b}$, 求证:
    $x$、$y$满足方程$kx-hy=0$.
    \item 已知$D$、$E$、$F$分别是$\triangle ABC$的边$\overline{AB}$、$\overline{BC}$、$\overline{AC}$上的一
    点且$\frac{\overline{AD}}{\overline{AB}}=\frac{\overline{BE}}{\overline{BC}}=\frac{\overline{CF}}{\overline{CA}}$

    求证:
    所有这样的
    $\triangle DEF$重心是一个定点.

\item 如图,已知平行六面体$OADB-CEFG$, 求证:若对角
线$\overline{AG}$与平面$(O,D,E)$相交于$H$, $OH$与侧面$ADFE$
相交于$K$, 求证$\overline{AH}=\frac{1}{3}\overline{AG}$,$\overline{OH}=\frac{2}{3}\overline{OK}$

\begin{figure}[htp]\centering
    \begin{minipage}[t]{0.48\textwidth}
    \centering
\begin{tikzpicture}[>=latex, scale=1]
\tkzDefPoints{0/0/O, 3/0/A, 3.5/1/D, .5/1/B, .4/3/C}
\tkzDefPointsBy[translation = from O to C](A,B,D){E,G,F}
\tkzDrawPolygon(C,G,F,E)
\tkzDrawSegments(A,E D,F E,D F,A E,O A,D)
\tkzDrawSegments[dashed](A,G B,G B,D O,D)
\tkzDrawSegments[->, thick](O,A O,C)
\tkzDrawSegments[->, dashed, thick](O,B)
\tkzInterLL(E,D)(F,A) \tkzGetPoint{K}
\tkzDrawSegments[dashed](O,K)

\node at (1.5,0)[below]{$\vec{a}$};
\node at (.25,0.5)[right]{$\vec{b}$};
\node at (.2,1.5)[left]{$\vec{c}$};

\tkzLabelPoints[below](O,A)
\tkzLabelPoints[above right](B,D)
\tkzLabelPoints[right](K)
\tkzLabelPoints[above left](C,E)
\tkzInterLL(A,G)(O,K) \tkzGetPoint{H}
\tkzLabelPoints[above ](G,F,H)
    \end{tikzpicture}
    \caption*{第5题}
    \end{minipage}
    \begin{minipage}[t]{0.48\textwidth}
    \centering
    \begin{tikzpicture}[>=latex, scale=1]
\tkzDefPoints{0/0/B, 3/0/C, 2/2.6/A}
\tkzDefMidPoint(A,B)   \tkzGetPoint{R}
\tkzDefMidPoint(A,C)   \tkzGetPoint{Q}
\tkzDefMidPoint(C,B)   \tkzGetPoint{P}

\tkzDrawPolygon(A,B,C)
\tkzDrawSegments(A,P B,Q C,R)
\tkzLabelPoints[below](P,C,B)
\tkzLabelPoints[above](A)
\tkzLabelPoints[left](R)
\tkzLabelPoints[right](Q)
    \end{tikzpicture}
    \caption*{第6题}
    \end{minipage}
    \end{figure}

\item  在$\triangle ABC$的三条边上各取一点$P$、$Q$、$R$, 若$\Vec{AR}=x\Vec{RB}$, $\Vec{BP}=y\Vec{PC}$, $\Vec{CQ}=z\Vec{QA}$. 求证:$AP$、$BQ$、$CR$共
点的充要条件是
$xyz=1$.

\item  在$\triangle ABC$三边所在的直线
上各取一点$P$、$Q$、$R$, 使
\[\Vec{AR}=x\Vec{RB},\qquad \Vec{BP}=y\Vec{PC},\qquad \Vec{CQ}=z\Vec{QA}\]
求证:$P$、$Q$、$R$三点共线的充要条件是$xyz=-1$.
\begin{figure}[htp]\centering
\begin{tikzpicture}[>=latex, scale=1]
\tkzDefPoints{0/0/B, 3/0/C, 2/2.6/A}
\tkzDefPointWith[linear, K=.7](A,B) \tkzGetPoint{R}
\tkzDefPointWith[linear, K=.4](A,C) \tkzGetPoint{Q}
\tkzInterLL(B,C)(R,Q)\tkzGetPoint{P}
\tkzDrawPolygon(A,B,C)
\tkzDrawSegments(Q,P B,P C,R)
\tkzLabelPoints[below](P,C,B)
\tkzLabelPoints[above](A)
\tkzLabelPoints[above left](R)
\tkzLabelPoints[right](Q)
    \end{tikzpicture}
    \caption*{第7题}
    \end{figure}
\end{enumerate}

\section{垂直与度量问题}
这节我们举例说明,用内积运算处理度量问题的一般方
法和技巧.

\begin{example}
    证明勾股定理(图4.11).
\end{example}

\begin{proof}
    在直角$\triangle ABC$中,$\angle B=90^{\circ}$, 则
\[\begin{split}
   \Vec{AC}&=\Vec{AB}+\Vec{BC}\\
   \Vec{AC}\cdot \Vec{AC}&= \left(\Vec{AB}+\Vec{BC}\right)\cdot \left(\Vec{AB}+\Vec{BC}\right)\\
   &=\Vec{AB}\cdot \Vec{AB}+\Vec{BC}\cdot \Vec{BC}
\end{split}\]
即:$\overline{AC}^2=\overline{AB}^2+\overline{BC}^2$
\end{proof}

\begin{figure}[htp]\centering
    \begin{minipage}[t]{0.48\textwidth}
    \centering
\begin{tikzpicture}[>=latex, scale=.8]
\tkzDefPoints{0/0/A, 3/0/B, 3/4/C}
\tkzDrawSegments[->](A,B B,C A,C)
\tkzLabelPoints[below](A,B)
\tkzLabelPoints[above](C)
    \end{tikzpicture}
    \caption{}
    \end{minipage}
    \begin{minipage}[t]{0.48\textwidth}
    \centering
    \begin{tikzpicture}[>=latex, scale=1]
        \tkzDefPoints{0/0/A, 3/0/B, 2/3/C}
        \tkzDrawSegments[->](A,B B,C A,C)
        \tkzLabelPoints[below](A,B)
        \tkzLabelPoints[above](C)
\node at (1.5,0)[below]{$c$};
\node at (1,1.5)[left]{$b$};
\node at (2.5,1.5)[right]{$a$};
    \end{tikzpicture}
    \caption{}
    \end{minipage}
    \end{figure}


\begin{example}
    证明余弦定理(图4.12).
\end{example}

\begin{proof}
在$\triangle ABC$中,
\[\begin{split}
    \Vec{AC}&=\Vec{AB}+\Vec{BC}\\
    \Vec{AC}\cdot  \Vec{AC}&=\left(\Vec{AB}+\Vec{BC}\right)\cdot \left(\Vec{AB}+\Vec{BC}\right)\\
    &=\Vec{AB}\cdot \Vec{AB}+\Vec{BC}\cdot \Vec{BC}+2\Vec{AB}\cdot \Vec{BC}
\end{split}\]
即
\[\begin{split}
    |\Vec{AC}|^2&=|\Vec{AB}|^2+|\Vec{BC}|^2-2|\Vec{AB}||\Vec{BC}|\cos B\\
b^2&=c^2+a^2-2ca\cos B
\end{split}\]
同理可证:
\[\begin{split}
  c^2&=a^2+b^2-2ab\cos C\\
a^2&=b^2+c^2-2bc\cos A  
\end{split}\]
\end{proof}

由例4.10、例4.11的证明过程可以看到,为了得到三角形的边
角关系,只要写出三角形中边向量所要满足的关系,然后作
内积运算,向量关系就可转化为数量关系.



\begin{example}
    证明射影定理与正弦定理.
\end{example}

\begin{proof}
过$A$点引单位向量$\vec{e}_1\parallel \Vec{AC}$, $\vec{e}_2
\bot \vec{e}_1$ (图4.13)
\begin{equation}
    \begin{split}
    \Vec{AC}&=\Vec{AB}+\Vec{BC}\\
    \Vec{AC}\cdot \vec{e}_1&=\Vec{AB}\cdot \vec{e}_1+\Vec{BC}\cdot \vec{e}_1\\
    |\Vec{AC}|&=|\Vec{AB}|\cos A+|\Vec{BC}|\cos C
\end{split}
\end{equation}
即:$b=c\cos A+a\cos C$
同理可证:
\[c=a\cos B+b\cos A,\qquad a=b\cos C+c\cos B\]
(4.5)式两边分别对$\vec{e}_2$取内积运算,则
\[\begin{split}
    \Vec{AC}\cdot \vec{e}_2&=\Vec{AB}\cdot \vec{e}_2+\Vec{BC}\cdot \vec{e}_2\\
    0&=|\Vec{AB}|\cos(90^{\circ}-A)+| \Vec{BC} |\cos(90^{\circ}+C)
\end{split}\]
即:$0=c\sin A-a\sin C\quad \Rightarrow\quad \frac{a}{\sin A}=\frac{c}{\sin C}$

同理可证:$\frac{a}{\sin A}=\frac{b}{\sin B}$,
所以
\[\frac{a}{\sin A}=\frac{b}{\sin B}=\frac{c}{\sin C}\]
\end{proof}

在例4.12的证明中,我们先写出三角形三个边向量所满足
的向量关系式,然后,分别对$\vec{e}_1$、$\vec{e}_2$两个互相垂直的单位向量取内积运算,这样就很容易地证明了射影定理和正弦定
理.适当选取单位向量,对题设条件所满足的向量关系式进
行内积运算是处理一些直线形边角关系的
基本方法之一.

\begin{figure}[htp]\centering
    \begin{minipage}[t]{0.48\textwidth}
    \centering
\begin{tikzpicture}[>=latex, scale=1]
    \tkzDefPoints{0/0/A, 3/0/C, 2/2/B}
    \tkzDrawSegments[->](A,B B,C A,C)
    \tkzLabelPoints[below](A,C)
    \tkzLabelPoints[above](B)
\node at (1.5,0)[below]{$b$};
\node at (1,1)[left]{$c$};
\node at (2.5,1)[right]{$a$};
\draw[thick,->](0,0)--node[below]{$\vec{e}_1$}(1,0);
\draw[thick,->](0,0)--node[left]{$\vec{e}_2$}(0,1);

    \end{tikzpicture}
    \caption{}
    \end{minipage}
    \begin{minipage}[t]{0.48\textwidth}
    \centering
    \begin{tikzpicture}[>=latex, scale=1]
\tkzDefPoints{0/0/B, 4/0/C, 3.5/2/A, 2.7/0/P}
\node at (1.75,1)[left]{$c$};
\node at (3.75,1)[right]{$b$};
\node at (2,0)[below]{$a$};
\tkzDrawSegments[->](A,B A,C A,P)
\tkzDrawSegments(B,C)
\tkzLabelPoints[below](B,P,C)
\tkzLabelPoints[above](A)
\draw[->](P)--node[above]{$\vec{a}$}+(160:1);

    \end{tikzpicture}
    \caption{}
    \end{minipage}
    \end{figure}
 
\begin{example}
    利用内积运算证明角平分线定理.
\end{example}


\begin{proof}
设$\overline{AP}$为$\triangle ABC$中内角$A$的平
分线,则$\angle BAP=\angle PAC=\alpha$, 取单位向量
$\vec{e}\bot \Vec{AP}$(图4.14)
\[\begin{split}
  \Vec{AB}\cdot \vec{e}&=|\Vec{AB} |\cos(90^{\circ}-\alpha)=|\Vec{AB} |\sin\alpha\\
  \Vec{AC}\cdot \vec{e}&=|\Vec{AC}|\cos(90^{\circ}+\alpha)=-|\Vec{AC}|\sin\alpha  
\end{split}\]
\[\frac{\Vec{AB}}{\Vec{AC}}=\frac{\Vec{AB}\cdot \vec{e}}{\Vec{AC}\cdot \vec{e}}=\frac{|(\Vec{AP}+\Vec{PB})\cdot \vec{e}|}{|(\Vec{AP}+\Vec{PC})\cdot \vec{e}|}=\frac{|\Vec{PB}\cdot \vec{e}|}{\Vec{PC}\cdot \vec{e}}=\frac{\Vec{PB}}{\Vec{PC}}\]
\end{proof}


\begin{example}
    在直角$\triangle ABC$中,$AD$是斜边$\overline{BC}$上的高,作
    $DE\bot AB$, $DF\bot AC$, $E$、$F$是垂足.
    
    求证:$\frac{\overline{BE}}{\overline{CF}}=\frac{\overline{AB}^3}{\overline{AC}^3}$
\end{example}

\begin{proof}
设$\frac{\overline{EB}}{\overline{AB}}=\lambda$, $\frac{\overline{FC}}{\overline{AC}}=\mu$,则:  
\[\Vec{EB}=\lambda\Vec{AB},\quad \Vec{FC}=\mu\Vec{AC},\quad \Vec{AE}=(1-\lambda)\Vec{AB},\quad \Vec{AF}=(1-\mu)\Vec{AC}\]
\[\Vec{AD}=\Vec{AE}+\Vec{AF}=(1-\lambda)\Vec{AB}+(1-\mu)\Vec{AC}\]
因为$D\in BC$,则:
\[(1-\lambda)+(1-\mu)=1\quad \Rightarrow\quad \lambda+\mu=1\]
又$\because\quad \Vec{AD}\bot \Vec{BC}$,因此:
\[\left[(1-\lambda)\Vec{AB}+(1-\mu)\Vec{AC}\right]\cdot \left(\Vec{AC}-\Vec{AB}\right)=0\]
由此可得:
\[\frac{\lambda}{\mu}=\frac{\overline{AB}^2}{\overline{AC}^2}\]
\[\frac{\overline{EB}}{\overline{FC}}=\frac{\lambda\overline{AB}}{\mu\overline{AC}}=\frac{\overline{AB}^3}{\overline{AC}^3}\]
\end{proof}

\begin{figure}[htp]\centering
    \begin{minipage}[t]{0.48\textwidth}
    \centering
\begin{tikzpicture}[>=latex, scale=1]
\tkzDefPoints{0/0/B, 4/0/C, 2.5/2/A, 2.5/0/D}
\tkzDefPointBy[projection =onto A--B](D) \tkzGetPoint{E}
\tkzDefPointBy[projection =onto A--C](D) \tkzGetPoint{F}
\tkzDrawPolygon(A,C,B)
\tkzDrawSegments(D,E D,F)
\tkzDrawSegments[->](A,E A,F  A,D)
\tkzLabelPoints[below](B,C,D)
\tkzLabelPoints[left](E)
\tkzLabelPoints[right](F)
\tkzLabelPoints[above](A)
    \end{tikzpicture}
    \caption{}
    \end{minipage}
    \begin{minipage}[t]{0.48\textwidth}
    \centering
    \begin{tikzpicture}[>=latex, scale=1]
\tkzDefPoints{0/0/A, 5/0/B, 2.5/0/O}
\tkzDefPoint(36.87:4){P}
\tkzDrawPolygon(A,B,P)  \tkzDrawSegments(P,O)
\tkzLabelPoints[below](B,A,O)
\tkzLabelPoints[above](P)
    \end{tikzpicture}
    \caption{}
    \end{minipage}
    \end{figure}

\begin{example}
    设一动点$P$, 到两点$A$、$B$的距
离的平方和等于常数$k$, 求$P$点的轨迹.
\end{example}

\begin{solution}
    取$\overline{AB}$的中点$O$, 则
\[\Vec{PA}=\Vec{PO}+\Vec{OA},\qquad \Vec{PB}=\Vec{PO}+\Vec{OB}\]
\[\begin{split}
    \overline{PA}^2&=\overline{PO}^2+\overline{OA}^2+2\Vec{PO}\cdot\Vec{OA}\\
    \overline{PB}^2&=\overline{PO}^2+\overline{OB}^2+2\Vec{PO}\cdot \Vec{OB}
\end{split}\]
因为$\overline{PA}^2+\overline{PB}^2=k,\quad \Vec{PO}\cdot\Vec{OA}=-\Vec{PO}\cdot \Vec{OB}$

所以:
\[2\left(\overline{PO}^2+\overline{OA}^2\right)=k,\qquad \overline{PO}=\sqrt{\frac{k}{2}-\overline{OA}^2}\]
于是,
$P$点到$O$点的距离是一个常数.即$P$点的轨迹是以$O$为
圆心,$\sqrt{\frac{k}{2}-\overline{OA}^2}$为半径的圆.
\end{solution}

\begin{example}
    已知正方形$ABCD$(图4.17), $P$为$\overline{BD}$上任一点,
$\overline{PE}\bot \overline{BC}$于$E$点,$\overline{PF}\bot \overline{CD}$于$F$点,求证:
$\overline{AP}=\overline{EF}$
且$AP\bot EF$.
\end{example}

\begin{proof}
设$\Vec{BP}=\lambda \Vec{BD}$, 正方形边长为$a$, 则
\[\begin{split}
    \Vec{AP}&=\Vec{AB}+\Vec{BP}+(1-\lambda)\Vec{AB}+\lambda\Vec{AD}\\
    \Vec{EF}&=\Vec{EC}+\Vec{CF}=(1-\lambda)\Vec{BC}+\lambda\Vec{CD}
\end{split}\]
\[\begin{split}
    \overline{AP}^2&=(1-\lambda )^2 a^2+\lambda a^2+2\lambda(1-\lambda)\Vec{AB}\Vec{AD}\\
&=(1-\lambda )^2 a^2+\lambda a^2\\
\overline{EF}^2&=(1-\lambda )^2 a^2+\lambda a^2
\end{split}\]
所以:$\overline{AP}=\overline{EF}$.因为
\[\Vec{AP}\cdot \Vec{EF}=\lambda (1-\lambda)\Vec{AB}\cdot \Vec{CD}+\lambda(1-\lambda)\Vec{AD}\cdot \Vec{BC}=0\]
所以
$\overline{AP} \bot \overline{EF}$.
\end{proof}

\begin{figure}[htp]\centering
    \begin{minipage}[t]{0.48\textwidth}
    \centering
\begin{tikzpicture}[>=latex, scale=1]
\tkzDefPoints{0/0/B, 3/0/C, 3/3/D, 0/3/A, 2.5/2.5/P, 3/2.5/F,2.5/0/E}
\tkzDrawPolygon(A,B,C,D)
\tkzDrawPolygon(P,E,F)
\tkzDrawSegments(B,D A,P)
\tkzLabelPoints[below](B,E,C)
\tkzLabelPoints[above](A,D,P)
\tkzLabelPoints[right](F)
    \end{tikzpicture}
    \caption{}
    \end{minipage}
    \begin{minipage}[t]{0.48\textwidth}
    \centering
    \begin{tikzpicture}[>=latex, scale=1]
\tkzDefPoints{0/0/B, 4/0/C, 2.5/2.5/A, 2/0/D}
\tkzDrawSegments(A,D)
\tkzDrawSegments[->](B,C A,B C,A)
\node at (1.25,1.25)[left]{$\vec{c}$};
\node at (3.25,1.25)[right]{$\vec{b}$};
\node at (1.5,0)[below]{$\vec{a}$};
\node at (2.25,1.25)[right]{$m_a$};
\tkzLabelPoints[below](B,D,C)
\tkzLabelPoints[above](A)
    \end{tikzpicture}
    \caption{}
    \end{minipage}
    \end{figure}

\begin{example}
    已知$\triangle ABC$ (图4.18), 
$\overline{BC}=a$, $\overline{CA}=b$, $\overline{AB}=c$,
求:
\begin{enumerate}
    \item 三边上的中线$m_a$, $m_b$, $m_c$;
    \item 三个角平分线$t_a$、$t_b$、$t_c$;
\item 三角形的面积$S$.
\end{enumerate}
\end{example}

\begin{solution}
\begin{enumerate}
    \item 如图:设$\Vec{BC}=\vec{a}$, $\Vec{CA}=\vec{b}$, $\Vec{AB}=\vec{c}$, 则$BC$边上的中线
$\Vec{AD}=\frac{1}{2}(-\vec{b}+\vec{c})$,
\[\begin{split}
    \Vec{AD}\cdot \Vec{AD}&=\frac{1}{4}(b^2+c^2-2\vec{b}\cdot \vec{c})\\
    |\Vec{AD}|^2&=\frac{1}{4}[b^2+c^2-(a^2-b^2-c^2)]=\frac{1}{4}(2b^2+2c^2-a^2)
\end{split}\]
因此
\[m_a=\frac{1}{2}\sqrt{2b^2+2c^2-a^2}\]
同理可得:
\[
    m_b=\frac{1}{2}\sqrt{2a^2+2c^2-b^2},\qquad
    m_c=\frac{1}{2}\sqrt{2a^2+2b^2-c^2}
\]

\item 因$\vec{t}_a=\frac{c}{b+c}\vec{b}+\frac{b}{b+c}\vec{c}$,则:
\[\begin{split}
\left|\vec{t_{a}}\right|^{2} &=\frac{c^{2} b^{2}}{(b+c)^{2}}+\frac{b^{2} c^{2}}{(b+c)^{2}}+2 \frac{b c}{(b+c)^{2}} \vec{b} \cdot \vec{c} \\
&=\frac{b c}{(b+c)^{2}}\left(2 b c+b^{2}+c^{2}-a^{2}\right) \\
&=\frac{b c}{(b+c)^{2}}\left[(b+c)^{2}-a^{2}\right] \\
&=\frac{b c}{(b+c)^{2}}[(a+b+c)(b+c-a)]
\end{split}\]  
令$a+b+c=2p$, 则
\[\begin{split}
    t^2_a&=\frac{4bc}{(b+c)^2}p(p-a)\\
t_a&=\frac{2\sqrt{bc}}{b+c}\sqrt{p(p-a)}
\end{split}\]
同理可得:
\[
    t_b=\frac{2\sqrt{ac}}{a+c}\sqrt{p(p-b)},\qquad
    t_c=\frac{2\sqrt{ab}}{a+b}\sqrt{p(p-c)}
\]

\item 
\[\begin{split}
    S^{2} &=\frac{1}{4} b^{2} c^{2} \sin ^{2} A=\frac{1}{4} b^{2} c^{2}\left(1-\cos ^{2} A\right) \\
    &=\frac{1}{4} b^{2} c^{2}\left[1-\left(\frac{\vec{b} \cdot \vec{c}}{b c}\right)^{2}\right] =\frac{1}{4} b^{2} c^{2}\left[1-\frac{\left(b^{2}+c^{2}-a^{2}\right)^{2}}{4 b^{2} c^{2}}\right] \\
    &=\frac{1}{16}\left[4 b^{2} c^{2}-\left(b^{2}+c^{2}-a^{2}\right)^{2}\right] \\
    &=\frac{1}{16}(a+b+c)(a+b-c)(a+c-b)(b+c-a) \\
    &=p(p-a)(p-b)(p-c) \\
    \end{split}\]
\[ S=\sqrt{p(p-a)(p-b)(p-c)}\]
\end{enumerate}
\end{solution}

\begin{example}
    证明定理:如果一条直线$a$垂直于平面$\pi$上的两条
相交直线$b$、$c$, 那么$a\bot\pi $ (图4.19).
\end{example}

\begin{proof}
    在平面$\pi$上,任取一条直
线$d$, 在$a,b,c,d$上分别取向量$\vec{a}$、$\vec{b}$、$\vec{c}$、$\vec{d}$. 由于$b$、$c$相交,
依共面向量定理,存在唯一的数偶
$(x,y)$, 使
\[\begin{split}
    \vec{d}&=x\vec{b}+y\vec{c}\\
    \vec{a}\cdot \vec{d}&=\vec{a}\cdot \left(x\vec{b}+y\vec{c}\right)=x\vec{a}\cdot \vec{b}+y\vec{a}\cdot \vec{c}=0
\end{split}\]
所以:$\vec{a}\bot \vec{d}$,即:$a\bot d$

这就是说$a$垂直于平面$\pi$上的任一条直线,所以$a\bot\pi$.
\end{proof}

\begin{figure}[htp]\centering
    \begin{minipage}[t]{0.48\textwidth}
    \centering
\begin{tikzpicture}[>=latex, yscale=.8]
\tkzDefPoints{0/0/A, 4/0/B, 5/2.5/C, 1/2.5/D}
\tkzDrawPolygon(A,B,C,D)
\draw[->](3,.5)--(1.4,1.7)node[above right]{$\vec{b}$};
\draw[->](1,.5)--(3.5,2)node[right]{$\vec{c}$};
\draw[->](3.5,.5)--node[right]{$\vec{d}$}(4,1.5);
\node at (0,0)[above right]{$\pi$};
\draw[->](2,2)--node[right]{$\vec{a}$}(2,3.5);
    \end{tikzpicture}
    \caption{}
    \end{minipage}
    \begin{minipage}[t]{0.48\textwidth}
    \centering
    \begin{tikzpicture}[>=latex, scale=1]
\tkzDefPoints{0/0/A, 4/0/C, 2.3/-1/B, 2.8/2/O}
\tkzDrawPolygon(O,A,B,C)
\tkzDrawSegments(O,B)
\tkzDefMidPoint(O,A)  \tkzGetPoint{E}
\tkzDefMidPoint(O,B)  \tkzGetPoint{F}
\tkzDefMidPoint(B,C)  \tkzGetPoint{G}
\tkzDefMidPoint(C,A)  \tkzGetPoint{H}
\tkzDrawSegments[dashed](A,C E,F F,G G,H E,H)
\tkzLabelPoints[above](O)
\tkzLabelPoints[right](F,G)
\tkzLabelPoints[left](E)
\tkzLabelPoints[below](A,B,C,H)
    \end{tikzpicture}
    \caption{}
    \end{minipage}
    \end{figure}

\begin{example}
    已知空间四边形$O-ABC$, $\overline{OA}=\overline{OB}$, $\overline{CA}=\overline{CB}$.
    $E$、$F$、$G$、$H$分别为$\overline{OA}$、$\overline{OB}$、$\overline{CB}$、$\overline{CA}$的中点,求证
    四边形$EFGH$是矩形(图4.20).
\end{example}

\begin{proof}
由于$E$、$F$、$G$、$H$分别是$\overline{OA}$
、$\overline{OB}$、$\overline{CB}$、$\overline{CA}$的中点,所以
\[\Vec{EF}=\frac{1}{2}\Vec{AB},\qquad \Vec{HG}=\frac{1}{2}\Vec{AB}\]
\[\Vec{EF}=\Vec{HG}\quad \Rightarrow\quad \overline{EF}=\overline{HG},\quad EF\parallel HG\]
$\therefore\quad EFGH$是平行四边形.

因为
$\Vec{EF}=\frac{1}{2}\Vec{AB}=\frac{1}{2}(\Vec{OB}-\Vec{OA}),\quad \Vec{EH}=\frac{1}{2}\Vec{OC}$,所以
\[\begin{split}
        \Vec{E F} \cdot \Vec{E H} &=\frac{1}{2}(\Vec{O B}-\Vec{O A}) \cdot \frac{1}{2} \Vec{O C} \\
        &=\frac{1}{4}(\Vec{O B} \cdot \Vec{O C}-\Vec{O A} \cdot \Vec{O C}) \\
        &=\frac{1}{4}\left(|\Vec{O B}|^{2}+|\Vec{O C}|^{2}-|\Vec{B C}|^{2}-|\Vec{O A}|^{2}-|\Vec{O C}|^{2}+|\Vec{C A}|^{2}\right) \\
\end{split}\]
但已知$\Vec{O A}=\Vec{O B},\qquad \Vec{C A}=\Vec{C B}$,所以
\[\Vec{E F} \cdot \Vec{E H} =0 \quad \Rightarrow\quad  E F \bot E H\]
故得四边形$EFGH$是矩形.
\end{proof}

由以上各例可看到在空间证明两线垂直,内积运算仍是
非常有效的工具.

\begin{example}
    一定长线段$\overline{AB}$的两个端
    点,沿互相垂直的两条异面直线$\ell$、
    $m$运动,求它的中点的轨迹.
\end{example}

\begin{solution}
    如图4.21设$\overline{MN}$为$\ell$、$m$的
公垂线,$\overline{AB}=a$, $\overline{MN}=b$, $O$、$P$分
别为$\overline{MN}$、$\overline{AB}$的中点,
则
\[\Vec{OP}=\frac{1}{2}\left(\Vec{OA}+\Vec{OB}\right)=\frac{1}{2}\left(\Vec{OM}+\Vec{MA}+\Vec{ON}+\Vec{NB}\right)\]
因为$\Vec{OM}=-\Vec{ON}$,所以
\[\begin{split}
    \Vec{OP}&=\frac{1}{2}(\Vec{MA}+\Vec{NB})\\
    \Vec{OP}\cdot\Vec{MN}&=\frac{1}{2}\left(\Vec{MA}+\Vec{NB}\right)\cdot \Vec{MN}=0
\end{split}    \]
因此:$P$点一定在$\overline{MN}$的垂直平分面上.

因为$\Vec{OP}\cdot \Vec{OP}=\frac{1}{4}(|\Vec{MA}|^2+|\Vec{NB}|^2)$,
连$AN$, 易证,$\triangle AMN$与$\triangle ABN$都是直角三角形.
所以
\[\begin{split}
    \Vec{OP}\cdot \Vec{OP}&=\frac{1}{4}\left(\overline{AN}^2-\overline{MN}^2+\overline{AB}^2-\overline{AN}^2\right)\\
    &=\frac{1}{4}\left(\overline{AB}^2-\overline{MN}^2\right)=\frac{1}{4}(a^2-b^2)
\end{split}\]
即:$|\Vec{OP}|^2=\frac{1}{4}(a^2-b^2)$

由上式可知$P$点在以$O$为圆心,以$\frac{1}{2}\sqrt{a^2-b^2}$为半径的
圆上,因此$P$点的轨迹是$\overline{MN}$的垂直平分面上的一个圆:
$\odot\left(O,\frac{1}{2}\sqrt{a^2-b^2}\right)$
\end{solution}

\begin{figure}[htp]\centering
    \begin{minipage}[t]{0.48\textwidth}
    \centering
\begin{tikzpicture}[>=latex, scale=1]
\tkzDefPoints{0/3/A, 0/2/M, 3/2/N, 1.5/0.5/B}
\tkzDefMidPoint(M,N) \tkzGetPoint{O}
\tkzDefMidPoint(A,B) \tkzGetPoint{P}
\tkzDrawSegments(A,B A,N M,N O,P)
\tkzDrawLines[add = 1 and 1](A,M) 
\tkzDrawLines[add = .5 and .5](B,N) 
 \node at (0,4)[right]{$\ell$};
 \node at (3.5,3)[right]{$m$};
\tkzLabelPoints[left](A,M,P)
\tkzLabelPoints[right](B,N)
 \tkzLabelPoints[above](O)
    \end{tikzpicture}
    \caption{}
    \end{minipage}
    \begin{minipage}[t]{0.48\textwidth}
    \centering
    \begin{tikzpicture}[>=latex, scale=1]
\tkzDefPoints{0/0/B, 4/0/D, 2.8/-1/C, 2.3/2/A}
\tkzDrawPolygon(D,A,B,C)
\tkzDrawSegments[dashed](D,B)
\tkzDrawSegments(A,C)
\tkzLabelPoints[below](D,B,C)
\tkzLabelPoints[above](A)
    \end{tikzpicture}
    \caption{}
    \end{minipage}
    \end{figure}

\begin{example}
    如图4.22, 已知四面体
$A-BCD$, $AB\bot CD$, $AC\bot BD$.

求证:$AD\bot BC$.
\end{example}

\begin{proof}
    $\because\quad AB\bot CD,\quad AC\bot BD$

    所以
\[    \Vec{AB}\cdot \Vec{CD}=0,\qquad    \Vec{AC}\cdot \Vec{BD}=0\]
\[\left(\Vec{AD}+\Vec{DB}\right)\cdot \Vec{CD}=0,\qquad    \left(\Vec{AD}+\Vec{DC}\right)\cdot \Vec{BD}=0\]
\[\Vec{AD}\cdot \Vec{CD}=\Vec{BD}\cdot \Vec{CD},\qquad \Vec{AD}\cdot \Vec{BD}=\Vec{BD}\cdot \Vec{CD}\]
\[\begin{split}
    \Vec{AD}\cdot \Vec{CD}&=\Vec{AD}\cdot \Vec{BD}
    \\
    \Vec{AD}\cdot \left(\Vec{BD}+\Vec{DC}\right)&=0\\
    \Vec{AD}\cdot \Vec{BC}&=0
\end{split}\]
即:$\Vec{AD}\bot \Vec{BC},\qquad AD\bot BC$
\end{proof}


\begin{example}
    如图4.23在直二面角的棱上有两点$A$、$B$, $AC$和
    $BD$各在这个二面角的一个面内,并且都垂直于棱$AB$,设
    $\overline{AB}=8$, $\overline{AC}=6$, $\overline{BD}=24$, 求$\overline{CD}$的长.
\end{example}

\begin{solution}    
 如图4.23, 
\[|\Vec{CD}|^2=\Vec{CB}\cdot \Vec{CD}=\left(\Vec{CA}+\Vec{AB}+\Vec{BD}\right)\cdot \left(Vec{CA}+\Vec{AB}+\Vec{BD}\right)\]
因为
$Vec{AC}$、$Vec{AB}$、$Vec{BD}$互相正交,
所以
\[|\Vec{CD}|^2=|\Vec{CA}|^2+|\Vec{AB}|^2+|\Vec{BD}|^2=6^2+8^2+24^2=676\]
$\therefore\quad \overline{CD}=\sqrt{676}=26$
\end{solution}

\begin{figure}[htp]\centering
    \begin{minipage}[t]{0.48\textwidth}
    \centering
\begin{tikzpicture}[>=latex, scale=1]
\tkzDefPoints{0/0/a1, .5/1.9/a2, 1.5/-1.8/a3, 3/0/a4}
\tkzDefPointsBy[translation= from a1 to a4](a2,a3){a2',a3'}
\tkzDrawPolygon(a1,a2,a2',a4)
\tkzDrawPolygon(a1,a3,a3',a4)
\draw(1-.5,0)node[below]{$A$}--(1.25-.5,0.95)node[above]{$C$}--(2.5+.75,-.9)node[below]{$D$}--(1+1.5,0)node[above]{$B$};
    \end{tikzpicture}
    \caption{}
    \end{minipage}
    \begin{minipage}[t]{0.48\textwidth}
    \centering
    \begin{tikzpicture}[>=latex, scale=.8]
\tkzDefPoints{0/0/B, 4/0/C, 5/1/D, 1/1/A, 0/4/F}
\tkzDefPointsBy[translation= from B to F](C,D,A){G,H,E}
\tkzDrawPolygon(E,F,G,H)
\tkzDrawSegments(B,C C,D B,F C,G H,D G,D)
\tkzDrawSegments[->, dashed](A,B A,D A,E A,G)

\tkzDefPointWith[linear, K=.65](G,D) \tkzGetPoint{Q}
\tkzDefPointWith[linear, K=.6](A,C) \tkzGetPoint{P}
\tkzDrawSegments[dashed](A,C P,Q)
\node at (.75,.75)[left]{$\vec{e_1}$};
\node at (1,2.5)[right]{$\vec{e_3}$};
\node at (3,1)[above]{$\vec{e_2}$};
\tkzLabelPoints[below](B,C)
\tkzLabelPoints[above](E,H,G)
\tkzLabelPoints[right](Q,D)
\tkzLabelPoints[left](F,A,P)
    \end{tikzpicture}
    \caption{}
    \end{minipage}
    \end{figure}

\begin{example}
    已知正方体$ABCD-EFGH$ (图4.24) 其棱长为
1. 求:$AC$与$DG$的公垂线的垂足$P$、$Q$的位置和$AC$与$DG$ 
间的距离.
\end{example}

\begin{solution}
设$\Vec{AB}=\vec{e}_1$, $\Vec{AD}=\vec{e}_2$, $\Vec{AE}=\vec{e}_3$, $\Vec{AP}=x\Vec{AC}$, $\Vec{DQ}=y\Vec{DG}$ 

由于$\Vec{AC}=\vec{e}_1+\vec{e}_2,\quad \Vec{DG}=\vec{e}_1+\vec{e}_3$,所以
\[\Vec{AP}=x\vec{e}_1+x\vec{e}_2,\qquad \Vec{AQ}=y\vec{e}_1+y\vec{e}_3\]
\[\begin{split}
    \Vec{AQ}&=\Vec{AD}+\Vec{DQ}=y\vec{e}_1+\vec{e}_2+y\vec{e}_3\\
    \Vec{PQ}&=\Vec{AQ}-\Vec{AP}=(y-x)\vec{e}_1+(1-x)\vec{e}_2+y\vec{e}_3\\
\end{split}\]
又因 $\Vec{P Q} \perp \Vec{A C}$, $\Vec{P Q} \perp \Vec{D G}$, 则
\[\begin{cases}
    {\left[(y-x) \vec{e}_{1}+(1-x) \vec{e}_{2}+y \vec{e}_{3}\right] \cdot\left(\vec{e}_{1}+\vec{e}_{2}\right)=0} \\
    {\left[(y-x) \vec{e}_{1}+(1-x) \vec{e}_{2}+y \vec{e}_{3}\right] \cdot\left(\vec{e}_{1}+\vec{e}_{3}\right)=0}
\end{cases}\]
由此可得方程组
\[\begin{cases}
    2 x-y=1 \\
x-2 y=0
\end{cases}\]
解之得:$x=\frac{2}{3}, \qquad y=\frac{1}{3}$

所以:$\Vec{AP}=\frac{2}{3}\Vec{AC},\quad \Vec{DQ}=\frac{1}{3}\Vec{DG}$
\[\Vec{PQ}=\frac{1}{3}\left(-\vec{e}_1+\vec{e}_2+\vec{e}_3\right)\]
于是$P$、$Q$两点的位置可定,且$AC$与$DG$的距离就是$|\Vec{PQ}|$. 
\[|\Vec{PQ}|=\sqrt{\Vec{PQ}\cdot \Vec{PQ}}=\sqrt{\frac{1}{9}(1+1+1)}=\frac{\sqrt{3}}{3}\]
\end{solution}


\section*{习题4.2}
\addcontentsline{toc}{subsection}{习题4.2}

试用向量运算证明以下各题:

\begin{enumerate}
    \item 试证:点$P$在$\overline{AB}$的垂直平分上的充要条件是$|\Vec{PA}|=|\Vec{PB}|$
\item 试证:对角线互相垂直的平行四边形是菱形.
\item 试证:平行四边形的对角线等长的充要条件是这个平行
四边形是矩形.
\item 求证:直角三角形斜边上的中线等于斜边的一半.
\item 等腰三角形顶角的平分线是底边上的高.
\item 求证:四边形$ABCD$中,对角线互相垂直的充要条件是
\[\overline{AB}^2 +\overline{CD}^2=\overline{AD}^2+\overline{BC}^2\]
\item 求证:平行四边形$ABCD$中的锐角$A$为$45^{\circ}$的充要条件
是$$\overline{AC}^2\cdot \overline{BD}^2=\overline{AB}^4+\overline{AD}^4$$
\item 已知$\ell$是一条直线,$A$、$B$为$\ell$外同侧的两个定点,$P\in\ell$, 

求
证:$|\Vec{AP}|+|\Vec{PB}|$取极小值的充要条件是
\[\frac{\vec{a}\cdot \vec{n}}{|\vec{a}|}=\frac{\vec{b}\cdot \vec{n}}{|\vec{b}|}\]
(其中$\vec{n}\bot \ell$且方向指向$A$、$B$所在的
那一侧)并解释其几何意义.
\item 求证:长方体的对角线的平方等于长、宽、高的平方
和.
\item 已知$\overline{AB}$在平面$\pi$内,$\overline{AC}\bot\pi$, $\overline{BD}\bot \overline{AB}$且与$\pi$所成
角为$30^{\circ}$, 若$\overline{AB}=a$, $\overline{AC}=\overline{BD}=b$, 求$C$和$D$间的
距离.
\item 线段
$\overline{DE}$
同时垂直于矩形$ABCD$的两边
$\overline{DA}$、$\overline{DC}$. 设
$\overline{AB}=12$cm, $\overline{BC}=9$cm, $\overline{DE}
=8$cm, 求$B$、$E$两点间
的距离.
\item 一点$P$在两个相交平面上的投影各为$A$、$B$, 求证连线$AB$
垂直于两个平面的交线.
\end{enumerate}


\section{圆}

一个以$O$点为圆心,$R$为半径的圆是平面$\pi$上所有满足
$\Vec{OX}\cdot\Vec{OX}=R^2$的点$X$的集合.即
\[\odot (O,R)=\left\{X:X\in \pi,\; \Vec{OX}\cdot\Vec{OX}=R^2\right\}\]

在平面几何中,关于圆有下列几个基本定理:
\begin{enumerate}
\item 圆周角定理;
\item 弦切角定理;
\item 圆幂定理.
\end{enumerate}

当时,这三个定理是按1、2、3的顺序来证明的,现在我们用向量的运算律,直接证明圆幂定理,
并由此再推出弦切角定理与圆周角定理.

\begin{blk}{定理1}
    如图4.25设平面上有$\odot (O,R)$及圆外一点$P$, 过
$P$点引圆割线交圆于$B$、$C$点,则有$\Vec{PB}\cdot \Vec{PC}=|OP|^2-R^2$.
\end{blk}

\begin{figure}[htp]
    \centering
\begin{tikzpicture}[>=latex, scale=1.6]
\tkzDefPoints{0/0/O, -2.5/0/P}
\tkzDefPoint(80:.5){C1}
\tkzDefPoint(60:1){C2}
\draw(O) circle(1);
\tkzInterLC(C1,P)(O,C2) \tkzGetPoints{B}{C}

\tkzDefPointWith[linear, K=0.8](B,C) \tkzGetPoint{X}
\tkzDrawSegments[->](P,C O,X)
\tkzLabelPoints[above](C,X)
\tkzLabelPoints[left](P)
\tkzLabelPoints[below](O)
\tkzLabelPoints[below left](B)
\draw(P)--(O);
\draw(P)--+(23.58:3.5);
\node at (-.4,.916)[above]{$T_1$};
\tkzDefPointWith[linear, K=0.4](P,B) \tkzGetPoint{X1}
\draw[thick,->](P)--(X1)node[above left]{$\vec{u}$};
\end{tikzpicture}   
    \caption{}
\end{figure}


\begin{proof}
    在$\Vec{PC}$上取单位向量
$\vec{u}$, 对$PC$上任意一点$X$, 都存在实数$x$, 使
$\Vec{PX}=x\vec{u}$.这里$x$就是$\Vec{PX}$
的长度.由向量加法,有
\[\Vec{OX}=\Vec{OP}+\Vec{PX}=\Vec{OP}+x\vec{u}\]
因此:
\[\begin{split}
    \Vec{OX}\cdot \Vec{OX}&=\left(\Vec{OP}+x\vec{u}\right)^2\\
    &=\Vec{OP}\cdot \Vec{OP}+2\left(\Vec{OP}\cdot \vec{u}\right)x+x^2(\vec{u}\cdot \vec{u})\\
    &=x^2+2\left(\Vec{OP}\cdot \vec{u}\right)+|\Vec{OP}|^2
\end{split}\]

当点$X$与点$B$或$C$重合时,$\Vec{OX}\cdot \Vec{OX}=R^2$,这时有
\[x^2+2\left(\Vec{OP}\cdot \vec{u}\right)x+\left(|\Vec{OP}|^2-R^2\right)=0\]
这里$x$是$\Vec{PB}$或$\Vec{PC}$的长度,若令$|\Vec{PB}|=\beta$,
 $|\Vec{PC}|=\gamma$,
则由韦达定理得:
\[\beta\gamma=|\Vec{OP}|^2-R^2\]

$\because\quad \Vec{PB}$与$\Vec{PC}$
同向

$\therefore\quad \Vec{PB}\cdot \Vec{PC}=\beta\gamma$, 因此:
\[\Vec{PB}\cdot \Vec{PC}=|\Vec{OP}|^2-R^2=|PT_1|^2\]
\end{proof}

\begin{analyze}
\begin{enumerate}
    \item 上面的证明对$P$为圆内一点也是适用的,不过
    这时设$\Vec{PX}=x\vec{u}$, 实数$x$可正可负,$|x|=|\Vec{PX}|$.
    
    由于$\Vec{PB}$与
    $\Vec{PC}$反向,则$\Vec{PB}\cdot \Vec{PC}$和$|OP|^2-R^2$都是负值.

\item 在证明过程中,我们还得到
\[\beta+\gamma=-2\Vec{OP}\cdot \vec{u}\]
我们令$B$、$C$、$P'$、$P$成调和点列,即此四点共线
且满足
\[\frac{\Vec{BP}}{\Vec{PC}}\cdot \frac{\Vec{CP'}}{\Vec{P'B}}=-1\]
令$\Vec{PP'}=y\vec{u}$,则有
\[-1=\frac{\Vec{BP}}{\Vec{PC}}\cdot \frac{\Vec{CP'}}{\Vec{P'B}}=\frac{-\beta}{\gamma}\x \frac{y-\gamma}{\beta-y}\]
得:$y=\frac{2\beta\gamma}{\beta+\gamma}$,因此:
\[\begin{split}
    \Vec{PP'}\cdot \Vec{PO}&=\left(\frac{2\beta \gamma}{\beta+\gamma}\vec{u}\right)\cdot \left(-\Vec{OP}\right)\\
    &=\frac{\beta \gamma}{\beta+\gamma}\left(-2\Vec{OP}\cdot \vec{u}\right)\\
    &=\beta\gamma =|\Vec{OP}|^2-R^2
\end{split}\]
这就是说$\Vec{PP'}$在$\Vec{PO}$上的投影的长度等于常数
$\frac{|\Vec{OP}|^2-R^2}{|\Vec{OP}|}$

由于
\[\frac{|\Vec{OP}|^2-R^2}{|\Vec{OP}|}=\frac{|\Vec{PT_1}|^2}{|\Vec{OP}|}=|\Vec{PT_1}|\cos\angle T_1PO=|PQ|\]
其中$Q$是线段$T_1T_2$的中点(图4.26)

$\therefore\quad $过$P$点的任一割线与$T_1T_2$的交
点$P'$与$B$、$C$、$P$成调和点列.
\end{enumerate}
\end{analyze}

下面,我们由圆幂定理推导
弦切角定理与圆周角定理.

\begin{figure}[htp]\centering
    \begin{minipage}[t]{0.48\textwidth}
    \centering
\begin{tikzpicture}[>=latex, scale=1.5]
\tkzDefPoints{0/0/O, -2/0/P}
\tkzDefPoint(120:1){T_1} \tkzDefPoint(-120:1){T_2}
\draw(0,0) circle (1);
\tkzDrawSegments(P,O P,T_1 P,T_2 T_1,T_2)
\tkzLabelPoints[above](P,T_1)
\tkzLabelPoints[below](O,T_2)
\node at (-.5,0)[above right]{$Q$};
    \end{tikzpicture}
    \caption{}
    \end{minipage}
    \begin{minipage}[t]{0.48\textwidth}
    \centering
    \begin{tikzpicture}[>=latex, scale=1.5]
\tkzDefPoints{0/0/O, -2.5/-1/P,0/-1/T}
\tkzDefPoint(-15:1){C} \tkzDefPoint(60:1){C'}
\draw(0,0) circle (1);
\tkzInterLC(C,P)(O,C')  \tkzGetPoints{B'}{B}
\tkzInterLL(C',B)(P,T)  \tkzGetPoint{P'}
\tkzDrawSegments[thick](C,T C',T C,B C',B P,T P,C B,T B,P')
\tkzLabelPoints[below](P,T,P')
\tkzLabelPoints[right](C,C')
\tkzLabelPoints[above](B)
    \end{tikzpicture}
    \caption{}
    \end{minipage}
    \end{figure}


如图4.27由圆幂定理,在
$\triangle PTB$与$\triangle PCT$中,
由于$\frac{\Vec{PT}}{\Vec{PB}}=\frac{\Vec{PC}}{\Vec{PT}}$
,$\angle P$是公共角,

$\therefore\quad \triangle PTB \backsim \triangle PCT$

得$\angle PTB=\angle PCT$,
, 这就是弦切角定理.

同理$\angle P'TB=\angle BC'T$,
则$\angle BCT=\angle BC'T$, 这就是圆周
角定理.


\section*{习题4.3}
\addcontentsline{toc}{subsection}{习题4.3}

\begin{enumerate}
    \item 试证圆的相交弦定理.
    \item 从圆$O$外一点$P$引圆的切线$PT_1$和$PT_2$, $T_1$、$T_2$为切点.
再引圆$O$的割线$PQR$, 交圆$O$于$Q$、$R$, 交$T_1T_2$于$T$, 
设$|PQ|=a$, $|PR|=b$, $|PT|=t$, 求证:
\[\frac{1}{a}+\frac{1}{b}=\frac{2}{t}\]
\end{enumerate}

\section*{复习题四}
\addcontentsline{toc}{section}{复习题四}

\begin{enumerate}

\item 在空间中,设有线性关系$\lambda_1\vec{a}_1+\lambda_2\vec{a}_2+\lambda_3\vec{a}_3+\lambda_4\vec{a}_4+\lambda_5\vec{a}_5=\vec{0}$,且$\lambda_1\lambda_2\lambda_3\lambda_4\lambda_5\ne 0$, 若
\begin{enumerate}
    \item $\vec{a}_1,\vec{a}_2,\vec{a}_3,\vec{a}_4,\vec{a}_5$都是非零向量;
    \item 有且只有$\vec{a}_4=\vec{a}_5=\vec{0}$;
    \item 有且只有三个零向量.
\end{enumerate}
问在各种情况下,它们的几何意义
分别是什么?
\item 试作一给定有向线段$\Vec{AB}$的定比分点,其比值分别为:
$\frac{1}{2},2,-2,-\frac{1}{2}$.
\item 设$P$、$A$、$B$是共线的相异三点,
$\Vec{AP}=\rho\Vec{PB}$, 
试用$\rho$去表达下列五个实数$\alpha$、$\beta$、$\gamma$、$\delta$、$\varepsilon$.
\[\begin{split}
    \Vec{BP}&=\alpha\Vec{PA},\qquad \Vec{PA}=\beta\Vec{AB},\qquad \Vec{BA}=\gamma\Vec{AP}\\
    \Vec{PB}&=\delta\Vec{BA},\qquad \Vec{AB}=\varepsilon\Vec{BP}
\end{split}\]
\item 如图,$\ell_1,\ell_2$交于$O$点,$\vec{u},\vec{v}$是$\ell_1,\ell_2$方向的单位向量,
设
\[\begin{split}
    \Vec{OA_1}=\alpha_1\vec{u},\qquad  \Vec{OB_1}=\beta_1\vec{u},\qquad  \Vec{OC_1}=\gamma_1\vec{u}\\
    \Vec{OA_2}=\alpha_2\vec{v},\qquad  \Vec{OB_2}=\beta_2\vec{v},\qquad  \Vec{OC_2}=\gamma_2\vec{v}
\end{split}\]
试用$\vec{u},\vec{v}$的线性组合表示
$\Vec{OP}$、$\Vec{OQ}$、$\Vec{OR}$. 并证明$P$、$Q$、$R$三点共线.

\begin{figure}[htp]
    \centering
\begin{tikzpicture}[>=latex]
\tkzDefPoints{0/0/O, 2/0/A_2, 3.5/0/B_2, 5/0/C_2}
\tkzDefPoint(40:2.5){A_1}\tkzDefPoint(40:4){B_1}
\tkzDefPoint(40:5.5){C_1}
\tkzLabelPoints[above](A_1,B_1,C_1)
\tkzLabelPoints[below](A_2,B_2,C_2)
\draw(0,0)node[left]{$O$}--(6,0)node[right]{$\ell_2$};
\draw(0,0)--(40:6)node[right]{$\ell_1$};
\tkzDrawSegments(A_1,C_2 A_1,B_2 B_1,A_2 B_1,C_2 C_1,A_2 C_1,B_2)
\draw[thick,->](0,0)--node[below]{$\vec{v}$}(1,0);
\draw[thick,->](0,0)--node[above]{$\vec{u}$}(40:1);

\tkzInterLL(A_1,B_2)(A_2,B_1) \tkzGetPoint{P};
\tkzInterLL(A_1,C_2)(A_2,C_1) \tkzGetPoint{Q};
\tkzInterLL(C_1,B_2)(C_2,B_1) \tkzGetPoint{R};
\tkzDrawPoints(P,Q,R)
\tkzLabelPoints[right](R)
\tkzLabelPoints[left](P)
\tkzLabelPoints[above](Q)

\end{tikzpicture}
    \caption*{第4题}
\end{figure}


\item 在$\triangle ABC$的外面作正方形$ABEF$和$ACGH$, 又设$D$为
$\Vec{BC}$的中点,求证:
\begin{enumerate}
    \item $\Vec{AF}\cdot \Vec{AH}=\Vec{AB}\cdot \Vec{AC}$
    \item $BH\bot CF$且$\overline{BH}=\overline{CF}$
    \item $AD\bot FH$且$\overline{AD}=\frac{1}{2} \overline{FH}$
\end{enumerate}

\item 已知四边形$ABCD$内接于圆且$AC\bot BD$于$E$, $F$是边
$\Vec{BC}$的中点,求证:$EF\bot AD$
\item 已知$O$、$M$、$H$三点分别是$\triangle ABC$的外心,重心和
垂心,求证:$O$、$M$、$H$三点共线且$\overline{OM}=\frac{1}{2}\overline{MH}$.
\item 求证:连结四面体的一个顶点和这个顶点所对的面的重
心的四条线段交于同一点,且这交点分线段的比例都
是3:1.
\item 求证平行六面体的四条对角线相交于一点.
\item 在四面体$ABCD$中,如果$AB\bot DC$且$AD\bot BC$, 试
证明:
\[|\Vec{AB}|^2+|\Vec{DC}|^2=|\Vec{AD}|^2+|\Vec{BC}|^2=|\Vec{AC}|^2+|\Vec{BD}|^2\]
\item 已知四面体$ABCD$, $G_1,G_2$分别是$\triangle ABC$和$\triangle ABD$的
重心,$M$是棱$CD$的中点,试确定过$G_1$、$G_2$、$M$三点的
平面与棱$AB$的交点的位置.
\item 已知正方体$ABCD-A_1B_1C_1D_1$的棱长为1, $E$、$F$分别
是棱$\overline{BC}$, $\overline{CC_1}$的中点,求下列各异面直线的距离.
\begin{multicols}{3}
    \begin{enumerate}
        \item $AA_1$与$BD_1$
        \item $AC$与$BD_1$
        \item $AC$与$EF$
    \end{enumerate}
\end{multicols}
\end{enumerate}

\setcounter{chapter}{4}
 
\chapter{向量的坐标运算~~直线与圆}



\section{向量的坐标运算}
\subsection{直角坐标系与向量的坐标}
在初中,我们已学习了平面直角坐标系,其要点如下:
选定一个长度单位,建立两条具有公共原点且互相垂直的数
轴(图5.1),通常一条为水平的数轴,称为横轴或$X$
轴,它的正向是由左到右,另一
条是和它垂直的轴称为纵轴或$Y$
轴,它的正向是从下到上。$X$
轴、$Y$轴总称为坐标轴、坐标轴的交点$O$称为坐标系的原点,这
样我们就说在平面上建立了直角
坐标系$OXY$, 这个平面就叫做
坐标平面,在坐标平面上任取一
点$P$, 过$P$引$X$轴、$Y$轴的垂线,设垂足分别是$M$、$N$, 如
果$M$在$X$轴上的坐标为$x$, $N$在$Y$轴上的坐标为$y$, 那么我
们就说$P$点的坐标是$(x,y)$, 记作$P(x,y)$, $x$称为
横坐标,$y$称为纵坐标。
\begin{figure}[htp]\centering
    \begin{minipage}[t]{0.48\textwidth}
    \centering
\begin{tikzpicture}[>=latex, scale=1]
\draw[->](-1,0)--(4,0)node[right]{$X$};
\draw[->](0,-1)--(0,3)node[right]{$Y$};
\node at (0,0)[below left]{$O$};
\draw[dashed](0,2)node[left]{$N$}--(3,2)node[right]{$P$}--(3,0)node[below]{$M$};
    \end{tikzpicture}
    \caption{}
    \end{minipage}
    \begin{minipage}[t]{0.48\textwidth}
    \centering
    \begin{tikzpicture}[>=latex, scale=1]

\draw[->](-1,0)--(4,0)node[right]{$X$};
\draw[->](0,-1)--(0,4)node[right]{$Y$};
\draw[<->, very thick](0,1)--node[left]{$\vec{e}_y$}(0,0)--node[below]{$\vec{e}_x$}(1,0);
\draw[dashed](0,1)--(3,1);
\draw[dashed](0,3)--(3,3);
\draw[dashed](1,0)--(1,3);
\draw[dashed](3,0)--(3,3);
\draw[->, very thick](1,1)--node[above]{$\vec{a}$}(3,3);
\draw(1.3,1) arc (0:45:.3)node[right]{$\alpha$};
\draw(1,1.5) arc (90:45:.5)node[above ]{$\beta$};
\node at (0,0)[below left]{$O$};

    \end{tikzpicture}
    \caption{}
    \end{minipage}
    \end{figure}


在建立直角坐标系$OXY$的平面上(图5.2),我们
沿$X$轴与$Y$轴的正方向分别取单位向量$\vec{e}_x$、$\vec{e}_y$, 由共面向量
定理可知,对坐标平面上任一向量
$\va$, 存在唯一的有序实数偶$(a_x,a_y)$使
\begin{equation}
    \va =a_x \eX+a_y \eY
\end{equation}
$(a_x,a_y)$就叫做$\va$在直角坐标系
$OXY$上的坐标,记作
\begin{equation}
    \va=(a_x,a_y)
\end{equation}
实质上(5.2)式是(5.1)式的缩写;其中$a_x$叫做$\va$在$X$轴上
的坐标分量,$a_y$叫做$\va$在$Y$轴上的坐标分量。

\begin{blk}
    {定理} 在坐标平面上,如果$\va=(a_x,a_y)$, 则
    \begin{equation}
    \begin{split}
        a_x&=\eX\cdot \va=|\va|\cos\langle\eX,\va\rangle\\
        a_y&=\eY\cdot \va=|\va|\cos\langle\eY,\va\rangle\\
    \end{split}
    \end{equation}
\end{blk}

\begin{proof}
    已知$\va =a_x \eX+a_y \eY$,则
\[\begin{split}
   \eX\cdot \va&=\eX\cdot (a_x\eX+a_y\eY)=a_x\eX\cdot \eX+a_y\eX\cdot \eY\\
   \eY\cdot \va&=\eY\cdot (a_x\eX+a_y\eY)=a_x\eY\cdot \eX+a_y\eY\cdot \eY\\ 
\end{split}\]
由于$\eX,\eY$是单位向量,且$\eX\bot \eY$, 所以,
\[\eX\cdot \eX=\eY\cdot \eY=1,\qquad \eX\cdot \eY=\eY\cdot \eX=0\]
于是得到
\[    \begin{split}
    a_x&=\eX\cdot \va=|\va|\cos\langle\eX,\va\rangle\\
    a_y&=\eY\cdot \va=|\va|\cos\langle\eY,\va\rangle\\
\end{split}\]
\end{proof}

这个定理说的是,\textbf{向量$\va$在$X$轴和$Y$轴上的坐标分量分
别是$\va$在坐标轴上的垂直投影量}。
















































































































































































































\subsection*{习题5.3}
\begin{enumerate}

\item 求通过点$(0,0)$、$(a,0)$、$(0,a)$的圆的方程。
\item 求下列每个圆的圆心和半径。
\begin{enumerate}
    \item $(x-a)(x-b)+(y-c)(y-d)=0$
    \item $(x+y+a)^2+(x-y-a)^2=8a^2$
\end{enumerate}

\item 已知$A(2,-1)$, $B(2,3)$, $C(4,-1)$, 求
$\triangle ABC$的外接圆的圆心和半径。
\item 证明通过$A(a,c)$, $B(b,c)$和$C(b,d)$的圆的方
程是$(x-a)(x-b)+(y-c)(y-d)=0$.
\item $P$是圆心在$A(a,b)$且通过原点的圆上的一动点,求
证$\triangle OAP$重心的轨迹方程是
\[3(x^2+y^2)-4ax-4by+a^2+b^2=0\]
\item 证明$A(6,3)$、$B(5,4)$、$C(1,-2)$和$D(6,
-1)$四点共圆。
\item 一动点到原点的距离的平方是它到定直线$x=1$距离的
4倍,求证这动点的轨迹是点圆$(2,0)$或圆$(x+2)^2
+y^2=8$.
\item 已知三角形由直线$x=2$, $y=4$和$4x+3y=32$围成,
求这三角形内切圆的方程。
\item 已知$A(x_1,y_1)$, $B(x_2,y_2)$, $C(x_3,y_3)$, 一动点
$P(x,y)$使$\overline{AP}^2+\overline{BP}^2+\overline{CP}^2=$常数,证明$P$点的
轨迹是一个圆且它的圆心是$\triangle ABC$的重心。
\item 已知$A(2a,0)$, $C(0,2a)$, $\overline{AC}$是正方形$OABC$的
对角线,一动点$P(x,y)$到这正方形四边距离的平方和
等于$12a^2$, 证明$P$点的轨迹是圆心在$(a,a)$, 半径是$2a$
的圆。
\item 已知$A(3,7)$, $B(-1,5)$, 动点$P(x,y)$使
$\overline{AP}^2+\overline{BP}^2=82$, 证明$P$点的轨迹是半径等于6的圆。
\item 已知$A(3,7)$, $B(1,-1)$, 动点$P$使$\overline{AP}=3\overline{BP}$. 
证明$P$点的轨迹是一个圆且这圆在Y轴上截出的弦长等
于6.
\item 已知$OC$是圆$x^2+y^2-2ax=0$的一条弦,直线$OC$的
斜率是$m$, 求证以$OC$作直径圆的方程是
\[(1+m^2)(x^2+y^2)-2a(x+my)=0\]
\item 如果$a,b$是常数,$\theta$是一动角,求证两直线$x\cos\theta+
y\sin\theta=a$与$x\cos\theta-y\sin\theta=b$的交点的轨迹是一个圆
圆心在原点.半径等于$\sqrt{a^2+b^2}$.
\item 已知一圆与$Y$轴相切于$A(0,-3)$且半径$r=2$, 求
此圆的方程。
\item 已知一圆与两坐标轴相切且通过$A(2,9)$, 求它的方
程。
\item 已知一圆与$X$轴相切于$(5,0)$且在$Y$轴上截出的弦长是10, 求此圆的方程。
\item 求圆:$x^2+y^2=25$与平行线系$x+5y+\lambda=0,\quad (\lambda\in\mathbb{R})$
相交所截弦中点的轨迹。
\item 求二圆$x^2-12x+y^2-10y+52=0$, $x^2+18x+y^2+
20y+60=0$的圆心距及原点到连心线的距离。
\item 在直线系$y-7+\lambda(x+1)=0$中,求与圆$x^2+y^2=2$
相切之直线。
\item 求通过点$(5,-2)$且与已知直线$3x-y-1=0$相切于
点$(1,2)$的圆的方程。
\end{enumerate}

\section*{复习题五}
\begin{enumerate}
    \item 一个正六边形边长是$a$, 中心在坐标原点,两个顶点在
    $X$轴上,求各顶点的坐标。
    \item 以原点为起点的三个力$\vec{F}_1=(9,7)$, $\vec{F}_2=(-6,4)$, 
    $\vec{F}_3=(1,2)$; 求它们的合力坐标和方向。
    \item 已知一个三角形三边中点的坐标分别是$(x_1,y_1)$, $(x_2,
    y_2)$, $(x_3,y_3)$, 求三个顶点的坐标。
    \item 已知$A(-1,3)$, $B(4,1)$, 直线$AB$与$X$轴,$Y$轴
    分别相交于$C$、$D$两点,求$C$、$D$两点分割$AB$的比
    值。
    \item 已知$P(x,y)$, $A(x_1,y_1)$, $B(x_2,y_2)$且
    \[x=x_1+t(x_2-xy),\qquad y=y_1+t(y_2-y_1)\]
    求证:$P$点按定比$\mu=\frac{t}{1-t}$
    分割$\Vec{AB}$.
    \item 已知直线$\ell:\; ax+by+c=0$及$P_1(x_1,y_1)$, $P_2(x_2,  y_2)$两点,
    求证直线$\ell$与直线$P_1P_2$的交点把$\Vec{P_1P_2}$按定比
$-\frac{ax_1+by_1+c}{ax_2+by_2+c}$分割。
\item 用坐标法证明:直角三角形斜边的中点到三顶点的距离
相等。
\item 用坐标法证明勾股定理的逆定理。
\item 已知:四边形一组对边的平方和等于另一组对边的平方
和,用坐标法证明:两条对角线互相垂直。
\item 已知$G$是$\triangle P_1P_2P_3$的重心,用坐标法证明:
\[S_{GP_2P_3}=\frac{1}{3}S_{P_1P_2P_3}\]
\item 已知$A(1,2)$、$B(8,4)$、$C(4,10)$, 求一点
使$\triangle PAB$、$\triangle PBC$、$\triangle PCA$的面积相等,并解释
这个结果的几何意义。
\item 一条直线经过点$P(1,-1)$, 它的倾角等于直线
$y=x$倾角的3倍,求这条直线的方程。
\item 已知$A(-3,2)$, $B(-2,-2)$, $C(4,0)$.求:
\begin{enumerate}
    \item $\triangle ABC$ $\overline{BC}$边上的中线方程;
    \item $\overline{AC}$边上的高线方程,并求出这条高的长。
\end{enumerate}

\item 一条直线经过$(2,4)$并且和直线$x+y-4=0$的夹角
是$\pi/4$,
求这条直线的方程。
\item 一条光线从$P_0(6,4)$射出和$X$轴正向交成锐角$\alpha$, 
遇到$X$轴反射,已知$\tan\alpha=2$, 求入射光线和反射光线
所在的直线方程。
\item 从原点向直线$3x-2y+7=0$作垂线,求垂线段的长和
垂足的坐标。
\item 在直线$2x-3y=0$上求一点,使这点和原点之间的距
离等于这点到直线$2x+3y-2=0$之间的距离。
\item 已知点$P(3,2)$, 直线$\ell:\; y=4x+3$, 求$P$点到直
线$\ell$的距离,垂线足的坐标,点$P$关于$\ell$的轴对称点的
坐标。
\item 已知直线$ax+by+c=0$和点$P_0(x_0,y_0)$, 求$P_0$点关
于直线$ax+by+c=0$的轴对称点的坐标。
\item 已知$\ell_1:\; 3x-4y-17=0$, $\ell_2:\; y=4$, $\ell_3:\; 12x+5y-12=0$, 求证点$(-4,-1)$是$\ell_1,\ell_2,\ell_3$两两相交
所构成三角形的内心。
\item 求直线$3x-4y+6=0$与$12x-5y-9=0$交角平分
线的方程。
\item 证明通过点$(a,b)$的直线方程可写为
\[\lambda_1(x-a)+\lambda_2(y-b)=0\]
\item 设$P_1(x_1,y_1)$及$P_2(x_2,y_2)$为两定点,过$P_1$作直线
交$Y$轴于$B$点,过$P_2$作直线与过$P_1$之直线垂直,交$X$
轴于点$A$, 求$AB$中点的轨迹。
\item 求下列各圆的方程。
\begin{enumerate}
 \item 过$O(0,0)$, $A(-5,0)$, $B(0,3)$;
\item 中心在$C(-3,4)$与$3x+8y-6=0$相切;
\item 过$A(4,3)$, $(-2,5)$, 圆心在$2x-3y=0$上;
\item 过$A(5,-2)$与直线$3x-y-1=0$相切于点$(1,
2)$;
\item 通过$O(0,0)$, 圆心在$x=2$上且与直线$x+
y-8=0$相切。
\end{enumerate}

\item 求两圆$x^2+y^2-x+2y=0$, $x^2+y^2+2x-y=9$的
交点的坐标。
\item 求两圆$x^2+y^2+ax+by=0$, $x^2+y^2+bx-ay=0$
的交点的坐标。
\item 求两圆$x^2+y^2=10$, $x^2+y^2-10x-10y+30=0$公
共弦所在直线的方程。
\item 已知圆$x^2+y^2-4x-5=0$和点$A(5,4)$, 求圆心
在$A$点且与已知圆外切的圆的方程。
\item 已知二圆$x^2+y^2-6x+8y=0$, $x^2+y^2+2x-12y
+1=0$, 求通过二圆圆心的直线方程。
\item 求两圆$(x-a_1)^2+(y-b_1)^2=r^2_1$, $(x-a_2)^2+(y-
b_2)^2=r^2_2$正交(即在两圆公共点处的切线互相垂直)
的条件是
\[(a_1-a_2)^2+(b_1-b_2)^2=r_1^2+r_2^2\]
\item 一条$AB=2a$的两个端点$A$和$B$分别在$X$轴和$Y$轴上滑
动,求$\overline{AB}$中点$M$的轨迹。
\item 已知$A(2,2)$, $\triangle OAC$是等边三角形,且$O$、$A$、
$C$构成反时针排列,求点$C$的坐标。
\item 已知$A(x_1,y_1)$, $B(x_2,y_2)$, $C(x_3,y_3)$, 求证
$\triangle ABC$的重心到三顶点的距离平方和为最小。
\item 已知$A(-5,4)$, 过点$A$作条直线使它与两坐标轴
相交所成的三角形的面积等于5个平方单位,求证这条
直线的方程是$8x+5y+20=0$或$2x+5y-10=0$.
\item 已知点$A(a,b)$在第I象限,过点$A$求一条直线使与坐
标轴交成的三角形面积最小,并求出最小面积的值。

\item 如果
\begin{multicols}{2}
    \begin{enumerate}
        \item $D=0$
        \item $E=0$
        \item $F=0$
        \item $D=0$, $E=0$
        \item $D=0$, $F=0$
        \item $E=0$, $F=0$
    \end{enumerate}
\end{multicols}
那么圆$x^2+y^2+2Dx+2Ey+F=0$对
坐标系的位置有什么特征。
\item 已知$P_0(x_0,y_0)$是圆:$x^2+y^2+2Dx+2Ey+F=0$
外任意一点,若自$P_0$向圆引切线$P_0T$, $T$为切点,求
证:$\overline{P_0T}^2=x_0^2+y_0^2+2Dx_0+2Ey_0+F$.
\item 为了使圆$x^2+y^2+2Dx+2Ey+F=0$
\begin{enumerate}
\item 不与$X$轴相交;    
\item 和$X$轴交于两点;    
\item 和$X$相切,
\end{enumerate}
问它的方程中的系数分别应该满足怎样的条件?

\item 求圆心在点$(4,0)$并与直线$3x-4y+1=0$相切的
圆的方程。
\item 已知$\odot C$的圆心$C$在直线$x-y-4=0$上,并经过两圆
$C_1:\; x^2+y^2-4x-3=0$和$C_2:\; x^2+y^2-4y-3
=0$的交点,求$\odot C$的方程。
\item 一动点到已知正方形的各顶点的距离平方和是一个常
数,求这动点的轨迹方程,并说明轨迹是什么图形。
\item 已知点$Q(4,0)$, 点$P(x,y)$是圆:$x^2+y^2=4$上一
动点,求$PQ$中点的轨迹方程。
\item 当$\lambda$为何值时,直线$\lambda x-y-\lambda-1=0$与圆$x^2+y^2-
4x-2y+1=0$相交,相切或相离。
\end{enumerate}

  \chapter{圆锥曲线}
远在古希腊时代,人们就开始研究一个平面和一个正瞬
锥的截线的性质,并且获得了丰硕的成果,这些截线分别有
椭圆、抛物线和双曲线,并统称为圆锥曲线,在第二章
附录里,我们已用球面切线长相等原理证明了它们分别具有
如下几何特征。

椭圆有两个焦点$F_1,F_2$, 对椭圆上任一点$P$有
\[\overline{PF_1}+\overline{PF_2}=\text{常数}\]

双曲线有两个焦点$F_1,F_2$, 对双曲线上任一点$P$有
\[\overline{PF_1}-\overline{PF_2}=\text{常数}\]

抛物线有一个焦点,对抛物线上任一点$P$到焦点$F$与到
一定直线$\ell$的距离$d$相等,即
\[\overline{PF}:d=1\]

这一章,我们将要根据上述圆锥曲线的几何特性来定义
椭圆、双曲线和抛物线,建立它们在平面直角坐标系中的标
准方程,并利用标准方程进一步研究圆锥曲线其它的几何特
性。

\section{圆锥曲线的标准方程及其性质}
\subsection{椭圆的标准方程和形状}
\begin{blk}
    {定义} 平面内与两定点的距离之和等于常数(这常数必
须大于两定点间的距离)的点的轨迹叫做椭圆。这两定点叫
做椭圆的\textbf{焦点}。两焦点间的距离叫做\textbf{焦距}。
\end{blk}

下面,我们根据椭圆的定义来建立椭圆的方程。

设$F_1$、$F_2$是椭圆的两个焦点,取射线$F_1F_2$作为$X$轴
的正半轴,$\overline{F_1F_2}$的垂直平分线作为$Y$轴(图6.1)。设
焦距$\overline{F_1F_2}=2c\; (c>0)$,则
\[F_1(-c,0),\qquad F_2(c,0)\]

\begin{figure}[htp]
    \centering
    \begin{tikzpicture}[>=latex]
\draw[->](-2.5,0)--(2.5,0)node[right]{$X$};
\draw[->](0,-2)--(0,2)node[right]{$Y$};
\draw[thick](0,0) ellipse [x radius=2, y radius=1.25];        
\tkzDefPoints{-1.56/0/F_1, 1.56/0/F_2, 1/1.08/P}
\tkzDrawSegments[thick](F_1,P P,F_2)
\tkzLabelPoints[below](F_1,F_2)
\tkzLabelPoints[above](P)
\node at (0,0)[below left]{$O$};
    \end{tikzpicture}
    \caption{}
\end{figure}


设$P(x,y)$是椭圆上的任一点,
它到$F_1$、$F_2$的距离之和等于常数
$2a\; (a>0)$, 则
\[\overline{PF_1}+\overline{PF_2}=2a\]
由求两点的距离公式得
\[\sqrt{(x+c)^2+y^2}+\sqrt{(x-c)^2+y^2}=2a\]
去根号,整理得
\begin{equation}
    (a^2-c^2)x^2+a^2y^2=a^2(a^2-c^2)
\end{equation}
因$\overline{PF_1}+\overline{PF_2}>\overline{F_1F_2}$, 所以$a>c,\; a^2-c^2>0$, 
设$a^2-c^2=b^2\; (b>0)$, 代入(6.1)式得
\[b^2x^2+a^2y^2=a^2b^2\]
两边同除$a^2b^2$得
\begin{equation}
 \boxed{\frac{x^2}{a^2}+\frac{y^2}{b^2}=1}   
\end{equation}
这就是说,椭圆上任一点的坐标都满足方程(6.2); 反过来,
设$P(x_1,y_1)$的坐标满足方程(6.2), 则
\[\frac{x_1^2}{a^2}+\frac{y_1^2}{b^2}=1\]
\[y^2_1=b^2\left(1-\frac{x^2_1}{a^2}\right)=(a^2-c^2)\left(1-\frac{x^2_1}{a^2}\right)\]
于是
\[\begin{split}
    \overline{PF_1}&=\sqrt{(x_1+c)^2+y^2_1}\\
    &=\sqrt{(x_1+c)^2+(a^2-c^2)\left(1-\frac{x^2_1}{a^2}\right)}\\
    &=\sqrt{a^2+2cx_1+\frac{c^2}{a^2}x^2_1}\\
    &=\left|a+\frac{c}{a}x_1\right|
\end{split}\]
由$\frac{x_1^2}{a^2}+\frac{y_1^2}{b^2}=1$,可推知
$\frac{x_1^2}{a^2}\le 1$,$|x_1|\le a$。

又因$c<a$,所以$\frac{c}{a}<1$,$\left|\frac{c}{a}x_1\right|<a$
,$a+\frac{c}{a}x_1>0$,因此:
\begin{equation}
    \overline{PF_1}=a+\frac{c}{a}x_1
\end{equation}
同理可证,
\[\overline{PF_2}=a-\frac{c}{a}x_1\]
所以:$\overline{PF_1}+\overline{PF_2}=2a$

这就是说,坐标满足方程(6.2)的点$P$也一定在椭圆上,由以
上证明,所以方程(6.2)是所求的椭圆方程,并把方程(6.2)
叫做\textbf{椭圆的标准方程}。

下面,用椭圆的标准方程来研究椭圆的几何形状。

首先,由于方程(6.2)中只含有$x,y$的平方,故把一
个坐标变号,对于方程没有影响,这就表明:如果$M(x,y)$
在椭圆上,那么,$M_1(x,-y)$, $M_2(-x,-y)$, $M_3(-x,
y)$各点也都在椭圆上,所以椭圆既是以$X$轴或$Y$轴为对称轴的
轴对称图形,又是以坐标原点为对称中心的中心对称图形,对
称中心又叫做椭圆的中心。

其次,由方程(6.2)得
\[\frac{x^2}{a^2}\le 1,\qquad \frac{y^2}{b^2}\le 1\]
\[-a\le x\le a,\qquad -b\le y\le b\]
这两个不等式表明椭圆全部包含在如图6.2所示的长方形
内。

最后,我们来讨论椭圆在第
I象限内的性态。由(6.2)得
\[y=\pm\frac{b}{a}\sqrt{a^2-x^2}\]
在第I象限,椭圆方程可写为
\[y=\frac{b}{a}\sqrt{a^2-x^2},\qquad 0\le x\le a\]
当$x=0$时,$y=b$, 当$x$递增时,
$y$递减,当$x=a$时,$y=0$, 
因此椭圆在第I象限内的轨迹大致是$B_2A_2$这部分曲线,由
对称性可画出整个椭圆的图象(图6.2)。

\begin{figure}[htp]
    \centering
    \begin{tikzpicture}[>=latex]
\draw[->](-2.5,0)--(2.75,0)node[right]{$X$};
\draw[->](0,-2)--(0,2.5)node[right]{$Y$};
\draw[thick](0,0) ellipse [x radius=2, y radius=1.25];        
\node at (0,0)[below left]{$O$};
\draw(-2,-1.25) rectangle (2,1.25);
\node at (-2,0)[below left]{$A_1$};
\node at (2,0)[below right]{$A_2$};
\node at (0,1.25)[above right]{$B_2$};
\node at (0,-1.25)[below right]{$B_1$};
\tkzDrawPoint(1.5,0)\tkzDrawPoint(-1.5,0)
    \end{tikzpicture}
    \caption{}
\end{figure}

当$y=0$, $x=\pm a$, 点$A_1(-a,0)$, $A_2(+a,0)$
是$X$轴上距$Y$轴最远的两个点,当$x=0$, $y=\pm b$, 点
$B_1(0,-b)$, $B_2(0,+b)$是$Y$轴上距$X$轴距离最远的
两个点,这四点,$A_1$、$A_2$、$B_1$、$B_2$叫做\textbf{椭圆的顶点}。
$\overline{A_1A_2},\overline{B_1B_2}$分别叫做椭圆的长和轴短轴。$\overline{A_1A_2}=2a$, 
$\overline{B_1B_2}=2b$, $a$和$b$分别叫做椭圆的\textbf{长半轴长和短半轴长}。
长轴和短轴的交点叫做\textbf{椭圆的中心}。

如果$a=b$, 那么方程(6.2)化为
\[x^2+y^2=a^2\]
这时椭圆成为圆,$c=\sqrt{a^2-b^2}=0$, 即椭圆的两个焦点重
合于圆心,因此可以说\textbf{圆是椭圆的特殊情形}。

由以上讨论可以看出,椭圆的形状依赖于$a$和$b$, 数量
$c=\sqrt{a^2-b^2}$可表示出椭圆离开圆的偏差。由$c^2=a^2-b^2$
可得
\[\frac{c}{a}=\sqrt{1-\left(\frac{b}{a}\right)^2},\qquad \frac{b}{a}=\sqrt{1-\left(\frac{c}{a}\right)^2}\]
比值
\[e=\frac{c}{a}=\frac{\sqrt{a^2-b^2}}{a}\]
叫做\textbf{椭圆的离心率},用它可同样来表出椭圆的形状。由$c<
a$, 可知$e<1$, 当离心率愈来愈大时,也就是愈来愈接近
1时,$1-e^2$就越小,椭圆的形状就愈扁平;反之,就愈接
近于圆,当$e=0$时,$a=b$椭圆就成为圆了。

如果椭圆的中心在原点,焦点在$Y$轴上,那么长轴也-
定在$Y$轴上,这时两个焦点$F_1,F_2$的坐标分别是$(0,-c)$,
$(0,c)$ (图6.3), 求得圆的标准方程是
\begin{equation}
    \boxed{\frac{x^2}{b^2}+\frac{y^2}{a^2}=1}\qquad a\ge b>0
\end{equation}
把方程(6.2)的变量$x$和$y$互换就可得到方程(6.6).

\begin{figure}[htp]\centering
    \centering
\begin{tikzpicture}[>=latex, scale=1]
    \draw[->](-1.5,0)--(1.75,0)node[right]{$X$};
    \draw[->](0,-2)--(0,2.5)node[right]{$Y$};
    \draw[thick](0,0) ellipse [x radius=.7, y radius=1.5];  
    \tkzDefPoints{0/-1.2/F_1, 0/1.2/F_2}
\tkzLabelPoints[right](F_1,F_2)
\tkzDrawPoints(F_1,F_2)
\node at (0,0)[below left]{$O$}; 
\node at (0,1.5)[above right]{$B(0,a)$}; 
\node at (0.7,0)[below right]{$A(b,0)$}; 
    \end{tikzpicture}
    \caption{}
    \end{figure}


\begin{example}
    已知椭圆的长轴长是10, 焦距是8, 求椭圆的标准方程。
\end{example}

\begin{solution}
    由已知条件得$2a=10,\quad 2c=8$,所以:
\[a=5,\qquad c=4,\qquad b^2=a^2-c^2=5^2-4^2=9\]
因此所求椭圆的标准方程为
\[\frac{x^2}{25}+\frac{y^2}{9}=1\]
\end{solution}

\begin{example}
    求椭圆$4x^2+9y^2=36$的长轴、短轴长、离心率、
焦点和顶点的坐标,并用描点法画出它的图形。
\end{example}

\begin{solution}
    已知方程可化为
    \[\frac{x^2}{9}+\frac{y^2}{4}=1\]
这是长轴在$X$轴上,中心在坐标原点的椭圆标准方程。
因此$a=3$, $b=2$, $c=\sqrt{3^2-2^2}=\sqrt{5}$, 顶点$A'(-3,
0)$, $A(3,0)$, $B'(0,-2)$, $B(0,2)$. 焦点
$F_1(-\sqrt{5},0)$, $F_2(\sqrt{5},0)$. 离心率$e=\frac{c}{a}=\frac{\sqrt{5}}{3}$。

在第I象限已知椭圆方程可写为
\[y=\frac{2}{3}\sqrt{9-x^2},\qquad 0\le x\le 3\]
算出一些满足所求椭圆方程的点的坐标$(x,y)$:
\begin{center}
    \begin{tabular}{cccccccc}
\hline
$x$&0&0.5&1&1.5&2&2.5&3\\
\hline
$y$&2&1.97&1.89&1.73&1.49&1.11&0\\
\hline
    \end{tabular}
\end{center}
描点画出椭圆在第I象限的图
象,然后根据椭圆的对称性就可
画出整个椭圆的图象(图6.4)。
\end{solution}

\begin{figure}[htp]\centering
    \begin{minipage}[t]{0.48\textwidth}
    \centering
\begin{tikzpicture}[>=latex, scale=.7]
    \draw[->](-4,0)--(4,0)node[right]{$X$};
    \draw[->](0,-3)--(0,3)node[right]{$Y$};
    \draw[thick](0,0)node[below left]{$O$} ellipse [x radius=3, y radius=2];  
    \node at (-3,0)[below left]{$A'$};
    \node at (3,0)[below right]{$A$};
    \node at (0,2)[above right]{$B$};
    \node at (0,-2)[below right]{$B'$};
\tkzDefPoints{0/2/A, .5/1.97/B, 1/1.89/C, 1.5/1.73/D, 2/1.49/E, 2.5/1.11/F, 3/0/G}
\tkzDrawPoints(A,B,C,D,E,F,G)

\tkzDefPoints{0/-2/A, .5/-1.97/B, 1/-1.89/C, 1.5/-1.73/D, 2/-1.49/E, 2.5/-1.11/F, 3/0/G}
\tkzDrawPoints(A,B,C,D,E,F,G)

\tkzDefPoints{0/2/A, -.5/1.97/B, -1/1.89/C, -1.5/1.73/D, -2/1.49/E, -2.5/1.11/F, -3/0/G}
\tkzDrawPoints(A,B,C,D,E,F,G)

\tkzDefPoints{0/-2/A, -.5/-1.97/B, -1/-1.89/C, -1.5/-1.73/D, -2/-1.49/E, -2.5/-1.11/F, -3/0/G}
\tkzDrawPoints(A,B,C,D,E,F,G)

\foreach \x in {-2.5,-2,...,2.5}
{
    \draw(\x,0)--(\x,.1);
}
\foreach \x in {-1.5,-1,...,1.5}
{
    \draw(0,\x)--(.1,\x);
}
    \end{tikzpicture}
    \caption{}
    \end{minipage}
    \begin{minipage}[t]{0.48\textwidth}
    \centering
    \begin{tikzpicture}[>=latex, scale=.7]
        \draw[->](-4,0)--(4,0)node[right]{$X$};
        \draw[->](0,-3)--(0,3)node[right]{$Y$};
    \draw(.5,0)node[above]{$F_2$} circle(1.5);
    \draw(-.5,0)node[above]{$F_1$} ellipse[x radius= 3,  y radius= 2.7];
    \tkzDrawPoint(.5,0)\tkzDrawPoint(-.5,0)
    \node at (0,0) [below left]{$O$};
    \node at (2.5,0) [below right]{$A_2$};
    \node at (-3.5,0) [below left]{$A_1$};
    \end{tikzpicture}
    \caption{}
    \end{minipage}
    \end{figure}


\begin{example}
    我国第一颗人造地球
卫星的运行轨道是以地球中心为
一焦点的椭圆,卫星的近地点与
地球表面距离为439公里;远地点
与地球表面距离为2384公里,已知地球半径约为6371公里,
试求卫星轨道的近似方程及其离心率。
\end{example}


\begin{solution}
    设地球中心$F_2$在$X$轴上(图6.5),所求方程为
\[\frac{x^2}{a^2}+\frac{y^2}{b^2}=1\]
依题意
\[\begin{split}
    \overline{A_1F_2}&=a+c=6371+2284=8755\\
    \overline{A_2F_2}&=a-c=6371+439=6810
\end{split}\]
由以上两式联立求解得
\[a=7782.5,\qquad c=972.5,\qquad b=\sqrt{a^2-c^2}=7721.5\]
所以,所求卫星轨道的近似方程为
\[\frac{x^2}{(7782.5)^2}+\frac{y^2}{(7721.5)^2}=1\]
其离心率
$e=\frac{c}{a}\approx 0.125$
\end{solution}

\begin{ex}
\begin{enumerate}
    \item 已知椭圆的长轴长是6, 短轴长是2, 焦点在$X$轴上,
    求这椭圆的标准方程并画出这椭圆的草图。
    \item 在第1题中,若焦点在$Y$轴,椭圆的标准方程为何?
    \item 已知椭圆的一个焦点是$F_1(-3,0)$与$X$轴一个交点
    $A(4,0)$, 求此椭圆的方程。
    \item 求以下椭圆的长轴长,短轴长、焦点的坐标及其离心率。
\begin{multicols}{2}
\begin{enumerate}
    \item $\frac{x^2}{100}+\frac{y^2}{36}=1$
    \item $25x^2+9y^2=100$
    \item $\frac{x^2}{16}+\frac{y^2}{64}=1$
    \item $49x^2+9y^2=2500$
\end{enumerate}
\end{multicols}

\item 已知椭圆中心在原点,一焦点是$F(3,0)$, 椭圆与$X$
轴相交于$A$、$A'$两点,$\overline{AF}=2$, $\overline{A'F}=8$, 求此椭圆的方程。
\item 已知地球的轨道是一个椭圆,太阳在它的一个焦点上,
长轴长约30亿公里,离心率$e=1/60$, 
求地球的轨道方程,地球的轨道中心与太阳的距离,以及近日点,远日
点到太阳的距离。
\item 试求平分圆$x^2+y^2=25$上各点的纵坐标,而横坐标不
变的点的轨迹方程。
\item 试求把圆$x^2+y^2=100$上各纵坐标分为2:3, 而横
坐标不变的点的轨迹方程。
\item 一动点与直线$x=8$的距离是它与点$(2,0)$的距离的
2倍,求这动点的轨迹方程。
\item 一定长为$a$的线段,两端在互相垂直的二直线上移动,
试求此线段上任意一点的轨迹方程。
\item 设一三角形的一边的两个端点为$(0,6)$, $(0,-6)$,其它两
边斜率的乘积是$-\frac{4}{9}$,
试求另一顶点的轨迹。
\item 已知$A>0$, $B>0$, 且$A<B$, 试求椭圆$Ax^2+By^2
=C$的焦点坐标。
\item 试证:椭圆的短半轴长是其中一焦点到长轴两顶点距离
的比例中项。
\item 在椭圆$\frac{x^2}{45}+\frac{y^2}{20}=1$上求一点,使它与两焦点连线互相垂直。
\end{enumerate}
\end{ex}

\subsection{双曲线的标准方程和形状}
\begin{blk}
    {定义} 平面内到两定点距离的差的绝对值等于常数(常
数小于两定点间的距离)的轨迹叫做双曲线。这两个定点叫
做双曲线的焦点,两个焦点间的距离叫做焦距。
\end{blk}

根据双曲线的定义,我们来求它的方程:









\begin{example}
    
\end{example}

\begin{solution}
    
\end{solution}

\begin{example}
    
\end{example}



\begin{solution}
    
\end{solution}

\begin{example}
    
\end{example}



\begin{solution}
    
\end{solution}

\begin{example}
    
\end{example}



\begin{solution}
    
\end{solution}

\begin{example}
    
\end{example}



\begin{solution}
    
\end{solution}

\begin{example}
    
\end{example}



\begin{solution}
    
\end{solution}

\begin{example}
    
\end{example}



\begin{solution}
    
\end{solution}




















\begin{ex}
\begin{enumerate}
    \item 对双曲线和抛物线情况证明本节定理。
    \item 已知椭圆$3x^2+4y^2=12$, 求倾角为$135^{\circ}$的椭圆平行弦
    中点所在的直线方程。
    \item 已知双曲线$2x^2-y^2=6$, 它的一族平行弦的倾角是
    $30^{\circ}$, 求这族平行弦中点所在的直线方程。
    \item 在练习2、3中写出与弦平行的直径和它的共轭直径的
    方程。
    \item 已知抛物线$y^2=6x$的一族平行弦的斜率是$1/2$, 
    求平分这族平行弦的直径方程。
    \item 设$P_0(x_0,y_0)$是双曲线
    $\frac{x^2}{a^2}-\frac{y^2}{b^2}=1$
    与它的一条直
    径$y=kx$的交点,求证:双曲线在$P_0(x_0,y_0)$的切线
    平行于这条直径的共轭直径。
\end{enumerate}
\end{ex}

\subsection*{习题6.1}
\begin{enumerate}
\item 在椭圆$24x^2+30y^2=720$上,求与短轴相距为5的点
的坐标。
\item 一椭圆以坐标轴为对称轴,坐标原点为对称中心且经过
点$M(\sqrt{3},-2)$, $N(-2\sqrt{3},1)$, 求此椭圆的方
程。
\item 点$P(x_1,y_1)$和点$Q(x_2,y_2)$分别位于椭圆的内部
和外部,求证
\[\frac{x_1^2}{a^2}+\frac{y_1^2}{b^2}<1,\qquad \frac{x_2^2}{a^2}+\frac{y_2^2}{b^2}>1\]
\item 已知一椭圆的准线方程是$x=\pm 8$, 短轴长等于8, 求
此椭圆的方程。
\item 已知椭圆$36x^2+100y^2=3600$, 在它上面求一点使这点
到右焦点的距离是这点到左焦点距离的4倍。
\item 已知椭圆中心在原点,它的一个焦点是$F_2(3,0)$, 求
其上一点$M(4,2.4)$到准线的距离。
\item 求下列各双曲线标准方程。
\begin{enumerate}
\item 两焦点间的距离是8, 两准线间的距离是6;
\item 已知两条准线方程是$x=\pm 3\sqrt{2}$, 两条渐近线的
夹角是直角。
\item 已知渐近线方程是$y=\pm 2x$, 两个焦点距中心的距
离是5.
\item 已知渐近线方程是$y=\pm \frac{5}{3}x$, 且双曲线通过点
$N(6,9)$.
\end{enumerate}

\item 根据下列已知条件,求双曲线$\frac{x^2}{a^2}-\frac{y^2}{b^2}=1$的渐近线
方程。
\begin{enumerate}
    \item 离心率$e=2$; 
    \item 两焦点间的距离是二准线间距离的2倍。
\end{enumerate}

\item 根据下列已知条件,求双曲线$\frac{x^2}{a^2}-\frac{y^2}{b^2}=1$的离心率。
\begin{enumerate}
    \item 两渐近线之间的夹角是$60^{\circ}$;
    \item 两渐近线之间的夹角是$90^{\circ}$.
\end{enumerate}

\item 已知等轴双曲线$x^2-y^2=8$, 求一抛物线方程使它与
已知双曲线有公共焦点且通过点$M(-5,3)$.
\item 通过点$A(2,-5)$引直线平行于双曲线$x^2-4y^2=4$的
渐近线,求此直线的方程。

\item 通过点$A(3,-1)$作双曲线$\frac{x^2}{4}-y^2=1$的弦且被$A$点
平分,求此弦的方程。
\item 求下列抛物线方程,已知
\begin{enumerate}
    \item 顶点在$(0,0)$, 焦点在$(2,0)$;
    \item 顶点在$(0,0)$, 准线是$2x+5=0$;
    \item 顶点在$(0,0)$, 准线是$2y-1=0$;
    \item 顶点在$(0,0)$, 焦点在$(0,-3/5)$.
\end{enumerate}

\item 一条抛物线顶点在原点,它的轴是$X$轴并且它通过点
$M(-1,1)$, 求它的方程。

\item 求椭圆$\frac{x^2}{6}+\frac{y^2}{3}=1$的内接正方形每边所在直线的方程。
\item 求直线$Ax+By+C=0$与椭圆$\frac{x^2}{a^2}+\frac{y^2}{b^2}=1$相切的条
件。

\item 已知椭圆$\frac{x^2}{25}-\frac{y^2}{9}=1$
的两个焦点到它的某条切线的距离之比是9, 求此切线方程。
\item 求双曲线$\frac{x^2}{8}-\frac{y^2}{9}=1$在下列各点的切线,$(2\sqrt{2},0)$, $(-4,3)$.
\item 一条双曲线在点$M(4,2)$与直线$x-y-2=0$相切.
求此双曲线的方程。
\item 求直线:$Ax+By+C=0$与双曲线
$\frac{x^2}{a^2}-\frac{y^2}{b^2}=1$相切
的条件。
\item 已知抛物线$y^2=12x$, 根据下列各条件,求它的切线方
程。
\begin{enumerate}
\item 切点的横坐标$x=3$;
\item 平行于直线$3x-y+5=0$;
\item 垂直于直线$2x+y-7=0$;
\item 与直线$4x-2y+9=0$交成$\pi/4$角。
\end{enumerate}

\item 求直线$y=kx+b$与抛物线$y^2=2px$相切的条件。
\item 直线$x+y=1$与椭圆相交于$C$和$D$两点,求弦$\overline{CD}$的
中点的坐标。
\item 已知椭圆$\frac{x^2}{a^2}+\frac{y^2}{b^2}=1$, 在其上一点$P$的切线和法线分
别与$X$轴相交于$T$点和$G$点,过焦点$F_1$、$F_2$和原点分
别作切线的垂线,设垂足分别为$V$、$U$、$K$, 过$P$点作
$X$轴的垂线,设垂足为$N$, 求证:
\begin{enumerate}
    \item $\overline{ON}\cdot \overline{OT}=a^2$
    \item $\overline{PG}\cdot \overline{OK}=b^2$
    \item $\overline{OG}=e^2\cdot \overline{ON}$
    \item $\overline{F_1V}\cdot  \overline{F_2U}=b^2$
\end{enumerate}
\end{enumerate}


\section{坐标变换}
\subsection{坐标轴的平移}
不改变坐标轴的方向和长度单位,只变换原点的位置,
这种坐标系的变换叫做\textbf{坐标轴的平移},简称\textbf{移轴}。

给定一坐标系$OXY$, 平移
坐标轴得到新坐标系$O'X'Y'$, 
下面我们来确定平面上任意一点
$P$的新坐标$(x',y')$与原坐标
$(x,y)$之间的关系(图6.27)。















































































\begin{ex}
\begin{enumerate}
    \item 试判别下列方程的类型
\begin{enumerate}
\item $16x^2-24xy+9y^2-38x-34y+71=0$
\item $x^2-5xy+13y^2-3x+21y=0$
\item $8x^2+8xy-7y^2+36y+36=0$
\item $4x^2+9y^2-16x-18y-11=0$
\item $2x^2+5xy-3y^2+3x+16y-5=0$
\end{enumerate}

    \item 判别下列方程的类型,并画出它们的图形
\begin{enumerate}
\item $5x^2-6xy+5y^2-4x-4y-4=0$
\item $7x^2-8xy+y^2+14x-8y-2=0$
\item $x^2-2xy+y^2+3x-y-4=0$
\item $3x^2-xy+5y^2-6x+y+3=0$
\item $4x^2+12xy+9y^2+2x+3y+2=0$   
\end{enumerate}

\end{enumerate}    
\end{ex}

\subsection*{习题6.3}
\begin{enumerate}
    \item 化简下列方程,求对称轴方程,并画出方程的图象。
\begin{enumerate}
    \item $11x^2+6xy+3y^2-12x+2y-12=0$
    \item $7x^2-8xy+y^2+14x-8y+16=0$
    \item $8x^2+8xy+2y^2-6x-3y-5=0$
    \item $x^2-2xy-6x+4y+4=0$
\end{enumerate}

\item 证明二元二次方程表示等轴双曲线或两条互相垂直的直
线的充要条件是$A+C=0$.
\item 证明抛物线$y=ax^2+bx+c\; (a\ne 0)$的对称轴平行于原
坐标轴。
\item 方程$2x^2+\lambda xy+4y^2-7x+\lambda^2y+3=0$中,$\lambda$取什么
值时,方程是:椭圆型;双曲线型;抛物
线型。
\item 设一二次曲线过点$(2,3)$, $(4,2)$, $(-1,-3)$, 且以
$(0,1)$为对称中心,求这曲线方程。
\end{enumerate}

\section*{复习题六}
\begin{enumerate}
    \item 已知椭圆的两个焦点分别是$F_1(2,4)$、$F_2(8,4)$并经
    点$A(5,0)$, 求此椭圆方程。
    \item 两条直线$3x\pm 4y=0$都是适合下列各条件的双曲线的渐
    近线,求各双曲线方程。
\begin{enumerate}
\item 焦点在点$(0,10)$;
\item 焦点在点$(5,0)$;
\item 经过点$(7,2)$.
\end{enumerate}

    \item 求适合下列条件的抛物线的方程式。
\begin{enumerate}
    \item 顶点在点$(2,4)$, 焦点在点$(3,4)$;
    \item 经过$(0,1)$, $(2,3)$, $(5,-1)$三点且它的轴
    平行于$Y$轴;
\item     顶点在原点,准线是$x=3$.
\end{enumerate}

\item    已知椭圆$\frac{x^2}{a^2}+\frac{y^2}{b^2}=1$,直线$\overline{OP}$与$\overline{OQ}$互相垂直并与
椭圆分别相交于$P$、$Q$两点,求证:
\[\frac{1}{\overline{OP}^2}+\frac{1}{\overline{OQ}^2}=\frac{1}{a^2}+\frac{1}{b^2}\]
\item    已知$P(x_1,y_1)$和$Q(x_2,y_2)$是椭圆$b^2x^2+a^2y^2=a^2b^2$
上任意两点,又知点$L(e_{x_1},0)$, 点$M(e_{x_2},0)$; 
求证:$\overline{PM}=\overline{QL}$.
\item    已知双曲线$\frac{x^2}{a^2}-\frac{y^2}{b^2}=1$,
求证:通过点$M(h,k)$且被
$M$点平分的弦的方程是
\[\frac{hx}{a^2}-\frac{ky}{b^2}=\frac{h^2}{a^2}-\frac{k^2}{b^2}\]
\item    证明方程
\[\frac{x^2}{9+\lambda}+\frac{y^2}{5+\lambda}=1\]
当$\lambda>-5$时,表示椭圆,当$-9<\lambda<-5$时,表示双
曲线,并证明所有这些椭圆和双曲线具有公共的焦点
$(\pm 2,0)$.
\item    已知方程
\[\frac{x^2}{a^2+\lambda}+\frac{y^2}{b^2+\lambda}=1,\qquad  a>b>0\]
问$\lambda$为何值时,表示椭圆;表示双曲线。并证明
所有这些椭圆和双曲线有公共焦点。

\item 已知双曲线的轴是坐标轴,且通过点$(1,4)$和点$(-2,
7)$, 求这双曲线的方程。
\item 证明由方程$4x^2-5y^2=c$($c$为非零常数)所确定的
双曲线具有公共的渐近线。
\item 设$\alpha$是双曲线
$\frac{x^2}{a^2}-\frac{y^2}{b^2}=1$的两条渐近线的夹角,证明
$\cos\alpha=2e^{-2}-1$.
\item 已知双曲线$\frac{x^2}{a^2}-\frac{y^2}{b^2}=1$, 如果与双曲线在$P$点的切线
与两条渐近线分别相交于$E$、$F$, 求证:
\begin{enumerate}
    \item $P$点是$EF$的中点;
    \item $\overline{OE}\cdot \overline{OF}=a^2+b^2$.
\end{enumerate}

\item 双曲线$x^2-y^2=a^2$在$P$点的法线与坐标轴相交于$C$、
$D$两点,求证:$P$点是通过$O$、$C$、$D$三点圆的中心。
\item 已知双曲线
$\frac{x^2}{a^2}-\frac{y^2}{b^2}=1$在$P$点的法线分别与$X$轴,$Y$轴
相交于$C$、$D$两点,求证$\overline{CD}$中点的轨迹是
\[4(a^2x^2-b^2y^2)=(a^2+b^2)^2\]
\item 求证椭圆只有一个内接正方形和一个外切正方形。
\item 证明通过点$M(a,b)$的椭圆$b^2x^2+a^2y^2=a^2b^2$的弦的
中点的轨迹是
\[\frac{x^2}{a^2}+\frac{y^2}{b^2}=\frac{x}{a}+\frac{y}{b}\]
\item 从椭圆外一点$P(x_1,y_1)$引椭圆的两条切线,求证:
通过两个切点的直线方程为
\[\frac{xx_1}{a^2}+\frac{yy_1}{b^2}=1\]

\item 证明:在过椭圆焦点弦的两个端点处的切线相交在椭圆
的准线上。
\item 求抛物线$y^2=8ax$和$x^2=ay$在公共点切线之间的交
角。
\item 求椭圆$\frac{x^2}{6}+\frac{y^2}{3}=1$
的外切正方形的边长。
\item 已知椭圆的轴平行于坐标轴且与X轴相切于点$(7,0)$, 
与$Y$轴相切于点$(0,4)$. 求这椭圆的方程。
\item 在抛物线$x^2=ay\; (a>0)$上求一点$N$, 使它到$M(0,ka)$
($k>0$且为定值)的距离最小;又当$a$变化时,求$N$点的
轨迹。
\item 求抛物线$4x^2+4x+3y-2=0$的顶点和焦点的坐标及其
对称轴和准线方程。
\item 证明:任何一个以椭圆$\frac{x^2}{a^2}+\frac{y^2}{b^2}=1$
的互为共轭直径的端点为顶点的平行四边形的面积都等
于常数$2ab$. 
\item 证明:外切椭圆的矩形,其对角线之长等于定量,
\item 试证明在抛物线上三点$P_1$、$P_2$、$P_3$各引切线,这三
条切线所围成的三角形面积等于$\triangle P_1P_2P_3$面积的一
半。
\item 判定下列二次曲线的类型,并把它们化为标准方程。
\begin{enumerate}
    \item $8x^2+4xy+5y^2+8x-16y-16=0$
    \item $x^2-4xy-2y^2+10x+4y=0$
    \item $4x^2-4xy+y^2+4x-2y=0$
\end{enumerate}
\end{enumerate}



  
\chapter{极坐标与参数方程}

\section{极坐标系与曲线的极坐标方程}
\subsection{极坐标的概念}

如果知道了一点相对于一定点的距离和方向,那么这个
点的位置就被唯一确定了.这就是说,我们可用角度和距离
来确定平面上点的位置.这节,我们研究如何利用角度和距
离来建立坐标系.

在平面内取一个定点$O$, 叫做\textbf{极点},引射线$OA$, 叫做
\textbf{极轴},再选定一个长度单位和角度的正向(通常取逆时针方
向).对于平面上任一点$P$, 但$P$不是极点,用$r$表示$\overline{OP}$
的长度,$\theta$表示从$OA$转到$OP$的角度.这时$r$叫做$P$点的
\textbf{极径},$\theta$叫做$P$点的\textbf{极角}.有序实数对$(r,\theta)$就叫做$P$点
的极坐标,并记作$P(r,\theta)$. 这样建立的坐标系叫做极坐
标系(图7.1).

在极坐标系中,$r=0$,
不论$\theta$是什么角,$(0,\theta)$
都表示极点,除去极点,显
然,不同的点对应不同的极
坐标;反过来任取一对实数
$(r,\theta)$, 其中$0<r<\infty$, $0\le 0\theta<2\pi$, 我们能够且只能够
在平面上找到一点$P$, 使它的坐标恰好是$(r,\theta)$. 由此可
见,平面上除了极点外的所有点和实数对集合:
\[\{(r,\theta)|0<r<\infty,\quad 0\le\theta<2\pi\}\]
可建立一一对应关系.

有时为了研究问题的需要,我们往往取消上述对$r,\theta$
的限制,规定$r$和$\theta$可取任何实数值.如果已知任意有序实
数对$(r,\theta)$, 那么,我们可按下面的方法,在极坐标系中
作出它的对应点.

\begin{figure}[htp]\centering
    \begin{minipage}[t]{0.48\textwidth}
    \centering
\begin{tikzpicture}[>=latex, scale=.8]
\draw[->, very thick](0,0)node[left]{$O$}--(4,0)node[right]{$A$}; 
\draw[very thick](0,0)--node[above]{$r$}(40:4)node[right]{$P$};
\draw[->](.75,0) arc (0:40:.75)node[right]{$\theta$};
\tkzDrawPoint(40:4)
    \end{tikzpicture}
    \caption{}
    \end{minipage}
    \begin{minipage}[t]{0.48\textwidth}
    \centering
    \begin{tikzpicture}[>=latex, scale=.8]
\draw[->, very thick](0,0)node[below]{$O$}--(4,0)node[right]{$A$}; 
\draw[very thick](0,0)--(40:3)node[right]{$M$};
\draw[very thick](0,0)--(-180+40:2);
\tkzDefPoint(40:1.5){P}
\tkzDefPoint(-180+40:1.5){P'}
\tkzDrawPoints(P,P')
\tkzLabelPoint[right=2pt](P){$P(r,\theta)$}
\tkzLabelPoint[right=2pt](P'){$P'(-r,\theta)$}
    \end{tikzpicture}
    \caption{}
    \end{minipage}
    \end{figure}


以极轴$OA$为始边,作有向角$\angle AOM=\theta$, 如果$r>0$, 
在射线$OM$上作$\overline{OP}=r$, 
如果$r<0$, 在射线$OM$的反
向延长线上作$\overline{OP}=|r|$. 这
样,对任一对有序实数$(r,\theta)$,我们总可以在平面上
确定一点$P$; 反过来,对平
面上任一点$P$, 都可对应无限多有序实数对组成的极坐标,
如果已知$P(r,\theta)$, 那么$(r,\theta+2k\pi)$, 当$k$为任意整数
时,都可表示$P$点的极坐标.

\begin{example}
    在极坐标系中,作出下列各点
\[B\left(4,\frac{\pi}{6}\right),\qquad C(2,0),\qquad D\left(4,\frac{5}{6}\pi\right),\qquad E\left(4,\frac{3}{2}\pi\right)\]
\[F\left(-4,\frac{\pi}{6}\right),\qquad G\left(3,-\frac{\pi}{3}\right),\qquad H\left(1,\frac{\pi}{2}\right)\]
\end{example}

\begin{solution}
    如图7.3所示.
\begin{figure}[htp]
    \centering
\begin{tikzpicture}[>=latex, scale=.7]
\draw[->, thick](0,0)--(5,0)node[right]{$A$};
\foreach \x in {1,2,3,4}
{
    \draw(0,0) circle(\x);
}    
\foreach \x in {30,90,...,330}
{
    \draw(0,0)--(\x:5);
}
\tkzDefPoint(30:4){B}
\tkzDefPoint(0:2){C}
\tkzDefPoint(150:4){D}
\tkzDefPoint(270:4){E}
\tkzDefPoint(30:-4){F}
\tkzDefPoint(-60:3){G}
\tkzDefPoint(90:1){H}
\tkzDefPoints{0/0/O}
\tkzDrawPoints(B,C,D,E,F,G,H)
\tkzAutoLabelPoints[center=O](G)
\node at (O)[below left]{$O$};
\tkzLabelPoints[above right](C,H,E)
\tkzLabelPoints[above](D,B)
\tkzLabelPoints[left](F)
\end{tikzpicture}   
    \caption{}
\end{figure}
\end{solution}

\begin{ex}
\begin{enumerate}
    \item 在极坐标系中,描出下各点.
  \[  L\left(3,\frac{\pi}{3}\right),\qquad  M(3,0),\qquad N\left(1,\frac{\pi}{2}\right),\qquad P\left(-3,\frac{\pi}{4}\right)\]
\[    Q\left(3, -\frac{\pi}{4}\right), \qquad R\left(-2,-\frac{2}{3}\pi\right),\qquad S\left(1,\frac{3}{4}\pi \right)\]
\item 在极坐标系中,描出下列各点和它们关于原点和极轴的
    对称点.
\[P_1\left(2,\frac{\pi}{3}\right),\qquad P_2\left(-3,\frac{\pi}{2}\right),\qquad P_3\left(\frac{5}{4},\frac{\pi}{4}\right)\]
    \item 已知一等边三角形边长为$a$, 它的中心与极点重合,一
    个顶点在极轴上,求三个顶点的极坐标.
    \item 已知一正方形边长是$2a$, 它的中心在极点,它的一边与
    极轴垂直.求它的四个顶点的极坐标.
\end{enumerate}
\end{ex}

\subsection{极坐标和直角坐标的关系}

\begin{figure}[htp]
    \centering
\begin{tikzpicture}[>=latex]
    \draw[-> ](-.5,0)--(4,0)node[right]{$X$};
    \draw[-> ](0,-.5)--(0,2.5)node[right]{$Y$};
\draw(0,0)node[below left]{$O$}--node[left]{$r$}(30:3)node[above]{$P$}--node[right]{$y$}+(0,-1.5);
\node at (1.5,0)[below]{$x$};
\draw(.6,0) arc(0:30:.6)node[right]{$\theta$};
\end{tikzpicture}
    \caption{}
\end{figure}

在平面上建立一直角坐标系$OXY$和一极坐标系,使极
点和坐标原点$O$重合,极轴$OA$与$X$轴的正半轴重合.设平
面上任一点$P$在$OXY$中的
坐标为$(x,y)$, 在极坐
标系中的坐标为$(r,\theta)$. 
若$P$点的极坐标为已知,且$r>0$, 则由三角学可知,
$P$点的直角坐标可由变换公式
\begin{equation}
    x=r\cos\theta,\qquad y=r\sin\theta
\end{equation}
求得.若$r=0$, 公式(7.1)显然成立,若$r<0$, 则因
$(r,\theta)$和$(-r,\theta+\pi)$表示同一点,故可用$(-r,\theta+\pi)$
代替$(r,\theta)$来求$(x,y)$, 于是
\[\begin{split}
    x&=-r\cos(\theta +\pi )=-r(-\cos\theta )=r\cos\theta \\
    y&=-r\sin(\theta +\pi )=(-r)(-\sin\theta )=r\sin\theta 
\end{split}\]
因此,当$r<0$时,点的直角坐标仍可由公式(7.1)求得.

反过来,如果$P$点的直角坐标为已知,我们可由公式
\begin{equation}
    r^2=x^2+y^2,\qquad \tan\theta=\frac{y}{x}\quad (x\ne 0)
\end{equation}
求得该点的极坐标,由上一小节可知,点$P$的极坐标可对应
无穷多对数值,其中任一对数值都可作为点$P$的极坐标.在
一般情况下,我们只求$r\ge 0$, $0\le 0<2\pi$ 的一对数值就可
以了.

\begin{example}
    把点$P$的极坐标$\left(3,\frac{\pi}{3}\right)$化为直角坐标.
\end{example}

\begin{solution}
    由于
\[\begin{split}
    x&=3\cdot \cos \frac{\pi}{3}=3\cdot \frac{1}{2}=\frac{3}{2}\\
    y&=3\cdot \sin \frac{\pi}{3}=3\cdot \frac{\sqrt{3}}{2}=\frac{3\sqrt{3}}{2}\\
\end{split}\]
因此:点$P$的直角坐标是$\left(\frac{3}{2},\frac{3\sqrt{3}}{2}\right)$
\end{solution}

\begin{example}
    把点$M(-1,-1)$化为极坐标.
\end{example}

\begin{solution}
\[\begin{split}
    r&=\sqrt{(-1)^2+(-1)^2}=\sqrt{2}\\
    \tan\theta&=\frac{-1}{-1}=1
\end{split}\]
由于点$M$在第三象限.因此,取$\theta=\frac{5\pi}{4}$

$\therefore\quad $点$M$的极坐标为$\left(\sqrt{2},\frac{5}{4}\pi\right)$
\end{solution}

\begin{ex}
\begin{enumerate}
    \item 试求下列各点的直角坐标.
\[M\left(-4,-\frac{\pi}{4}\right),\qquad N\left(-4,-\frac{5}{4}\pi\right),\qquad P(-3,8\pi)\]
\[Q(7,0^{\circ}),\qquad R\left(5,-\frac{\pi}{2}\right),\qquad S\left(-2,-\frac{2}{3}\pi\right)\]
    \item 试求下列各点的极坐标.
\[B(1,-1),\qquad C\left(3,\sqrt{3}\right),\qquad D(-1,1)\]
\end{enumerate}
\end{ex}

\subsection{点的轨迹的极坐标方程}
我们已知,可用一对有序实数$(r,\theta)$来确定平面上
一点的位置,因此,平面上点的轨迹有时可用含有$r,\theta$两个
变量的方程来表示,这个方程叫做点的轨迹的\textbf{极坐标方程}或
简称\textbf{极方程}.


\begin{example}
    求通过极点$O$且与极轴成定角$\alpha$的直线的极坐标
    方程(图7.5).
\end{example}

\begin{solution}
    设点$P(r,\theta)$为已知直线上的任一点,则点$P$
    满足极方程
 \begin{equation}
     \theta=\alpha
 \end{equation}
    反之,对任一实数$r$, 以$(r,\alpha)$为极坐标的点也一定满
    足方程(7.3). 因此方程(7.3)就是所求的直线的极方程.
\end{solution}

\begin{figure}[htp]\centering
    \begin{minipage}[t]{0.48\textwidth}
    \centering
\begin{tikzpicture}[>=latex, scale=.8]
    \draw[->, thick](0,0)node[below]{$O$}--(3,0)node[right]{$A$}; 
    \draw[very thick](0,0)--(45:2.5)node[right]{$P$};
    \draw[very thick](0,0)--(-180+45:2);
\draw[->](.75,0) arc (0:45:.75)node[right]{$\alpha$};

    \end{tikzpicture}
    \caption{}
    \end{minipage}
    \begin{minipage}[t]{0.48\textwidth}
    \centering
    \begin{tikzpicture}[>=latex, scale=.8]
\draw[->, thick](0,0)node[below]{$O$}--(3,0)node[right]{$A$};
\draw[very thick](0,0) circle (1.5); 
\draw[->](0,0)--node[right]{$a$}(120:1.5);
    \end{tikzpicture}
    \caption{}
    \end{minipage}
    \end{figure}

\begin{example}
    求圆心在极点$O$, 半径为$a$的圆的极坐标方程
(图7.6).
\end{example}

\begin{solution}
    因为对任一点$P(r,\theta)$,当且仅当
\begin{equation}
    r=a
\end{equation}
时,$P$点才在已知圆上,所以(7.4)式就是所求圆的极方程.
\end{solution}

\begin{example}
    试求以$C(a,0)$为圆心,以$a$为半径的圆的
    极坐标方程(图7.7). 
\end{example}

\begin{figure}[htp]
    \centering
\begin{tikzpicture}[>=latex, scale=.8]
\draw[->, thick](0,0)--(5,0)node[right]{$A$};
\draw[very thick](1.5,0) circle (1.5); 
\draw (0,0)node[left]{$O$}--(60:1.5)node[above]{$P$}--(3,0)node[below right]{$B$};
\tkzDefPoint(1.5,0){C}
\tkzDrawPoint(C)
\tkzLabelPoints[below](C)
    \end{tikzpicture}
    \caption{}
\end{figure}

\begin{solution}
    由已知条件,圆心在
极轴上,圆经过极点$O$. 设圆
和极轴的另一个交点是$B$. 得
知$P(r,\theta)$在已知圆上的充要条件是$\angle OPB=\pi/2$, 即
\[|\Vec{OP}|=|\Vec{OB}|\cos\theta\]
\begin{equation}
    r=2a\cos\theta
\end{equation}
因此(7.5)式就是所求圆的极方程.
\end{solution}

如果某动点的轨迹在直角坐标系中的方程为已知,那
么,利用变换公式
\[x=r\cos\theta,\qquad y=r\sin\theta\]
可求得该动点轨迹的极坐标方程;反之,若一动点的轨迹的
极方程为已知,我们也可用上节变换公式(7.2), 把它化为
在直角坐标系中的方程.

\begin{example}
    设一圆的方程为
\[x^2+y^2-8y=0\]
如果以原直角坐标系的原点为极点,$X$轴的正半轴为极轴,
求这圆的极方程.
\end{example}

\begin{solution}
    将$x=r\cos\theta$, $y=r\sin\theta $, 代入已知圆的方程,得
\[r^2-8r\sin\theta =0\]
即$r=8\sin\theta$.
这就是已知圆的极坐标方程.
\end{solution}

\begin{example}
    已知直线的极方程为$r\sin\theta=2$, 把它化为直角
    坐标方程.
\end{example}

\begin{solution}
    将$r=\sqrt{x^2+y^2}$, $\sin\theta =\frac{y}{r}$代入已知直线的极方
程,得
\[\sqrt{x^2+y^2}\cdot \frac{y}{\sqrt{x^2+y^2}}=2\]
即
$y=2$. 
这就是已知直线的直角坐标方程.
\end{solution}

我们可根据极方程用描点法近似地作出这个极方程的图
象,下面举例说明.

\begin{example}
    描出方程$r=a\theta\; (a>0)$的图象.
\end{example}

\begin{solution}
    与直角坐标系中的作图步骤一样,先给出$\theta$一系列
的允许值,算出$r$的对应值,由此得到一对应值表.然后再
根据对应值表描点作图(图7.8).
\begin{center}
\begin{tabular}{c|cccccccccc}
\hline
$\theta$ & 0&$\frac{\pi}{4}$&$\frac{\pi}{2}$&$\frac{3\pi}{4}$&$\pi$&$\frac{5\pi}{4}$&$\frac{3\pi}{2}$&$\frac{7\pi}{4}$&$2\pi$&$\cdots$\\
\hline
$r$ & 0&  $0.78a$ & $1.57a$ & $2.36a$ & $3.14a$ & $3.93a$ & $4.71a$ & $5.50a$ & $6.28a$ & $\cdots$\\
\hline
\end{tabular}
\end{center}
\begin{figure}[htp]
    \centering
\begin{tikzpicture}[>=latex, scale=.8]
    \draw[->](0,0)--(5,0)node[right]{$A$};
\draw[domain=0:2.25*pi, samples=1000, thick]plot({\x r}: {.5*\x});
\draw[domain=-2.25*pi:0, samples=1000, thick, dashed]plot({\x r}: {.5*\x});
\end{tikzpicture}
    \caption{}
\end{figure}

方程$r=a\theta,\; (a>0)$的图象,叫做\textbf{阿基米德螺线}.
\end{solution}

\begin{ex}
\begin{enumerate}
    \item 求通过极点$O$且与极轴$OA$成$\pi/6$
    角的直线的极方程.
    \item 求以$C(3,\pi/3)$
    为圆心,半径等于2的圆的极方程.
    \item 求通过点$P(r,\theta )$且与原点距离等于$d$的直线的极方
    程.
    \item 求$M(r_1,\theta_1)$, $N(r_2,\theta_2)$两点间的距离.
    \item 把下列直角坐标方程化为极方程.
\begin{multicols}{2}
\begin{enumerate}
\item $x=6$
\item $y=2x$
\item $x^2+y^2-9=0$
\item $x^2+y^2-4x+8=0$
\item $xy=4$
\item $x^2-y^2=1$
\item $y^2=4x$
\item $(x^2+y^2)^2=a^2(x^2-y^2)$
\end{enumerate}
\end{multicols}

    \item 把下列极坐标方程,化为直角坐标方程.
\begin{multicols}{2}
\begin{enumerate}
\item $r=3$
\item $\theta=\frac{\pi}{4}$
\item $r=3\cos\theta$
\item $r=5\tan\theta$
\item $r^2\cos 2\theta=16$
\item $r=\frac{6}{1+2\cos\theta}$
\end{enumerate}
\end{multicols}
    \item 画出心脏线$r=a(1+\cos\theta )$的图象.
    \item 画出双纽线$p^2=a^2\cos2\theta$的图象.
\end{enumerate}
\end{ex}

\subsection{圆锥曲线的极坐标方程}
在第六章中,我们给圆锥曲线下了一个统一的定
义,现在我们根据这个定义来求圆锥曲线统一的极方程.

已知圆锥曲线的焦点$F$和准线$\ell$, 过$F$作$\ell$的垂线,设
垂足为$D$. 以$F$为极点,$\Vec{DF}$的方向作为极轴的方向建立极
坐标系(图7.9).设$P(r,\theta)$是曲线上任一点,作$\overline{PF}$, 
再作$PQ\bot\ell$, $PM$垂直极轴,垂足分别为$Q,M$. 设$F$到
准线$\ell$的距离$\overline{FD}=p$, 则由圆锥曲线的定义,得
\[\frac{PF}{PQ}=e\]
即:$r=e\cdot PQ$

$\because\quad PQ=DF+r\cos\theta=p+r\cos\theta$

$\therefore\quad r=e(p+r\cos\theta)$

解出$r$,得:
\begin{equation}
\boxed{ r=\frac{ep}{1-e\cos\theta}}   
\end{equation}

这就是圆锥曲线的极方程.当$0<e<1$时,方程(7.6)
表示椭圆,定点$F$是它的左焦点,定直线$\ell$是它的左准线;
当$e=1$时,方程(7.6)表示开口向右的抛物线;当$e>1$
时,方程(7.6)表示双曲线,定点$F$是它的右焦点,定直线
$\ell$是它的右准线(图7.10).

\begin{figure}[htp]\centering
    \begin{minipage}[t]{0.48\textwidth}
    \centering
\begin{tikzpicture}[>=latex, scale=1]
    \draw[->](-1,0)--(4,0)node[right]{$A$};
    \draw[->](-.5,-2)--(-.5,2.5)node[right]{$\ell$};
    \draw[domain=-1.75:1.75, samples=100, thick] plot({\x*\x}, \x);
    \draw(1.5,0)node[below]{$M$}--(1.5,1.22)node[above]{$P$}--(-.5,1.22)node[left]{$Q$};        
    \draw(.5,0)node[below]{$F$}--node[left]{$r$}(1.5,1.22);
\draw(.75,0) arc (0:50.77:.25)node[right]{$\theta$};
\node at (-.5,0)[below left]{$D$};
    \end{tikzpicture}
    \caption{}
    \end{minipage}
    \begin{minipage}[t]{0.48\textwidth}
    \centering
    \begin{tikzpicture}[>=latex, scale=1]
\draw[->](-2,0)--(3,0);
\draw(-1.3,-2.5)node[below]{$\ell$}--(-1.3,2.5);
\draw[domain=0:2*pi, samples=300]plot({\x r}: {.6*1.3/(1-.5*cos(\x r))})node[below right]{$e<1$};
 \draw[domain=0.4*pi:1.6*pi, samples=300, thick]plot({\x r}: {1*1.3/(1-1*cos(\x r))})node[right]{$e=1$};
\draw[domain=0.45*pi:1.55*pi, samples=300, very thick]plot({\x r}: {1.5*1.3/(1-1.5*cos(\x r))})node[right]{$e>1$};
    \end{tikzpicture}
    \caption{}
    \end{minipage}
    \end{figure}

\begin{ex}
\begin{enumerate}
    \item 求证圆锥曲线$r=\frac{ep}{1-e\cos\theta}$,
    当$0<e\le 1$时的直角
    坐标方程是
    \[(1-e^2)x^2+y^2-2e^2px-e^2p^2=0\]
    当$e>1$时,直角坐标方程是$\sqrt{x^2+y^2}=e(x+p)$.
    \item 一颗慧星的轨道是抛物线,太阳位于这条抛物线的焦点
    上,已知这颗慧星在距太阳为$1.6\x10^8$公里时,它的极
    径和轨道轴成$60^{\circ}$角.求这颗慧星的轨道的极方程,并
    且求出它的近日点与太阳的距离.
    \item 说明下列方程的图形是什么,并且画出草图.
\begin{multicols}{2}
    \begin{enumerate}
        \item $r=\frac{5}{1-\cos\theta}$
        \item $r=\frac{5}{3-4\cos\theta}$
        \item $r=\frac{1}{2-\cos\theta}$
        \item $r=\frac{4}{1+\cos\theta}$
    \end{enumerate}
\end{multicols}
\end{enumerate}
\end{ex}

\section*{习题7.1}
\addcontentsline{toc}{subsection}{习题7.1}

\begin{enumerate}
    \item 作出下列极方程的图象,并说明它们各是什么曲线.
\begin{multicols}{2}
    \begin{enumerate}
        \item $r=1$
        \item $\theta=\frac{\pi}{3}$
        \item $r\cos\theta =2$
        \item $r\sin\theta =1$
        \item $r=6\cos\theta$
        \item $r=10\sin\theta$
    \end{enumerate}
\end{multicols}

    \item 求满足下列条件的各图形的极方程.
\begin{enumerate}
 \item 经过点$P\left(2,\frac{\pi}{4}\right)$, 
    垂直于极轴的直线;
    \item 经过点$Q\left(3,\frac{\pi}{3}\right)$, 
    平行于极轴的直线;
    \item 圆心在极点,半径等于$a$的圆;
    \item 圆心在点$B\left(a,\frac{\pi}{2}\right)$
    半径等于3的圆;
    \item 圆心在$C(a,\pi)$, 半径等于$a$的圆;
    \item 经过点$P(a,0)$和极轴相交成$\alpha$角的直线.  
\end{enumerate}

    \item 从极点作圆$r=2a\cos\theta$的各弦,求各弦中点的轨迹的极
    坐标方程.
    \item 从极点$O$作直线和直线$r\cos\theta=4$相交于$M$点,在$\overline{OM}$
    上取一点$P$, 使$\overline{OM}\cdot\overline{OP}=12$, 求$P$点轨迹的极坐标
    方程.
    \item 求适合于下列条件的轨迹的极坐标方程,并且画出轨迹
    的草图.
\begin{enumerate}
    \item 点的极径和极角成正比例;
    \item 点的极径和极角成反比例.
\end{enumerate}
\item 把下列各直角坐标方程化为极方程.
\begin{multicols}{2}
\begin{enumerate}
    \item $y^2=12x$
    \item $x^2+y^2=4y$
    \item $x^2-2xy+y^2=x-4$
    \item $y^2=4(1-x)$
    \item $y^2=2px$
    \item $\frac{x^2}{a^2}+\frac{y^2}{b^2}=1$
    \item $\frac{x^2}{a^2}-\frac{y^2}{b^2}=1$
\end{enumerate}
\end{multicols}
\item 判别下列各极方程表示什么曲线.
\begin{multicols}{2}
    \begin{enumerate}
    \item $r^2=\frac{400}{25\sin^2\theta+16\cos^2\theta}$
    \item $r^2=\frac{4}{\cos^2\theta-\sin^2\theta}$
    \item $r^2=\frac{6}{2\cos^2\theta+3\sin^2\theta}$
    \item $r^2=\frac{1}{\cos^2\theta-2\sin^2\theta}$
    \item $r^2=\frac{4\cos\theta}{\sin^2\theta}$
    \item $r^2=\frac{4\sin\theta}{\cos^2\theta}$
\end{enumerate}
\end{multicols}

\item 把下列极坐标方程化成直角坐标方程.
\begin{multicols}{2}
\begin{enumerate}
    \item $r=64\sin2\theta$
    \item $r+6\cot\theta\cdot \cos\theta=0$
\end{enumerate}
\end{multicols}

\end{enumerate}


\section{参数方程}

\subsection{参数方程的概念}
在直角坐标系$OXY$中,已知直线$\ell$过点$P_0(x_0,y_0)$
且平行于已知向量$\vec{a}=(a_1,a_2)$ (图7.11). 如果$P(x,
y)$是$\ell$上一动点,那么$\ell$的向量方程为
\[\Vec{OP}=\Vec{OP_0}+t\vec{a}\]
换成坐标形式,即为
\begin{equation}
    \begin{cases}
        x=x_0+a_1t\\
        y=y_0+a_2t
    \end{cases}
\end{equation}


\begin{figure}[htp]
    \centering
\begin{tikzpicture}[>=latex]
    \draw[->](-2,0)--(2.5,0)node[right]{$X$}; 
\draw[->](0,-1.5)--(0,2.5)node[right]{$Y$};
\node at (0,0)[below left]{$O$};
\draw[domain=-1.5:2, samples=10, very thick]plot(\x, {-\x+1})node[right]{$\ell$};
\draw[thick,->](0,0)--node[left]{$\vec{a}$}(.9,-.9);
\draw[thick,->](0,0)--(-.4,1.4)node[above]{$P$};
\draw[thick,->](0,0)--(.25,.75)node[above]{$P_0$};
\end{tikzpicture}
    \caption{}
\end{figure}


这就是说,直线$\ell$上的
点可以和实数$t$建立一一对
应关系.

一般来说,在取定的坐标系中,如果曲线上任一点的坐
标$x,y$都是某个变数$t$的函数时,
\begin{equation}
    x=f(t),\qquad y=\varphi(t)
\end{equation}
并且对于$t$的每一允许值,由方程(7.8)所确定的点$P(x,y)$
都在这条曲线上.那么方程(7.8)就叫做这条\textbf{曲线的参数方
程}.例如,方程(7.7)就是通过$P_0(x_0,y_0)$且平行于已知
向量$\vec{a}$的直线的参数方程.

上面我们用参数来表示直角坐标系中点的坐标$x,y$, 
同样我们也可用参数来表示在极坐标系中,点的极坐标$r,
\theta$. 即
\[  r=f(t),\qquad \theta=\varphi(t)\]

相对于参数方程来说,直接给出点的坐标间的关系的曲
线方程,叫做曲线的\textbf{普通方程}.如我们学过的直角坐标方程
和极方程.

\subsection{曲线的参数方程}
除上节建立的直线参数方程外,现在我们来建立几种常
见的曲线的参数方程.

\subsubsection{圆的参数方程}


以原点为圆心,$R$为半
径的圆,可以看作是一个质
点作等速圆周运动的轨迹
(图7.12).设质点的运
动的角速度为$\omega$, 从圆周与
$X$轴的正半轴的交点$A$的位
置开始按逆时针方向运动,
经过时间$t$后,质点到达圆周上一点$P(x,y)$的位置.由于
$\angle AOP=\omega t$,所以
\begin{equation}
    \begin{cases}
    x=R\cos\omega t\\
    y=R\sin\omega t
    \end{cases}
\end{equation}
在方程(7.9)中,对应$t$的每一个值,圆周上就有一点
$P(x,y)$与它对应.当$t$的值从0逐渐增加到
$2\pi/\omega$时,$P$
点就从$A$点开始按逆时针方向描出一个圆,所以(7.9)式就
是表示以原点为中心,$R$为半径的圆的参数方程.如果直接
取$\angle AOP=\theta$作为参数,那么圆的参数方程是
\[  \begin{cases}
    x=R\cos\theta\\
    y=R\sin\theta
    \end{cases}\]

\begin{figure}[htp]\centering
    \begin{minipage}[t]{0.48\textwidth}
    \centering
\begin{tikzpicture}[>=latex, scale=1]
\draw[->](-2,0)--(2.5,0)node[right]{$X$}; 
\draw[->](0,-2)--(0,2.5)node[right]{$Y$};
\node at (0,0)[below left]{$O$};
\draw[very thick](0,0) circle (1.5);
\draw[thick](0,0)--node[left]{$R$}(45:1.5)node[above]{$P$}--(1.06,0);
\node at (1.5,0)[below right]{$A$};
\draw[->](.5,0) arc (0:45:.5);  
    \end{tikzpicture}
    \caption{}
    \end{minipage}
    \begin{minipage}[t]{0.48\textwidth}
    \centering
    \begin{tikzpicture}[>=latex, scale=1]
\draw[->](-2,0)--(2.5,0)node[right]{$X$}; 
\draw[->](0,-2)--(0,2.5)node[right]{$Y$};
\node at (0,0)[below left]{$O$};
\draw(0,0) circle (1.5);
\draw(0,0) circle (1);
\draw[very thick] (0,0) ellipse [x radius=1.5, y radius=1];
\draw(0,.707)--(1.06,.707)node[right]{$P$};
\draw(0,0)--(45:1.5)node[above right]{$M$}--(1.06,0)node[below]{$Q$};

\tkzDefPoints{1.06/0/Q, 1.06/.707/P, 1.06/1.06/M, .707/.707/N}
\tkzDrawPoints(Q,P,M,N)
\node at (45:1)[above]{$N$};

\draw(.2,0) arc (0:45:.2);
\draw(.707-.2,.707) arc (180:180+45:.2);
    \end{tikzpicture}
    \caption{}
    \end{minipage}
    \end{figure}


\subsubsection{椭圆的参数方程}
    设$P(x,y)$是椭圆
$\frac{x^2}{a^2}+\frac{y^2}{b^2}=1$上任一点,以$O$为圆心,
分别以$a,b$为半径作两个辅助圆(图7.13).
过$P$点作直线$PQ$垂直于
$X$轴,垂足为$Q$点,并交大
辅助圆于$M$点,作$\overline{OM}$.

设$\angle MOX\varphi$,则$x=\overline{OQ}=a\cos\varphi$.
把上式代入椭圆方程,得$y=b\sin\varphi$,因此:
\begin{equation}
    \begin{cases}
        x=a\cos\varphi\\
        y=b\sin\varphi     
    \end{cases}
\end{equation}
就是椭圆的一个参数方程,其中$\varphi$叫做\textbf{离心角}.

\subsubsection{双曲线的参数方程}

由三角公式
\[\sec^2\varphi-\tan^2\varphi=1\]
我们可得双曲线$\frac{x^2}{a^2}-\frac{y^2}{b^2}=1$的一个参数方程为
\begin{equation}
    \begin{cases}
        x=a\sec\varphi\\
       y=b\tan\varphi
    \end{cases}
\end{equation}

\subsubsection{抛物线$y^2=2px$的参数方程}

如果令$y=2pt$,则$x=2pt^2$,所以
\begin{equation}
    \begin{cases}
        x=2pt^2\\ y=2pt
    \end{cases}
\end{equation}
可作为抛物线$y^2=2px$的一个参数方程.

\subsubsection{旋轮线的参数方程}
一个半径是$a$的车轮,沿一条直线轨道滚动,轮周上一
点$P$的轨迹叫做\textbf{旋轮线}(图7.14).
\begin{figure}[htp]
    \centering
	\begin{tikzpicture}[scale=1]
		\coordinate (O) at (0,0);
		\coordinate (Y) at (0,3);
		\def\r{1.5} % radius
		\def\p{1}
		\coordinate (B) at (\r, \r);
		\coordinate (A) at (\r, 0);
		\coordinate (P) at ({\r * (\p - sin(\p*180/pi))}, {\r * (1 - cos(\p*180/pi))});
		
		\draw[-latex] (0,-0.5) -- (Y) node[anchor=west] {$Y$};
		\draw[-latex] (-0.5,0) -- (2.4*pi*\r,0) node[anchor=west] {$X$};
		\draw[domain=0:{2*pi},samples=50,variable=\t] plot ({\r * (\t - sin(\t*180/pi)))},{\r * (1 - cos(\t*180/pi)))});
		\draw (B) circle (\r);
		\draw (B)--(A) (B)--(0,\r) coordinate (F);
		\draw (P) -- ({\r*(\p - sin(\p*180/pi))}, 0) coordinate (D);
		\draw (P) -- (\r, {\r*(\p * (1- cos(\p*180/pi)))}) coordinate (C);
		\draw (P) -- (B);
		
		\tkzLabelPoint[below left](O){$O$}
		\tkzLabelPoints[below](D,A)	
		\tkzLabelPoints[right](B,C)
		\tkzLabelPoints[above](P)
		\tkzMarkAngle[size=0.3](P,B,A) \tkzLabelAngle[pos=0.5](P,B,A){$\varphi$};
		\tkzLabelSegment(B,F){$a$}
	\end{tikzpicture}
    \caption{}
\end{figure}

下面我们来建立旋轮线的参数方程.

取$P$点落在轨道上的个一位置作为原点.轨道所在直线
作为$X$轴,当车轮从开始起转过了$\varphi$角,设这时$P$点的坐标
是$(x,y)$, 车轮的圆心在$B$点,与轨道相切于$A$点,于
是$\wideparen{AP}$的长等于$\overline{OA}$的长,我们引入参数$\varphi$(弧度).(叫
做滚动角)来表示$x$和$y$.

作$PD\bot OX$于$D$点,$PC\bot BA$于$C$点,则
\[\begin{split}
    x&=\overline{OD}=\overline{OA}-\overline{DA}=\wideparen{AP}-\overline{PC}=a\varphi-a\sin\varphi=a(\varphi-\sin\varphi)\\
    y&=\overline{DP}=\overline{AC}=\overline{AB}-\overline{BC}=a -a\cos\varphi=a(1-\cos\varphi)
\end{split}\]
因此,$P$点的轨迹的参数方程是
\begin{equation}
    \begin{cases}
        x=a(\varphi-\sin\varphi)\\
        y=a(1-\cos\varphi)
    \end{cases}
\end{equation}
(7.13)式就是旋轮线的一个参数方程.

\subsubsection{圆的渐开线参数方程}

把一条没有伸缩性的绳子绕在一个固定的圆盘的侧面
上,拉开绳子的一端并拉直,使绳子和圆周始终相切,然后
逐渐展开.绳子端点的轨迹叫做圆的\textbf{渐开线}(图7.15).
这个圆叫做渐开线的\textbf{基圆}.
\begin{figure}[htp]
    \centering
	\begin{tikzpicture}[scale=1]
		\coordinate (O) at (0,0);
		\coordinate (Y) at (0,3.5);
		\def\r{1} % radius
		\def\p{1.2	}
		\coordinate (A) at (\r, 0);
		\pgfmathsetmacro{\u}{\r*(cos(\p*180/pi)+\p*sin(\p*180/pi))}
		\pgfmathsetmacro{\v}{\r * (sin(\p*180/pi) - \p*cos(\p*180/pi))}
		\coordinate (B) at (\u,\v);
		
		\draw[-latex] (0,-1.5) -- (Y) ;
		\draw[-latex] (-1.6*pi*\r,0) -- (2	*\r,0) ;
		\draw[domain=0:{4.49341},samples=50,variable=\t] plot ({\r*(cos(\t*180/pi)+\t*sin(\t*180/pi))},{\r * (sin(\t*180/pi) - \t*cos(\t*180/pi))});
		\draw (O) circle (\r);
		
		\pgfmathsetmacro{\uvsqsum}{\u * \u + \v * \v}
		\pgfmathsetmacro{\Cx}{(\u - \v * sqrt(\uvsqsum - 1)) / \uvsqsum}
		\pgfmathsetmacro{\Cy}{sqrt(1 - \Cx * \Cx)}
		\coordinate (C) at (\Cx,\Cy);
		\draw (B)--(C)--(O)--(B);
		\tkzMarkAngle[size=0.3](A,O,B)
		\tkzLabelSegment[above](O,C){$a$}
		\tkzLabelSegment[above=-0.1](O,B){$r$}
		\tkzMarkRightAngle(O,C,B)
		\tkzLabelPoints[below right](A)
		\tkzLabelPoints[right](B)
		\tkzLabelPoints[above](C)
		\tkzLabelPoints[below left](O)
	\end{tikzpicture}
    \caption{}
\end{figure}

下面我们来建立渐开线的参数方程.

设定圆的圆心为$O$, 半径为$a$, 开始时绳子的外端在$A$
点,$O$为极点,以射线$OA$
为极轴建立极坐标系,设$B$
是渐开线上任一点,$(r,\theta)$是
它的极坐标,其中$\theta$的单位
是弧度.$\overline{OA}=a$, 
$\angle BOC=\alpha$, 则$r=\frac{a}{\cos\alpha}$,
$\overline{BC}=a\tan\alpha$. 根据题设应有
\[\overline{BC}=\wideparen{AC}=a(\alpha+\theta)\]
解出$\theta$, 得
\[\theta=\frac{\overline{BC}}{a}-\alpha=\tan\alpha-\alpha\]
因此,渐开线的极坐标的参数方程为
\begin{equation}
    \begin{cases}
        r=\frac{a}{\cos\alpha}\\
    \theta=\tan\alpha-\alpha
    \end{cases}
\end{equation}

以$OA$为$X$轴的正半轴,建立直角坐标系.
取$\angle AOC=\varphi$ 作为参数,由于
$\varphi =\alpha+\theta$, 
应用公式(7.14)式的第二式可得
$$\varphi  =\tan\alpha$$
设$B$点在$OXY$中的坐标为$(x,y)$, 则
\begin{equation}
\begin{split}
    x&=r\cos\theta=a(\cos\varphi+\varphi\sin\varphi)\\
    y&=r\sin\theta=a(\sin\varphi-\varphi\cos\varphi)
\end{split}
\end{equation}
这是渐开线在直角坐标系中的参数方程.

由以上几种常见曲线的参数方程的推导可知,通常建立
曲线的参数方程有两种方法:一种是像(一)那样,把曲线
看作动点的轨迹,选取时间参数$t$, 使得曲线上的点的动坐
标$x,y$分别用$t$的函数来表示,另一种是像(二)、(三)那
样,从已知曲线的直角坐标方程引入适当的参数,从而求得
曲线的参数方程.最后,我们指出,一条曲线的参数方程不
是唯一的.

以后我们将会看到,利用参数方程研究曲线的形状和性
质比普通方程更加方便.

\begin{example}
    画出参数方程$\begin{cases}
        x=t^2\\ y=t^3
    \end{cases}$
    所表示的曲线.
\end{example}


\begin{solution}
列表
\begin{center}
\begin{tabular}{ccccccccc}
\hline
$t$&$\cdots$ &$-3$ &$-2$&0&1&2&3&$\cdots$\\
\hline
$x$&$\cdots$ & 9&4&0&1&4&9             &$\cdots$ \\
$y$&$\cdots$ & $-27$&$-8$ &0&1&8&27  &$\cdots$ \\
\hline
\end{tabular}
\end{center}

用表中的数对$(x,y)$描点作图,就可
得到方程的曲线(图7.16).这条曲线叫做\textbf{半立方抛物线}.

\begin{figure}[htp]
    \centering
\begin{tikzpicture}[>=latex, scale=.3]
\draw[->](-2,0)--(8,0)node[right]{$X$};
\draw[->](0,-8)--(0,8)node[right]{$Y$};
\draw[domain=0:4, samples=100, very thick]plot(\x, {\x*sqrt(\x)});
\draw[domain=0:4, samples=100, very thick]plot(\x, {-\x*sqrt(\x)});
\foreach \x in {2,4,6}
{
   \draw(0,\x)node[left]{\x}--(.2,\x);
   \draw(\x,0)node[below]{\x}--(\x,.2);
}
\foreach \x in {-2,-4,-6}
{
   \draw(0,\x)node[left]{$\x$}--(.2,\x);
}


\end{tikzpicture}
    \caption{}
\end{figure}

\end{solution}

\begin{ex}
\begin{enumerate}
    \item 写出下列直线的参数方程.
\begin{enumerate}
    \item 过$P(2,3)$且平行于已知向量$\vec{a}=(1,4)$的
    直线
    \item 过$P_1(3,4)$, $P_2(4,-3)$两点的直线
    \item $y=kx$
    \item $y=kx+b$
\end{enumerate}
\item 一质点沿方向$\vec{S}=\left(\cos\frac{\pi}{6},\sin\frac{\pi}{6}\right)$, 
从点$P_0(1,2)$, 
以3m/s的速率运动,写出运动轨迹的参数方程.
\item 已知一条直线上两点$M_1(x_1,y_1)$, $M_2(x_2,y_2)$, 以
分点$M(x,y)$分$\overline{M_1M_2}$所成的比$\lambda$为参数.写出这条直线的参数方程.
\item 已知抛物线$x=2pt^2$, $y=2pt$, 写出通过对应于参数
$t_1,t_2$两点的弦的方程.
\item 作下列参数方程的图形.
\begin{enumerate}
\item $x=t$, $y=3t$
\item $x=3\sin\theta$, $y=4\cos\theta$
\item $x=4t^2$, $y=2t$
\end{enumerate}
\end{enumerate}
\end{ex}

\subsection{参数方程和普通方程的互化}
设曲线的参数是
\[\begin{cases}
   x=f(t)\\
y=\varphi(t) 
\end{cases}\]
如果我们能从这个方程消去参数$t$, 那么我们就可求出当线
的普通方程.
 
\begin{example}
    把参数方程$\begin{cases}
        x=5\cos t+2\\
y=2\sin t-3
    \end{cases}$化为普通方程.
\end{example}

\begin{solution}
    由已知参数方程可得
\[\frac{x-2}{5}=\cos t,\qquad \frac{y+3}{2}=\sin t\]
两式两边平方后相加,得
\[\frac{(x-2)^2}{25}+\frac{(y+3)^2}{4}=1\]
这就是已知曲线的普通方程.
\end{solution}

\begin{example}
    把参数方程
\begin{numcases}{}
    x=at^2\\
    y=a^2t^3
\end{numcases}
化为普通方程.
\end{example}

\begin{solution}
(7.16)式两边立方,(7.17)式两边平方,得
\begin{align}
    x^3=a^3t^6\\
    y^2=a^4t^6
\end{align}
由(7.18), (7.19)两式可得
\[y^2=ax^3\]
\end{solution}

\begin{example}
    化直线的点斜式方程$y-y_0=k(x-x_0)$为参数
方程.
\end{example}

\begin{solution}
    直线的点斜式方程可变为
\[kx-y+y_0-kx_0=0\]
因此直线具有方向向量为$\vec{S}=(1,k)$, 所以,直线方程的
参数方程可写为
\[\begin{cases}
    x=x_0+t\\ y=y_0+kt
\end{cases}\]
\end{solution}




\begin{ex}
\begin{enumerate}
    \item 把下列曲线的参数方程化为普通方程.
    \begin{multicols}{2}
\begin{enumerate}
    \item $\begin{cases}
        x=3+2t\\y=2-3t
    \end{cases}$
    \item $\begin{cases}
        x=2+3\cos\theta\\
        y=4-3\sin\theta
    \end{cases}$
    \item $\begin{cases}
        x=t\\y=4t^2
    \end{cases}$
    \item $\begin{cases}
        x=\cos^2 t\\ y=\sin t
    \end{cases}$
\end{enumerate}
    \end{multicols}

\item 把下列普通方程化为参数方程.
\begin{multicols}{2}
\begin{enumerate}
    \item $\frac{x-2}{4}=\frac{x+5}{3}$
    \item $y=4x$
    \item $\frac{(x-h)^2}{a^2}+\frac{(y-k)^2}{b^2}=1$
    \item $\frac{(x-h)^2}{a^2}-\frac{(y-k)^2}{b^2}=1$
    \item $y=6x^2$
    \item $y^2=8x$
\end{enumerate}
\end{multicols}
\end{enumerate}
\end{ex}

\section*{习题7.2}
\addcontentsline{toc}{subsection}{习题7.2}

\begin{enumerate}
\item 已知直线$\ell$通过点$P_0(x_0,y_0)$并且与已知单位向量
$\vec{e}=(\cos\alpha,\sin\alpha)$平行,求直线$\ell$的参数方程.
\item 求经过点$P(1,3)$, 倾角是$\pi/4$
的参数方程.
\item 已知$M(x,y)$从原点以常速度向量$\vec{v}(v_x,v_y)$运动.
求$M$点的轨迹的参数方程.并且把它化为普通方程.如
果$M$点从$A(a,b)$点开始运动,$M$点的轨迹的参数方
程怎样?
\item 把下列参数方程化成普通方程.
\begin{multicols}{2}
\begin{enumerate}
    \item $\begin{cases}
        x=t^2-2t\\y=t^2+2
    \end{cases}$
    \item $\begin{cases}
        x=\frac{a(1-t^2)}{1+t^2}\\
        y=\frac{2bt}{1+t^2}
    \end{cases}$
    \item $\begin{cases}
        x=a\sec\varphi\\
        y=b\tan\varphi
    \end{cases}$
    \item $\begin{cases}
        x=5t^2-1\\y=10t^2+4
    \end{cases}$
    \item $\begin{cases}
        x=\frac{a}{2}\left(t+\frac{1}{t}\right)\\
        y=\frac{b}{2}\left(t-\frac{1}{t}\right)
    \end{cases}$
\end{enumerate}
\end{multicols}
\item 已知抛物线$x=2pt^2$, $y=2pt$, 求证:通过对应$t_1$,
$t_2$两点的直线方程是
\[x-(t_1+t_2)y+2pt_1t_2=0\]
\item 已知抛物线$x=2pt^2$, $y=2pt$, 求证:抛物线在点$t_1$
处的切线方程为
\[x-2t_1y+2pt_1^2=0\]
\item 求证:抛物线$x=2pt^2$, $y=2pt$, 在点$t_1$、$t_2$处的切
线交点的坐标是
\[\big(2pt_1t_2,\; p(t_1+t_2)\big)\]
\item 利用第7题的结论,证明:抛物线通过焦点的弦的两个端
点处的切线相交在准线上.

\end{enumerate}

\section*{复习题七}
\addcontentsline{toc}{section}{复习题七}

\begin{enumerate}
    \item 画出下列各极坐标方程的图形.
\begin{multicols}{2}
    \begin{enumerate}
        \item $r\theta =a$
        \item $r=2\theta$
        \item $r=5(1-\cos\theta )$
        \item $r=a\sin3\theta$
        \item $r=a\cos\theta +b$
        \item $r^2=16\sin 2\theta$
    \end{enumerate}
\end{multicols}

\item 把下列各曲线的极坐标方程化为直角坐标方程.
\begin{multicols}{2}
\begin{enumerate}
\item $r\sin\theta =10$
\item $r=4\sin\theta$
\item $r(5+3\cos\theta )=16$
\item $r(4+5\cos\theta )=9$
\item $r^2\cos2\theta =-1$
\item $r(\sin\theta +2\cos\theta )=6$
\item $r=2\cos\theta +3\sin\theta $
\item $\theta =45^{\circ}$
\item $r=\frac{3}{2+3\sin\theta}$ 
\item $r^2=9\cos2\theta $
\end{enumerate}
\end{multicols}

\item 把下列各直角坐标方程化为极坐标方程.
\begin{multicols}{2}
  \begin{enumerate}
\item $\frac{x^2}{a^2}-\frac{y^2}{b^2}=1$
\item $\frac{x^2}{a+x}=\frac{y^2}{a-x}$
\item $x^2+y^2=3xy$
\item $y^2=\frac{x^3}{2a-x}$
\item $x^2+y^2+2Dx+2Ey+F=0$
\item $(x^2+y^2)^3=4x^2y^2$
\item $x^4+x^2y^2-(x+y)^2=0$
\item $(x^2+y^2)^3=16x^2y^2(x^2-y^2)^2$
\end{enumerate}  
\end{multicols}


\item 说明下列两条直线的位置关系.
\begin{enumerate}
    \item $\theta =\alpha$ 和$r \cos(\theta-\alpha)=a$ ($a>0$且为定值);
\item $\theta =\alpha$ 和$r\sin(\theta -\alpha)=a$.
\end{enumerate}

\item 求证:经$P(r_1,\theta_1)$点和极轴交成$\alpha$角的直线方程是
\[r\sin(\theta -\alpha )=r_1\sin(\theta_1-\alpha)\]
\item $O$点是原点,$P$点是椭圆$x=3\cos\varphi$, $y=2\sin\varphi$上相
当于$\varphi=\pi/6$的一点,求直线$OP$的倾角.
\item 已知椭圆
$x=a\cos\varphi$, $y=b\sin\varphi$, 求证:通过对应
$\varphi =\alpha$和$\varphi =\beta$ 椭圆上两点的直线方程是
\[\frac{x}{a}\cos\frac{1}{2}(\alpha+\beta )+\frac{y}{b}\sin\frac{1}{2}(\alpha+\beta )=\cos\frac{1}{2}(\alpha-\beta )\]
\item 已知椭圆$x=a\cos\varphi$, $y=b\sin\varphi$. 求证:椭圆上对应
$\varphi_1$点的切线方程是
\[\frac{x}{a}\cos\varphi_1+\frac{y}{b}\sin\varphi_1=1\]
\item 一个圆的圆心在$C(a,b)$, 半径为$R$. 求这个圆以圆
心角$\theta$(从$X$轴的正方向算起)为参数的参数方程.
\item 已知$P$、$Q$是椭圆
$x=a\cos\varphi$, $y=b\sin\varphi$ 上分别对应
$\varphi_1$和$\varphi_2$的两点.求证:直线$OP$和$OQ$为椭圆共轭直
径的条件是$|\varphi_1-\varphi_2|=90^{\circ}$.
\item 已知椭圆$x=a\cos\varphi$, $y=b\sin\varphi$, $P$、$Q$是椭圆上对
应$\varphi$ 和$\varphi +90^{\circ}$的两点,求证:
\[|\Vec{OP}|^2+|\Vec{OQ}|^2=a^2+b^2\]

\item 画出下列参数方程表示的图形.
\begin{multicols}{2}
\begin{enumerate}
    \item $\begin{cases}
        x=3t-5\\y=t^3-t
    \end{cases}$
    \item $\begin{cases}
        x=t-\sin t\\ y=1-\cos t
    \end{cases}$
\end{enumerate}
\end{multicols}
\end{enumerate}



  
\chapter{空间解析几何初步}
\section{空间向量的坐标运算}
\subsection{空间直角坐标系与向量运算}
任取一点$O$(图8.1), 一个单位长,通过$O$点建立
三条互相垂直的数轴,$X$轴、$Y$轴、$Z$轴,并且使这三个数
轴的正方向构成右手系。这样我们
就说在空间建立了一个空间右手坐
标系,并用$OXYZ$来表示。$O$点
叫做坐标系的原点。$X$轴、$Y$轴、
$Z$轴总称为坐标轴。三个坐标轴每
两个决定一平面叫做坐标平面。坐标平面共有三个$OXY$、$OYZ$、
$OZX$,它们互相垂直并且把空间分为八个区域,每个区域叫做一个\textbf{卦限}。

\begin{figure}[htp]\centering
    \begin{minipage}[t]{0.48\textwidth}
    \centering
\begin{tikzpicture}[>=latex, scale=1]
\draw[<->](0,3.5)node[right]{$Z$}--(0,0)node [below right]{$O$}--(3,0)node[right]{$Y$};  
\draw[dashed](-2,0)--(0,0)--(1.5,1.5);
\draw[dashed](0,0)--(0,-1);
\draw[->](0,0)--(-1.5,-1.5)node[right]{$X$};
    \end{tikzpicture}
    \caption{}
    \end{minipage}
    \begin{minipage}[t]{0.48\textwidth}
    \centering
    \begin{tikzpicture}[>=latex, scale=1]
\draw[<->](0,3.5)node[right]{$Z$}--(0,0)--(3,0)node[right]{$Y$};  
\draw[->](0,0)--(-1.5,-1.5)node[left]{$X$};
\tkzDefPoints{0/0/O, 2/0/B, 2/2.5/P', 0/2.5/C, -1/-1/A}
\tkzDefPointsBy[translation= from O to A](B,P',C){B',P,C'}
\tkzDrawPolygon(B',P,C',A)
\tkzDrawPolygon(B,P',C,O)
\tkzDrawSegments(P,P' C,C' B,B')
\tkzLabelPoints[below](A,O,B)
\tkzLabelPoints[right](P)
\tkzLabelPoints[left](C)
    \end{tikzpicture}
    \caption{}
    \end{minipage}
    \end{figure}

设$P$是空间中任一点,通过$P$点作平面分别与坐标平面
$OYZ$、$OZX$、$OXY$平行(图8.2),并且分别与$X$
轴、$Y$轴、$Z$轴相交于$A$、$B$、$C$三点,如果$A$、$B$、$C$在
各坐标轴上的坐标分别为$x$、$y$、$z$, 则这三个有序实数组
$(x,y,z)$叫$P$点的\textbf{空间坐标}。简称坐标。$P$
点的坐标是$(x,y,z)$, 
记作$P(x,y,z)$. $x$、$y$、$z$分别叫做$P$点
的$X$坐标,$Y$坐标,$Z$坐标。

\begin{figure}[htp]
    \centering
\begin{tikzpicture}[>=latex]
    \draw[->](0,2.6)--(0,4)node[right]{$Z$};
    \draw[->](0,0)--(4.5,0)node[right]{$Y$};  
    \draw[->](0,0)--(-1.5,-1.5)node[left]{$X$};
\draw[->, very thick](0,0)--node[right]{$\eZ$}(0,1);
\draw[->, very thick](0,0)--(1,0)node[below]{$\eY$};
\draw[->, very thick](0,0)--(-.5,-.5)node[right]{$\eX$};
\node at (0,0)[below right]{$O$};
\tkzDefPoints{-1/1/A, 2/1/B, 3.3/1.75/C, .3/1.75/D, -1/2/A'}
\tkzDefPointsBy[translation = from A to A'](B,C,D){B',C',D'}
\tkzDrawPolygon[thick](A',B',C',D')
\tkzDrawPolygon[dashed](A,B,C,D)
\tkzDrawSegments[dashed](D,D')
\tkzDrawSegments[thick](A,A' B,B'  A,B)
\draw[dashed](0,1)--(0,2.6);
\draw[very thick, ->](B)--node[below]{$a_x\eX$}(C);
\draw[very thick, ->](C)--node[right]{$a_z\eZ$}(C');
\draw[very thick, ->](C')--node[above]{$a_y\eY$}(D');
\draw[thick, ->, dashed](B)--node[below]{$\vec{a}$}(D');
\end{tikzpicture}
    \caption{}
\end{figure}


如果沿$X$轴、$Y$轴、$Z$轴的正方向分别引单位向量$\eX$、$\eY$、$\eZ$(图8.3), 那么对空间任一向量$\vec{a}$, 存在唯一的有序数组
$(a_x,a_y,a_z)$使
\[\vec{a}=a_x\eX+a_y\eY+a_z\eZ\]
$(a_x,a_y,a_z)$就叫做$\vec{a}$在
$OXYZ$中的坐标。并简记作
\[\vec{a}=(a_x,a_y,a_z)\]
其中$a$叫做$\vec{a}$在$X$轴上的坐标分量。
$a_y$叫做$\vec{a}$在$Y$轴上的坐标分量。$a_z$叫做$\vec{a}$在$Z$轴上的坐标分量。

如果$\vec{a}=a_x\eX+a_y\eY+a_z\eZ$,那么分别对这个表示式两
边对$\eX,\eY,\eZ$取内积运算,就可得到
\[\begin{split}
    a_x&=\eX\cdot \vec{a}=|\vec{a}|\cos\langle \eX,\vec{a} \rangle \\
    a_y&=\eY\cdot \vec{a}=|\vec{a}|\cos\langle \eY,\vec{a} \rangle \\
    a_z&=\eZ\cdot \vec{a}=|\vec{a}|\cos\langle \eZ,\vec{a} \rangle \\
\end{split}\]

如果$\langle \eX,\vec{a} \rangle=\alpha$, $\langle \eY,\vec{a} \rangle=\beta$, $\langle \eZ,\vec{a} \rangle=\gamma$, 那
么$\alpha$、$\beta$、$\gamma$确定了$\vec{a}$在空间中的方向。$\alpha$、$\beta$、$\gamma$叫做
$\vec{a}$的方向角,$\cos\alpha$、$\cos\beta$、$\cos\gamma$叫做$\vec{a}$的方向余弦,于是
$\vec{a}$的单位向量
\[\vec{a}_0=(\cos\alpha, \cos\beta, \cos\gamma)\]

对空间任一点$P$, 它被相对于$O$点的位置向量所唯一确
定(图8.4)。设
\[\Vec{OP}=x\eX+y\eY+z\eZ\]
由上述点的坐标和向量坐标的定
义,$\Vec{OP}$的坐标$(x,y,z)$
也就是$P$点的坐标;反之$P$点的
坐标也是$\Vec{OP}$的坐标。由此可
见,给定了原点$O$和三个互相垂
直且构成右手系的单位向量$\eX,\eY,\eZ$,坐标系$OXYZ$也就完全确定了。因此,坐标系
$OXYZ$也可用$[O:\eX,\eY,\eZ]$来表示,$\eX,\eY,\eZ$叫
做$OXYZ$的基向量。

\begin{figure}[htp]
    \centering
\begin{tikzpicture}[>=latex]
\tkzDefPoints{0/0/A, 2/0/B, 2/2.5/C, 0/2.5/D, -.8/-.8/A'}
\tkzDefPointsBy[translation= from A to A'](B,C,D){B',C',D'}
\tkzDrawPolygon(A',B',C',D')
\tkzDrawPolygon[dashed](A,B,C,D)
\tkzDrawSegments(B,C C,D B,B' C,C' D,D')
\tkzDrawSegments[dashed](A,B A,D  A,A')
\draw[->](A')--(-1.5,-1.5)node[left]{$X$};
\draw[->](B)--(3,0)node[right]{$Y$};
\draw[->](D)--(0,3.5)node[right]{$Z$};
\draw[->, dashed](A)--(C');
\draw[->, very thick](0,0)--(0,1)node[left]{$\eZ$};
\draw[->, very thick](0,0)--node[above]{$\eY$}(1,0);
\draw[->, very thick](0,0)--(-.5,-.5)node[above]{$\eX$};
\node at (0,0)[below right]{$O$};
\node at (C')[right]{$P$};
\end{tikzpicture}
    \caption{}
\end{figure}

已知$A(x_1,y_1,z_1)$, $B(x_2,y_2,z_2)$, 则:
\[\begin{split}
   \Vec{AB}&=\Vec{OB}-\Vec{OA}\\
&=x_2\eX+y_2\eY+z_2\eZ-(x_1\eX+y_1\eY+z_1\eZ)\\
&=(x_2-x_1)\eX+(y_2-y_1)\eY+(z_2-z_1)\eZ\\
&=(x_2-x_1, y_2-y_1, z_2-z_1)
\end{split}\]
这就是说\textbf{一个向量的坐标,等于表示它的有向线段终点的坐
标减去起点的坐标}。例如,已知$A(2,-1,5)$、$B(3,
2,-7)$, 则
\[\Vec{AB}=[3-2,\; 2-(-1),\; -7-5]=(1,3,-12)\]

\begin{ex}
\begin{enumerate}
    \item 问在$OXYZ$中,哪个坐标平面与$X$轴垂直,哪个坐标
    平面与$Y$轴垂直,哪个坐标平面与$Z$轴垂直?
    \item 写出点$P(2,4,3)$在$OXYZ$的三个坐标平面上投
    影点的坐标。
    \item 求点$P(3,5,4)$关于坐标平面$OXY$的对称点的坐
    标。
    \item 点$P$在$OXYZ$中的坐标平面$OXY$上,若$P$点在平面
    直角坐标系$OXY$中的坐标是$(2,3)$, 求它在
    $OXYZ$中的坐标。
    \item 写出基向量$\eX,\eY,\eZ$的坐标。
    \item 已知$\vec{a}=12$, $\langle\eX ,\vec{a}\rangle=30^{\circ}$, $\langle\eY ,\vec{a}\rangle=45^{\circ}$, $\langle\eZ ,\vec{a}\rangle=60^{\circ}$,
求$\vec{a}$的坐标。
    \item 已知$P(-3,2,4)$, $Q(5,7,-2)$, 求$\Vec{PQ}$与$\Vec{QP}$的坐标。
    \item 已知$A(2,-1,5)$, $B(3,2,-1)$用基向量$\eX,\eY,\eZ$表示向量$\Vec{AB}$.
\end{enumerate}
\end{ex}

\subsection{向量的坐标运算}

\begin{blk}{定理}
     如果$\vec{a}=(a_x,a_y,a_z)$, $\vec{b}=(b_x, b_y,b_z)$, 
$\vec{c}=(c_x,c_y,c_z)$, 那么
\[\begin{split}
    \vec{a}\pm \vec{b}&=(a_x,a_y,a_z)\pm (b_x,b_y,b_z)
=(a_x\pm b_x,a_y\pm b_y,a_z\pm b_z)\\
\lambda\vec{a}&=\lambda(a_x, a_y,a_z)=(\lambda a_x,\lambda 
a_y,\lambda a_z)\\
\vec{a}\cdot \vec{b}&=(a_x, a_y, a_z)\cdot (b_x,b_y,b_z)
=a_xb_x+a_yb_y+a_zb_z\\
\end{split}\]
\[\vec{a}\x \vec{b}=\begin{vmatrix}
  \eX&\eY&\eZ\\
  a_x&a_y&a_z\\
  b_x&b_y&b_z  
\end{vmatrix},\qquad \left(\vec{a},\vec{b},\vec{c}\right)=\begin{vmatrix}
    a_x&a_y&a_z\\
    b_x&b_y&b_z\\  
    c_x&c_y&c_z
\end{vmatrix}\]
\end{blk}


证明留给同学作为练习。

下面我们研究如何用向量的坐标来表示向量垂直、平行
与共面的条件。

已知$\vec{a}\parallel \vec{b}\quad (\vec{b}\ne 0)$的充要条件是存在一实数$\lambda$,使
$$\vec{a}=\lambda\vec{b}$$
如果
$\vec{a}=(a_x,a_y,a_z)$, $\vec{b}=(b_x,b_y,b_z)$, 那么上面
条件用坐标表示,即为
\begin{equation}
    a_x=\lambda b_x,\qquad  a_y=\lambda b_y,\qquad  a_z=\lambda b_z
\end{equation}
或
\begin{equation}
    a_x:b_x=a_y:b_y=a_z:b_z
\end{equation}
这就是说\textbf{两个向量平行的充要条件是它们的坐标成比例}。

已知$\vec{a}\bot \vec{b}\quad \Longleftrightarrow \quad \vec{a}\cdot \vec{b}=0$
用坐标表示,即为
\begin{equation}
    \vec{a}\bot \vec{b}\quad \Longleftrightarrow \quad a_xb_x+a_yb_y+a_zb_z=0
\end{equation}

已知$\vec{a}=(a_x,a_y,a_z)$, $\vec{b}=(b_x,b_y,b_z)$, $\vec{c}=(c_x,c_y,c_z)$, 则
\[\vec{a}, \vec{b}, \vec{c}\text{ 共面} \quad \Longleftrightarrow \quad  (\vec{a}, \vec{b}, \vec{c})=0\]
即
\[\vec{a}, \vec{b}, \vec{c}\text{ 共面} \quad \Longleftrightarrow \quad \begin{vmatrix}
    a_x&a_y&a_z\\b_x&b_y&b_z\\c_x&c_y&c_z
\end{vmatrix}=0\]

\begin{example}
    已知 $\vec{a}=(1,1,1)$, $\vec{b}=(3,-1,2)$, 
$\vec{c}=(1,-3,0)$。
求证:$\vec{a},\vec{b},\vec{c}$共面。
\end{example}

\begin{solution}
\[\because\quad (\vec{a},\vec{b},\vec{c})=\begin{vmatrix}
    1&1&1\\3&-1&2\\1&-3&0
\end{vmatrix}=0\]

$\therefore\quad \vec{a},\vec{b},\vec{c}$共面。
\end{solution}


\begin{ex}
\begin{enumerate}
    \item 已知$\vec{a}=(-1,2,3)$, $\vec{b}=(2,-4,-6)$, 求证:
$\vec{a}\parallel \vec{b}$.
\item 试证下面各对向量线性相关。
\begin{enumerate}
    \item $\vec{a}=(2,-1,-2),\qquad \vec{b}=(6,-3,-6)$
    \item $\vec{a}=(-3,-5,4),\qquad   \vec{b}=(6,10,-8)$
\end{enumerate}

 \item 已知$\vec{a}=(2,3,4)$, $\vec{b}=(-3,-6,6)$, 求证$\vec{a}\bot \vec{b}$.

 \item 设$\vec{a}=(2,-1,4)$, $\vec{b}=(-4,-5.-1)$, 求使
 $\vec{a}-k\vec{b}$垂直于$\vec{b}$的实数$k$的值。

 \item 已知$\vec{a}=(-5,2,3)$, $\vec{b}=(0,-3,2)$, $\vec{c}=(5,
 -2,-3)$, 求证:$\vec{a},\vec{b},\vec{c}$三个向量共面。
\item 已知$P(x,y,z)$, $P_1(x_1,y_1,z_1)$, $P_2(x_2,y_2,
z_2)$, $P_3(x_3, У_3,z_3)$. 求证这四点共面的充要条件是
\[\begin{vmatrix}
    x-x_1 &y-y_1&z-z_1\\
    x_2-x_1 &y_2-y_1&z_2-z_1\\
    x_3-x_1 &y_3-y_1&z_3-z_1\\
\end{vmatrix}=0\quad \text{或}\quad \begin{vmatrix}
    x&y&z&1\\x_1 & y_1&z_1& 1\\
    x_2 & y_2&z_2& 1\\x_3 & y_3&z_3& 1
\end{vmatrix}=0\]
\end{enumerate}   
\end{ex}

\subsection{空间解析几何的基本问题}
\begin{blk}{问题1}
    求有向线段定比分点的坐标。
\end{blk}
 
已知$P_1(x_1,y_1,z_1)$、$P_2(x_2,y_2,z_2)$, 如果
$P(x,y,z)$, 按定比$\mu$分割$\Vec{P_1P_2}$, 那么
\[\Vec{OP}=\frac{\Vec{OP_1}+\mu \Vec{OP_2}}{1+\mu}\]
换用坐标表示,即为
\begin{equation}
    \begin{cases}
        x=\frac{x_1+\mu x_2}{1+\mu}\\
        y=\frac{y_1+\mu y_2}{1+\mu}\\
        z=\frac{z_1+\mu z_2}{1+\mu}\\
    \end{cases}
\end{equation}
(8.4)式就是求$\Vec{P_1P_2}$的\textbf{定比分点坐标的计算公式}。当
$\mu=1$时,$P$点是$\overline{P_1P_2}$的中点,$P$点的坐标是
\begin{equation}
    \begin{cases}
        x=\frac{x_1+ x_2}{2}\\
        y=\frac{y_1+ y_2}{2}\\
        z=\frac{z_1+ z_2}{2}\\
    \end{cases}
\end{equation}
公式(8.5)又叫做中点公式。


\begin{example}
已知$A(-1,2,2)$, $B(-4,2,5)$, 点$P$按定比
$\mu=2$分割$\Vec{AB}$, 求$P(x,y,z)$.
\end{example}

\begin{solution}
由于$\mu=2$,因此:
\[\begin{split}
    x&=\frac{-1+2\x (-4)}{1+2}=-3\\
    y&=\frac{2+2\x 2}{1+2}=2\\
    z&=\frac{2+2\x 5}{1+2}=4\\
\end{split}\]    
即:$P(-3,2,4)$
\end{solution}

\begin{blk}{问题2}
    求向量长度和两点间距离公式。
\end{blk}

若$\vec{a}=(a_x,a_y,a_z)$, 则
\begin{align}
    |\vec{a}|^2&=\vec{a}\cdot \vec{a} =a^2_x+a^2_y+a^2_z\nonumber\\
|\vec{a}|&=\sqrt{a^2_x+a^2_y+a^2_z}
\end{align}
(8.6)式就是求\textbf{向量$\vec{a}$的长度的计算公式}。

若$A(x_1,y_1,z_1)$、$B(x_2,y_2,z_2)$,则:
\begin{equation}
   |\Vec{AB}|=\sqrt{(x_2-x_1)^2+(y_2-y_1)^2+(z_2-z_1)^2} 
\end{equation}
(8.7)式就是求空间任意\textbf{两点间的距离公式}。


\begin{blk}
    {问题3} 求一向量的方向余弦。
\end{blk}

若$\vec{a}=(a_x,a_y,a_z)$, $\alpha,\beta,\gamma$为$\vec{a}$的方向角,则:
\[a_x=|\vec{a}|\cos\alpha,\qquad a_y=|\vec{a}|\cos\beta,\qquad a_z=|\vec{a}|\cos\gamma\]
于是得:
\begin{equation}
    \begin{cases}
    \cos\alpha=\frac{a_x}{|\vec{a}|}=\frac{a_x}{\sqrt{a^2_x+a^2_y+a^2_z}}\\
    \cos\beta=\frac{a_y}{|\vec{a}|}=\frac{a_y}{\sqrt{a^2_x+a^2_y+a^2_z}}\\
    \cos\gamma=\frac{a_z}{|\vec{a}|}=\frac{a_z}{\sqrt{a^2_x+a^2_y+a^2_z}}\\        
    \end{cases}
\end{equation}
(8.8)式就是\textbf{向量$\vec{a}$的方向余弦的计算公式}。

把(8.7)式两边平方加起来,得
\[\cos^2\alpha+\cos^2\beta+\cos^2\gamma=1\]
这就是说,\textbf{任何一个向量的方向余弦的平方和恒等于1}.

若$\vec{a}=(a_x,a_y,a_z)$,则$\vec{a}$的单位向量
\[\begin{split}
    \vec{a}_0=\frac{\vec{a}}{|\vec{a}|}&=\left(\frac{a_x}{|\vec{a}|},\frac{a_y}{|\vec{a}|},\frac{a_z}{|\vec{a}|}\right)\\
    &=(\cos\alpha,\cos\beta,\cos\gamma)
\end{split}\]
这就是说,\textbf{空间任一向量的单位向量的坐标分量正好等于它
的方向余弦}。

\begin{example}
    求$\vec{a}=(2,-3,1)$的方向余弦和它的单位向
量$\vec{a}_0$的坐标。
\end{example}

\begin{solution}
由于:$|\vec{a}|=\sqrt{2^2+(-3)^2+1^2}=\sqrt{14}$

$\therefore\quad \cos\alpha=\frac{2}{\sqrt{14}},\qquad \cos\beta=\frac{-3}{\sqrt{14}},\qquad \cos\gamma=\frac{1}{\sqrt{14}}$
\[\vec{a}_0=\left(\frac{2}{\sqrt{14}},\frac{-3}{\sqrt{14}},\frac{1}{\sqrt{14}}\right)\]
\end{solution}

\begin{blk}
    {问题4}
求两个向量的夹角。
\end{blk}

如果$\vec{a}=(a_x,a_y,a_z)$, $\vec{b}=(b_x,b_y,b_z)$,那么
\begin{equation}
    \begin{split}
\cos\langle \vec{a},\vec{b}\rangle &=\frac{\vec{a}\cdot \vec{b}}{|\vec{a}||\vec{b}|}\\
&=\frac{a_xb_x+a_yb_y+a_zb_z}{\sqrt{a^2_x+a^2_y+a^2_z}\cdot \sqrt{b^2_x+b^2_y+b^2_z}}
    \end{split}
\end{equation}
公式(8.9)就是求向量夹角的计算公式。



\begin{example}
    已知$\vec{a}=(1,1,0)$, $\vec{b}=(1,0,1)$, 
求$\langle \vec{a},\vec{b}\rangle$.
\end{example}

\begin{solution}
\[\cos\langle \vec{a}\cdot \vec{b} \rangle=\frac{1\x 1+1\x 0+0\x 1}{\sqrt{1^2+1^2+0^2}\cdot \sqrt{1^2+1^2+0^2}}=\frac{1}{2} \]
$\therefore\quad \langle \vec{a},\vec{b}\rangle=\frac{\pi}{3}$    
\end{solution}

\begin{ex}
\begin{enumerate}
    \item 已知$A(3,5,-7)$, $B(-2,4,3)$, 点$P$按定
    比$\mu=-2$分割$\Vec{AB}$, 求$P$点的坐标。
    \item 已知$P(3,-4,1)$, $Q(0,2,-3)$, 点$A$按定
    比$\mu=2$分割$\Vec{QP}$, 求$A$点的坐标。
    \item 已知$A(0,-1,1)$, $B(2,1,-3)$, 求$\overline{AB}$中点的坐
    标。
    \item 已知$\triangle ABC$的两个顶点$A(-4,-1,2)$, $B(3,
    5,-16)$, $\overline{AC}$边的中点在$Y$轴上,
    $\overline{BC}$边的中点在
    $OZX$平面上,求第三顶点$C$的坐标。
    \item 已知$A(3,-1,0)$、$B(2,1,-3)$, 求$A$、$B$
    两点间的距离。
    \item 已知$|\vec{a}|=10$, $\langle \eX,\vec{a} \rangle=60^{\circ}$, $\langle \eY,\vec{a} \rangle=60^{\circ}$

    求$\langle \eZ,\vec{a} \rangle=60^{\circ}$和$\vec{a}$的坐标。
\item 已知$\vec{a}\parallel \vec{b}$, $|\vec{a}|=10$, $\vec{b}=(3,-3,3)$, 求$\vec{a}$的坐标。

\item 已知$\vec{a}=(-1,2,3)$, $\vec{b}=(2,5,4)$, 求$\langle \vec{a},\vec{b} \rangle$

\item 已知$\vec{a}=(2,-3,5)$, $\vec{b}=(-4,2,6)$, 求证
$\vec{a}\nparallel \vec{b}$.
\end{enumerate}
\end{ex}

\subsection{习题8.1}

\begin{enumerate}
    \item 如果向量$\vec{a}$、$\vec{b}$、$\vec{c}$分别平行于$X$轴、$Y$轴、$Z$轴,问
    它们的坐标各有什么特点?
    \item 
    如果$\vec{a}$的$x$坐标是0, 那么$\vec{a}$与哪个平面平行。
    \item 已知$\vec{a}=(2,-1,3)$、$\vec{b}=(-3,0,4)$, 求满足下
    列关系的向量$\vec{c}$的坐标。
\begin{multicols}{2}
\begin{enumerate}
    \item $3\vec{a}+2\vec{c}=\vec{b}$
    \item $\vec{a}-3\vec{c}=2\vec{b}$
    \item $\vec{a}-2\vec{c}=3\vec{b}-\vec{c}$
    \item $2(3\vec{a}-\vec{c})+\vec{b}=0$
\end{enumerate}
\end{multicols}
    \item 已知$A(2,-1,7)$, $B(4,5,-2)$, 求每个坐标平
    面分割$\Vec{AB}$的比值。
    \item 已知$A(2,3,6)$, $B(5,2,8)$, 直线$AB$上有$C$点
    使$B$点为$\overline{AC}$的中点,求$C(x,y,z)$.
    \item 已知$A=(x_1,y_1,z_1)$、$B=(x_2,y_2,z_2)$、
    $C=(x_3,y_3,z_3)$, 求$\triangle ABC$的重心。

    \item 已知$\vec{a}=(1,2,-2)$, $\vec{b}=(3,4,2)$,
    $\vec{c}=(-2,-4,4)$, 求证:$\vec{a}$、$\vec{b}$、$\vec{c}$线性相关。

    \item 已知$A(4,1,3)$, $B(2,-5,1)$, $C(3,7,-5)$. 求向量$\Vec{AB},\Vec{BA},\Vec{AC},\Vec{BC}$的坐标和长度
    (精确到0.01)。
    \item 已知$A(1,1,\sqrt{2})$, 求$\Vec{OA}$与三个坐标轴的夹角。
    \item 已知$\vec{a}=(-1,1,0)$, $\vec{b}=(1,-2,2)$,
    求$\langle \vec{a},\vec{b}\rangle$.
\end{enumerate}

\section{空间的平面~~直线与球面方程}

\subsection{空间的平面方程}

已知非零向量$\vec{n}=(a,b,c)$和定点$P_0(x_0,y_0,z_0)$,
过$P_0$点作平面$\pi$与$\vec{n}$垂直,求平面
$\pi$的方程。

\begin{figure}[htp]
    \centering
\begin{tikzpicture}[>=latex]
\draw[->](0,0)node[below]{$O$}--(4,0)node[right]{$Y$};
\draw[->](0,0)--(0,4)node[right]{$Z$};
\draw[->](0,0)--(-1,-1)node[right]{$X$};
\tkzDefPoints{.5/2.5/A, 3/2/B, -.5/1/A', 0/0/O}
\tkzDefPointsBy[translation = from A to A'](B){B'}
\tkzDrawPolygon[fill=white](A,B,B',A')
\tkzDefPoint(30:2.6){P}
\tkzDefPoint(60:2){P_0}
\tkzDrawSegments[dashed, ->](O,P O,P_0)
\tkzDrawSegments[->](P_0,P)
\tkzLabelPoints[above](P_0,P)
\tkzDefPointWith[linear, K=.5](O,P) \tkzGetPoint{P1}
\tkzDefPointWith[linear, K=.43](O,P_0) \tkzGetPoint{P2}


\draw[->, thick](.65,2.2)--node[left]{$\vec{n}$}+(72:1);

\tkzDrawSegments (O,P1 O,P2)
\end{tikzpicture}
    \caption{}
\end{figure}

设$P(x,y,z)$为平面$\pi$上一动点,因为
$\Vec{P_0P}\bot\vec{n}$, 所以
$\Vec{P_0P}\cdot \vec{n}=0$,
即:
\begin{equation}
    \left(\Vec{OP}-\Vec{OP_0}\right)\cdot\vec{n}=0
\end{equation}
反之,如果$P(x,y,z)$满足(8.10)式,那么$P$点一定在
平面$\pi$上,所以(8.10)式就是\textbf{平面$\pi$的向量方程}。

(8.10)式用坐标表示即可写为
\begin{equation}
    a(x-x_0)+b(y-y_0)+c(z-z_0)=0
\end{equation}
(8.11)式就叫做\textbf{平面的点法向式方程}。其中$\vec{n}=(a,b,c)$, 
叫做平面$\pi$的一条法线向量。

如果令$d=-(ax_0+by_0+cz_0)$, 那么(8.11)式又可
写为
\begin{equation}
    ax+by+cz+d=0
\end{equation}
方程(8.12)又叫做\textbf{平面的普通方程},其中$a,b,c$至少有
一个不为零。

显然,如果$\vec{n}=(a,b,c)$是平面$\pi$的一个法线向
量,那么对任何非零常数$k$, $k\vec{n}$也是$\pi$的法线向量。这
样,若取$k\vec{n}$作为平面的法线向量,则$\pi$的方程还可写为
\[k(ax+by+cz+d)=0\]
因此,同一个平面方程,仅仅相差一个常数因子。

由方程(8.12)可以看出,平面的方程是$x,y,z$的一
次方程;反之,如果设$(x_0,y_0,z_0)$是三元一次方程
$ax+by+cz+d=0$
的一个解,则
\[ax_0+by_0+cz_0+d=0\]
两式相减,得
\begin{equation}
    a(x-x_0)+b(y-y_0)+c(z-z_0)=0
\end{equation}
如果建立空间直角坐标系,作$\Vec{OP_0}=(x_0,y_0,z_0)$,
 $\vec{n}=(a,b,c)$, 那么(8.13)式就是通过$P_0$且垂直于$\vec{n}$
的一个平面方程,这就是说,\textbf{任何一个三元一次方程都表示
一个平面}。这样,在空间解析几何中,一个平面和一个三元
一次方程是同一码事。

由以上分析,我们还可得到一个结论,即,\textbf{任给一个平
面$\pi:\; ax+by+cz+d=0$, 其中$x,y,z$的系数向量  
$\vec{n}=(a,b,c)$是平面$\pi$的一个法线向量。
}

\begin{example}
    求通过点$P(2,-1,3)$且垂直于$\vec{n}=(2,-1,5)$
的平面方程。
\end{example}

\begin{solution}
    由平面的点法式方程,得所求平面方程为
\[2(x-2)+(-1)[y-(-1)]+5(z-3)=0\]
整理得
\[2x-y+5z-20=0\]
\end{solution}




\begin{example}
    已知$A(x_1,y_1,z_1)$, $B(x_2,y_2,z_2)$, 
$C(x_3,y_3,z_3)$三点不共线。求通过$A$、$B$、$C$的平面方
程。
\end{example}

\begin{solution}
    设$P(x,y,z)$为所求平面的一个动点,则$P$点
与$A$、$B$、$C$三点共面的充要条件是
\[\begin{vmatrix}
    x-x_1&y-y_1&z-z_1\\
    x_2-x_1&y_2-y_1&z_2-z_1\\
    x_3-x_1&y_3-y_1&z_3-z_1\\
\end{vmatrix}=0\]
这就是\textbf{通过$A$、$B$、$C$三点的平面方程},叫做平面方程的三
点式。
\end{solution}




\begin{example}
    求通过原点和两点$(2,0,1)$, $(0,1,3)$
的平面方程。
\end{example}

\begin{solution}
\textbf{方法1:} 由平面方程的三点式,得
\[\begin{vmatrix}
x-0&y-0&z-0\\    
2-0&0-0&1-0\\ 
0-0&1-0&3-0\\ 
\end{vmatrix}=0\]
展开化简,得
\[x+6y-2z=0\]
\textbf{方法2:}
设所求的平面方程为$ax+by+cz+d=0$, 
把已知三点的坐标,代入上面方程,得
\[\begin{cases}
    d=0\\
    2a+c=0\\
    b+3c=0
\end{cases}\]
解此方程组,得
\[a=-\frac{1}{2}c,\qquad b=-3c,\qquad d=0\]
所以,所求的平面方程为
\[\frac{1}{2}cx-3cy+cz=0\]
即:$x+6y-2z=0$.
\end{solution}

\begin{example}
    求通过点$(1,2,3)$且平行于平面
$2x+y-z+3=0$的平面方程。
\end{example}

\begin{solution}
    已知平面的一个法线向量是$\vec{n}=(2,1,-1)$, 
它与所求平面垂直,由平面的点法向式方程,得所求方程为
\[2(x-1)+1(y-2)+(-1)(z-3)=0\]
整理,得
\[2x+y-z-1=0\]
\end{solution}

\begin{example}
    求点$P_1(x_1,y_1,z_1)$到平面$\pi:\; ax+by+cz+d=0$的距离$d$(图8.6).
\end{example}

\begin{figure}[htp]
    \centering
\begin{tikzpicture}[>=latex, scale=1.2]
\draw(0,0)--(3,0)--(3.5,1)--(.5,1)--(0,0);
\draw[->, thick](3,.5)--(3,1.5)node[right]{$\vec{n}$};
\tkzDefPoints{1/0/A, 1.5/1/B, 1.9/.5/P_0}
\tkzDefPointWith[linear, K=1.6](A,B)  \tkzGetPoint{P_1}
\tkzDefPointWith[linear, K=1.6](B,A)  \tkzGetPoint{S}
\tkzDefPointWith[linear, K=0.6](S,P_0)  \tkzGetPoint{S1}
\tkzDrawSegments[dashed](A,B P_0,S1)
\tkzDrawSegments(P_1,B S,A P_1,P_0 S,S1)
\tkzLabelPoints[right](P_1,P_0)
\end{tikzpicture}
    \caption{}
\end{figure}

\begin{solution}
    过$P_1$作$P_1P_0$垂直平面$\pi$
于$P_0$点,则
\[d=|\Vec{P_0P_1}|\]
设$P_0$的坐标为$(x_0,y_0,z_0)$. 则
\[|\Vec{P_0P_2}|=|\Vec{P_0P_1} \cdot \vec{n}_0|\]
其中$\vec{n}_0$是$\pi$的单位法向量。换用坐标表示,即为
\[|\Vec{P_0P_1}|=\frac{a(x_1-x_0)+b(y_1-y_0)+c(z_1-z_0)}{\sqrt{a^2+b^2+c^2}}\]
因为$P_0\in \pi$, 所以
\[ax_0+by_0+cz_0+d=0\]
其中: $d=-(ax_0+by_0+cz_0)$, 代入上式,得
\[|\Vec{P_0P_1}|=\frac{ax_1+by_1+cz_1+d}{\sqrt{a^2+b^2+c^2}}\]
\[d=\frac{|ax_1+by_1+cz_1+d|}{\sqrt{a^2+b^2+c^2}}\]
\end{solution}

例8.9说明,如果要求一点到一平面的距离,只要把这
一点的坐标代入平面方程。取绝对值,再除以系数向量的长
度就可求出。

\begin{ex}
\begin{enumerate}
    \item  求三个坐标平面的方程。
    \item  求过点$A(1,2,-3)$, 以$\vec{n}=(1,-3,2)$为法
    线向量的方程。
    \item  求过点$P_0(x_0,y_0,z_0)$且垂直于$X$轴的平面方程。
    \item  已知两点$A(2,3,4)$, $B(-2,4,3)$, 求$\overline{AB}$
    的垂直平分面的方程。
    \item  求通过点$P_0(x_0,y_0,z_0)$且平行于$OXY$平面的方程。
    \item  证明方程$ax+by+cz=0$, 是通过原点的平面,其中
    $a$、$b$、$c$至少有一个不为零
    \item  求过原点和两点$(1,0,-1)$, $(0,2,3)$的平
    面方程。
    \item  求过点$(3,5,-2)$且平行于平面$2x-y+3z=0$
    的平面方程。
    \item  求点$(3,-2,5)$到平面$3x-4y-z+3=0$的距
    离。
\end{enumerate}
\end{ex}


\subsection{空间的直线方程}
已知,一定点$P_0(x_0,y_0,z_0)$
和一向量$\vec{a}=(a_1,a_2,a_3)$, 求过
$P_0$且平行于向量$\vec{a}$的直线方程。

\begin{figure}[htp]
    \centering
\begin{tikzpicture}[>=latex]
\draw[->](0,0)node[below]{$O$}--(3.5,0)node[right]{$Y$};
\draw[->](0,0)--(0,3)node[right]{$Z$};
\draw[->](0,0)--(-1.25,-1.25)node[right]{$X$};
\draw[domain=-1:2.5, samples=10, thick]plot(\x, {1.5-\x})node[right]{$\ell$};

\tkzDefPoints{0/0/O, 0.6/0.9/P_0, -.5/2/P}
\tkzDrawSegments[->, thick](O,P O,P_0)
\draw[thick,->](0,0)--node[below]{$\vec{a}$}(135:1);
\tkzLabelPoints[above](P,P_0)
\end{tikzpicture}
    \caption{}
\end{figure}

设$P(x,y,z)$是所求直线
$\ell$上一动点,则存在一实数$t$使
\[\Vec{P_0P}=t\vec{a},\qquad \Vec{OP}=\Vec{OP_0}+t\vec{a}\]
换用坐标表示,即为
\begin{equation}
    \begin{cases}
        x=x_0+a_1t\\
y = y_0+a_2t\\
z=z_0+a_3t
    \end{cases}
\end{equation}
(8.14)式叫做直线$\ell$的\textbf{参数方程}。$t$叫做\textbf{参数}。

如果$a_1,a_2,a_3$都不为零,从(8.14)式消去参数$t$, 得
\[\frac{x-x_0}{a_1}=\frac{y-y_0}{a_2}=\frac{z-z_0}{a_3}\]
(8.15)式叫做$\ell$的\textbf{点、方向式方程}又叫\textbf{对称式方程}。其中
$\vec{a}=(a_1,a_2,a_3)$叫做$\ell$的方向向量。如果取$\vec{a}$的单位向量
\[\vec{a}_0=(\cos\alpha, \cos\beta,\cos\gamma)\]
作为方向向量,则$\ell$的方程为
\[\frac{x-x_0}{\cos\alpha}=\frac{y-y_0}{\cos\beta}=\frac{z-z_0}{\cos\gamma}\]
$\cos\alpha, \cos\beta,\cos\gamma$又叫做有向\textbf{直线$\ell$的方向余弦}。

如果直线$\ell$通过两点$P_1(x_1,y_1,z_1)$, $P_2(x_2,y_2,z_2)$, 
则直线$\ell$的方向向量可取
\[\Vec{P_1P_2}=(x_2-x_1,y_2-y_1,z_2-z_1)\]
这时直线$\ell$的方程可写为
\begin{equation}
    \frac{x-x_1}{x_2-x_1}=\frac{y-y_1}{y_2-y_1}=\frac{z-z_1}{z_2-z_1}
\end{equation}
方程(8.16)一般叫做直线的\textbf{两点式方程}。



\begin{example}
    求通过$P_0(1,-1,2)$, 且和向量$\vec{a}=(2,3,1)$平
行的直线$\ell$的方程。
\end{example}


\begin{solution}
    由直线的对称式方程可得直线$\ell$的方程为
\[\frac{x-1}{2}=\frac{y+1}{3}=z-2\]
\end{solution}

\begin{example}
    求通过两个不同点$P_1(x_1,y_1,z_1)$, $P_2(x_2,
y_2,z_2)$的直线的参数方程。
\end{example}


\begin{solution}
    取直线$P_1P_2$的方向向量
\[    \Vec{P_1P_2}=(x_2-x_1, y_2-У_1,z_2-z_1)\]
    由直线的参数方程。可得直线$P_1P_2$的参数方程为
\[\begin{cases}
    x=x_1+(x_2-x_1)t\\
    y=y_1+(y_2-y_1)t\\
    z=z_1+(z_2-z_1)t
\end{cases}\]
\end{solution}    

\begin{example}
    已知$P_1(5,0,1)$, $P_2(5,6,4)$, 求直线
$P_1P_2$的参数方程。
\end{example}


\begin{solution}
    由例8.11, 可知直线$P_1P_2$的参数方程为
    \[\begin{cases}
     x=5\\
y=6t\\
z=1+3t\\
    \end{cases}\]
\end{solution}

在例8.12中,由于$\Vec{P_1P_2}=(0,6,3)$, 其中$x$坐标为
零,因此直线$P_1P_2$不能写为对称式方程,但确能用参数方
程来表达。由此可看到,直线的参数方程比较优越。


\begin{ex}
\begin{enumerate}
    \item 求通过点$P_0(-1,2,-3)$且平行于向量$\vec{s}=(2,3,-5)$
    的直线方程。
    \item 求通过$P_0(2,3,1)$, $P_1(-1,-2,3)$的直线方程。
    \item 求通过点$(2,3,1)$且和$X$轴平行的直线方程。
    \item 求过点$(2,-3,7)$, 其方向向量为$(2,0,3)$的直
    线方程。
    \item 求直线$2x-6=4-y=2-5$的方向向量。
    \item 求平行于两平面$x-2y+5z+2=0$和$3x+y-z
    +5=0$的交线,且通过原点的直线方程。
\end{enumerate}
\end{ex}


\subsection{球面方程}
空间一动点$P(x,y,z)$在以$A(a,b,c)$为球
心,$R$为半径的球面上的充要条件是
\[|\Vec{OP}-\Vec{OA}|=R\]
或
\[(\Vec{OP}-\Vec{OA})\cdot (\Vec{OP}-\Vec{OA})=R^2\]
换用坐标表示,条件可写为
\begin{equation}
    (x-a)^2+(y-b)^2+(z-c)^2=R^2
\end{equation}

\begin{figure}[htp]
    \centering
\begin{tikzpicture}[>=latex]
\draw[<->](0,4)node[right]{$Z$}--(0,0)node[below right]{$O$}--(3.5,0)node[right]{$Y$};
\draw[->](0,0)--(-1,-1)node[right]{$X$};
\tkzDefPoints{1.5/1.75/A, 0/0/O, 1/2.62/P}

\draw[thick](A) circle (1);
\draw[dashed](A) ellipse[x radius=1, y radius=.4];
\draw[thick](2.5,1.75) arc [x radius=1, y radius=.4,start angle =0, end angle =-180];
\tkzDrawSegments[dashed, ->](O,P A,P O,A)
\tkzLabelPoints[right](A)
\tkzLabelPoints[above](P)
\end{tikzpicture}
    \caption{}
\end{figure}


方程(8.16)就是以$A(a,b,c)$为球心,以$R$为半径的\textbf{球面
方程}。当$A$点在原点,球面方程变为
\[    x^2+y^2+z^2=R^2\]



\begin{ex}
\begin{enumerate}
    \item 求以$A(1,2,-2)$为球心,3为半径的球面方程。
    \item 求球心在原点,半径等于5的球面方程。
    \item 设一动点$Q$在以$A(0,4,0)$为球心,2为半径的球面
    上变动,求$\overline{OQ}$中点的轨迹。
\end{enumerate}
\end{ex}

\subsection*{习题8.2}
\begin{enumerate}
    \item 求满足以下条件的平面方程。
\begin{enumerate}
\item 通过点$P_0(5,3,4)$且垂直于向量$\vec{n}=(1,1,1)$;
\item 通过坐标原点且垂直于$\vec{n}=\left(-\frac{1}{3},\frac{2}{3},-\frac{2}{3}\right)$;
\item 垂直于$\vec{n}=\left(\frac{1}{2},\frac{\sqrt{3}}{2},0\right)$且与原点的距离等于5.
\end{enumerate}

    \item 说出如下方程表示的平面的几何特征。
\begin{multicols}{3}
\begin{enumerate}
    \item $x=2$ \item $x=y$ \item $x+y+z=1$
\end{enumerate}
\end{multicols}

    \item 求证通过三点$A(a,0,0)$, $B(0,b,0)$, $C(0,
    0,c)$的平面方程为
\[\frac{x}{a}+\frac{y}{b}+\frac{z}{c}=1\] 

\item 如图,试写出长方体$ABC
D-A'B'C'D'$的各侧面,底面的平面方程以及各
棱所在的直线方程。已知
$\overline{AB}=a$, $\overline{AD}=b$, 
$\overline{AA'}=c$.

\begin{figure}[htp]
    \centering
\begin{tikzpicture}[>=latex, scale=1.3]
    \tkzDefPoints{0/0/A, -.5/-.5/B, 1/-.5/C, 1.5/0/D, 0/2/A'}
    \tkzDefPointsBy[translation= from A to A'](B,C,D){B',C',D'}
    \tkzDrawPolygon[thick](A',B',C',D')
    \tkzDrawPolygon[dashed](A,B,C,D)
    \tkzDrawSegments[thick](B,C C,D B,B' C,C' D,D')
    \tkzDrawSegments[dashed](A,B A,D  A,A')
    \draw[->](B)--(-1,-1)node[left]{$X$};
    \draw[->](D)--(2.5,0)node[right]{$Y$};
    \draw[->](A')--(0,3)node[right]{$Z$};
    \node at (0,0)[left]{$O$};
\tkzLabelPoints[below](B,C,A,D)
\tkzLabelPoints[left](B',A')
\tkzLabelPoints[right](C',D')
\end{tikzpicture}
    \caption*{第4题}
\end{figure}


\item 分别求两点$P_1(3,9,1)$, $P_2(4,1,5)$到平面$S:\;
x-2y+2z-3=0$的距离。
\item 求两条直线
\[\begin{split}
    \ell_1:&\quad \frac{x-1}{3}=\frac{y+2}{6}=\frac{z-5}{2}\\
    \ell_2:&\quad \frac{x}{2}=\frac{y-3}{9}=\frac{z+1}{6}\\
\end{split}\]
的夹角。
\item 求通过点$(1,-1,2)$并与已知平面:$x+y+z=1$
垂直的直线方程。并求这条直线与平面交点的坐标。
\item 在直线$\ell:\; x=1+2t,\; y=8+t,\; z=8+3t$上求
一点使它和原点的距离等于35.
\item 求满足下列条件的球面方程。
\begin{multicols}{2}
    \begin{enumerate}
    \item 球心在$(-2,3,-6)$, 半径是7
    \item 球心在$(4,0,0)$, 半径是2
    \item 球心在$(0,-4,3)$, 半径是5
    \item 球心在$(0,-5,0)$, 半径是2
    \item 球心在$\left(\frac{2}{3},-\frac{1}{3},0\right)$,
半径是1
\end{enumerate}
\end{multicols}

\item 求球面:$x^2+y^2+z^2+4x-6y-2z+5=0$的球心和
半径。
\end{enumerate}

\section*{复习题八}
\begin{enumerate}
\item 已知点$P(3,-1,2)$和$M(a,b,c)$, 求$P$、$M$两点
分别关于坐标平面、坐标轴以及原点的对称点的坐标。
\item 求点$P(2,5,6)$到坐标原点以及三条坐标轴的距离。
\item 已知$\Vec{OA}=(6,2,9)$,求$\Vec{OA}$与三个坐标平面的夹角.
\item 已知$A(-2,1,3)$, $B(0,-1,2)$,求与$A$、$B$两点
距离相等点的轨迹方程。
\item 已知$A(a,0,0)$, $B(0,b,0)$, $C(0,0,c)$, 原
点到平面$(A,B,C)$的距离为$d$, 求证
\[\frac{1}{a^2}+\frac{1}{b^2}+\frac{1}{c^2}=\frac{1}{d^2}\]
\item 在$Z$轴上求一点,使得到$A(-4,1,7)$, $B(3,5,-2)$两点的距离相等。
\item 已知四面体$ABCD$,且$A(x_1,y_1,z_1)$, $B(x_2,
y_2,z_2)$, $C(x_3,y_3, z_3)$, $D(x_4, y_4,z_4)$, 
求它的重心的坐标。
\item 已知$\vec{a}=(a_1,a_2,a_3)$, $\vec{b}=(b_1,b_2,b_3)$, 求证:以$\vec{a}$, $\vec{b}$为邻边的平行四边形面积
\[S=\sqrt{\begin{vmatrix}
    a_2&a_3\\b_2&b_3
\end{vmatrix}^2+\begin{vmatrix}
    a_3&a_1\\b_3&b_1
\end{vmatrix}^2+\begin{vmatrix}
    a_1&a_2\\b_1&b_2
\end{vmatrix}^2}\]

\item 已知点$P(1,3,5)$和$\vec{a}(-2,1,1)$. 求通过$P$点
具有方向$\vec{a}$的直线与平面$2x+3y-z=1$的交点。

\item 求通过如下三点的平面方程。
\begin{enumerate}
    \item $(2,1,1),\qquad (3,-1,1),\qquad (4,1,-1)$
    \item $(-2,3,-1),\qquad (2,2,3),\qquad (-4,-1,1)$
    \item $(-5,-1,-2),\qquad (1,2,-1),\qquad (3,-1,2)$
\end{enumerate}

\item 求下列通过已知点$P$且以$\vec{a}$为方向向量的直线方程。
\begin{multicols}{2}
    \begin{enumerate}
    \item $P(2,1,3),\quad \vec{a}=(1,1,-2)$
    \item $P(-5,3,4),\quad \vec{a}=(-2,2,1)$
    \item $P(4,-3,2),\quad \vec{a}=(5,0,3)$
    \item $P(0,0,0),\quad \vec{a}=(2,-3,5)$
    \item $P(a,b,c),\quad \vec{a}=(\ell,m,n)$
\end{enumerate}
\end{multicols}

\end{enumerate}





\end{document}


\begin{figure}[htp]\centering
  \begin{minipage}[t]{0.48\textwidth}
  \centering
\begin{tikzpicture}[>=latex, scale=1]
     
  \end{tikzpicture}
  \caption{}
  \end{minipage}
  \begin{minipage}[t]{0.48\textwidth}
  \centering
  \begin{tikzpicture}[>=latex, scale=1]
    
  \end{tikzpicture}
  \caption{}
  \end{minipage}
  \end{figure}

