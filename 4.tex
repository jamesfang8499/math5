\chapter{向量几何初步}
在第三章,我们学习了向量运算与运算律,这一章我们
要用向量代数的方法来研究几何学,把对几何学的研究推进
到有效能算的定量的水平.

\section{平行与相似}
\subsection{直线的向量方程}
\begin{figure}[htp]
    \centering
\begin{tikzpicture}[>=latex, yscale=1.5]
    \tkzDefPoints{2/.5/A, .5/2/P, 0/0/O}
    \tkzDefMidPoint(A,P)
    \tkzGetPoint{B}
\draw[<->,  thick](P)--(O)--(A);
\draw[->,  thick](O)--(B);
\tkzDrawLines[add =.25 and .25](A,P)
\tkzLabelPoints[right](A,B,P)
\tkzLabelPoint[below](O){$O$}
\end{tikzpicture}
    \caption{}
\end{figure}

给定空间任意两点$A$、$B$(图4.1), 由平行向量基本定理可知,对空
间中一点$P$与$A$、$B$两点共线的充要条件是存在一实数$t$, 使
\[\Vec{AP}=t\Vec{AB}\]
对空间任一点$O$, 这个条件还可
写为
\[\Vec{OP}-\Vec{OA}=t\left(\Vec{OB}-\Vec{OA}\right)\]
\begin{equation}
    \Vec{OP}=(1-t)\Vec{OA}+t\Vec{OB}
\end{equation}
这就是说,如果$P$点在直线$AB$上,则一定存在实数$t$使(4.1)
式成立,反之,任给一实数$t$, 由等式(4.1)所确定的$P$点也
一定在直线$AB$上,方程(4.1)通常叫做\textbf{直线$AB$的向量方
程}.其中\textbf{参数}$t$的几何意义是,$|t|=|\Vec{AP}|:|\Vec{AB}|$, 当$P$点在
射线$AB$上,$t\ge 0$. 当$P$在射线$AB$的反向延长线上,$t<0$.

由直线$AB$的向量方程(4.1)可知,如果
\[\Vec{OP}=x\Vec{OA}+y\Vec{OB}\]
那么点$P$在直线$AB$上的充要条件是$x+y=1$

\begin{example}
    已知$\vec{a},\vec{b}$是两个线性无关的向量,
$\Vec{OA}=\alpha\vec{a}$, $\Vec{OB}=\beta\vec{b}\quad (\alpha\ne 0,\;\beta\ne 0)$,$\Vec{OC}=x\vec{a}+y\vec{b}$(图4.2),
则$A$、$B$、$C$三点共线的充要条件是
\[\frac{x}{\alpha}+\frac{y}{\beta}=1\]
\end{example}

\begin{figure}[htp]
    \centering
    \begin{tikzpicture}[>=latex]
  \draw(0,0)--(-40:4);
\draw[->](0,-3)node[below]{$O$}--(-40:.5)node[right]{$B$}node[below left]{$\beta\vec{b}$};
\draw[->](0,-3)--(-40:1.5)node[right]{$C$};
\draw[->](0,-3)--(-40:2.5)node[right]{$A$}node[below]{$\alpha\vec{a}$};  

\tkzDefPoint(0,-3){O}
\tkzDefPoint(-40:.5){B}
\tkzDefPoint(-40:2.5){A}
\tkzDefMidPoint(O,B) \tkzGetPoint{B1}
\tkzDefMidPoint(O,A) \tkzGetPoint{A1}
\draw[->](O)--(A1)node[right]{$\vec{a}$};
\draw[->](O)--(B1)node[left]{$\vec{b}$};

\end{tikzpicture}
    \caption{}
\end{figure}

\begin{proof}
    由直线的向量方程可知,点$C$在直线$AB$上的充要条
    件是存在实数$t$使
\[\Vec{OC}=(1-t)\Vec{OA}+t\Vec{OB}\]
即\[\Vec{OC}=(1-t)\alpha\vec{a}+t\beta\vec{b}\]
但已知$\Vec{OC}=x\vec{a}+y\vec{b}$且$\vec{a}\nparallel\vec{b}$,所以
\[x=(1-t)\alpha,\qquad y=t\beta\]
消去$t$则可得$A$、$B$、$C$三点共线的充要条件为
\[\frac{x}{\alpha}+\frac{y}{\beta}=1\]
\end{proof}

\begin{example}
    如图4.3, 设$O$、$A$、$B$三点不共线,
$\Vec{OA}=\vec{a}$, $\Vec{OB}=\vec{b}$, $\Vec{OA_1}=\alpha_1\vec{a}$, $\Vec{OA_2}=\alpha_2\vec{a}$, $\Vec{OB_1}=\beta_1\vec{b}$, $\Vec{OB_2}=\beta_2\vec{b}$
且$\alpha_1$、$\alpha_2$、$\beta_1$、$\beta_2$都不为零,又设$A_1B_2$与$A_2B_1$交于$C$点,试以$\vec{a}$、$\vec{b}$、$\alpha_1$、$\alpha_2$、$\beta_1$、$\beta_2$, 表示$\vec{c}=\Vec{OC}$
\end{example}

\begin{figure}[htp]
    \centering
\begin{tikzpicture}[>=latex]
\tkzDefPoints{0/0/O, 1/0/A, 2/0/A_1,  4.5/0/A_2}
\tkzDefPoint(45:1){B}
\tkzDefPoint(45:2.5){B_1}\tkzDefPoint(45:4.5){B_2}
\tkzDrawSegments[->, thick](O,B_1 O,B_2  O,A_1 O,A_2 A_1,B_2 B_1,A_2)
\tkzInterLL(A_1,B_2)(B_1,A_2) \tkzGetPoint{C}
\tkzDrawSegments[->, thick](O,C)
\tkzLabelPoints[below](A,A_1,A_2)
\tkzLabelPoints[above](B,B_1,B_2)
\tkzLabelPoints[right](C)
\tkzLabelPoints[left](O)
\tkzDrawPoints(A,B)
\end{tikzpicture}
    \caption{}
\end{figure}


\begin{solution}
    设$\vec{c}=x\vec{a}+y\vec{b}$, 因为$A_1$、$C$、$B_2$三点共线,$A_2$、$C$、$B_1$三点共线,由例4.1有方程组
\[\begin{cases}
    \frac{x}{\alpha_1}+\frac{y}{\beta_2}=1\\
    \frac{x}{\alpha_2}+\frac{y}{\beta_1}=1\\
\end{cases}\]
由于$A_1B_2$与$A_2B_1$相交于$C$点,可知$\alpha_1\beta_1-\alpha_2\beta_2\ne 0$, 解这个方程组可得
\[x=\alpha_1\alpha_2\frac{\beta_2-\beta_1}{\alpha_2\beta_2-\alpha_1\beta_1},\qquad y=\beta_1\beta_2\frac{\alpha_2-\alpha_1}{\alpha_2\beta_2-\alpha_1\beta_1}\]
\end{solution}

\begin{ex}
\begin{enumerate}
    \item 已知$\vec{a}$、$\vec{b}$线性无关,$\Vec{OA}=\vec{a}$, $\Vec{OB}=\vec{b}$, $\Vec{AP_1}=\frac{1}{3}\Vec{AB}$,
    $\Vec{AP_2}=\frac{1}{2}\Vec{AB}$, $\Vec{AP_3}=\frac{3}{2}\Vec{AB}$,
    试用向量$\vec{a}$、$\vec{b}$表示
    $\Vec{OP_1},\Vec{OP_2},\Vec{OP_3}$.

    \item 已知$\vec{a}$、$\vec{b}$线性无关,$\Vec{OA}=\vec{a}$, $\Vec{OB}=\vec{b}$, $P$点满足
    $\Vec{AP}=\mu \Vec{PB}\;  (\mu\in\mathbb{R})$, $P$点叫做$AB$的\textbf{定比分点}.求证:
\[\Vec{OP}=\frac{1}{1+\mu}\vec{a}+\frac{\mu}{1+\mu}\vec{b}\]
    \item 已知$\vec{a}$、$\vec{b}$线性无关,$\Vec{OA}=\vec{a}$, $\Vec{OB}=\vec{b}$,$\Vec{AP_1}=\frac{1}{2}\Vec{P_1B}$,
    $\Vec{AP_2}=-\frac{1}{2}\Vec{P_2B}$, $\Vec{AP_3}=-\frac{3}{2}\Vec{P_3B}$, 
    试用向量$\vec{a}$、$\vec{b}$表
    示$\Vec{OP_1},\Vec{OP_2},\Vec{OP_3}$.
    \item 已知$\vec{a}$、$\vec{b}$不平行且$\Vec{OP_1}=
    x_1\vec{a}+y_1\vec{b}$, $\Vec{OP_2}=x_2\vec{a}+y_2\vec{b}$, 
    $P$点在直线$P_1P_2$上且以$k$为比值定比分割$\Vec{P_1P_2}$,
    若$\Vec{OP}=x\vec{a}+y\vec{b}$, 
    试用$k$、$x_1$、
    $x_2$、$y_1$、$y_2$表示$x$、$y$.
    \item 如果$\Vec{OA}=\vec{a}$, $\Vec{OB}=\vec{b}$, $\Vec{OC}=\vec{c}$, 那么,$A$、$B$、
    $C$三点共线的充要条件是存在三个不全为零的实数$\alpha$、
    $\beta$、$\gamma$使
  \[  \alpha\vec{a}+\beta\vec{b}+\gamma\vec{c}=0\quad \text{且}\quad \alpha+\beta+\gamma=0\]
    (提示:应用直线的向量方程(4.1))
\end{enumerate}
\end{ex}

\subsection{几何证明举例}
 在上一章我们已详细的分析了平
行、相似与向量的加法、倍积运算之间的密切关系,下面我们
举例说明向量加法与倍积运算在几何证题中的应用.


\begin{example}
    证明三角形中位线定理.

已知:在$\triangle ABC$中,$D$、$E$分别是$\overline{AB}$、$\overline{AC}$的中点.

求证:$DE\parallel BC$, 且$\overline{DE}=\frac{1}{2}\overline{BC}$ (图4.4)
\end{example}

\begin{figure}[htp]
    \centering
\begin{tikzpicture}[>=latex, scale=.7]
\tkzDefPoints{0/0/B, 4/0/C, 3/4/A}
\tkzDefMidPoint(A,B) \tkzGetPoint{D}
\tkzDefMidPoint(A,C) \tkzGetPoint{E}
\tkzDrawSegments[->, thick](A,B A,C D,E B,C)
\tkzLabelPoints[below](B,C)
\tkzLabelPoints[left](D)
\tkzLabelPoints[right](E)
\tkzLabelPoints[above](A)
\end{tikzpicture}
    \caption{}
\end{figure}

\begin{proof}
    因$D$、$E$分别是$\overline{AB}$、$\overline{AC}$的中点,所以
\[\Vec{AD}=\frac{1}{2} \Vec{AB},\qquad \Vec{AE}=\frac{1}{2}\Vec{AC}\]
\[\Vec{DE}=\Vec{AE}-\Vec{AD}=\frac{1}{2}\left(\Vec{AC}-\Vec{AB}\right)=\frac{1}{2}\Vec{BC}\]
即:$DE\parallel BC$,  $\overline{DE}=\frac{1}{2}\overline{BC}$
\end{proof}

由例4.3的证明,大致可以看出用向量运算证明几何题的
主要步骤:
\begin{enumerate}
\item 选择基底向量
$\Vec{AB}$、$\Vec{AC}$, 
把已知条件($D$、$E$
是$\Vec{AB}$、$\Vec{AC}$的中点)写为向量形式($\Vec{AD}=\frac{1}{2}\Vec{AB}$, $\Vec{AE}=\frac{1}{2}\Vec{AC}$)
\item 进行向量运算,算出结果($\Vec{DE}=\frac{1}{2}\Vec{BC}$)
\item 把结果转化为几何结论.
\end{enumerate}

对例4.3的进一步分析,我们还会看到,证明中应用了倍
积分配律和向量平行的条件,这正好与几何中应用平行四边
形定理(或相似形定理)相对应.

\begin{example}
 已知五边形$ABCDE$, $M$、$N$、$P$、$Q$分别是$\overline{AB}$、
 $\overline{CD}$、$\overline{BC}$、$\overline{DE}$的中点,$K$、$L$是$\overline{MN}$与$\overline{PQ}$的中点,求
证:$\overline{KL}=\frac{1}{4}\overline{AE}$ 且$KL\parallel AE$(图4.5).
\end{example}

\begin{proof}
    在平面上任选一点$O$作为基点,则
\[\begin{split}
    \overline{KL}&=\overline{OL}-\overline{OK}=\frac{1}{2}\left(\overline{OP}+\overline{OQ}\right)-\frac{1}{2}\left(\overline{OM}+\overline{ON}\right)\\
&=\frac{1}{2}\left[\frac{1}{2}\left(\overline{OB}+\overline{OC}\right)+\frac{1}{2}\left(\overline{OD}+\overline{OE}\right)\right]-\frac{1}{2}\left[\frac{1}{2}\left(\overline{OA}+\overline{OB}\right)+\frac{1}{2}\left(\overline{OC}+\overline{OD}\right)\right]\\
&=\frac{1}{4}\left(\overline{OB}+\overline{OC}+\overline{OD}+\overline{OE}\right)-\frac{1}{4}\left(\overline{OA}+\overline{OB}+\overline{OC}+\overline{OD}\right)\\
&=\frac{1}{4}\left(\overline{OE}-\overline{OA}\right)=\frac{1}{4}\overline{AE}
\end{split}\]
即:$KL\parallel AE$且$\overline{KL}=\frac{1}{4}\overline{AE}$
\end{proof}

在例4.4中,基点选为任一点
$O$, 这样对题中各点的位置向量
表达就比较对称,计算起来就较为方便.

\begin{figure}[htp]\centering
    \begin{minipage}[t]{0.48\textwidth}
    \centering
\begin{tikzpicture}[>=latex, scale=1]
\tkzDefPoints{0/0/C, 2/0/D, 1.9/1.7/E, 0/2.5/A, -1/1.2/B, 1.3/3/O}
\tkzDefMidPoint(A,B) \tkzGetPoint{M}
\tkzDefMidPoint(C,B) \tkzGetPoint{P}
\tkzDefMidPoint(C,D) \tkzGetPoint{N}
\tkzDefMidPoint(D,E) \tkzGetPoint{Q}
\tkzDefMidPoint(M,N) \tkzGetPoint{K}
\tkzDefMidPoint(P,Q) \tkzGetPoint{L}

\tkzDrawPolygon(A,B,C,D,E)
\tkzDrawSegments(M,N P,Q)
\tkzDrawSegments[->](K,L)
\tkzDrawSegments[dashed, ->](O,K O,L)

\tkzLabelPoints[left](A,B,M,K,P)
\tkzLabelPoints[right](E,Q)
\tkzLabelPoints[below](C,N,D,L)
\tkzLabelPoints[above](O)

    \end{tikzpicture}
    \caption{}
    \end{minipage}
    \begin{minipage}[t]{0.48\textwidth}
    \centering
    \begin{tikzpicture}[>=latex, scale=1]
\tkzDefPoints{0/0/B, 3/0/C, 2.2/2.2/A}
\tkzDefPointWith[linear, K=.6](A,B) \tkzGetPoint{E}
\tkzDefPointWith[linear, K=.6](C,A) \tkzGetPoint{F}
\tkzDefPointWith[linear, K=.6](C,B) \tkzGetPoint{D}
\tkzDrawSegments[->, thick](A,D A,E A,F)
\tkzDrawSegments(E,B E,D F,D F,C B,C)
\tkzLabelPoints[below](B,D,C)
\tkzLabelPoints[left](E)
\tkzLabelPoints[right](F)
\tkzLabelPoints[above](A)
    \end{tikzpicture}
    \caption{}
    \end{minipage}
    \end{figure}

\begin{example}
    已知$\triangle ABC$, $D$是$\overline{BC}$上任一点,$\overline{DE}\parallel \overline{CA}$, 
$\overline{DF}\parallel\overline{BA}$, 

求证:$\frac{\overline{ED}}{\overline{AC}}+\frac{\overline{FD}}{\overline{AB}}=1$ (图4.6).
\end{example}

\begin{proof}
设$\frac{\overline{ED}}{\overline{AC}}=x$, $\frac{\overline{FD}}{\overline{AB}}=y$,则:
\[\Vec{ED}=x\Vec{AC},\qquad \Vec{FD}=y\Vec{AB}\]
由求和法则可得
\[\Vec{AD}=\Vec{AE}+\Vec{AF}=\Vec{ED}+\Vec{FD}=x\Vec{AC}+y\Vec{AB}\]
又因$D$在直线$BC$上,所以$x+y=1$,即:
$$\frac{\overline{ED}}{\overline{AC}}+\frac{\overline{FD}}{\overline{AB}}=1$$
\end{proof}

例4.5中的结论,实际上就是一点在直线上的一个必要条
件,换用向量表达式就一目了然了,应注意,题中若设$D$点
是直线$BC$上任一点,结论同样成立.


\begin{example}
    已知$\parallelogram ABCD$, $M$是$\overline{AB}$的中点,$\overline{DM}$交对角线
$\overline{AC}$于$H$点,

求证:
$\overline{AH}=\frac{1}{3} \overline{AC},\qquad \overline{MH}=\frac{1}{3} \overline{MD}$ (图4.7).
\end{example}

\begin{proof}
设$\Vec{AH}=x\Vec{AC}$,$\Vec{MH}=y\Vec{MD}$,则
\[\begin{split}
    \Vec{AH}&=X\Vec{AC}=x\left(\Vec{AB}+\Vec{AD}\right)=x\Vec{AB}+x\Vec{AD}\\
    \Vec{AH}&=(1-y)\Vec{AM}+y\Vec{AD}=\frac{1}{2}(1-y)\Vec{AB}+y\Vec{AD}
\end{split}\]
由于$\Vec{AB}$、$\Vec{AD}$线性无关,所以有
\[\begin{cases}
    2x+y=1\\
    y=x
\end{cases}\]
解方程组可得:$x=\frac{1}{3},\quad y=\frac{1}{3}$

所以:
\[\Vec{AH}=\frac{1}{3}\Vec{AC},\qquad \Vec{MH}=\frac{1}{3}\Vec{MD}\]
\[\overline{AH}=\frac{1}{3}\overline{AC},\qquad \overline{MH}=\frac{1}{3}\overline{MD}\]
\end{proof}

例4.6中,证明的关键是设未知数,根据已知条件,用基
向量$\Vec{AB}$, $\Vec{AD}$写出$\Vec{AH}$的两个表达式,然后由基向量的线
性无关性转化为方程组来求解.这和在代数中设未知数列方
程的解问题的方法相似.

\begin{figure}[htp]\centering
    \begin{minipage}[t]{0.48\textwidth}
    \centering
\begin{tikzpicture}[>=latex, scale=1]
\tkzDefPoints{0/0/A, 3/0/B, 1/2/D, 4/2/C}
\tkzDefMidPoint(A,B)   \tkzGetPoint{M}
\tkzDrawSegments[->](A,D A,M A,B A,C M,D)
\tkzDrawSegments(B,C D,C)
\tkzLabelPoints[below](A,M,B)
\tkzLabelPoints[above](C,D)
\tkzInterLL(A,C)(D,M) \tkzGetPoint{H}
\tkzLabelPoints[above](H))
\tkzDrawSegments[->](M,H A,H)
    \end{tikzpicture}
    \caption{}
    \end{minipage}
    \begin{minipage}[t]{0.48\textwidth}
    \centering
    \begin{tikzpicture}[>=latex, scale=1]
\tkzDefPoints{0/0/B, 3/0/C, 2/2.6/A}
\tkzDefMidPoint(A,B)   \tkzGetPoint{F}
\tkzDefMidPoint(A,C)   \tkzGetPoint{E}
\tkzDefMidPoint(C,B)   \tkzGetPoint{D}

\tkzDrawPolygon(A,B,C)
\tkzDrawSegments[->, thick](A,D B,E C,F)
\tkzInterLL(A,D)(B,E) \tkzGetPoint{G}

\tkzLabelPoints[below](D,C,B)
\tkzLabelPoints[above](A,G)
\tkzLabelPoints[left](F)
\tkzLabelPoints[right](E)
    \end{tikzpicture}
    \caption{}
    \end{minipage}
    \end{figure}


\begin{example}
    已知$\triangle ABC$, 证明:三条中线$\overline{AD}$、$\overline{BE}$、$\overline{CF}$相交于一点$G$且$\overline{AG}=\frac{2}{3} \overline{AD}$, $\overline{BG}=\frac{2}{3}\overline{BE}$, $\overline{CG}=\frac{2}{3}\overline{CF}$ (图4.8).
\end{example}

\begin{proof}
    设$\overline{AD},\overline{BE}$相交于
    $G$点,$\Vec{AG}=x\Vec{AD}$, $\Vec{BG}=y\Vec{BE}$,$\Vec{AB}=a$, $\Vec{AC}=\vec{b}$,则:
\[\Vec{AG}=x\Vec{AD}=x\x \frac{1}{2}(\vec{a}+\vec{b})=\frac{1}{2}x\vec{a}+\frac{1}{2}x\vec{b}\]
又    
\[\Vec{AG}=(1-y)\Vec{AB}+y\Vec{AE}=(1-y)\vec{a}+\frac{1}{2}y\vec{b}\]
由$\vec{a},\vec{b}$线性无关,得:
\[\begin{cases}
    \frac{1}{2}x=1-y\\
    \frac{1}{2}x=\frac{1}{2}y
\end{cases}\]
解之得:$x=\frac{2}{3},\quad y=\frac{2}{3}$,所以
\[\Vec{AG}=\frac{2}{3}\Vec{AD},\qquad \Vec{BG}=\frac{2}{3}\Vec{BE}\]
\[\overline{AG}=\frac{2}{3}\overline{AD},\qquad \overline{BG}=\frac{2}{3}\overline{BE}\]

如果$\overline{BE},\overline{CF}$相交于$G'$点,那么同样可证,
\[\overline{BG'}=\frac{2}{3} \overline{BE},\qquad \overline{CG'}=\frac{2}{3}\overline{CF}\]
于是$G$点与$G'$点重合,题中结论得证.
\end{proof}

\begin{example}
    已知三棱锥$S-ABC$, $K_1$、$L_1$、$M_1$分别是侧棱
    $\overline{SA}$、$\overline{SB}$、$\overline{SC}$的中点,$K_2$、$L_2$、$M_2$分别是$\overline{BC}$、$\overline{CA}$、
    $\overline{AB}$的中点,求证:$\overline{K_1K_2}$, $\overline{L_1L_2}$, $\overline{M_1M_2}$相交于一点,并在
    这点互相平分(图4.9).
\end{example}

\begin{proof}
设$\Vec{SA}=\vec{a}$, $\Vec{SB}=\vec{b}$, $\Vec{SC}=\vec{c}$, $O_1$为$\overline{K_1K_2}$的中点,则
\[\Vec{SO_1}=\frac{1}{2}(\Vec{SK_1}+\Vec{SK_2})=\frac{1}{2}\left[\frac{1}{2}\vec{a}+\frac{1}{2}(\vec{b}+\vec{c})\right]
=\frac{1}{4}\left(\vec{a}+\vec{b}+\vec{c}\right)\]
取$L_1L_2$的中点$O_2$, $M_1M_2$的中点$O_3$, 同理可证
\[\Vec{SO_2}=\Vec{SO_3}=\frac{1}{4}\left(\vec{a}+\vec{b}+\vec{c}\right)\]
因此三点$O_1$、$O_2$、$O_3$必重合于一点$O$并在$O$点互相平分.
\end{proof}

\begin{figure}[htp]\centering
    \begin{minipage}[t]{0.48\textwidth}
    \centering
\begin{tikzpicture}[>=latex, scale=.8]
\tkzDefPoints{-.5/0/A, 5.5/0/C, 3.5/3/S, 3.5/-1/B}
\tkzDrawPolygon(A,B,C,S)

\tkzDrawSegments(S,B)
\tkzDefMidPoint(S,A) \tkzGetPoint{K_1}
\tkzDefMidPoint(S,B) \tkzGetPoint{L_1}
\tkzDefMidPoint(S,C) \tkzGetPoint{M_1}

\tkzDefMidPoint(B,C) \tkzGetPoint{K_2}
\tkzDefMidPoint(C,A) \tkzGetPoint{L_2}
\tkzDefMidPoint(B,A) \tkzGetPoint{M_2}

\tkzDefMidPoint(M_1,M_2) \tkzGetPoint{O_1}
\tkzDrawSegments[dashed](M_1,M_2 K_1,K_2 A,C L_1,L_2)

\tkzLabelPoints[below](A,B,C,M_2,K_2,L_2)
\tkzLabelPoints[above](M_1,K_1,L_1,S,O_1)
    \end{tikzpicture}
    \caption{}
    \end{minipage}
    \begin{minipage}[t]{0.48\textwidth}
    \centering
    \begin{tikzpicture}[>=latex, scale=1]
\tkzDefPoints{0/0/O'}
\tkzDefPoint(180-15:3.5){A'}\tkzDefPoint(180-15:2.5){A}
\tkzDefPoint(180:5){C'}\tkzDefPoint(180:1.8){C}
\tkzDefPoint(180+15:4.2){B'}\tkzDefPoint(180+15:2.2){B}

\tkzDrawPolygon[pattern=north east lines](A,B,C)
\tkzDrawPolygon[pattern=north west lines](A',B',C')
\tkzLabelPoints[left](A',B',C')
\tkzLabelPoints[right](A,B,C)

\tkzInterLL(A',B')(A,B) \tkzGetPoint{P}
\tkzInterLL(A',C')(A,C) \tkzGetPoint{R}
\tkzInterLL(C',B')(C,B) \tkzGetPoint{Q}
\tkzLabelPoints[right](O')
\tkzLabelPoints[above](P,R)
\tkzLabelPoints[below](Q)
\tkzDrawSegments(Q,B' Q,B Q,P P,A' P,A R,A' R,A)
\tkzDrawSegments[dashed](O',B' O',A' O',C')

    \end{tikzpicture}
    \caption{}
    \end{minipage}
    \end{figure}
    
\begin{example}
    试证 Desargues 定理:如图4.10, 设$\triangle ABC$与$\triangle A'B'C'$的顶点连
    线$AA'$、$BB'$、$CC'$相交于一点$O$, 则对应边$AB$与$A'B'$、
    $BC$与$B'C'$、$CA$与$C'A'$的延长线分别相交于$P$、$Q$、$R$, 
    
    试证$P$、$Q$、$R$共线.
\end{example}

\begin{proof}
    设$\Vec{OA}=\vec{a}$, $\Vec{OB}=\vec{b}$, $\Vec{OC}=\vec{c}$, 则存在$\alpha,\beta,\gamma\in \mathbb{R}$,使$\Vec{OA'}=\alpha\vec{a}$, $\Vec{OB'}=\beta\vec{b}$, $\Vec{OC'}=\gamma\vec{c}$

    因$P$点既在$AB$上又在$A'B'$上,所以存在$x,y\in\mathbb{R}$,使
\[\begin{split}
    \Vec{OP}&=(1-x)\vec{a}+x\vec{b}\\
    \Vec{OP}&=(1-y)\alpha\vec{a}+\beta y\vec{b}\\
\end{split}\]
由此可得
\[x=\frac{\beta(\alpha-1)}{\alpha-\beta},\qquad y=\frac{\alpha-1}{\alpha-\beta}\]
\begin{equation}
    \Vec{OP}=\frac{\alpha(1-\beta)}{\alpha-\beta}\vec{a}+\frac{\beta(\alpha-1)}{\alpha-\beta}\vec{b}
\end{equation}
同理可求得
\begin{align}
    \Vec{OQ}&=\frac{\beta(1-\gamma)}{\beta-\gamma}\vec{b}+\frac{\gamma(\beta-1)}{\beta-\gamma}\vec{c}\\
    \Vec{OR}&=\frac{\gamma(1-\alpha)}{\gamma-\alpha}\vec{c}+\frac{\alpha(\gamma-1)}{\gamma-\alpha}\vec{a}
\end{align}
\[\begin{split}
    \Vec{PR}&=\Vec{OR}-\Vec{OP}=\frac{\alpha(\alpha-1)(\gamma-\beta)}{(\gamma-\alpha)(\alpha-\beta)}\vec{a}+\frac{\beta(1-\alpha)}{\alpha-\beta}\vec{b}+\frac{\gamma(1-\alpha)}{\gamma-\alpha}\vec{c}\\
    \Vec{PQ}&=\Vec{OQ}-\Vec{OP}=\frac{\alpha(\beta-1)}{\alpha-\beta}\vec{a}+\frac{\beta(\beta-1)(\gamma-\alpha)}{(\alpha-\beta)(\beta-\gamma)}\vec{b}+\frac{\gamma(\beta-1)}{\beta-\gamma}\vec{c}
\end{split}\]

容易验证$\vec{a},\vec{b},\vec{c}$的系数成比例,所以
$\Vec{PR}$与$\Vec{PQ}$共
线,即$P$、$Q$、$R$共线.

此题还可另证,设$x=(y-1)(\alpha-\beta)$,
 $y=(\alpha-1)(\beta-\gamma)$, $z=(\beta-1)(\gamma-\alpha)$, 则可得:
\[x\Vec{OP}+y\Vec{OQ}+z\Vec{OR}=\vec{0}\]
并且$x+y+z=0$.

$x$、$y$、$z$至少有一不为0, 
这便证明了$P$、$Q$、$R$三点共线.
\end{proof}

\begin{ex}
    试用向量运算证明以下各题.
\begin{enumerate}
\item  试证平行四边形的对角线互相平分.
\item 设$\triangle ABC$和$\triangle A'B'C'$的对应顶点连线$AA'$、$BB'$、
    $CC'$相交于一点$O$. 试证若 $AB\parallel A'B'$, $BC\parallel B'C'$, 则
    $AC\parallel A'C'$.

\item 设$\ell$、$\ell'$相交于$O$点,$A,B,C\in\ell$, $A',B',C'\in\ell'$. 试
证如果$AB'\parallel A'B$, $BC'\parallel B'C$, 则有$AC'\parallel A'C$.
\item 证明梯形中位线定理.
\item 已知:$\triangle ABC$中,$D$是$\overline{BC}$的中点,过$D$任作一直线分别交$\overline{AC}$于$E$, 交$AB$的延长线于$F$, 求证:
$\overline{AE}:\overline{EC}=\overline{AF}:\overline{FB}$.
\item 已知:梯形$ABCD$, $AB\parallel DC$, $\overline{AB}=2\overline{CD}$, $\overline{AC}$、$\overline{BD}$
相交于$E$, 求证:$\overline{CE}=\frac{1}{3}\overline{AC}$.
\item 已知:梯形$ABCD$中,$E$、$F$是上、下底$\overline{AD}$、$\overline{BC}$的
中点,$\overline{AC}$、$\overline{BD}$相交于$G$, 求证:$E$、$G$、$F$三点共线.
\end{enumerate}
\end{ex}

\begin{figure}[htp]\centering
    \begin{minipage}[t]{0.48\textwidth}
    \centering
\begin{tikzpicture}[>=latex, scale=1]
\tkzDefPoints{0/0/O, 4/0/A', 4.2/1.5/B', 3/2.5/C'}
\tkzDefPointWith[linear, K=.6](O,A') \tkzGetPoint{A}
\tkzDefPointWith[linear, K=.6](O,B') \tkzGetPoint{B}
\tkzDefPointWith[linear, K=.6](O,C') \tkzGetPoint{C}
\tkzDrawSegments(O,C' O,A' O,B')
\tkzDrawPolygon(A,B,C)
\tkzDrawPolygon(A',B',C')
\tkzLabelPoints[below](A,A')\tkzLabelPoints[above](C,C')
\tkzLabelPoints[below right](B,B')\tkzLabelPoints[left](O)
    \end{tikzpicture}
    \caption*{第2题}
    \end{minipage}
    \begin{minipage}[t]{0.48\textwidth}
    \centering
    \begin{tikzpicture}[>=latex, scale=.7]
\tkzDefPoints{0/0/O, 2/0/A, 3.5/0/B, 5/0/C}
\tkzDefPoint(40:2){C'}\tkzDefPoint(40:3.1){B'}
\tkzDefPoint(40:5){A'}
\tkzLabelPoints[above](A',B',C')
\tkzLabelPoints[below](A,B,C)
\draw(0,0)node[left]{$O$}--(6,0)node[right]{$\ell$};
\draw(0,0)--(40:6)node[right]{$\ell'$};
\tkzDrawSegments(A,B' A,C' B,C' B,A' C,B' C,A')
    \end{tikzpicture}
    \caption*{第3题}
    \end{minipage}
    \end{figure}

\section*{习题4.1}
\addcontentsline{toc}{subsection}{习题4.1}

\begin{enumerate}
    \item 已知:$O$是一个定点,$\vec{a}$、$\vec{b}$是两个线性无关的向量,
$\Vec{OP_1}=2\vec{a}+\vec{b}$, $\Vec{OP_2}=\vec{a}-\vec{b}$, $P$是直线$P_1P_2$上任意一
    点且$\Vec{OP}=x\vec{a}+y\vec{b}$, 求$x$、$y$满足的代数关系式.
    \item 已知:在$\triangle ABC$中,$D$、$E$分别是$\overline{BC}$、$\overline{AC}$边上的点且
    $\overline{BD}:\overline{DC}=1:2$, $\overline{CE}:\overline{EA}=1:1$, $\overline{AD}$与$\overline{BE}$相交于$O$
    点,设$\Vec{AB}=\vec{a}$, $\Vec{AC}=6$, $\Vec{CO}=x\vec{a}+y\vec{b}$, 求$x$、$y$.
    \item 设$\vec{a}$、$\vec{b}$是两个线性无关的向量,$\Vec{OD}=h\vec{a}+k\vec{b}\ne \vec{0}$.
    $P$是直线$OD$上任意一点,令$\Vec{OP}=x\vec{a}+y\vec{b}$, 求证:
    $x$、$y$满足方程$kx-hy=0$.
    \item 已知$D$、$E$、$F$分别是$\triangle ABC$的边$\overline{AB}$、$\overline{BC}$、$\overline{AC}$上的一
    点且$\frac{\overline{AD}}{\overline{AB}}=\frac{\overline{BE}}{\overline{BC}}=\frac{\overline{CF}}{\overline{CA}}$

    求证:
    所有这样的
    $\triangle DEF$重心是一个定点.

\item 如图,已知平行六面体$OADB-CEFG$, 求证:若对角
线$\overline{AG}$与平面$(O,D,E)$相交于$H$, $OH$与侧面$ADFE$
相交于$K$, 求证$\overline{AH}=\frac{1}{3}\overline{AG}$,$\overline{OH}=\frac{2}{3}\overline{OK}$

\begin{figure}[htp]\centering
    \begin{minipage}[t]{0.48\textwidth}
    \centering
\begin{tikzpicture}[>=latex, scale=1]
\tkzDefPoints{0/0/O, 3/0/A, 3.5/1/D, .5/1/B, .4/3/C}
\tkzDefPointsBy[translation = from O to C](A,B,D){E,G,F}
\tkzDrawPolygon(C,G,F,E)
\tkzDrawSegments(A,E D,F E,D F,A E,O A,D)
\tkzDrawSegments[dashed](A,G B,G B,D O,D)
\tkzDrawSegments[->, thick](O,A O,C)
\tkzDrawSegments[->, dashed, thick](O,B)
\tkzInterLL(E,D)(F,A) \tkzGetPoint{K}
\tkzDrawSegments[dashed](O,K)

\node at (1.5,0)[below]{$\vec{a}$};
\node at (.25,0.5)[right]{$\vec{b}$};
\node at (.2,1.5)[left]{$\vec{c}$};

\tkzLabelPoints[below](O,A)
\tkzLabelPoints[above right](B,D)
\tkzLabelPoints[right](K)
\tkzLabelPoints[above left](C,E)
\tkzInterLL(A,G)(O,K) \tkzGetPoint{H}
\tkzLabelPoints[above ](G,F,H)
    \end{tikzpicture}
    \caption*{第5题}
    \end{minipage}
    \begin{minipage}[t]{0.48\textwidth}
    \centering
    \begin{tikzpicture}[>=latex, scale=1]
\tkzDefPoints{0/0/B, 3/0/C, 2/2.6/A}
\tkzDefMidPoint(A,B)   \tkzGetPoint{R}
\tkzDefMidPoint(A,C)   \tkzGetPoint{Q}
\tkzDefMidPoint(C,B)   \tkzGetPoint{P}

\tkzDrawPolygon(A,B,C)
\tkzDrawSegments(A,P B,Q C,R)
\tkzLabelPoints[below](P,C,B)
\tkzLabelPoints[above](A)
\tkzLabelPoints[left](R)
\tkzLabelPoints[right](Q)
    \end{tikzpicture}
    \caption*{第6题}
    \end{minipage}
    \end{figure}

\item  在$\triangle ABC$的三条边上各取一点$P$、$Q$、$R$, 若$\Vec{AR}=x\Vec{RB}$, $\Vec{BP}=y\Vec{PC}$, $\Vec{CQ}=z\Vec{QA}$. 求证:$AP$、$BQ$、$CR$共
点的充要条件是
$xyz=1$.

\item  在$\triangle ABC$三边所在的直线
上各取一点$P$、$Q$、$R$, 使
\[\Vec{AR}=x\Vec{RB},\qquad \Vec{BP}=y\Vec{PC},\qquad \Vec{CQ}=z\Vec{QA}\]
求证:$P$、$Q$、$R$三点共线的充要条件是$xyz=-1$.
\begin{figure}[htp]\centering
\begin{tikzpicture}[>=latex, scale=1]
\tkzDefPoints{0/0/B, 3/0/C, 2/2.6/A}
\tkzDefPointWith[linear, K=.7](A,B) \tkzGetPoint{R}
\tkzDefPointWith[linear, K=.4](A,C) \tkzGetPoint{Q}
\tkzInterLL(B,C)(R,Q)\tkzGetPoint{P}
\tkzDrawPolygon(A,B,C)
\tkzDrawSegments(Q,P B,P C,R)
\tkzLabelPoints[below](P,C,B)
\tkzLabelPoints[above](A)
\tkzLabelPoints[above left](R)
\tkzLabelPoints[right](Q)
    \end{tikzpicture}
    \caption*{第7题}
    \end{figure}
\end{enumerate}

\section{垂直与度量问题}
这节我们举例说明,用内积运算处理度量问题的一般方
法和技巧.

\begin{example}
    证明勾股定理(图4.11).
\end{example}

\begin{proof}
    在直角$\triangle ABC$中,$\angle B=90^{\circ}$, 则
\[\begin{split}
   \Vec{AC}&=\Vec{AB}+\Vec{BC}\\
   \Vec{AC}\cdot \Vec{AC}&= \left(\Vec{AB}+\Vec{BC}\right)\cdot \left(\Vec{AB}+\Vec{BC}\right)\\
   &=\Vec{AB}\cdot \Vec{AB}+\Vec{BC}\cdot \Vec{BC}
\end{split}\]
即:$\overline{AC}^2=\overline{AB}^2+\overline{BC}^2$
\end{proof}

\begin{figure}[htp]\centering
    \begin{minipage}[t]{0.48\textwidth}
    \centering
\begin{tikzpicture}[>=latex, scale=.8]
\tkzDefPoints{0/0/A, 3/0/B, 3/4/C}
\tkzDrawSegments[->](A,B B,C A,C)
\tkzLabelPoints[below](A,B)
\tkzLabelPoints[above](C)
    \end{tikzpicture}
    \caption{}
    \end{minipage}
    \begin{minipage}[t]{0.48\textwidth}
    \centering
    \begin{tikzpicture}[>=latex, scale=1]
        \tkzDefPoints{0/0/A, 3/0/B, 2/3/C}
        \tkzDrawSegments[->](A,B B,C A,C)
        \tkzLabelPoints[below](A,B)
        \tkzLabelPoints[above](C)
\node at (1.5,0)[below]{$c$};
\node at (1,1.5)[left]{$b$};
\node at (2.5,1.5)[right]{$a$};
    \end{tikzpicture}
    \caption{}
    \end{minipage}
    \end{figure}


\begin{example}
    证明余弦定理(图4.12).
\end{example}

\begin{proof}
在$\triangle ABC$中,
\[\begin{split}
    \Vec{AC}&=\Vec{AB}+\Vec{BC}\\
    \Vec{AC}\cdot  \Vec{AC}&=\left(\Vec{AB}+\Vec{BC}\right)\cdot \left(\Vec{AB}+\Vec{BC}\right)\\
    &=\Vec{AB}\cdot \Vec{AB}+\Vec{BC}\cdot \Vec{BC}+2\Vec{AB}\cdot \Vec{BC}
\end{split}\]
即
\[\begin{split}
    |\Vec{AC}|^2&=|\Vec{AB}|^2+|\Vec{BC}|^2-2|\Vec{AB}||\Vec{BC}|\cos B\\
b^2&=c^2+a^2-2ca\cos B
\end{split}\]
同理可证:
\[\begin{split}
  c^2&=a^2+b^2-2ab\cos C\\
a^2&=b^2+c^2-2bc\cos A  
\end{split}\]
\end{proof}

由例4.10、例4.11的证明过程可以看到,为了得到三角形的边
角关系,只要写出三角形中边向量所要满足的关系,然后作
内积运算,向量关系就可转化为数量关系.



\begin{example}
    证明射影定理与正弦定理.
\end{example}

\begin{proof}
过$A$点引单位向量$\vec{e}_1\parallel \Vec{AC}$, $\vec{e}_2
\bot \vec{e}_1$ (图4.13)
\begin{equation}
    \begin{split}
    \Vec{AC}&=\Vec{AB}+\Vec{BC}\\
    \Vec{AC}\cdot \vec{e}_1&=\Vec{AB}\cdot \vec{e}_1+\Vec{BC}\cdot \vec{e}_1\\
    |\Vec{AC}|&=|\Vec{AB}|\cos A+|\Vec{BC}|\cos C
\end{split}
\end{equation}
即:$b=c\cos A+a\cos C$
同理可证:
\[c=a\cos B+b\cos A,\qquad a=b\cos C+c\cos B\]
(4.5)式两边分别对$\vec{e}_2$取内积运算,则
\[\begin{split}
    \Vec{AC}\cdot \vec{e}_2&=\Vec{AB}\cdot \vec{e}_2+\Vec{BC}\cdot \vec{e}_2\\
    0&=|\Vec{AB}|\cos(90^{\circ}-A)+| \Vec{BC} |\cos(90^{\circ}+C)
\end{split}\]
即:$0=c\sin A-a\sin C\quad \Rightarrow\quad \frac{a}{\sin A}=\frac{c}{\sin C}$

同理可证:$\frac{a}{\sin A}=\frac{b}{\sin B}$,
所以
\[\frac{a}{\sin A}=\frac{b}{\sin B}=\frac{c}{\sin C}\]
\end{proof}

在例4.12的证明中,我们先写出三角形三个边向量所满足
的向量关系式,然后,分别对$\vec{e}_1$、$\vec{e}_2$两个互相垂直的单位向量取内积运算,这样就很容易地证明了射影定理和正弦定
理.适当选取单位向量,对题设条件所满足的向量关系式进
行内积运算是处理一些直线形边角关系的
基本方法之一.

\begin{figure}[htp]\centering
    \begin{minipage}[t]{0.48\textwidth}
    \centering
\begin{tikzpicture}[>=latex, scale=1]
    \tkzDefPoints{0/0/A, 3/0/C, 2/2/B}
    \tkzDrawSegments[->](A,B B,C A,C)
    \tkzLabelPoints[below](A,C)
    \tkzLabelPoints[above](B)
\node at (1.5,0)[below]{$b$};
\node at (1,1)[left]{$c$};
\node at (2.5,1)[right]{$a$};
\draw[thick,->](0,0)--node[below]{$\vec{e}_1$}(1,0);
\draw[thick,->](0,0)--node[left]{$\vec{e}_2$}(0,1);

    \end{tikzpicture}
    \caption{}
    \end{minipage}
    \begin{minipage}[t]{0.48\textwidth}
    \centering
    \begin{tikzpicture}[>=latex, scale=1]
\tkzDefPoints{0/0/B, 4/0/C, 3.5/2/A, 2.7/0/P}
\node at (1.75,1)[left]{$c$};
\node at (3.75,1)[right]{$b$};
\node at (2,0)[below]{$a$};
\tkzDrawSegments[->](A,B A,C A,P)
\tkzDrawSegments(B,C)
\tkzLabelPoints[below](B,P,C)
\tkzLabelPoints[above](A)
\draw[->](P)--node[above]{$\vec{a}$}+(160:1);

    \end{tikzpicture}
    \caption{}
    \end{minipage}
    \end{figure}
 
\begin{example}
    利用内积运算证明角平分线定理.
\end{example}


\begin{proof}
设$\overline{AP}$为$\triangle ABC$中内角$A$的平
分线,则$\angle BAP=\angle PAC=\alpha$, 取单位向量
$\vec{e}\bot \Vec{AP}$(图4.14)
\[\begin{split}
  \Vec{AB}\cdot \vec{e}&=|\Vec{AB} |\cos(90^{\circ}-\alpha)=|\Vec{AB} |\sin\alpha\\
  \Vec{AC}\cdot \vec{e}&=|\Vec{AC}|\cos(90^{\circ}+\alpha)=-|\Vec{AC}|\sin\alpha  
\end{split}\]
\[\frac{\Vec{AB}}{\Vec{AC}}=\frac{\Vec{AB}\cdot \vec{e}}{\Vec{AC}\cdot \vec{e}}=\frac{|(\Vec{AP}+\Vec{PB})\cdot \vec{e}|}{|(\Vec{AP}+\Vec{PC})\cdot \vec{e}|}=\frac{|\Vec{PB}\cdot \vec{e}|}{\Vec{PC}\cdot \vec{e}}=\frac{\Vec{PB}}{\Vec{PC}}\]
\end{proof}


\begin{example}
    在直角$\triangle ABC$中,$AD$是斜边$\overline{BC}$上的高,作
    $DE\bot AB$, $DF\bot AC$, $E$、$F$是垂足.
    
    求证:$\frac{\overline{BE}}{\overline{CF}}=\frac{\overline{AB}^3}{\overline{AC}^3}$
\end{example}

\begin{proof}
设$\frac{\overline{EB}}{\overline{AB}}=\lambda$, $\frac{\overline{FC}}{\overline{AC}}=\mu$,则:  
\[\Vec{EB}=\lambda\Vec{AB},\quad \Vec{FC}=\mu\Vec{AC},\quad \Vec{AE}=(1-\lambda)\Vec{AB},\quad \Vec{AF}=(1-\mu)\Vec{AC}\]
\[\Vec{AD}=\Vec{AE}+\Vec{AF}=(1-\lambda)\Vec{AB}+(1-\mu)\Vec{AC}\]
因为$D\in BC$,则:
\[(1-\lambda)+(1-\mu)=1\quad \Rightarrow\quad \lambda+\mu=1\]
又$\because\quad \Vec{AD}\bot \Vec{BC}$,因此:
\[\left[(1-\lambda)\Vec{AB}+(1-\mu)\Vec{AC}\right]\cdot \left(\Vec{AC}-\Vec{AB}\right)=0\]
由此可得:
\[\frac{\lambda}{\mu}=\frac{\overline{AB}^2}{\overline{AC}^2}\]
\[\frac{\overline{EB}}{\overline{FC}}=\frac{\lambda\overline{AB}}{\mu\overline{AC}}=\frac{\overline{AB}^3}{\overline{AC}^3}\]
\end{proof}

\begin{figure}[htp]\centering
    \begin{minipage}[t]{0.48\textwidth}
    \centering
\begin{tikzpicture}[>=latex, scale=1]
\tkzDefPoints{0/0/B, 4/0/C, 2.5/2/A, 2.5/0/D}
\tkzDefPointBy[projection =onto A--B](D) \tkzGetPoint{E}
\tkzDefPointBy[projection =onto A--C](D) \tkzGetPoint{F}
\tkzDrawPolygon(A,C,B)
\tkzDrawSegments(D,E D,F)
\tkzDrawSegments[->](A,E A,F  A,D)
\tkzLabelPoints[below](B,C,D)
\tkzLabelPoints[left](E)
\tkzLabelPoints[right](F)
\tkzLabelPoints[above](A)
    \end{tikzpicture}
    \caption{}
    \end{minipage}
    \begin{minipage}[t]{0.48\textwidth}
    \centering
    \begin{tikzpicture}[>=latex, scale=1]
\tkzDefPoints{0/0/A, 5/0/B, 2.5/0/O}
\tkzDefPoint(36.87:4){P}
\tkzDrawPolygon(A,B,P)  \tkzDrawSegments(P,O)
\tkzLabelPoints[below](B,A,O)
\tkzLabelPoints[above](P)
    \end{tikzpicture}
    \caption{}
    \end{minipage}
    \end{figure}

\begin{example}
    设一动点$P$, 到两点$A$、$B$的距
离的平方和等于常数$k$, 求$P$点的轨迹.
\end{example}

\begin{solution}
    取$\overline{AB}$的中点$O$, 则
\[\Vec{PA}=\Vec{PO}+\Vec{OA},\qquad \Vec{PB}=\Vec{PO}+\Vec{OB}\]
\[\begin{split}
    \overline{PA}^2&=\overline{PO}^2+\overline{OA}^2+2\Vec{PO}\cdot\Vec{OA}\\
    \overline{PB}^2&=\overline{PO}^2+\overline{OB}^2+2\Vec{PO}\cdot \Vec{OB}
\end{split}\]
因为$\overline{PA}^2+\overline{PB}^2=k,\quad \Vec{PO}\cdot\Vec{OA}=-\Vec{PO}\cdot \Vec{OB}$

所以:
\[2\left(\overline{PO}^2+\overline{OA}^2\right)=k,\qquad \overline{PO}=\sqrt{\frac{k}{2}-\overline{OA}^2}\]
于是,
$P$点到$O$点的距离是一个常数.即$P$点的轨迹是以$O$为
圆心,$\sqrt{\frac{k}{2}-\overline{OA}^2}$为半径的圆.
\end{solution}

\begin{example}
    已知正方形$ABCD$(图4.17), $P$为$\overline{BD}$上任一点,
$\overline{PE}\bot \overline{BC}$于$E$点,$\overline{PF}\bot \overline{CD}$于$F$点,求证:
$\overline{AP}=\overline{EF}$
且$AP\bot EF$.
\end{example}

\begin{proof}
设$\Vec{BP}=\lambda \Vec{BD}$, 正方形边长为$a$, 则
\[\begin{split}
    \Vec{AP}&=\Vec{AB}+\Vec{BP}+(1-\lambda)\Vec{AB}+\lambda\Vec{AD}\\
    \Vec{EF}&=\Vec{EC}+\Vec{CF}=(1-\lambda)\Vec{BC}+\lambda\Vec{CD}
\end{split}\]
\[\begin{split}
    \overline{AP}^2&=(1-\lambda )^2 a^2+\lambda a^2+2\lambda(1-\lambda)\Vec{AB}\Vec{AD}\\
&=(1-\lambda )^2 a^2+\lambda a^2\\
\overline{EF}^2&=(1-\lambda )^2 a^2+\lambda a^2
\end{split}\]
所以:$\overline{AP}=\overline{EF}$.因为
\[\Vec{AP}\cdot \Vec{EF}=\lambda (1-\lambda)\Vec{AB}\cdot \Vec{CD}+\lambda(1-\lambda)\Vec{AD}\cdot \Vec{BC}=0\]
所以
$\overline{AP} \bot \overline{EF}$.
\end{proof}

\begin{figure}[htp]\centering
    \begin{minipage}[t]{0.48\textwidth}
    \centering
\begin{tikzpicture}[>=latex, scale=1]
\tkzDefPoints{0/0/B, 3/0/C, 3/3/D, 0/3/A, 2.5/2.5/P, 3/2.5/F,2.5/0/E}
\tkzDrawPolygon(A,B,C,D)
\tkzDrawPolygon(P,E,F)
\tkzDrawSegments(B,D A,P)
\tkzLabelPoints[below](B,E,C)
\tkzLabelPoints[above](A,D,P)
\tkzLabelPoints[right](F)
    \end{tikzpicture}
    \caption{}
    \end{minipage}
    \begin{minipage}[t]{0.48\textwidth}
    \centering
    \begin{tikzpicture}[>=latex, scale=1]
\tkzDefPoints{0/0/B, 4/0/C, 2.5/2.5/A, 2/0/D}
\tkzDrawSegments(A,D)
\tkzDrawSegments[->](B,C A,B C,A)
\node at (1.25,1.25)[left]{$\vec{c}$};
\node at (3.25,1.25)[right]{$\vec{b}$};
\node at (1.5,0)[below]{$\vec{a}$};
\node at (2.25,1.25)[right]{$m_a$};
\tkzLabelPoints[below](B,D,C)
\tkzLabelPoints[above](A)
    \end{tikzpicture}
    \caption{}
    \end{minipage}
    \end{figure}

\begin{example}
    已知$\triangle ABC$ (图4.18), 
$\overline{BC}=a$, $\overline{CA}=b$, $\overline{AB}=c$,
求:
\begin{enumerate}
    \item 三边上的中线$m_a$, $m_b$, $m_c$;
    \item 三个角平分线$t_a$、$t_b$、$t_c$;
\item 三角形的面积$S$.
\end{enumerate}
\end{example}

\begin{solution}
\begin{enumerate}
    \item 如图:设$\Vec{BC}=\vec{a}$, $\Vec{CA}=\vec{b}$, $\Vec{AB}=\vec{c}$, 则$BC$边上的中线
$\Vec{AD}=\frac{1}{2}(-\vec{b}+\vec{c})$,
\[\begin{split}
    \Vec{AD}\cdot \Vec{AD}&=\frac{1}{4}(b^2+c^2-2\vec{b}\cdot \vec{c})\\
    |\Vec{AD}|^2&=\frac{1}{4}[b^2+c^2-(a^2-b^2-c^2)]=\frac{1}{4}(2b^2+2c^2-a^2)
\end{split}\]
因此
\[m_a=\frac{1}{2}\sqrt{2b^2+2c^2-a^2}\]
同理可得:
\[
    m_b=\frac{1}{2}\sqrt{2a^2+2c^2-b^2},\qquad
    m_c=\frac{1}{2}\sqrt{2a^2+2b^2-c^2}
\]

\item 因$\vec{t}_a=\frac{c}{b+c}\vec{b}+\frac{b}{b+c}\vec{c}$,则:
\[\begin{split}
\left|\vec{t_{a}}\right|^{2} &=\frac{c^{2} b^{2}}{(b+c)^{2}}+\frac{b^{2} c^{2}}{(b+c)^{2}}+2 \frac{b c}{(b+c)^{2}} \vec{b} \cdot \vec{c} \\
&=\frac{b c}{(b+c)^{2}}\left(2 b c+b^{2}+c^{2}-a^{2}\right) \\
&=\frac{b c}{(b+c)^{2}}\left[(b+c)^{2}-a^{2}\right] \\
&=\frac{b c}{(b+c)^{2}}[(a+b+c)(b+c-a)]
\end{split}\]  
令$a+b+c=2p$, 则
\[\begin{split}
    t^2_a&=\frac{4bc}{(b+c)^2}p(p-a)\\
t_a&=\frac{2\sqrt{bc}}{b+c}\sqrt{p(p-a)}
\end{split}\]
同理可得:
\[
    t_b=\frac{2\sqrt{ac}}{a+c}\sqrt{p(p-b)},\qquad
    t_c=\frac{2\sqrt{ab}}{a+b}\sqrt{p(p-c)}
\]

\item 
\[\begin{split}
    S^{2} &=\frac{1}{4} b^{2} c^{2} \sin ^{2} A=\frac{1}{4} b^{2} c^{2}\left(1-\cos ^{2} A\right) \\
    &=\frac{1}{4} b^{2} c^{2}\left[1-\left(\frac{\vec{b} \cdot \vec{c}}{b c}\right)^{2}\right] =\frac{1}{4} b^{2} c^{2}\left[1-\frac{\left(b^{2}+c^{2}-a^{2}\right)^{2}}{4 b^{2} c^{2}}\right] \\
    &=\frac{1}{16}\left[4 b^{2} c^{2}-\left(b^{2}+c^{2}-a^{2}\right)^{2}\right] \\
    &=\frac{1}{16}(a+b+c)(a+b-c)(a+c-b)(b+c-a) \\
    &=p(p-a)(p-b)(p-c) \\
    \end{split}\]
\[ S=\sqrt{p(p-a)(p-b)(p-c)}\]
\end{enumerate}
\end{solution}

\begin{example}
    证明定理:如果一条直线$a$垂直于平面$\pi$上的两条
相交直线$b$、$c$, 那么$a\bot\pi $ (图4.19).
\end{example}

\begin{proof}
    在平面$\pi$上,任取一条直
线$d$, 在$a,b,c,d$上分别取向量$\vec{a}$、$\vec{b}$、$\vec{c}$、$\vec{d}$. 由于$b$、$c$相交,
依共面向量定理,存在唯一的数偶
$(x,y)$, 使
\[\begin{split}
    \vec{d}&=x\vec{b}+y\vec{c}\\
    \vec{a}\cdot \vec{d}&=\vec{a}\cdot \left(x\vec{b}+y\vec{c}\right)=x\vec{a}\cdot \vec{b}+y\vec{a}\cdot \vec{c}=0
\end{split}\]
所以:$\vec{a}\bot \vec{d}$,即:$a\bot d$

这就是说$a$垂直于平面$\pi$上的任一条直线,所以$a\bot\pi$.
\end{proof}

\begin{figure}[htp]\centering
    \begin{minipage}[t]{0.48\textwidth}
    \centering
\begin{tikzpicture}[>=latex, yscale=.8]
\tkzDefPoints{0/0/A, 4/0/B, 5/2.5/C, 1/2.5/D}
\tkzDrawPolygon(A,B,C,D)
\draw[->](3,.5)--(1.4,1.7)node[above right]{$\vec{b}$};
\draw[->](1,.5)--(3.5,2)node[right]{$\vec{c}$};
\draw[->](3.5,.5)--node[right]{$\vec{d}$}(4,1.5);
\node at (0,0)[above right]{$\pi$};
\draw[->](2,2)--node[right]{$\vec{a}$}(2,3.5);
    \end{tikzpicture}
    \caption{}
    \end{minipage}
    \begin{minipage}[t]{0.48\textwidth}
    \centering
    \begin{tikzpicture}[>=latex, scale=1]
\tkzDefPoints{0/0/A, 4/0/C, 2.3/-1/B, 2.8/2/O}
\tkzDrawPolygon(O,A,B,C)
\tkzDrawSegments(O,B)
\tkzDefMidPoint(O,A)  \tkzGetPoint{E}
\tkzDefMidPoint(O,B)  \tkzGetPoint{F}
\tkzDefMidPoint(B,C)  \tkzGetPoint{G}
\tkzDefMidPoint(C,A)  \tkzGetPoint{H}
\tkzDrawSegments[dashed](A,C E,F F,G G,H E,H)
\tkzLabelPoints[above](O)
\tkzLabelPoints[right](F,G)
\tkzLabelPoints[left](E)
\tkzLabelPoints[below](A,B,C,H)
    \end{tikzpicture}
    \caption{}
    \end{minipage}
    \end{figure}

\begin{example}
    已知空间四边形$O-ABC$, $\overline{OA}=\overline{OB}$, $\overline{CA}=\overline{CB}$.
    $E$、$F$、$G$、$H$分别为$\overline{OA}$、$\overline{OB}$、$\overline{CB}$、$\overline{CA}$的中点,求证
    四边形$EFGH$是矩形(图4.20).
\end{example}

\begin{proof}
由于$E$、$F$、$G$、$H$分别是$\overline{OA}$
、$\overline{OB}$、$\overline{CB}$、$\overline{CA}$的中点,所以
\[\Vec{EF}=\frac{1}{2}\Vec{AB},\qquad \Vec{HG}=\frac{1}{2}\Vec{AB}\]
\[\Vec{EF}=\Vec{HG}\quad \Rightarrow\quad \overline{EF}=\overline{HG},\quad EF\parallel HG\]
$\therefore\quad EFGH$是平行四边形.

因为
$\Vec{EF}=\frac{1}{2}\Vec{AB}=\frac{1}{2}(\Vec{OB}-\Vec{OA}),\quad \Vec{EH}=\frac{1}{2}\Vec{OC}$,所以
\[\begin{split}
        \Vec{E F} \cdot \Vec{E H} &=\frac{1}{2}(\Vec{O B}-\Vec{O A}) \cdot \frac{1}{2} \Vec{O C} \\
        &=\frac{1}{4}(\Vec{O B} \cdot \Vec{O C}-\Vec{O A} \cdot \Vec{O C}) \\
        &=\frac{1}{4}\left(|\Vec{O B}|^{2}+|\Vec{O C}|^{2}-|\Vec{B C}|^{2}-|\Vec{O A}|^{2}-|\Vec{O C}|^{2}+|\Vec{C A}|^{2}\right) \\
\end{split}\]
但已知$\Vec{O A}=\Vec{O B},\qquad \Vec{C A}=\Vec{C B}$,所以
\[\Vec{E F} \cdot \Vec{E H} =0 \quad \Rightarrow\quad  E F \bot E H\]
故得四边形$EFGH$是矩形.
\end{proof}

由以上各例可看到在空间证明两线垂直,内积运算仍是
非常有效的工具.

\begin{example}
    一定长线段$\overline{AB}$的两个端
    点,沿互相垂直的两条异面直线$\ell$、
    $m$运动,求它的中点的轨迹.
\end{example}

\begin{solution}
    如图4.21设$\overline{MN}$为$\ell$、$m$的
公垂线,$\overline{AB}=a$, $\overline{MN}=b$, $O$、$P$分
别为$\overline{MN}$、$\overline{AB}$的中点,
则
\[\Vec{OP}=\frac{1}{2}\left(\Vec{OA}+\Vec{OB}\right)=\frac{1}{2}\left(\Vec{OM}+\Vec{MA}+\Vec{ON}+\Vec{NB}\right)\]
因为$\Vec{OM}=-\Vec{ON}$,所以
\[\begin{split}
    \Vec{OP}&=\frac{1}{2}(\Vec{MA}+\Vec{NB})\\
    \Vec{OP}\cdot\Vec{MN}&=\frac{1}{2}\left(\Vec{MA}+\Vec{NB}\right)\cdot \Vec{MN}=0
\end{split}    \]
因此:$P$点一定在$\overline{MN}$的垂直平分面上.

因为$\Vec{OP}\cdot \Vec{OP}=\frac{1}{4}(|\Vec{MA}|^2+|\Vec{NB}|^2)$,
连$AN$, 易证,$\triangle AMN$与$\triangle ABN$都是直角三角形.
所以
\[\begin{split}
    \Vec{OP}\cdot \Vec{OP}&=\frac{1}{4}\left(\overline{AN}^2-\overline{MN}^2+\overline{AB}^2-\overline{AN}^2\right)\\
    &=\frac{1}{4}\left(\overline{AB}^2-\overline{MN}^2\right)=\frac{1}{4}(a^2-b^2)
\end{split}\]
即:$|\Vec{OP}|^2=\frac{1}{4}(a^2-b^2)$

由上式可知$P$点在以$O$为圆心,以$\frac{1}{2}\sqrt{a^2-b^2}$为半径的
圆上,因此$P$点的轨迹是$\overline{MN}$的垂直平分面上的一个圆:
$\odot\left(O,\frac{1}{2}\sqrt{a^2-b^2}\right)$
\end{solution}

\begin{figure}[htp]\centering
    \begin{minipage}[t]{0.48\textwidth}
    \centering
\begin{tikzpicture}[>=latex, scale=1]
\tkzDefPoints{0/3/A, 0/2/M, 3/2/N, 1.5/0.5/B}
\tkzDefMidPoint(M,N) \tkzGetPoint{O}
\tkzDefMidPoint(A,B) \tkzGetPoint{P}
\tkzDrawSegments(A,B A,N M,N O,P)
\tkzDrawLines[add = 1 and 1](A,M) 
\tkzDrawLines[add = .5 and .5](B,N) 
 \node at (0,4)[right]{$\ell$};
 \node at (3.5,3)[right]{$m$};
\tkzLabelPoints[left](A,M,P)
\tkzLabelPoints[right](B,N)
 \tkzLabelPoints[above](O)
    \end{tikzpicture}
    \caption{}
    \end{minipage}
    \begin{minipage}[t]{0.48\textwidth}
    \centering
    \begin{tikzpicture}[>=latex, scale=1]
\tkzDefPoints{0/0/B, 4/0/D, 2.8/-1/C, 2.3/2/A}
\tkzDrawPolygon(D,A,B,C)
\tkzDrawSegments[dashed](D,B)
\tkzDrawSegments(A,C)
\tkzLabelPoints[below](D,B,C)
\tkzLabelPoints[above](A)
    \end{tikzpicture}
    \caption{}
    \end{minipage}
    \end{figure}

\begin{example}
    如图4.22, 已知四面体
$A-BCD$, $AB\bot CD$, $AC\bot BD$.

求证:$AD\bot BC$.
\end{example}

\begin{proof}
    $\because\quad AB\bot CD,\quad AC\bot BD$

    所以
\[    \Vec{AB}\cdot \Vec{CD}=0,\qquad    \Vec{AC}\cdot \Vec{BD}=0\]
\[\left(\Vec{AD}+\Vec{DB}\right)\cdot \Vec{CD}=0,\qquad    \left(\Vec{AD}+\Vec{DC}\right)\cdot \Vec{BD}=0\]
\[\Vec{AD}\cdot \Vec{CD}=\Vec{BD}\cdot \Vec{CD},\qquad \Vec{AD}\cdot \Vec{BD}=\Vec{BD}\cdot \Vec{CD}\]
\[\begin{split}
    \Vec{AD}\cdot \Vec{CD}&=\Vec{AD}\cdot \Vec{BD}
    \\
    \Vec{AD}\cdot \left(\Vec{BD}+\Vec{DC}\right)&=0\\
    \Vec{AD}\cdot \Vec{BC}&=0
\end{split}\]
即:$\Vec{AD}\bot \Vec{BC},\qquad AD\bot BC$
\end{proof}


\begin{example}
    如图4.23在直二面角的棱上有两点$A$、$B$, $AC$和
    $BD$各在这个二面角的一个面内,并且都垂直于棱$AB$,设
    $\overline{AB}=8$, $\overline{AC}=6$, $\overline{BD}=24$, 求$\overline{CD}$的长.
\end{example}

\begin{solution}    
 如图4.23, 
\[|\Vec{CD}|^2=\Vec{CB}\cdot \Vec{CD}=\left(\Vec{CA}+\Vec{AB}+\Vec{BD}\right)\cdot \left(Vec{CA}+\Vec{AB}+\Vec{BD}\right)\]
因为
$Vec{AC}$、$Vec{AB}$、$Vec{BD}$互相正交,
所以
\[|\Vec{CD}|^2=|\Vec{CA}|^2+|\Vec{AB}|^2+|\Vec{BD}|^2=6^2+8^2+24^2=676\]
$\therefore\quad \overline{CD}=\sqrt{676}=26$
\end{solution}

\begin{figure}[htp]\centering
    \begin{minipage}[t]{0.48\textwidth}
    \centering
\begin{tikzpicture}[>=latex, scale=1]
\tkzDefPoints{0/0/a1, .5/1.9/a2, 1.5/-1.8/a3, 3/0/a4}
\tkzDefPointsBy[translation= from a1 to a4](a2,a3){a2',a3'}
\tkzDrawPolygon(a1,a2,a2',a4)
\tkzDrawPolygon(a1,a3,a3',a4)
\draw(1-.5,0)node[below]{$A$}--(1.25-.5,0.95)node[above]{$C$}--(2.5+.75,-.9)node[below]{$D$}--(1+1.5,0)node[above]{$B$};
    \end{tikzpicture}
    \caption{}
    \end{minipage}
    \begin{minipage}[t]{0.48\textwidth}
    \centering
    \begin{tikzpicture}[>=latex, scale=.8]
\tkzDefPoints{0/0/B, 4/0/C, 5/1/D, 1/1/A, 0/4/F}
\tkzDefPointsBy[translation= from B to F](C,D,A){G,H,E}
\tkzDrawPolygon(E,F,G,H)
\tkzDrawSegments(B,C C,D B,F C,G H,D G,D)
\tkzDrawSegments[->, dashed](A,B A,D A,E A,G)

\tkzDefPointWith[linear, K=.65](G,D) \tkzGetPoint{Q}
\tkzDefPointWith[linear, K=.6](A,C) \tkzGetPoint{P}
\tkzDrawSegments[dashed](A,C P,Q)
\node at (.75,.75)[left]{$\vec{e_1}$};
\node at (1,2.5)[right]{$\vec{e_3}$};
\node at (3,1)[above]{$\vec{e_2}$};
\tkzLabelPoints[below](B,C)
\tkzLabelPoints[above](E,H,G)
\tkzLabelPoints[right](Q,D)
\tkzLabelPoints[left](F,A,P)
    \end{tikzpicture}
    \caption{}
    \end{minipage}
    \end{figure}

\begin{example}
    已知正方体$ABCD-EFGH$ (图4.24) 其棱长为
1. 求:$AC$与$DG$的公垂线的垂足$P$、$Q$的位置和$AC$与$DG$ 
间的距离.
\end{example}

\begin{solution}
设$\Vec{AB}=\vec{e}_1$, $\Vec{AD}=\vec{e}_2$, $\Vec{AE}=\vec{e}_3$, $\Vec{AP}=x\Vec{AC}$, $\Vec{DQ}=y\Vec{DG}$ 

由于$\Vec{AC}=\vec{e}_1+\vec{e}_2,\quad \Vec{DG}=\vec{e}_1+\vec{e}_3$,所以
\[\Vec{AP}=x\vec{e}_1+x\vec{e}_2,\qquad \Vec{AQ}=y\vec{e}_1+y\vec{e}_3\]
\[\begin{split}
    \Vec{AQ}&=\Vec{AD}+\Vec{DQ}=y\vec{e}_1+\vec{e}_2+y\vec{e}_3\\
    \Vec{PQ}&=\Vec{AQ}-\Vec{AP}=(y-x)\vec{e}_1+(1-x)\vec{e}_2+y\vec{e}_3\\
\end{split}\]
又因 $\Vec{P Q} \perp \Vec{A C}$, $\Vec{P Q} \perp \Vec{D G}$, 则
\[\begin{cases}
    {\left[(y-x) \vec{e}_{1}+(1-x) \vec{e}_{2}+y \vec{e}_{3}\right] \cdot\left(\vec{e}_{1}+\vec{e}_{2}\right)=0} \\
    {\left[(y-x) \vec{e}_{1}+(1-x) \vec{e}_{2}+y \vec{e}_{3}\right] \cdot\left(\vec{e}_{1}+\vec{e}_{3}\right)=0}
\end{cases}\]
由此可得方程组
\[\begin{cases}
    2 x-y=1 \\
x-2 y=0
\end{cases}\]
解之得:$x=\frac{2}{3}, \qquad y=\frac{1}{3}$

所以:$\Vec{AP}=\frac{2}{3}\Vec{AC},\quad \Vec{DQ}=\frac{1}{3}\Vec{DG}$
\[\Vec{PQ}=\frac{1}{3}\left(-\vec{e}_1+\vec{e}_2+\vec{e}_3\right)\]
于是$P$、$Q$两点的位置可定,且$AC$与$DG$的距离就是$|\Vec{PQ}|$. 
\[|\Vec{PQ}|=\sqrt{\Vec{PQ}\cdot \Vec{PQ}}=\sqrt{\frac{1}{9}(1+1+1)}=\frac{\sqrt{3}}{3}\]
\end{solution}


\section*{习题4.2}
\addcontentsline{toc}{subsection}{习题4.2}

试用向量运算证明以下各题:

\begin{enumerate}
    \item 试证:点$P$在$\overline{AB}$的垂直平分上的充要条件是$|\Vec{PA}|=|\Vec{PB}|$
\item 试证:对角线互相垂直的平行四边形是菱形.
\item 试证:平行四边形的对角线等长的充要条件是这个平行
四边形是矩形.
\item 求证:直角三角形斜边上的中线等于斜边的一半.
\item 等腰三角形顶角的平分线是底边上的高.
\item 求证:四边形$ABCD$中,对角线互相垂直的充要条件是
\[\overline{AB}^2 +\overline{CD}^2=\overline{AD}^2+\overline{BC}^2\]
\item 求证:平行四边形$ABCD$中的锐角$A$为$45^{\circ}$的充要条件
是$$\overline{AC}^2\cdot \overline{BD}^2=\overline{AB}^4+\overline{AD}^4$$
\item 已知$\ell$是一条直线,$A$、$B$为$\ell$外同侧的两个定点,$P\in\ell$, 

求
证:$|\Vec{AP}|+|\Vec{PB}|$取极小值的充要条件是
\[\frac{\vec{a}\cdot \vec{n}}{|\vec{a}|}=\frac{\vec{b}\cdot \vec{n}}{|\vec{b}|}\]
(其中$\vec{n}\bot \ell$且方向指向$A$、$B$所在的
那一侧)并解释其几何意义.
\item 求证:长方体的对角线的平方等于长、宽、高的平方
和.
\item 已知$\overline{AB}$在平面$\pi$内,$\overline{AC}\bot\pi$, $\overline{BD}\bot \overline{AB}$且与$\pi$所成
角为$30^{\circ}$, 若$\overline{AB}=a$, $\overline{AC}=\overline{BD}=b$, 求$C$和$D$间的
距离.
\item 线段
$\overline{DE}$
同时垂直于矩形$ABCD$的两边
$\overline{DA}$、$\overline{DC}$. 设
$\overline{AB}=12$cm, $\overline{BC}=9$cm, $\overline{DE}
=8$cm, 求$B$、$E$两点间
的距离.
\item 一点$P$在两个相交平面上的投影各为$A$、$B$, 求证连线$AB$
垂直于两个平面的交线.
\end{enumerate}


\section{圆}

一个以$O$点为圆心,$R$为半径的圆是平面$\pi$上所有满足
$\Vec{OX}\cdot\Vec{OX}=R^2$的点$X$的集合.即
\[\odot (O,R)=\left\{X:X\in \pi,\; \Vec{OX}\cdot\Vec{OX}=R^2\right\}\]

在平面几何中,关于圆有下列几个基本定理:
\begin{enumerate}
\item 圆周角定理;
\item 弦切角定理;
\item 圆幂定理.
\end{enumerate}

当时,这三个定理是按1、2、3的顺序来证明的,现在我们用向量的运算律,直接证明圆幂定理,
并由此再推出弦切角定理与圆周角定理.

\begin{blk}{定理1}
    如图4.25设平面上有$\odot (O,R)$及圆外一点$P$, 过
$P$点引圆割线交圆于$B$、$C$点,则有$\Vec{PB}\cdot \Vec{PC}=|OP|^2-R^2$.
\end{blk}

\begin{figure}[htp]
    \centering
\begin{tikzpicture}[>=latex, scale=1.6]
\tkzDefPoints{0/0/O, -2.5/0/P}
\tkzDefPoint(80:.5){C1}
\tkzDefPoint(60:1){C2}
\draw(O) circle(1);
\tkzInterLC(C1,P)(O,C2) \tkzGetPoints{B}{C}

\tkzDefPointWith[linear, K=0.8](B,C) \tkzGetPoint{X}
\tkzDrawSegments[->](P,C O,X)
\tkzLabelPoints[above](C,X)
\tkzLabelPoints[left](P)
\tkzLabelPoints[below](O)
\tkzLabelPoints[below left](B)
\draw(P)--(O);
\draw(P)--+(23.58:3.5);
\node at (-.4,.916)[above]{$T_1$};
\tkzDefPointWith[linear, K=0.4](P,B) \tkzGetPoint{X1}
\draw[thick,->](P)--(X1)node[above left]{$\vec{u}$};
\end{tikzpicture}   
    \caption{}
\end{figure}


\begin{proof}
    在$\Vec{PC}$上取单位向量
$\vec{u}$, 对$PC$上任意一点$X$, 都存在实数$x$, 使
$\Vec{PX}=x\vec{u}$.这里$x$就是$\Vec{PX}$
的长度.由向量加法,有
\[\Vec{OX}=\Vec{OP}+\Vec{PX}=\Vec{OP}+x\vec{u}\]
因此:
\[\begin{split}
    \Vec{OX}\cdot \Vec{OX}&=\left(\Vec{OP}+x\vec{u}\right)^2\\
    &=\Vec{OP}\cdot \Vec{OP}+2\left(\Vec{OP}\cdot \vec{u}\right)x+x^2(\vec{u}\cdot \vec{u})\\
    &=x^2+2\left(\Vec{OP}\cdot \vec{u}\right)+|\Vec{OP}|^2
\end{split}\]

当点$X$与点$B$或$C$重合时,$\Vec{OX}\cdot \Vec{OX}=R^2$,这时有
\[x^2+2\left(\Vec{OP}\cdot \vec{u}\right)x+\left(|\Vec{OP}|^2-R^2\right)=0\]
这里$x$是$\Vec{PB}$或$\Vec{PC}$的长度,若令$|\Vec{PB}|=\beta$,
 $|\Vec{PC}|=\gamma$,
则由韦达定理得:
\[\beta\gamma=|\Vec{OP}|^2-R^2\]

$\because\quad \Vec{PB}$与$\Vec{PC}$
同向

$\therefore\quad \Vec{PB}\cdot \Vec{PC}=\beta\gamma$, 因此:
\[\Vec{PB}\cdot \Vec{PC}=|\Vec{OP}|^2-R^2=|PT_1|^2\]
\end{proof}

\begin{analyze}
\begin{enumerate}
    \item 上面的证明对$P$为圆内一点也是适用的,不过
    这时设$\Vec{PX}=x\vec{u}$, 实数$x$可正可负,$|x|=|\Vec{PX}|$.
    
    由于$\Vec{PB}$与
    $\Vec{PC}$反向,则$\Vec{PB}\cdot \Vec{PC}$和$|OP|^2-R^2$都是负值.

\item 在证明过程中,我们还得到
\[\beta+\gamma=-2\Vec{OP}\cdot \vec{u}\]
我们令$B$、$C$、$P'$、$P$成调和点列,即此四点共线
且满足
\[\frac{\Vec{BP}}{\Vec{PC}}\cdot \frac{\Vec{CP'}}{\Vec{P'B}}=-1\]
令$\Vec{PP'}=y\vec{u}$,则有
\[-1=\frac{\Vec{BP}}{\Vec{PC}}\cdot \frac{\Vec{CP'}}{\Vec{P'B}}=\frac{-\beta}{\gamma}\x \frac{y-\gamma}{\beta-y}\]
得:$y=\frac{2\beta\gamma}{\beta+\gamma}$,因此:
\[\begin{split}
    \Vec{PP'}\cdot \Vec{PO}&=\left(\frac{2\beta \gamma}{\beta+\gamma}\vec{u}\right)\cdot \left(-\Vec{OP}\right)\\
    &=\frac{\beta \gamma}{\beta+\gamma}\left(-2\Vec{OP}\cdot \vec{u}\right)\\
    &=\beta\gamma =|\Vec{OP}|^2-R^2
\end{split}\]
这就是说$\Vec{PP'}$在$\Vec{PO}$上的投影的长度等于常数
$\frac{|\Vec{OP}|^2-R^2}{|\Vec{OP}|}$

由于
\[\frac{|\Vec{OP}|^2-R^2}{|\Vec{OP}|}=\frac{|\Vec{PT_1}|^2}{|\Vec{OP}|}=|\Vec{PT_1}|\cos\angle T_1PO=|PQ|\]
其中$Q$是线段$T_1T_2$的中点(图4.26)

$\therefore\quad $过$P$点的任一割线与$T_1T_2$的交
点$P'$与$B$、$C$、$P$成调和点列.
\end{enumerate}
\end{analyze}

下面,我们由圆幂定理推导
弦切角定理与圆周角定理.

\begin{figure}[htp]\centering
    \begin{minipage}[t]{0.48\textwidth}
    \centering
\begin{tikzpicture}[>=latex, scale=1.5]
\tkzDefPoints{0/0/O, -2/0/P}
\tkzDefPoint(120:1){T_1} \tkzDefPoint(-120:1){T_2}
\draw(0,0) circle (1);
\tkzDrawSegments(P,O P,T_1 P,T_2 T_1,T_2)
\tkzLabelPoints[above](P,T_1)
\tkzLabelPoints[below](O,T_2)
\node at (-.5,0)[above right]{$Q$};
    \end{tikzpicture}
    \caption{}
    \end{minipage}
    \begin{minipage}[t]{0.48\textwidth}
    \centering
    \begin{tikzpicture}[>=latex, scale=1.5]
\tkzDefPoints{0/0/O, -2.5/-1/P,0/-1/T}
\tkzDefPoint(-15:1){C} \tkzDefPoint(60:1){C'}
\draw(0,0) circle (1);
\tkzInterLC(C,P)(O,C')  \tkzGetPoints{B'}{B}
\tkzInterLL(C',B)(P,T)  \tkzGetPoint{P'}
\tkzDrawSegments[thick](C,T C',T C,B C',B P,T P,C B,T B,P')
\tkzLabelPoints[below](P,T,P')
\tkzLabelPoints[right](C,C')
\tkzLabelPoints[above](B)
    \end{tikzpicture}
    \caption{}
    \end{minipage}
    \end{figure}


如图4.27由圆幂定理,在
$\triangle PTB$与$\triangle PCT$中,
由于$\frac{\Vec{PT}}{\Vec{PB}}=\frac{\Vec{PC}}{\Vec{PT}}$
,$\angle P$是公共角,

$\therefore\quad \triangle PTB \backsim \triangle PCT$

得$\angle PTB=\angle PCT$,
, 这就是弦切角定理.

同理$\angle P'TB=\angle BC'T$,
则$\angle BCT=\angle BC'T$, 这就是圆周
角定理.


\section*{习题4.3}
\addcontentsline{toc}{subsection}{习题4.3}

\begin{enumerate}
    \item 试证圆的相交弦定理.
    \item 从圆$O$外一点$P$引圆的切线$PT_1$和$PT_2$, $T_1$、$T_2$为切点.
再引圆$O$的割线$PQR$, 交圆$O$于$Q$、$R$, 交$T_1T_2$于$T$, 
设$|PQ|=a$, $|PR|=b$, $|PT|=t$, 求证:
\[\frac{1}{a}+\frac{1}{b}=\frac{2}{t}\]
\end{enumerate}

\section*{复习题四}
\addcontentsline{toc}{section}{复习题四}

\begin{enumerate}

\item 在空间中,设有线性关系$\lambda_1\vec{a}_1+\lambda_2\vec{a}_2+\lambda_3\vec{a}_3+\lambda_4\vec{a}_4+\lambda_5\vec{a}_5=\vec{0}$,且$\lambda_1\lambda_2\lambda_3\lambda_4\lambda_5\ne 0$, 若
\begin{enumerate}
    \item $\vec{a}_1,\vec{a}_2,\vec{a}_3,\vec{a}_4,\vec{a}_5$都是非零向量;
    \item 有且只有$\vec{a}_4=\vec{a}_5=\vec{0}$;
    \item 有且只有三个零向量.
\end{enumerate}
问在各种情况下,它们的几何意义
分别是什么?
\item 试作一给定有向线段$\Vec{AB}$的定比分点,其比值分别为:
$\frac{1}{2},2,-2,-\frac{1}{2}$.
\item 设$P$、$A$、$B$是共线的相异三点,
$\Vec{AP}=\rho\Vec{PB}$, 
试用$\rho$去表达下列五个实数$\alpha$、$\beta$、$\gamma$、$\delta$、$\varepsilon$.
\[\begin{split}
    \Vec{BP}&=\alpha\Vec{PA},\qquad \Vec{PA}=\beta\Vec{AB},\qquad \Vec{BA}=\gamma\Vec{AP}\\
    \Vec{PB}&=\delta\Vec{BA},\qquad \Vec{AB}=\varepsilon\Vec{BP}
\end{split}\]
\item 如图,$\ell_1,\ell_2$交于$O$点,$\vec{u},\vec{v}$是$\ell_1,\ell_2$方向的单位向量,
设
\[\begin{split}
    \Vec{OA_1}=\alpha_1\vec{u},\qquad  \Vec{OB_1}=\beta_1\vec{u},\qquad  \Vec{OC_1}=\gamma_1\vec{u}\\
    \Vec{OA_2}=\alpha_2\vec{v},\qquad  \Vec{OB_2}=\beta_2\vec{v},\qquad  \Vec{OC_2}=\gamma_2\vec{v}
\end{split}\]
试用$\vec{u},\vec{v}$的线性组合表示
$\Vec{OP}$、$\Vec{OQ}$、$\Vec{OR}$. 并证明$P$、$Q$、$R$三点共线.

\begin{figure}[htp]
    \centering
\begin{tikzpicture}[>=latex]
\tkzDefPoints{0/0/O, 2/0/A_2, 3.5/0/B_2, 5/0/C_2}
\tkzDefPoint(40:2.5){A_1}\tkzDefPoint(40:4){B_1}
\tkzDefPoint(40:5.5){C_1}
\tkzLabelPoints[above](A_1,B_1,C_1)
\tkzLabelPoints[below](A_2,B_2,C_2)
\draw(0,0)node[left]{$O$}--(6,0)node[right]{$\ell_2$};
\draw(0,0)--(40:6)node[right]{$\ell_1$};
\tkzDrawSegments(A_1,C_2 A_1,B_2 B_1,A_2 B_1,C_2 C_1,A_2 C_1,B_2)
\draw[thick,->](0,0)--node[below]{$\vec{v}$}(1,0);
\draw[thick,->](0,0)--node[above]{$\vec{u}$}(40:1);

\tkzInterLL(A_1,B_2)(A_2,B_1) \tkzGetPoint{P};
\tkzInterLL(A_1,C_2)(A_2,C_1) \tkzGetPoint{Q};
\tkzInterLL(C_1,B_2)(C_2,B_1) \tkzGetPoint{R};
\tkzDrawPoints(P,Q,R)
\tkzLabelPoints[right](R)
\tkzLabelPoints[left](P)
\tkzLabelPoints[above](Q)

\end{tikzpicture}
    \caption*{第4题}
\end{figure}


\item 在$\triangle ABC$的外面作正方形$ABEF$和$ACGH$, 又设$D$为
$\Vec{BC}$的中点,求证:
\begin{enumerate}
    \item $\Vec{AF}\cdot \Vec{AH}=\Vec{AB}\cdot \Vec{AC}$
    \item $BH\bot CF$且$\overline{BH}=\overline{CF}$
    \item $AD\bot FH$且$\overline{AD}=\frac{1}{2} \overline{FH}$
\end{enumerate}

\item 已知四边形$ABCD$内接于圆且$AC\bot BD$于$E$, $F$是边
$\Vec{BC}$的中点,求证:$EF\bot AD$
\item 已知$O$、$M$、$H$三点分别是$\triangle ABC$的外心,重心和
垂心,求证:$O$、$M$、$H$三点共线且$\overline{OM}=\frac{1}{2}\overline{MH}$.
\item 求证:连结四面体的一个顶点和这个顶点所对的面的重
心的四条线段交于同一点,且这交点分线段的比例都
是3:1.
\item 求证平行六面体的四条对角线相交于一点.
\item 在四面体$ABCD$中,如果$AB\bot DC$且$AD\bot BC$, 试
证明:
\[|\Vec{AB}|^2+|\Vec{DC}|^2=|\Vec{AD}|^2+|\Vec{BC}|^2=|\Vec{AC}|^2+|\Vec{BD}|^2\]
\item 已知四面体$ABCD$, $G_1,G_2$分别是$\triangle ABC$和$\triangle ABD$的
重心,$M$是棱$CD$的中点,试确定过$G_1$、$G_2$、$M$三点的
平面与棱$AB$的交点的位置.
\item 已知正方体$ABCD-A_1B_1C_1D_1$的棱长为1, $E$、$F$分别
是棱$\overline{BC}$, $\overline{CC_1}$的中点,求下列各异面直线的距离.
\begin{multicols}{3}
    \begin{enumerate}
        \item $AA_1$与$BD_1$
        \item $AC$与$BD_1$
        \item $AC$与$EF$
    \end{enumerate}
\end{multicols}
\end{enumerate}